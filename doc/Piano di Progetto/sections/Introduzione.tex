\newpage
\section{Introduzione} \label{Introduzione}
	
	\subsection{Scopo del documento}
	Questo documento ha l'intento di specificare la pianificazione e l'approccio che AlphaSix adotterà per portare a termine il progetto Butterfly.
	All'interno vengono illustrate le strategie, le suddivisioni dei compiti, l'utilizzo delle risorse, la gestione dei rischi e le attività secondo le quali il gruppo ha intenzione di lavorare.
	
	\subsection{Scopo del prodotto}	
	Il prodotto che AlphaSix si incarica di realizzare è composto da ... tool per prendere i messaggi , tool per analizzarli e mandarli verso gestore interno o direttamente alle perosne interessate tramite canale di informazione che è stato scelto ...
	
	\subsection{Glossario}
		\subsubsection{Riferimenti Normativi}
			\begin{itemize}
				\item Capitolato d'appalto C1 (aggiungere link)
				\item Vincoli di organigramma (aggiungere link a sito di tullio)
				\item Norme di progetto interne di AlphaSsix 
				\item The twelve factor app, https://12factor.net/, consigliato fortemente dall'azienda
			\end{itemize}
		
		\subsubsection{Riferimenti Informativi}
			\begin{itemize}
				\item Software engineering di sommerville
				\item slide di tullio
			\end{itemize}
		
	\subsection{Scadenze}
	Il gruppo ha deciso di rispettare le scadenze indicate dal professor Vardanega e riportate di seguito:
	\begin{itemize}
		\item Revisione dei Requisiti: 2012-12-21;
		\item Revisione di Progetto: 2013-02-03;
		\item Revisione di Qualifica: 2013-03-06;
		\item Revisione di Accettazione: 2013-03-16.
	\end{itemize}
	
	\subsection{Ciclo di vita}
	Il gruppo ha deciso di adottare il modello incrementale come ciclo di vita da adottare per i processi in quanto si adatta particolarmente per questo progetto.
	Ogni ripetizione del ciclo identifica un "ciclo di incremento": questo verrà ripetuto fino a quando il prodotto non arriverà a soddisfare interamente i requisiti richiesti dal cliente.
	Questa decisione è stata presa in quanto:
	\begin{itemize}
		\item le richieste presentate nel capitolato sono facilmente scomponibili in sottoproblemi categorizzabili in base alla loro importanza e di conseguenza possono essere soddisfatti partendo da quelli più critici a quelli meno critici anche parallelamente
		\item il passaggio da un incremento ad un altro consolida il processo appena concluso riducendo il rischio di fallimento e facilitando la verifica parziale del prodotto
		\item la ripartizione in cicli permette un controllo più efficente delle risorse pianificandone l'uso
	\end{itemize}
	Addottando questa modalità il gruppo può rilasciare al committente un prototipo (funzionale) che soddisfa requisiti più importanti fra quelli presentati nel capitolato.
	
	Inizialmente quindi si potranno spendere le risorse nella realizzazione di una base (baseline) che possa presentare il core del prodotto finale.
	
	A tale base(line) saranno successivamente integrate le funzionalità secondarie richieste dal cliente nel capitolato e secondo la pianificazione e le risorse disponibili potranno essere aggiunte altre funzionalità opzionali e desidarabili che possono sorgere nello svuiluppo del prodotto o nel durante delle analisi dei requisiti (da rivedere tutto il paragrafo)
	