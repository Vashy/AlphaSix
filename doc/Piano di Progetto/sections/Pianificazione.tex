\newpage
\section{Pianificazione} \label{Pianificazione}
	La fase di Pianificazione consiste nella suddivisione del lavoro. Questsa deve fare in modo che ogni componente del gruppo abbia la possibilità di ricoprire almeno una volta tutti i ruoli del progetto. È stato deciso dal gruppo di utilizzare lo standard ISO/IEC 12207:1995, questo prevede l'utilizzo della seguente struttura:
	\begin{itemize}
			\item Analisi
			\item Progettazione
			\item Realizzazione
			\item Manutenzione
		\end{itemize}
	Per rendere più chiara l'idea utilizzeremo vari diagrammi di Gantt dove sarà chiaro chi ha svolto qualsiasi attività.	
		\subsection{Analisi vista dall'alto / generale}
		
		Questa attività ha inizio il 15-11-2018 e termina il 21-01-2019. In questa fase i ruoli attivi saranno: 
		\begin{itemize}
			\item Project Manager
			\item Amministratore
			\item Analista
			\item Verificatore
		\end{itemize}
		In questo periodo le attività svolte saranno:
		\begin{itemize}
			\item \textbf{Norme di progetto:} questa attività spetta all'Amministratore. Egli dovrà stipulare una serie di norme che dovrannno essere rispettate dai componenti del gruppo per tutta la durata del progetto. Le quali sono interne al gruppo e non legate al capitolato scelto. Questa attività deve essere svolta a priori così da garantire uno standard per la stesura dei doumenti e per quanto riguarda le tecnologie usate.
			\item \textbf{Studio di fattibilità:} questa attività valuta rischi, costi e benefici dei vari capitolati proposti. Viene svolta dagli Analisti e in base ad un accurato studio emergono pro e contro di ogni capitolato. Questi dati sono vari e spesso incerti. Una volta analizzati, portano alla scelta definitiva del capitolato da implementare. L'attività risulta bloccante per l'inizio della stesura dell'Analisi dei requisiti.
			\item \textbf{Analisi dei requisiti:} questa attività richiede un grande approfondimento dei requisiti. questi devono essere soddisfacibili, necessari e sufficenti. Questa responsabilità spetta agli Analisti. 
			\item \textbf{Piano di progetto:} questa attività viene svolta dal Responsabile che avendo bene in mente le scadenze analizza le attività necessarie, l'Amministratore invece studierà i rischi che il gruppo potrà dover affrontare durante lo svolgimento del progetto. L'obbiettivo è quello di organizzare attività con efficenza per produrre risultati efficaci.
			\item \textbf{Piano di qualifica:} questa attività è svolta dagli Amministratori, dove vengono studiate le strategie di verifca e validazione adottate.
			\item \textbf{Glossario:} questa attività viene svlota dai redattori. Consiste nell'inserimento di termini che protrebbero risultare ambigui, così aiuta a garantire una terminologia consistente.
			\item \textbf{Lettera di presentazione:} questa attività consiste nella stesura del documento dove il gruppo AlphaSix si presenta come Fornitore del prodotto richiesto.
		\end{itemize}
		
			\subsubsection{Diagramma di gantt}
			\subsubsection{WBS (Work Breakdown Structure)}
			\subsubsection{Ripartizione delle ore}
				vedere don't panic, hanno messo una tabella con ore ripartite
				Secondo me questa sezione comprende la parte di studio personale messa da sweefty
				
		\subsection{Analisi in dettaglio}
			\subsubsection{Diagramma di gantt}
			\subsubsection{WBS (Work Breakdown Structure)}
			\subsubsection{Ripartizione delle ore}

		\subsection{Progettazione architetturale}
			\subsubsection{Diagramma di gantt}
			\subsubsection{WBS (Work Breakdown Structure)}
			\subsubsection{Ripartizione delle ore}
			
		\subsection{progettazione di dettaglio e codifica}
			\subsubsection{Diagramma di gantt}
			\subsubsection{WBS (Work Breakdown Structure)}
			\subsubsection{Ripartizione delle ore}
			
		\subsection{Verifica e validazione}
			\subsubsection{Diagramma di gantt}
			\subsubsection{WBS (Work Breakdown Structure)}
			\subsubsection{Ripartizione delle ore}
			
		
