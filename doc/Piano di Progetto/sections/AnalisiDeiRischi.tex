\newpage
\section{Analisi dei rischi} \label{AnalisiDeiRischi}
	
	Meglio così a paragrafi (e alla fine di ciascuno mettere le contromisure) o fare una tabella con tutti i rischi?
	
	In modo tale da sviluppare il progetto in maniera efficente ed efficace (da mettere queste parole a glossario?) viene effettuata un'analisi preliminare dei rischi che possono "mettere i bastoni fra le ruote" / deviare il processo di sviluppo.
	E' quindi fondamentale avere un sistema che ne permetta la gestione in maniera da poter agire immediatamente e con un piano ben preciso in caso (se e quando) dovessero presentarsi.
	Per poterli identificare e classificare viene seguita la seguente procedura (rivedere):
	
	DA FARE IN TABELLA
	
	\begin{itemize}
		\item Identificazione
		\begin{itemize}
			\item Progetto
			\item Prodotto
			\item Business
		\end{itemize}
		\item Analisi
		\item Valutazione
		\item Pianificazione di controllo
		\item Trattamento / Mitigazione
		\item Controllo e revisione dei rischi (cosa abbiamo imparato?)
	\end{itemize}
	
	\subsection{Classi di gravità}
	Qualitative risk assessment
	Prendere spunto da questa pagina
	\href{https://www.google.com/url?sa=i&rct=j&q=&esrc=s&source=images&cd=&cad=rja&uact=8&ved=2ahUKEwiKvMy9-4rfAhVBMewKHYJYDuIQjRx6BAgBEAU&url=https%3A%2F%2Fstudy.com%2Facademy%2Flesson%2Fqualitative-risk-analysis-benefits-limitations.html&psig=AOvVaw2oFLtWl4og20ZR80HOAo79&ust=1544177442887418}{qui}
	
	I rischi che possono esserci all'interno del progetto rientrano in queste categorie
	
	\subsection{Livello tecnologico}
	\subsubsection{Tecnologie adottate}
	\subsubsection{Rotture Hardware}
	
	\subsection{Livello del personale}
	\subsubsection{Problemi dei componenti del gruppo}
	\subsubsection{Problemi tra componenti del gruppo}
	\subsubsection{Inesperienza del gruppo}
	
	\subsection{Livello organizzativo e di valutazione dei costi}
	\subsection{Livello dei requisiti}
