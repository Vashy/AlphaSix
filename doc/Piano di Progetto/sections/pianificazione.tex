\newpage
\section{Pianificazione}\label{Pianificazione}
    La fase di Pianificazione\G consiste nella suddivisione del lavoro tra i vari membri del gruppo. Essa deve fare in modo che ogni componente abbia la possibilità di ricoprire almeno una volta tutti i ruoli del progetto.
    È stato deciso dal gruppo di utilizzare lo standard ISO/IEC 12207:1995\GAlt, che prevede l'utilizzo della seguente struttura:
    \begin{itemize}
        \item Analisi
        \item Progettazione
        \item Realizzazione
        \item Manutenzione
    \end{itemize}
    Per rendere più chiara l'idea utilizzeremo vari diagrammi di Gantt\G dove sarà chiaro chi ha svolto qualsiasi attività.

    \subsection{Analisi vista dall'alto / generale}
        Questa fase ha inizio il 15-11-2018 e termina il 14-01-2019 con la consegna dei documenti in ingresso per la Revisione dei Requisiti. I ruoli attivi saranno: 
        \begin{itemize}
            \item Responsabile
            \item Amministratore
            \item Analista
            \item Verificatore
        \end{itemize}
        In questo periodo i documenti da redigere sono:
		\begin{itemize}
			\item \textbf{Norme di progetto}: spetta all'Amministratore. Egli dovrà stipulare una serie di norme
            che dovrannno essere rispettate dai componenti del gruppo per tutta la durata del progetto\GAlt,
            le quali sono interne e non legate al capitolato scelto. La normazione deve essere svolta a priori così da garantire uno standard
            per la stesura dei doumenti e per quanto riguarda le tecnologie usate.
			\item \textbf{Studio di fattibilità}: valuta rischi, costi e benefici dei vari capitolati proposti.
            Viene svolto dagli Analisti e in base ad un accurato studio emergono pro e contro di ogni capitolato. Questi dati
            sono vari e spesso incerti. Una volta analizzati, portano alla scelta definitiva del capitolato da implementare.
            L'attività risulta bloccante per l'inizio della stesura dell'analisi dei requisiti.
			\item \textbf{Analisi dei requisiti}: richiede un grande approfondimento dei requisiti, che devono essere
            soddisfacibili, necessari e sufficenti. Questa responsabilità spetta agli analisti.
			\item \textbf{Piano di progetto}: viene svolta dal responsabile che avendo bene in mente le scadenze analizza le
            attività necessarie, l'amministratore invece studierà i possibili rischi che potrebbero verificarsi durante lo
            svolgimento del progetto. L'obiettivo è quello di organizzare attività con efficenza per produrre risultati efficaci.
			\item \textbf{Piano di qualifica}: è svolta dagli amministratori. Vengono studiate le strategie di verifica
            e validazione adottate.
			\item \textbf{Glossario}: è scritto dai redattori. Consiste nell'inserimento di termini che protrebbero
            risultare ambigui, aiutando a garantire una terminologia coerente.
			\item \textbf{Lettera di presentazione}: consiste nella stesura del documento dove il gruppo AlphaSix si presenta
            come fornitore\G del prodotto richiesto.
		\end{itemize}
		
            \subsubsection{Diagramma di gantt}
            \subsubsection{WBS (Work Breakdown Structure)}
            \subsubsection{Ripartizione delle ore}
                \textbf{TODO} vedere don't panic, hanno messo una tabella con ore ripartite
                Secondo me questa sezione comprende la parte di studio personale messa da sweefty
				
        \subsection{Analisi in dettaglio}
            \subsubsection{Diagramma di gantt}
            \subsubsection{WBS (Work Breakdown Structure)}
            \subsubsection{Ripartizione delle ore}

        \subsection{Progettazione architetturale}
            \subsubsection{Diagramma di gantt}
            \subsubsection{WBS (Work Breakdown Structure)}
            \subsubsection{Ripartizione delle ore}

        \subsection{progettazione di dettaglio e codifica}
            \subsubsection{Diagramma di gantt}
            \subsubsection{WBS (Work Breakdown Structure)}
            \subsubsection{Ripartizione delle ore}

        \subsection{Verifica e validazione}
            \subsubsection{Diagramma di gantt}
            \subsubsection{WBS (Work Breakdown Structure)}
            \subsubsection{Ripartizione delle ore}
