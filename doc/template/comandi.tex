% Generali
\newcommand{\progetto}{Butterfly}
\newcommand{\gruppo}{AlphaSix}
\newcommand{\email}{alpha.six.unipd@gmail.com}

% Documenti
\newcommand{\AdR}{Analisi dei Requisiti}
\newcommand{\NdP}{Norme di Progetto}
\newcommand{\PdP}{Piano di Progetto}
\newcommand{\SdF}{Studio di Fattibilità}
\newcommand{\PdQ}{Piano di Qualifica}
\newcommand{\VI}{Verbale Interno}
\newcommand{\VE}{Verbale Esterno}
\newcommand{\ST}{Specifica Tecnica}
\newcommand{\DDP}{Definizione di Prodotto}
\newcommand{\MU}{Manuale Utente}
\newcommand{\Gl}{Glossario}
\newcommand{\LdP}{Lettera di Presentazione}
\newcommand{\AdRv}{AnalisiDeiRequisiti v2.0.0}
\newcommand{\NdPv}{NormeDiProgetto v2.0.0}
\newcommand{\PdPv}{PianoDiProgetto v2.0.0}
\newcommand{\PdQv}{PianoDiQualifica v2.0.0}
\newcommand{\SdFv}{StudioDiFattibilità v1.0.0}
\newcommand{\DdPv}{DefinizioneDiprodotto v1.0.0}
\newcommand{\Glv}{Glossario v2.0.0}

% Componenti del gruppo
\newcommand{\LC}{Laura Cameran}
\newcommand{\TG}{Timoty Granziero}
\newcommand{\CV}{Ciprian Voinea}
\newcommand{\SG}{Samuele Gardin}
\newcommand{\NC}{Nicola Carlesso}
\newcommand{\MM}{Matteo Marchiori}

% Ruoli
\newcommand{\RdP}{Responsabile di Progetto}
\newcommand{\Res}{Responsabile}
\newcommand{\Red}{Redattore}
\newcommand{\Amm}{Amministratore}
\newcommand{\Ver}{Verificatore}
\newcommand{\Prog}{Progettista}
\newcommand{\Progr}{Programmatore}
\newcommand{\Ana}{Analista}
\newcommand{\RdPs}{Responsabili di Progetto}
\newcommand{\Ress}{Responsabile}
\newcommand{\Amms}{Amministratori}
\newcommand{\Vers}{Verificatori}
\newcommand{\Progs}{Progettisti}
\newcommand{\Progrs}{Programmatori}
\newcommand{\Anas}{Analisti}

% Professori e proponente
\newcommand{\TV}{Prof. Tullio Vardanega}
\newcommand{\RC}{Prof. Riccardo Cardin}
\newcommand{\LuC}{Luca Cappelletti}
\newcommand{\DZ}{Davide Zanetti}
\newcommand{\II}{Imola Informatica}
\newcommand{\proponente}{Imola Informatica}

% Comando per una nuova riga nella tabella del diario delle modifiche
\newcommand{\specialcell}[2][c]{%
	\begin{tabular}[#1]{@{}c@{}}#2\end{tabular}}

\renewcommand*\sectionmark[1]{\markboth{#1}{}}
\renewcommand*\subsectionmark[1]{\markright{#1}}

% Pediodi di lavoro 
\newcommand{\AR}{Analisi dei Requisiti}
\newcommand{\AD}{Analisi dei Requisiti in Dettaglio}
\newcommand{\PA}{Progettazione Architetturale}
\newcommand{\PD}{Progettazione di Dettaglio}
\newcommand{\CO}{Codifica}
\newcommand{\VV}{Validazione}

% Revisioni
\newcommand{\RR}{Revisione dei Requisiti}
\newcommand{\RP}{Revisione di Progettazione}
\newcommand{\RQ}{Revisione di Qualifica}
\newcommand{\RA}{Revisione di Accettazione}

\newcommand{\myincludegraphics}[2][]{%
	\setbox0=\hbox{\phantom{X}}%
	\vtop{
		\hbox{\phantom{X}}
		\vskip-\ht0
		\hbox{\includegraphics[#1]{#2}}}}

% Ridefinizione linea per le note a piè di pagina
\renewcommand{\footnoterule}{%
  \kern -3pt
  \hrule width \textwidth height 0.4pt
  \kern 2pt
}

% \colorlet{punct}{red!60!black}
% \definecolor{background}{HTML}{EEEEEE}
% \definecolor{delim}{RGB}{20,105,176}
% \colorlet{numb}{magenta!60!black}
% \lstdefinelanguage{json}{
% 	basicstyle=\small\ttfamily,
% 	numbers=left,
% 	numberstyle=\scriptsize,
% 	stepnumber=1,
% 	numbersep=8pt,
% 	showstringspaces=false,
% 	breaklines=true,
% 	frame=lines,
% 	backgroundcolor=\color{background},
% 	literate=
% 	*{0}{{{\color{numb}0}}}{1}
% 	{1}{{{\color{numb}1}}}{1}
% 	{2}{{{\color{numb}2}}}{1}
% 	{3}{{{\color{numb}3}}}{1}
% 	{4}{{{\color{numb}4}}}{1}
% 	{5}{{{\color{numb}5}}}{1}
% 	{6}{{{\color{numb}6}}}{1}
% 	{7}{{{\color{numb}7}}}{1}
% 	{8}{{{\color{numb}8}}}{1}
% 	{9}{{{\color{numb}9}}}{1}
% 	{:}{{{\color{punct}{:}}}}{1}
% 	{,}{{{\color{punct}{,}}}}{1}
% 	{\{}{{{\color{delim}{\{}}}}{1}
% 	{\}}{{{\color{delim}{\}}}}}{1}
% 	{[}{{{\color{delim}{[}}}}{1}
% 	{]}{{{\color{delim}{]}}}}{1},
% }
% \lstset{language=json}
% \lstset{literate=%
% 	{Ö}{{\"O}}1
% 	{Ä}{{\"A}}1
% 	{Ü}{{\"U}}1
% 	{é}{{\"s}}1
% 	{è}{{\"e}}1
% 	{à}{{\"a}}1
% 	{ö}{{\"o}}1
% }


\definecolor{listinggray}{gray}{0.9}
\definecolor{lbcolor}{rgb}{0.9,0.9,0.9}

\lstset{
  backgroundcolor=\color{lbcolor},
  tabsize=4,
  language=Python,
  captionpos=b,
  frame=single,
  numbers=left,
  numberstyle=\tiny,
  numbersep=5pt,
  breaklines=true,
  showstringspaces=false,
  basicstyle=\footnotesize,
  % identifierstyle=\color{magenta},
  keywordstyle=\bfseries\color[rgb]{0,0,1},
  commentstyle=\color[rgb]{0,0.6,0},
  stringstyle=\color{red}
}

% \definecolor{mygreen}{rgb}{0,0.6,0}
% \definecolor{mygray}{rgb}{0.5,0.5,0.5}
% \definecolor{mymauve}{rgb}{0.58,0,0.82}

% \lstset{ 
%   backgroundcolor=\color{white},   % choose the background color; you must add \usepackage{color} or \usepackage{xcolor}; should come as last argument
%   basicstyle=\footnotesize,        % the size of the fonts that are used for the code
%   breakatwhitespace=false,         % sets if automatic breaks should only happen at whitespace
%   breaklines=true,                 % sets automatic line breaking
%   captionpos=b,                    % sets the caption-position to bottom
%   commentstyle=\color{mygreen},    % comment style
%   deletekeywords={...},            % if you want to delete keywords from the given language
%   escapeinside={\%*}{*)},          % if you want to add LaTeX within your code
%   extendedchars=true,              % lets you use non-ASCII characters; for 8-bits encodings only, does not work with UTF-8
%   firstnumber=1000,                % start line enumeration with line 1000
%   frame=single,	                   % adds a frame around the code
%   keepspaces=true,                 % keeps spaces in text, useful for keeping indentation of code (possibly needs columns=flexible)
%   keywordstyle=\color{blue},       % keyword style
%   language=Octave,                 % the language of the code
%   morekeywords={*,...},            % if you want to add more keywords to the set
%   numbers=left,                    % where to put the line-numbers; possible values are (none, left, right)
%   numbersep=5pt,                   % how far the line-numbers are from the code
%   numberstyle=\tiny\color{mygray}, % the style that is used for the line-numbers
%   rulecolor=\color{black},         % if not set, the frame-color may be changed on line-breaks within not-black text (e.g. comments (green here))
%   showspaces=false,                % show spaces everywhere adding particular underscores; it overrides 'showstringspaces'
%   showstringspaces=false,          % underline spaces within strings only
%   showtabs=false,                  % show tabs within strings adding particular underscores
%   stepnumber=2,                    % the step between two line-numbers. If it's 1, each line will be numbered
%   stringstyle=\color{mymauve},     % string literal style
%   tabsize=2,	                   % sets default tabsize to 2 spaces
%   title=\lstname                   % show the filename of files included with \lstinputlisting; also try caption instead of title
% }

\newcommand{\impl}{\textcolor{Green}{Implementato}}
\newcommand{\implno}{\textcolor{Red}{Non Implementato}}

% G di glossario a pedice, con e senza spazio
\newcommand{\GAlt}{\ped{\tiny{G}}}
\newcommand{\G}{\ped{\tiny{G }}}

% e.g. \gloss{progetto}
\newcommand{\gloss}[1]{%
    {\small \textsc{#1}}\GAlt%
}

% D di documento a pedice, con e senza spazio
% \newcommand{\DAlt}{\ped{\tiny{D}}}
\newcommand{\D}{\ped{\tiny{D}}}

% e.g. \Doc{Norme di Progetto}
\newcommand{\Doc}[1]{\textit{#1}\D}

% Comandi per applicare \Doc con un comando unico
\newcommand{\PdQd}{\Doc{\PdQv}}
\newcommand{\PdPd}{\Doc{\PdPv}}
\newcommand{\NdPd}{\Doc{\NdPv}}
\newcommand{\AdRd}{\Doc{\AdRv}}
\newcommand{\SdFd}{\Doc{\SdFv}}
\newcommand{\Gld}{\Doc{\Gld}}

% Le sottosezioni paragraph, subparagraph ecc.. vengono visualizzate come section
\titleformat{\paragraph}{\normalfont\normalsize\bfseries}{\theparagraph}{1em}{}
\titlespacing*{\paragraph}{0pt}{3.25ex plus 1ex minus .2ex}{1.5ex plus .2ex}

\titleformat{\subparagraph}{\normalfont\normalsize\bfseries}{\thesubparagraph}{1em}{}
\titlespacing*{\subparagraph}{0pt}{3.25ex plus 1ex minus .2ex}{1.5ex plus .2ex}

\titleformat{\subsubparagraph}{\normalfont\normalsize\bfseries}{\thesubsubparagraph}{1em}{}
\titlespacing*{\subsubparagraph}{0pt}{3.25ex plus 1ex minus .2ex}{1.5ex plus .2ex}

\titleformat{\subsubsubparagraph}{\normalfont\normalsize\bfseries}{\thesubsubsubparagraph}{1em}{}
\titlespacing*{\subsubsubparagraph}{0pt}{3.25ex plus 1ex minus .2ex}{1.5ex plus .2ex}


% Indentazione paragrafi rimossa. Per metterla manualmente, precedere il paragrafo con il comando /indent
\newlength\tindent
\setlength{\tindent}{\parindent}
\setlength{\parindent}{0pt}
\renewcommand{\indent}{\hspace*{\tindent}}


% Generazione automatica dei numeri per le versioni
\newcounter{vX} % valore per X in X.Y.Z
\newcounter{vY} % valore per Y in X.Y.Z
\newcounter{vZ} % valore per Z in X.Y.Z
\newcommand{\decrvX}{\addtocounter{vX}{-1}} % Comando per il decremento automatico del counter vZ
\newcommand{\decrvY}{\addtocounter{vY}{-1}} % Comando per decrementare vY
\newcommand{\decrvZ}{\addtocounter{vZ}{-1}} % Comando per decrementare vZ
\newcommand{\addToDiary}[4]{\thevX.\thevY.\thevZ & #1 & #2 & #3 & #4\decrvZ\\} % Comando per generare una riga di diario delle modifiche (\addToDiary{desc}{ruolo}{nominativo}{data})

% Colore righe grigie
\newcommand{\tablegray}{gray!20}

% Stile liste
% \renewcommand\labelitemi{$\circ$} % Bullet, primo livello
% \renewcommand\labelitemii{$\diamond$} % Bullet, primo livello
% \renewcommand\labelitemii{\normalfont\bfseries \textendash} % --, secondo livello
% \renewcommand\labelitemiii{\textasteriskcentered} % *, terzo livello
% \renewcommand\labelitemiv{\textperiodcentered} % ., quarto livello
% \setlist[itemize,2]{label=$\circ$}
% \setlist[itemize,2]{label=$\diamond$}

% Placeholder sui diari
\newcommand{\pl}{Placeholder}

%Comandi per le versioni delle tecnologie
\newcommand{\python}{Python 3.6.7}
\newcommand{\gitlab}{GitLab 11.7}
\newcommand{\redmine}{Redmine 4.0.1}
\newcommand{\kafka}{Apache Kafka 2.12}
\newcommand{\docker}{Docker 18.09}
\newcommand{\telegram}{Telegram (Bot API 4.0)}
\newcommand{\slack}{Slack}
\newcommand{\jenkins}{Jenkins 2.146}


