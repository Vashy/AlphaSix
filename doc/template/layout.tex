\documentclass[a4paper,11pt]{article}

\usepackage{ifthen}
\usepackage[english,italian]{babel}
\usepackage[utf8]{inputenc}
\usepackage[T1]{fontenc}
\usepackage{float}
\usepackage{chapterbib}
\usepackage{graphicx}
\usepackage[a4paper,top=2.5cm,bottom=2.5cm,left=2.5cm,right=2.5cm]{geometry}

\usepackage{hyperref}
\hypersetup{%
	colorlinks=true,
	citecolor=black,
	linkcolor=black,
	urlcolor=black
}

\usepackage{booktabs}
\usepackage{fancyhdr}
\usepackage{totpages}
\usepackage{tabularx, array}
\usepackage{dcolumn}
\usepackage{epstopdf}
\usepackage{booktabs}
\usepackage{fancyhdr}
\usepackage{longtable}
\usepackage{calc}
\usepackage{datatool}
\usepackage[bottom]{footmisc}
\usepackage{listings}
\usepackage{textcomp}
\usepackage{titlesec}
\usepackage{rotating}
\usepackage{multirow}
\usepackage{placeins}
\usepackage{color}

\usepackage[table,usenames,dvipsnames]{xcolor}
% Definizione di nuovi colori da poter usare per le tabelle
\definecolor{lightgray}{gray}{0.9}
\definecolor{lightblue}{rgb}{0.93,0.95,1.0}

% Ridefinizione dell'env tabularx. Il vecchio è utilizzabile con l'env oldtabularx
\let\oldtabularx\tabularx
\let\endoldtabularx\endtabularx
\renewenvironment{tabularx}{\rowcolors{2}{white}{lightgray}\oldtabularx}{\endoldtabularx}

% Ridefinizione dell'env tabular. Il vecchio è utilizzabile con l'env oldtabular
\let\oldtabular\tabular
\let\endoldtabular\endtabular
\renewenvironment{tabular}{\rowcolors{2}{white}{lightgray}\oldtabular}{\endoldtabular}


\usepackage{makecell}
% \usepackage{breakurl}

% ***STILE PAGINA***
\pagestyle{fancy}

% ***INTESTAZIONE***
\rhead{\Large{\progetto} \\ \footnotesize{\documento}}
\lhead{\includegraphics[keepaspectratio = true, width = 25px]{../template/icons/a6(1).png}}

% ***PIÈ DI PAGINA***
\lfoot{\textit{\gruppoLink} \\
\footnotesize{\email}}

\rfoot{\thepage} % per le prime pagine: mostra solo il numero romano
\cfoot{}
\renewcommand{\footrulewidth}{0.4pt}   % Linea sopra il piè di pagina
\renewcommand{\headrulewidth}{0.4pt}  % Linea sotto l'intestazione

% ***INSERIMENTO DI NUOVE SOTTOSEZIONI
\setcounter{secnumdepth}{7} %mostra nel documento fino al livello 8 (1.2.3.4.5.6.7.8)
\setcounter{tocdepth}{7}    % mostra nell'indice fino al livello 8 (1.2.3.4.5.6.7.8)


\makeatletter
\newcounter{subsubparagraph}[subparagraph]
\renewcommand\thesubsubparagraph{%
	\thesubparagraph.\@arabic\c@subsubparagraph}
\newcommand\subsubparagraph{%
	\@startsection{subsubparagraph}    % counter
	{6}                              % level
	{\parindent}                     % indent
	{3.25ex \@plus 1ex \@minus .2ex} % beforeskip
	{0.75em}                           % afterskip
	{\normalfont\normalsize\bfseries}}
\newcommand\l@subsubparagraph{\@dottedtocline{6}{10em}{5.5em}} %gestione dell'indice
\newcommand{\subsubparagraphmark}[1]{}
\makeatother

\makeatletter
\newcounter{subsubsubparagraph}[subsubparagraph]
\renewcommand\thesubsubsubparagraph{%
	\thesubsubparagraph.\@arabic\c@subsubsubparagraph}
\newcommand\subsubsubparagraph{%
	\@startsection{subsubsubparagraph}    % counter
	{7}                              % level
	{\parindent}                     % indent
	{3.25ex \@plus 1ex \@minus .2ex} % beforeskip
	{0.75em}                           % afterskip
	{\normalfont\normalsize\bfseries}}
\newcommand\l@subsubsubparagraph{\@dottedtocline{7}{10em}{6.5em}} %gestione dell'indice
\newcommand{\subsubsubparagraphmark}[1]{}
\makeatother
