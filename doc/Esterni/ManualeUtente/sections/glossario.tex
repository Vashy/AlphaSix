\section{Glossario}\label{glossario}

\lettera{B}

    \parola{Broker}{%
    Componente che gestisce i messaggi inviati da Publisher a Subscriber nel relativo modello architetturale.
    Mette a disposizione i \gloss{Topic} nei quali i Publisher inviano i messaggi, mentre i Subscriber possono
    iscriversi a essi per ricevere i messaggi.
    }\label{Broker}

\lettera{C}

    \parola{Cluster}{
    Cluster è un insieme di computer connessi tra loro tramite una rete telematica.
    }

    \parola{Consumer}{
    Componente applicativa che ha il compito di abbonarsi a determinati Topic su Kafka, recuperandone i messaggi.
    }

    \parola{Container}{
    Con container si intende la componente con la capacità di eseguire più processi e applicazioni in modo separato per sfruttare al meglio l’infrastruttura esistente pur conservando il livello di sicurezza che sarebbe garantito dalla presenza di sistemi separati.
    }

\lettera{D}

    \parola{Docker}{
    Piattaforma che consente di automatizzare il deployment di applicazioni all’interno di container software.
    }

    \parola{Docker Compose}{
    Definisce le configurazioni necessarie per come devono essere eseguite le immagini contenute nei container Docker, i link e le porte esposte verso l’esterno.
    }

\lettera{I}

    \parola{ID}{
    Termine che indica ``identificativo''.
    }

\lettera{J}

    \parola{JSON}{
    Acronimo di ``JavaScript Object Notation'', è un formato adatto all’interscambio di dati fra applicazioni client/server. È basato sul linguaggio JavaScript Standard ma ne è indipendente.
    }

\lettera{K}

    \parola{Kubernetes}{
    Kubernetes è uno strumento open source di orchestrazione e gestione di container. È stato sviluppato dal team di Google ed è uno dei tool più utilizzati a questo scopo.
    }

\lettera{M}

    \parola{Metadato}{
    Particolare dato che descrive insiemi di altri dati.
    }

\lettera{N}

    \parola{Namespace}{Ambiente per organizzare i file sorgente secondo una gerarchia specifica o il loro tipo. Un namespace, ad esempio, risulta essere simile alla gerarchia delle cartelle di un qualunque computer.}

\lettera{P}

    \parola{Payload}{
    Il ``carico utile'' (in inglese payload) è un termine utilizzato (per analogia dal mondo dei trasporti) per indicare la parte di dati trasmessi effettiva che è destinata all'utilizzatore, in contrasto con i metadati e con gli header che servono esclusivamente a far funzionare il protocollo di comunicazione.
    }

    \parola{Plug and play}{Termine usato per indicare una tecnologia di cui non è necessario conoscere il funzionamento per eseguirla, basta collegarla al sistema che la utilizza.}

    \parola{Plugin}{\`E un programma non autonomo che interagisce con un altro programma per ampliarne o estenderne le funzionalità originarie.}

    \parola{Pod}{All'interno di Kubernates un Pod è un insieme di più container con una rete e memoria condivisa. All'interno di un Pod sono contenute anche le varie configurazioni per eseguire i container al suo interno.}

    \parola{Prodotto}{
    Risultato finito e funzionante del lavoro fatto durante tutto lo svolgimento di un progetto. Il prodotto è ciò che viene consegnato al cliente alla fine.
    }

    \parola{Producer}{
    Componente applicativa che ha il compito di raccogliere dei messaggi per pubblicarli su determinati Topic di Kafka.
    }

    \parola{Push}{
    Comando che aggiorna i riferimenti remoti di una o più repository con i riferimenti locali, mandando gli oggetti necessari per allineare il branch remoto a quello locale.
    }

\lettera{R}

    \parola{Rancher}{
    Rancher è un software di gestione di oggetti di Kubernetes che fornisce un’interfaccia grafica più ricca rispetto a quella di Dockstation per gestire i container Docker e offre ulteriori funzionalità.
    }

    \parola{Root path}{
    Con il termine inglese ``root path'' si intende il percorso di root (dalla radice).
    }

\lettera{T}

    \parola{Topic}{
    Equivalente di ``argomento'' in italiano.
    }

\lettera{U}

    \parola{Update}{
    Equivalente di ``aggiornamento'' in italiano.
    }

\lettera{W}

    \parola{Webhook}{Metodo per aumentare o modificare il comportamento di una pagina o applicazione Web con chiamate HTTP esterne in modo semplice, standardizzato e intelligente (callback).}

    % \parola{}{

    % }
    % \parola{}{

    % }
    % \parola{}{

    % }
    % \parola{}{

    % }
    % \parola{}{

    % }
    % \parola{}{

    % }
