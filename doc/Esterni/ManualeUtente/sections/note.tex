\section{Note sulla gestione delle notifiche}\label{note}
Considerando che un messaggio appartiene ad un progetto e possiede un  Topic, un utente riceve una notifica se:
\begin{itemize}
    \item L'utente è iscritto al Topic del messaggio
    \item L'utente è iscritto al progetto del messaggio e possiede la priorità del progetto più alta tra tutti gli utenti iscritti al progetto indicato
    \item L'utente è disponibile il giorno di invio del messaggio
\end{itemize}
In Kafka è presente una coda chiamata ``lostmessages'' nella quale vengono salvati i seguenti messaggi:
\begin{itemize}
    \item Nel momento in cui viene inviata la segnalazione di un progetto o di una label non ancora registrato nel sistema, tale segnalazione non viene inviata a nessuno, questo perché il progetto viene salvato nel Gestore Personale solo dopo la prima segnalazione. Bisogna dunque tenere presente che la prima segnalazione di un nuovo progetto o label di una Issue non verrà inviata ad alcun utente
    \item Nel momento in cui nessun utente è interessato o disponibile a un dato progetto il giorno dell'invio della segnalazione 
\end{itemize}
I messaggi nella coda ``lostmessages'' non possono essere direttamente letti dall'interfaccia del Gestore Personale, ma solo attraverso un Consumer apposito che l'attuale versione di \progetto\ non possiede.

\subsection{Caso di persona non disponibile}
Seguendo le tre regole ad inizio paragrafo è comprensibile capire la logica per l'inoltro delle notifiche per indisponibilità. Per chiarezza ne esplichiamo comunque i meccanismi. \par 
Nel caso in cui una persona a cui dovrebbe arrivare una notifica non sia disponibile (e quindi lo abbia precedentemente notificato nel calendario come descritto in \S\ref{preferenze}), la procedura del possibile inoltro del messaggio avviene nel seguente modo:
\begin{itemize}
    \item Se è presente almeno un altro utente iscritto al Topic legato alla segnalazione, disponibile
    il giorno dell’invio della notifica e con la stessa priorità di progetto, non viene inviata la
    notifica all’utente in questione perché almeno un altro utente riceverà la segnalazione
    \item Se non è disponibile nessun altro utente con la stessa priorità iscritto al Topic legato alla
    segnalazione, la notifica viene inoltrata ad un utente sempre iscritto allo stesso Topic e
    disponibile il giorno dell’invio della segnalazione, ma con una priorità di progetto inferiore
    \item Se non è disponibile alcun utente iscritto al Topic legato alla segnalazione e disponibile
    il giorno dell’invio della notifica, verrà notificata una persona iscritta al progetto indicato
    con priorità massima tra quelli disponibili
\end{itemize}
