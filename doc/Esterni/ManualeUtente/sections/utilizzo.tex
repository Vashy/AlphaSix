\section{Utilizzo di Butterfly}\label{utilizzo}

% TODO: Resto delle attività utile all'amministratore che configura l'applicazione

\subsection{Gestore Personale}

Il Gestore Personale è la componente principale di Butterfly. %che tiene la logica di business di Butterfly.
Esso si può distinguere in due sotto-componenti:

\begin{itemize}
    \item Interfaccia utente
    \item Message processor, che tiene la logica di business di Butterfly
\end{itemize}

Il Message processor non è di pertinenza di questo manuale, per cui verrà discusso l'utilizzo del sistema tramite
l'interfaccia utente.

% \subsection{Interfaccia utente}

\subsubsection{Iscrizione a Butterfly}

Per iscriversi a Butterfly, accedere all'interfaccia web tramite il link

\begin{center}
    \url{link/pannello/utente}
\end{center}

Tramite il form apposito, selezionare il bottone ``Iscrizione'' e inserire delle credenziali valide.

È necessario inserire almeno un campo tra e-mail e Telegram, per avere un'identificativo univoco
all'interno del sistema.

È anche possibile inserire un nuovo utente tramite \hyperref[APIRest]{API Rest}. Vedere la sezione apposita.


\subsubsection{Modifica preferenze}

\subsubsection{Inserimento giorni di indisponibilità}

\subsubsection{Aggiunta progetti e priorità}

\subsubsection{Disiscrizione}


\subsection{API Rest}\label{APIRest}

\newcommand{\homeUrl}{home\_url}

Per Butterfly abbiamo utilizzato lo standard API Rest per la gestione delle risorse.
Viene descritto nelle prossime sezioni come interagire con le API del sistema.

Il root path sottinteso sarà sempre \texttt{\homeUrl/api/v1/}

Ad esempio, per la GET degli user, l'indirizzo sarà:
\begin{center}
    \texttt{GET \homeUrl/api/v1/users}
\end{center}

\subsubsection{Users}

\texttt{Users} è la risorsa che corrisponde agli utenti.
È possibile visualizzare, aggiungere, modificare o rimuovere gli utenti tramite una semplice
richiesta HTTP.

\paragraph{Visualizzazione}

\begin{itemize}
    \item \textbf{Payload di tutti gli utenti}: \texttt{GET /users}
    \item \textbf{Utente specifico}: \texttt{GET /users/<id>}
\end{itemize}

\paragraph{Inserimento}
È possibile inserire un nuovo utente tramite la richiesta
    \begin{center}
        \texttt{POST /users}
    \end{center}

È possibile dare i seguenti campi di tipo stringa alla richiesta, per aggiungere in fase di creazione
i dati:
\begin{itemize}[noitemsep]
    \item \texttt{name}
    \item \texttt{surname}
    \item \texttt{telegram}
    \item \texttt{email}
\end{itemize}
Almeno uno tra i campi \texttt{email} e \texttt{telegram} vanno fornite insieme al payload.


\paragraph{Modifica}

È possibile modificare un utente tramite la richiesta
\begin{center}
    \texttt{PUT /users/<id>}
\end{center}
È possibile dare i seguenti campi di tipo stringa alla richiesta, per aggiungere in fase di creazione
i dati:
\begin{itemize}[noitemsep]
    \item \texttt{name}
    \item \texttt{surname}
    \item \texttt{telegram}
    \item \texttt{email}
\end{itemize}


\paragraph{Rimozione}

È possibile rimuovere un utente dal sistema Butterfly con la richiesta
\begin{center}
    \texttt{DELETE /users/<id>}
\end{center}

Se il campo \texttt{<id>} corrisponde a un ID presente nel sistema, esso verrà rimosso.


\paragraph{Riepilogo}

\begin{table}[H]
    \begin{paddedtablex}[1.3]{\textwidth}{cYY}
        \thead{Metodo HTTP} & \thead{URI} & \thead{Action}\\\toprule
        \texttt{GET} & \texttt{/users} & Restituisce un payload in JSON di tutti gli utenti\\
        \texttt{GET} & \texttt{/users/<id>} & Restituisce un payload in JSON dell'utente che corrisponde a \texttt{<id>}\\
        \texttt{POST} & \texttt{/users} & Inserisce un nuovo utente. È necessario fornire uno tra i campi telegram o email\\
        \texttt{PUT} & \texttt{/users/<id>} & Modifica l'utente corrispondente a \texttt{<id>} con i campi passati nella richiesta\\
        \texttt{DELETE} & \texttt{/users/<id>} & Elimina l'utente corrispondente a \texttt{<id>} dal sistema\\
        \bottomrule
    \end{paddedtablex}
    \caption{Riepilogo delle Rest API per gli Users}
\end{table}


\subsubsection{Projects}

\paragraph{Visualizzazione}


\begin{itemize}
    \item \textbf{Payload di tutti i progetti}: \texttt{GET /projects}
    \item \textbf{Progetto specifico}: \texttt{GET /projects/<id>}
\end{itemize}

\paragraph{Inserimento}
È possibile inserire un nuovo progetto tramite la richiesta
    \begin{center}
        \texttt{POST /projects}
    \end{center}

È possibile dare i seguenti campi di tipo stringa alla richiesta, per aggiungere in fase di creazione
i dati:
\begin{itemize}[noitemsep]
    \item \texttt{url}
\end{itemize}


\paragraph{Modifica}

È possibile modificare un progetto tramite la richiesta
\begin{center}
    \texttt{PUT /projects/<id>}
\end{center}
È possibile dare i seguenti campi di tipo stringa alla richiesta, per aggiungere in fase di creazione
i dati:
\begin{itemize}[noitemsep]
    \item \texttt{url}
\end{itemize}


\paragraph{Rimozione}

È possibile rimuovere un progetto dal sistema Butterfly con la richiesta
\begin{center}
    \texttt{DELETE /projects/<id>}
\end{center}

Se il campo \texttt{<id>} corrisponde a un ID presente nel sistema, esso verrà rimosso.


\paragraph{Riepilogo}

\begin{table}[H]
    \begin{paddedtablex}[1.3]{\textwidth}{cYY}
        \thead{Metodo HTTP} & \thead{URI} & \thead{Action}\\\toprule
        \texttt{GET} & \texttt{/projects} & Restituisce un payload in JSON di tutti i progetti\\
        \texttt{GET} & \texttt{/projects/<id>} & Restituisce un payload in JSON del progetto che corrisponde a \texttt{<id>}\\
        \texttt{POST} & \texttt{/projects} & Inserisce un nuovo progetto\\
        \texttt{PUT} & \texttt{/projects/<id>} & Modifica il progetto corrispondente a \texttt{<id>} con i campi passati nel payload\\
        \texttt{DELETE} & \texttt{/projects/<id>} & Elimina il progetto corrispondente a \texttt{<id>} dal sistema\\
        \bottomrule
    \end{paddedtablex}
    \caption{Riepilogo delle Rest API per i progetti}
\end{table}



\subsection{Piattaforma di messaggistica}

\subsubsection{Email}

Per ricevere i messaggi di Butterfly tramite e-mail, è sufficiente fornire tramite l'interfaccia del Gestore Personale l'e-mail sulla
quale si vuole ricevere la notifica.

\subsubsection{Telegram}

Per ricevere le notifiche via Telegram, è necessario fare un passaggio addizionale. Va fornita l'autorizzazione al bot per poter inviare messaggi
agli utenti. Il bot è raggiungibile al seguente link:
\begin{center}
    \url{http://t.me/ButterflyBot}
\end{center}

Dare il comando \texttt{/start} per dare l'autorizzazione di inoltro dei messaggi al bot.
È necessario inoltre aggiungere tramite l'interfaccia del Gestore Personale il proprio account Telegram.

In qualsiasi momento sarà possibile bloccare il bot in caso non si voglia più ricevere messaggi relativi a
Butterfly su Telegram, tramite le funzionalità dell'applicazione.
