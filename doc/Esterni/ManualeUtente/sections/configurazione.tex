\section{Configurazione}\label{configurazione}

Su richiesta di \II, tutti i servizi sono stati configurati sotto forma di container. È quindi possibile avviarli su macchine fisiche differenti, ma collegate in rete fra loro, con Docker.\\
Sempre sotto richiesta di \II viene utilizzato Rancher come software per la gestione dei container, quindi questa guida verterà principalmente sulla configurazione di \progetto utilizzando Rancher. %TODO riverere frase per ripetizione

\subsection{Requisiti di sistema}

	Non sono necessari particolari requisiti in modo da poter configurare e utilizzare il nostro prodotte, tuttavia valgono i requisiti minimi dei sistemi di terze parti che vengono utilizzati da \progetto.

	\subsubsection{Software}
		\begin{itemize}
			\item Docker\footnote{\url{https://docs.docker.com/v17.09/datacenter/ucp/2.1/guides/admin/install/system-requirements}}: è necessario avere installata e configurata correttamente almeno la versione v18.09.
			\item Docker Compose\footnote{\url{https://docs.docker.com/compose/install/}} :  nel caso si decidesse di utilizzare Docker Compose per l'avvio dei container è consigliata l'installazione e configurazione corretta della versione v3.7.
			\item Kubernetes\footnote{\url{https://kubernetes.io/docs/setup/independent/install-kubeadm}}: nel caso si decidesse di utilizzare Kubernetes per la gestione dei container è consigliata l'installazione e configurazione corretta della versione v1.13.
			\item Rancher\footnote{\url{https://rancher.com/docs/rancher/v2.x/en/installation/requirements/}}: nel caso si decidesse di utilizzare Rancher per la gestione grafica di oggetti Kubernetes contenenti i container Docker, è consigliata l'installazione e configurazione della versione v2.1.4.
			\item Kafka\footnote{\url{https://docs.confluent.io/current/installation/system-requirements.html}}: è necessario avere installata e configurata correttamente almeno la versione v2.12.
			\item GitLab\footnote{\url{https://git.ucd.ie/help/install/requirements.md}} 11.7
			\item Redmine\footnote{\url{https://www.easyredmine.com/faq/technical-info/176-hardware-and-software-requirements-for-server-solution}}: durante lo sviluppo di \progetto abbiamo utilizzato 4.0.1
		\end{itemize}
	
	%TODO rivedere
	GitLab e Redmine sono componenti esterne a \progetto, tuttavia vengono citate in quanto è possibile configurare tutto il progetto in un ambiente unico.\\
	Le versioni specificate possono essere trovate anche nella sezione ``Requisiti di vincolo'' del documento \AdRv.
	
	\subsubsection{Hardware}
	Per \progetto~non sono necessari ulteriori requisiti a livello hardware particolari se non quelli di cui hanno bisogno i software precedentemente elencati.

\subsection{Mappatura delle porte}
Per la configurazione di ciascun tipo servizio abbiamo deciso di dare un range di porte da poter esporre in modo tale da effettuare una separazione a logico.\\
Questa regola non influisce col corretto funzionamento di \progetto ma è solamente per non assegnare le porte in modalità casuale e facilitare le analisi di eventuali errori come anche la facile manutenzione e aggiunta di servizi.

La suddivisione delle porte è la seguente:

\begin{table}[H]
	\centering
	\begin{paddedtablex}[1.3]{\textwidth}{YYY}
		\thead{Servizio} & \thead{Porta inizio} & \thead{Porta fine}\\\toprule
		Software di terze parti& 30000& 30029\\\hline
		Kafka e servizi correlati& 30030& 30059\\\hline
		Producer& 30060& 30089\\\hline
		Consumer& 30090& 30119\\\hline
		Gestore personale& 30120& 30149\\
	\end{paddedtablex}
	\caption{Suddivisione delle porte}
\end{table}

La suddivisione che abbiamo utilizzato durante lo sviluppo ed alla consegna del progetto prevede la seguente esposizione delle porte:

\begin{table}[H]
	\centering
	\begin{paddedtablex}[1.3]{\textwidth}{YYY}
		\thead{Servizio} & \thead{Porta interna} & \thead{Porta esposta}\\\toprule
		Redmine& 3000& 30000\\\hline
		GitLab& 80& 30001\\\hline
		Kafka& 9092& 30030\\\hline
		Producer GitLab& 5000& 30060\\\hline
		Producer Redmine& 5000& 30061\\\hline
		Consumer Telegram& 30090& \\\hline
		Consumer Email& 30091& \\\hline
		Producer Gestore personale& 30120& 5000\\\hline
		Consumer Gestore personale& 30121& \\
	\end{paddedtablex}
	\caption{Configurazione delle porte in fase di sviluppo e consegna}
\end{table}

Le specifiche relative alla configurazione delle porte in Rancher possono essere trovate nella pagina dedicata\footnote{\url{https://rancher.com/docs/rancher/v2.x/en/installation/references/}} sulla sezione della documentazione presente nel loro sito.

\subsection{Configurazione servizi principali container}

	\subsubsection{Redmine} + sottoparagrafo plugin
	gitlab
	kafka
	
	
\subsection{Servizi aggiuntivi}
Aggiungere servizio necessario per il monitoraggio di kafka
Altri servizi per monitoraggio ?