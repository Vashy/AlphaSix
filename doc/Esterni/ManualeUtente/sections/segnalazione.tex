\section{Segnalazione problematiche}

Nel caso dovessero venire riscontrati bug o problematiche relative a \progetto, si prega di segnalarlo tramite una delle seguenti procedure:
\begin{itemize}
    \item Inviare una mail all'indirizzo \href{mailto:alpha.six.unipd@gmail.com}{alpha.six.unipd@gmail.com}. Questa deve essere nel seguente formato:
    \begin{itemize}
    	\item Oggetto: \texttt{[BUTTERFLY PROJECT]: <Nome significativo dell'evento da segnalare>}
    	\item Corpo:
    	\begin{itemize}
    		\item \texttt{[SUMMARY]: <Riepilogo di come è accaduto il fatto da segnalare>}
    		\item \texttt{[TYPE]: <Di tipo: BUG | FIX | ENHANCEMENT >}
    		\item \texttt{[PRIORITY]: <Di tipo LOW | MEDIUM | HIGH>}    		
    		\item \texttt{[SEVERITY]: <Di tipo LOW | MEDIUM | HIGH>}    		
    		\item \texttt{[DATE]: <Data in cui è stato riscontrato in formato ANNO-MESE-GIORNO>}
    		\item \texttt{[DESCRIPTION]: <Descrizione completa del fatto da segnalare>}
    		\item \texttt{[ENVIRONMENT]: <Descrizione del sistema sul quale è successo il fatto>}
   			\item \texttt{[OTHER]: <Altre informazioni utili al report e descrizione di eventuali allegati esemplificativi>}
    	\end{itemize}
    \end{itemize}
    \item Aprire una issue nel progetto \progetto\ su GitHub seguendo, come per le segnalazioni via Email, il seguente formato:
    \begin{itemize}
    	\item Titolo: \texttt{<Nome significativo dell'evento da segnalare>}
    	\item Corpo:
    	\begin{itemize}
    		\item \texttt{[SUMMARY]: <Riepilogo di come è accaduto il fatto da segnalare>}
    		\item \texttt{[TYPE]: <Di tipo: BUG | FIX | ENHANCEMENT >}
    		\item \texttt{[PRIORITY]: <Di tipo LOW | MEDIUM | HIGH>}    		
    		\item \texttt{[SEVERITY]: <Di tipo LOW | MEDIUM | HIGH>}    		
    		\item \texttt{[DATE]: <Data in cui è stato riscontrato in formato ANNO-MESE-GIORNO>}
    		\item \texttt{[DESCRIPTION]: <Descrizione completa del fatto da segnalare>}
    		\item \texttt{[ENVIRONMENT]: <Descrizione del sistema sul quale è successo il fatto>}
    		\item \texttt{[OTHER]: <Altre informazioni utili al report e descrizione di eventuali screenshot esemplificativi presenti>}
    	\end{itemize}
    \end{itemize}
    % TODO nella seconda versione da rilasciare mettere un footnote con il link al progetto in github
\end{itemize}
