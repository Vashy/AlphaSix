\section{Tecnologie interessate}

\section{Strumenti per lo sviluppo} % Non saprei, omettibile?

\subsection{Python}

\subsection{MongoDB}




\subsection{Strumenti per la gestione dei container}

\subsubsection{Docker}

\subsubsection{Rancher}


\subsection{Librerie esterne}

Le librerie esterne usate per Python3 scelte sono tutte installabili con il gestore dei pacchetti di Python \gloss{pip}.

Nota: saranno già installate nel cluster fornito dalla proponente.

\subsubsection{Requests}
Libreria per la gestione delle richieste HTTP. Grazie a essa è possibile effettuare ogni tipo di richiesta (GET, POST, DELETE, PUT).
Viene utilizzata principalmente per effettuare richieste tramite la API di Telegram.

Maggiori informazioni al link

\begin{center}
    \url{http://docs.python-requests.org/en/master/}
\end{center}

Installabile con il comando \texttt{pip install requests}.

\subsubsection{Flask}
Flask è un framework lightweight per l'installazione di web server, scritto in Python.
Viene utilizzato dai \gloss{Producer} per restare in ascolto degli \gloss{webhook} e dal \gloss{Gestore Personale} per istanziare l'interfaccia
web per la gestione del personale.

Maggiori informazioni al link

\begin{center}
    \url{http://flask.pocoo.org/docs/1.0/}
\end{center}

Installabile con il comando \texttt{pip install flask}.

\subsubsection{Flask-restful}
Libreria che estende Flask e semplifica la gestione di API Rest, automatizzando il collegamento tra risorse e le relative tipologie di richiesta.
Viene usata nel Gestore Personale per implementare lo standard API Rest.

Maggiori informazioni al link

\begin{center}
    \url{https://flask-restful.readthedocs.io/en/latest/}
\end{center}

Installabile con il comando \texttt{pip install flask-restful}.

\subsubsection{PyMongo}

Libreria per l'utilizzo di MongoDB con Python. Il database viene utilizzato nel Gestore Personale.

Maggiori informazioni al link

\begin{center}
    \url{https://api.mongodb.com/python/current/}
\end{center}

Installabile con il comando \texttt{pip install pymongo}.

\subsubsection{Pytest}
Pytest è uno dei framework più utilizzati in Python per l'esecuzione dei test. Semplifica di molto la loro gestione e implementazione rispetto al modulo \texttt{unittest}
della libreria standard.

Maggiori informazioni al link

\begin{center}
    \url{https://docs.pytest.org/en/latest/}
\end{center}

Installabile con il comando \texttt{pip install pytest}
