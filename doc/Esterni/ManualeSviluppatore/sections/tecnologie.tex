\section{Tecnologie interessate}
In questa sezione descriviamo le tecnologie che dopo una fase di analisi, abbiamo deciso di utilizzare per lo sviluppo di \progetto
\subsection{Strumenti per lo sviluppo} % Non saprei, omettibile?

\subsubsection{Python}

\subsubsection{MongoDB}
Abbiamo deciso di utilizzare \gloss{MongoDB} Rispetto ad un database relazionale, quale \gloss{MySQL}, perchè non sfrutta una struttura tradizionale basata su tabelle ma si basa sui documenti in stile \gloss{JSON} con schema dinamico, rendendo l’integrazione di dati di alcuni tipi di applicazioni più facile e performante.

\subsection{Strumenti per la gestione dei container}

\subsubsection{Docker}
Abbiamo deciso di utilizzare \gloss{Docker} per la semplicità di utilizzo e per l'adattamento all'architettura a microservizi.
La configurazione avverrà tramite un \gloss{Dockerfile} in cui verranno specificate informazioni come sistema operativo, script di avvio,
numero di istanze ed altri parametri specifici. \\
Docker oltretutto ci consente di creare dei container aventi la capacità di simulare un ambiente virtuale dov'è possibile testare e mantenere le proprie applicazioni, permettendo di aumentare l'efficienza riducendone i costi e simulando l'esecuzione di sistema operativo su una macchina con \gloss{risorse} condivise.\\
A differenza delle \gloss{macchine virtuali}, dove lo stato dell'ambiente viene salvato su disco, occupando memoria, i container si adattano in maniera più performante all'applicativo richiesto, in quanto il loro scopo è quello di massimizzare la quantità delle applicazioni in esecuzione riducendo al minimo il numero delle macchine per eseguirla.
Sono quindi più leggeri, occupando meno memoria su disco e impiegando meno risorse.

\subsubsection{Rancher}
Per la gestione dei container in remoto, la proponte \II~ci ha messo a disposizione un \gloss{cluster} con due server sui quali è stato installato Rancher. Questo è un software di gestione di oggetti di \gloss{Kubernetes}.
Da qui possiamo quindi gestire i nostri container installando le immagini direttamente da \gloss{DockerHub} senza aver bisogno di file di configurazione come ad esempio quello necessario al docker-compose.


\subsection{Librerie esterne}

Le librerie esterne usate per Python3 scelte sono tutte installabili con il gestore dei pacchetti di Python \gloss{pip}.

Nota: saranno già installate nel cluster fornito dalla proponente.

\subsubsection{Requests}
Libreria per la gestione delle richieste HTTP. Grazie a essa è possibile effettuare ogni tipo di richiesta (GET, POST, DELETE, PUT).
Viene utilizzata principalmente per effettuare richieste tramite la API di Telegram.

Maggiori informazioni al link

\begin{center}
    \url{http://docs.python-requests.org/en/master/}
\end{center}

Installabile con il comando \texttt{pip install requests}.

\subsubsection{Flask}
Flask è un framework lightweight per l'installazione di web server, scritto in Python.
Viene utilizzato dai \gloss{Producer} per restare in ascolto degli \gloss{webhook} e dal Gestore Personale per istanziare l'interfaccia
web per la gestione del personale.

Maggiori informazioni al link

\begin{center}
    \url{http://flask.pocoo.org/docs/1.0/}
\end{center}

Installabile con il comando \texttt{pip install flask}.

\subsubsection{Flask-restful}
Libreria che estende Flask e semplifica la gestione di API Rest, automatizzando il collegamento tra risorse e le relative tipologie di richiesta.
Viene usata nel Gestore Personale per implementare lo standard API Rest.

Maggiori informazioni al link

\begin{center}
    \url{https://flask-restful.readthedocs.io/en/latest/}
\end{center}

Installabile con il comando \texttt{pip install flask-restful}.

\subsubsection{PyMongo}

Libreria per l'utilizzo di MongoDB con Python. Il database viene utilizzato nel Gestore Personale.

Maggiori informazioni al link

\begin{center}
    \url{https://api.mongodb.com/python/current/}
\end{center}

Installabile con il comando \texttt{pip install pymongo}.

\subsubsection{Pytest}
Pytest è uno dei framework più utilizzati in Python per l'esecuzione dei test. Semplifica di molto la loro gestione e implementazione rispetto al modulo \texttt{unittest}
della libreria standard.

Maggiori informazioni al link

\begin{center}
    \url{https://docs.pytest.org/en/latest/}
\end{center}

Installabile con il comando \texttt{pip install pytest}
