\section{Glossario}\label{glossario}

\lettera{A}

    \parola{Applicativo}{%
        Programma software con lo scopo di rendere possibili una o più funzionalità, servizi o strumenti utili.
    }\label{Applicativo}

    \parola{API Rest}{%
        Metodi con cui è possibile fornire l'interazione con le componenti del sistema basate su representational state transfer,
        dove le risorse sono uniche e indirizzabili mediante URI.
    }\label{API Rest}

\lettera{B}

    \parola{Broker}{%
        Componente che gestisce i messaggi inviati da Publisher a Subscriber nel relativo modello architetturale.
        Mette a disposizione i \gloss{Topic} nei quali i Publisher inviano i messaggi, mentre i Subscriber possono
        iscriversi a essi per ricevere i messaggi.
    }\label{Broker}

\lettera{C}

    \parola{Cluster}{%
        È un insieme di computer connessi tra loro tramite una rete telematica. Scopo del cluster è distribuire un'elaborazione molto complessa tra i vari computer, aumentando la potenza di calcolo del sistema e/o garantendo una maggiore disponibilità di servizio, a prezzo di un maggior costo e complessità di gestione dell'infrastruttura: per essere risolto il problema che richiede molte elaborazioni viene infatti scomposto in sottoproblemi separati i quali vengono risolti ciascuno in parallelo.
    }\label{Cluster}

\lettera{D}

    \parola{Docker}{%
        Piattaforma che consente di automatizzare il \gloss{deployment} di applicazioni all'interno di \gloss{container} software.
    }\label{Docker}

    \parola{Dockerfile}{%
        Definisce le configurazioni necessarie per il container sul quale si andrà a eseguire l'\gloss{applicativo}.%
    }\label{Dockerfile}

	\parola{DockerHub}{%
        Repository di Docker che permette il versionamento delle immagini e la gestione delle build in maniera automatica innescata al push delle modifiche sulla repository del codice (necessita che queste siano collegate fra loro).
    }\label{DockerHub}

\lettera{E}

    \parola{Event Driven}{%
        È un paradigma di programmazione dell'informatica. Mentre in un programma tradizionale l'esecuzione delle istruzioni segue percorsi fissi, che si ramificano soltanto in punti ben determinati predefiniti dal programmatore, nei programmi scritti utilizzando la tecnica a eventi il flusso del programma è largamente determinato dal verificarsi di eventi esterni.
    }\label{EVM}

\lettera{F}

    \parola{Factory Method}{%
        TODO
    }\label{Factory Method}


\lettera{J}

    \parola{JSON}{%
        Acronimo di JavaScript Object Notation, è un formato adatto all'interscambio di dati fra applicazioni client/server.
        È basato sul linguaggio JavaScript Standard ma ne è indipendente.
    }\label{JSON}

\lettera{K}

	\parola{Kubernetes}{%
        Kubernetes è uno strumento open source di orchestrazione e gestione di container. È stato sviluppato dal team di Google ed è uno dei tool più utilizzati a questo scopo. Kubernetes permette di eliminare molti dei processi manuali coinvolti nel deployment e nella scalabilità di applicazioni contenute in container e di gestire in maniera semplice ed efficiente cluster di host su cui questi vengono eseguiti.
    }\label{Kubernetes}


\lettera{M}

    \parola{Macchina virtuale}{%
        Una macchina virtuale (VM) indica un software che, attraverso un processo di virtualizzazione, crea un ambiente virtuale che emula il comportamento di una macchina fisica (PC client o server) grazie all'assegnazione di risorse hardware ed in cui alcune applicazioni possono essere eseguite come se interagissero con tale macchina.
    }\label{Macchina virtuale}

	\parola{Metadato}{%
        Particolare dato che descrive insiemi di altri dati.
    }\label{Metadato}

	\parola{MongoDB}{%
        È un \gloss{DBMS} non relazionale, orientato ai documenti. MongoDB si allontana dalla struttura tradizionale basata su tabelle dei database relazionali in favore di documenti in stile \gloss{JSON} con schema dinamico.
    }\label{MongoDB}

    \parola{MySQL}{%
        È un software di tipo server avente il compito di gestire uno o più database. Il suo compito è quello di intervenire, in qualità di intermediario, in ogni operazione sui database, gestendo gli accessi ai dati (filtrando quelli non autorizzati), ad eseguire le interrogazioni ed a restituirne il risultato ove previsto.
    }\label{MySQL}

\lettera{O}

    \parola{Open World Assumption}{%
        TODO
    }\label{Open World Assumption}

\lettera{P}

    \parola{PIP}{%
        Il Python Package Index è un repository che contiene decine di migliaia di package scritti in Python. Chiunque può scaricare package esistenti o condividere nuovi package su PIP.
    }\label{PIP}

   	\parola{Producer}{%
        Componente di Butterfly con lo scopo di raccogliere i messaggi e pubblicarli sotto forma di messaggi all'interno dei Topic adeguati.
    }\label{Producer}

\lettera{R}

	\parola{Risorsa}{%
        Soggetto consumabile che può essere di varia natura. Nel caso di un progetto software una risorsa può consistere in ore di lavoro o tecnologie utilizzabili.
    }\label{Risorsa}

\lettera{T}

    \parola{Template Method}{%
        TODO
    }\label{Template Method}

    \parola{Topic}{%
        Equivalente di ``argomento'' in italiano. Nel contesto di un \gloss{Broker} si intende un canale di messaggi associato a uno specifico argomento.
    }\label{Topic}

\lettera{W}

	\parola{Webhook}{
        Metodo per aumentare o modificare il comportamento di una pagina o applicazione Web con chiamate HTTP esterne in modo semplice, standardizzato e intelligente (\gloss{callback}).
    }\label{Webhook}
