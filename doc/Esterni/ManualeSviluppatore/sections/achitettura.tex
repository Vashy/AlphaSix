\section{Architettura}

In questa sezione descriviamo brevemente l'architettura del sistema Butterfly.

L'architettura adottata è \gloss{event-driven}: ogni componente è isolato e asincrono,
e comunica interfacciandosi e scambiandosi messaggi tramite il broker Apache Kafka.

Il sistema si divide principalmente in tre insiemi di componenti:

\begin{itemize}
    \item Producers
    \item Gestore Personale
    \item Consumers
\end{itemize}

\subsection{Producers}
Il Producer è il componente che resta in ascolto degli webhook provenienti dal suo applicativo specifico (e.g. Redmine, GitLab).
Ha lo scopo di immettere i messaggi su Kafka in formato JSON, conservando solo i campi di interesse e aggiungendone eventualmente di propri.

Al momento della stesura di questo manuale, i Producer implementati sono due:
\begin{itemize}
    \item RedmineProducer
    \item GitlabProducer
\end{itemize}

\subsubsection{RedmineProducer}



\subsection{Gestore Personale}
Il Gestore Personale è la componente con la logica più complessa del sistema Butterfly. Ha un proprio KafkaProducer e KafkaConsumer
che si occupano rispettivamente di ricevere i messaggi da Kafka e re-immetterli nelle code relative ai Consumer finali, passando per la logica
del personale, in cui ogni utente può selezionare i propri giorni di indisponibilità, la priorità dei progetti e la piattaforma di messaggistica sul
quale ricevere la notifica.

\subsection{Consumers}
Il Consumer è la componente finale. Esso resta in ascolto della sua coda specifica, specifica per applicativo (e.g. telegram, email).
Si occupa di inoltrare il messaggio al destinatario finale, tramite le API della piattaforma specifica.

Al momento della stesura di questo manuale, i Consumer implementati sono due:
\begin{itemize}
    \item TelegramConsumer
    \item EmailConsumer
\end{itemize}
