\newpage

\subsubsection{Consuntivo di periodo}	\label{consuntivo}
Nella fase di verifica al termine del periodo dell'analisi dei requisiti è stato redatto un consuntivo di periodo al fine di verificare se il numero di ore assegnate ad ogni ruolo è stato rispettato o meno. 

	\paragraph{Analisi dei requisiti}
	Vengono riportate in seguito le ore di lavoro effettive relative al periodo di analisi dei requisiti.
	
	\begin{table}[H]
		\begin{detailtable}{\columnwidth}{m{3cm}YYYYYYY}
			\thead{Membro} & 
			\thead{Re} &
			\thead{Am} &
			\thead{An} &
			\thead{Pj} &
			\thead{Pr} &
			\thead{Ve} &
			\thead{Totale}\\\toprule\rowcolor{\tablegray}
			Ciprian Voinea & 11 (+3) & & 7 (-2) & & & 5 (-2) & 23 (-1)\\
			Laura Cameran & & 8 & 9 & & & 7 & 24\\\rowcolor{\tablegray}
			Matteo Marchiori & 7 (-1) & & 9 & & & 8 (+1) & 24\\
			Nicola Carlesso & 8 & & 10 (+1) & & & 7 & 25 (+1)\\\rowcolor{\tablegray} 
			Samuele Gardin & & 7 (-1) & 9 & & & 7 & 23 (-1)\\ 
			Timoty Graziero & & 7 (-1) & 8 (-1) & & & 9 (+2) & 24\\\bottomrule
		\end{detailtable}
		\caption{Ore consuntivate nel periodo di analisi dei requisiti}
	\end{table}
	
	Viene riportato in seguito il consuntivo relativo al periodo di analisi dei requisiti:
	
	\begin{table}[H]
		\begin{detailtable}{\columnwidth}{m{3cm}YY}
			\thead{Ruolo} & 
			\thead{Totale ore} &
			\thead{Costo in \euro}\\\toprule\rowcolor{\tablegray}
			Responsabile & 26 (+2) & 780,00 (+60,00)\\
			Amministratore & 22 (-2) & 440,00 (-40,00)\\\rowcolor{\tablegray}
			Analista & 52 (-2) & 1300,00 (-50,00)\\
			Progettista & & \\\rowcolor{\tablegray}
			Programmatore & &\\
			Verificatore & 43 (+1) & 645,00 (+15,00)\\\rowcolor{\tablegray}
			Totale & 143 (-1) & 3165,00 (-15,00)\\\bottomrule
		\end{detailtable}
		\caption{Consuntivo del periodo di analisi dei requisiti}
	\end{table}
	
	\paragraph{Conclusioni}
	Non si ritiene opportuno aggiungere osservazioni in merito ai discostamenti di orario rispetto a quanto preventivato.\\
	Questo primo consuntivo di periodo è di utilità per \gruppo\ in modo da poter osservare in modo oggettivo l'andamento del lavoro.\\
	%Inoltre il preventivo iniziale viene presentato con questa prima versione del documento al proponente.
	