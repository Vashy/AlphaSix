\section{Qualità di processo}\label{qualità di processo}

\subsection{Scopo}
La qualità di un prodotto è fortemente influenzata dai processi utilizzati nell'arco di creazione del prodotto stesso: da un tubo sporco non può uscire acqua pulita.

Per questo è necessario operare con un buon \gloss{ciclo di vita} che
determina quali processi attivare e che devono essere verificati e valutati adeguatamente. A questo scopo viene seguito lo schema del Ciclo di Deming e dell'\gloss{ISO 15504} descritti nelle \NdPd.

\subsection{Nomenclatura metriche e obiettivi di qualità}  \label{nomenclatura}
La nomenclatura degli obiettivi di qualità, delle \gloss{metriche} ed il loro funzionamento è spiegato in dettaglio nelle \NdP. In questa sezione gli obiettivi e le metriche vengono sinteticamente descritte:

	\begin{itemize}
		\item \textbf{Obiettivi}:

		\begin{center}
			\texttt{QPR[ID] [Nome]}
		\end{center}

		\begin{itemize}
			\item \textbf{QPR}: indica ``Qualità di processo''.
			\item \textbf{ID}: numero progressivo di tre cifre.
			\item \textbf{Nome}: descrive brevemente il processo.
		\end{itemize}

		\item \textbf{Metriche}:

		\begin{center}
			\texttt{MPR[ID] [Nome]}
		\end{center}

		\begin{itemize}
			\item \textbf{MPR}: indica ``Metrica di processo''.
			\item \textbf{ID}: numero progressivo di tre cifre.
			\item \textbf{Nome}: descrive brevemente la metrica.
		\end{itemize}

	\end{itemize}




\subsection{Processi}
I processi saranno elencati nel seguente modo:

\begin{center}
	\texttt{PROC[ID] [Nome]}
\end{center}

\begin{itemize}
	\item \textbf{PROC}: sta a indicare ``Processo''.
	\item \textbf{ID}: un numero incrementale di tre cifre per indicare in modo univoco il processo.
	\item \textbf{Nome}: una breve frase per indicare la funzione del processo.
\end{itemize}

Per ogni processo sono elencate le sue funzioni principali, gli obiettivi che ci prefiggiamo per ottenere la qualità desiderata e le metriche adottate per raggiungere quell'obiettivo (quando previste).
%Gli obiettivi di qualità elencati, quando è possibile, sono affiancati da una particolare metrica.

	\subsubsection{PROC001 Pianificazione del progetto, organizzazione e struttura}
	Tale \gloss{macro-processo} ha lo scopo di pianificare il lavoro da svolgere per soddisfare i \gloss{requisiti} richiesti dal \gloss{progetto}. È in questo processo che viene messo in atto il nostro \gloss{way of working} che ha un'importanza particolarmente alta perché i suoi risultati vanno a condizionare l'esito della qualità dell'intero progetto.

		\paragraph{Funzioni}

		\begin{itemize}
			\item \textbf{Sviluppare \gloss{sotto-processi}}: i vari obiettivi devono poter essere associati ad azioni ben precise, ognuna delle quali appartiene ad un sotto-processo.
			\item \textbf{Suddividere i compiti}: assegnazione dei compiti che ognuno di noi deve realizzare.
			\item \textbf{Calendarizzare i documenti}: stabilire delle \gloss{baseline} durante il progetto.
			\item \textbf{\gloss{Formazione} personale}: data la nostra inesperienza è richiesto un periodo di formazione personale più lungo del normale. Questo periodo deve essere contato all'interno della calendarizzazione.
			\item \textbf{Standard}: vengono scelti gli standard più convenienti da seguire.
			\item \textbf{\gloss{Budget}}: è necessario conoscere le proprie \gloss{risorse} in termini di tempo $\frac{\text{costo}}{\text{persona}}$ in modo tale da restare il più fedeli possibile al \gloss{preventivo} stilato.
		\end{itemize}

		\paragraph{Metriche}

		\begin{itemize}
			\item MPR001 Varianza della \gloss{pianificazione}
			\item MPR002 Varianza dei costi
			\item MPR003 Aderenza agli standard
			\item MPR004 Frequenza \gloss{commit} nella \gloss{repository}
		\end{itemize}

		\paragraph{Obiettivi}

		\begin{itemize}
			\item \textbf{QPR001 Rispetto dei periodi della pianificazione}: all'interno della presente nel \PdPd, sono presenti le date di scadenza delle varie attività. L'obiettivo è rispettarle il più possibile per proseguire al meglio la realizzazione del progetto.
			\item \textbf{QPR002 Variazione del budget}: le risorse messe a disposizione all'inizio del progetto devono potersi mantenere tali in tutta la sua durata.
			\item \textbf{QPR003 Rispetto delle fasi del ciclo di vita}: ogni processo deve rispettare le fasi del Ciclo di Deming.
			\item \textbf{QPR004 \gloss{Versionamento}}: durante ogni processo, i prodotti che vengono realizzati devono possedere un numero di versione in modo da tener traccia delle modifiche. In questo modo è possibile sapere le cause di certi comportamenti dovuti ad un mutamento del prodotto in modo molto rapido. L'obiettivo è quindi tenere sotto controllo l'andamento dei cambiamenti.
		\end{itemize}

	\subsubsection{PROC002 Analisi}
	Per processo di analisi si intende in questo caso un'attività di analisi generica, di più contenuti, non riferendosi
	solo all'analisi dei requisiti (consultabile al documento \AdRd), ma anche all'analisi dei rischi (presente nel \PdPd)
	e altre attività di identificazione e analisi riportate qui di seguito tra le funzioni.
	La prima parte di questo processo consiste nel poter dire se il progetto risulta fattibile per le sue possibilità in
	termini di competenze e risorse. Tale processo non deve essere rendicontato per il preventivo a finire.

		\paragraph{Funzioni}

		\begin{itemize}
			\item \textbf{Individuare i requisiti}: il capitolato che viene presentato dal cliente può possedere un lessico più discorsivo che tecnico.
			È necessario dunque effettuare un lavoro più approfondito per individuare i requisiti, espliciti ed impliciti, e classificarli.
			%\item \textbf{Individuare le tecnologie da utilizzare}: in base al tipo di progetto le tecnologie da utilizzare migliori per realizzarlo possono cambiare da progetto a progetto.
			\item \textbf{Comprendere la quantità di risorse richieste}: sviluppare un preventivo è fondamentale, sia per il cliente che vuole sapere il costo del progetto, che per il \gloss{fornitore} che deve sapere se il progetto per lui è fattibile.
			\item \textbf{Individuare i rischi}: lungo il corso del progetto possono accadere degli imprevisti, perciò è necessario poterli prevedere ed organizzare fin da subito una reazione ad essi.
			%\item \textbf{Stabilire la fattibilità del progetto}: dopo aver fatto l'analisi dei requisiti, dei rischi e dei costi, è necessario essere consapevoli se si è in grado di iniziare lo sviluppo del progetto o meno.
		\end{itemize}

		\paragraph{Metriche}

		\begin{itemize}
			\item MPR005 Requisiti obbligatori non soddisfatti
			\item MPR006 Requisiti desiderabili non soddisfatti
			\item MPR007 Requisiti opzionali non soddisfatti
			\item MPR008 Rischi non previsti avvenuti
		\end{itemize}
		I risultati di tutte queste metriche possono essere pubblicate ed analizzate solo a progetto terminato, prima avrebbero poca utilità.
		%compiuto; prima avrebbe poca utilità.

		\paragraph{Obiettivi}

		\begin{itemize}
			\item \textbf{QPR005 Soddisfacimento dei requisiti obbligatori}: tutti i requisiti obbligatori devono essere soddisfatti a fine progetto.
			\item \textbf{QPR006 Soddisfacimento dei requisiti opzionali e desiderabili}: i requisiti opzionali e desiderabili possono essere soddisfatti
				solo se si avanzano ancora risorse in termini di $ \frac{\text{tempo}}{\text{persona}}$ prima del definitivo termine del progetto.
				I requisiti inoltre devono essere stabiliti in modo corretto: hanno l'obbligo di rispecchiare quanto chiesto dal cliente e non sovrapporsi tra loro.
			\item \textbf{QPR007 Verificarsi dei rischi previsti}: lo sviluppo del progetto dovrebbe procedere senza incertezze e non presentare imprevisti non precedentemente studiati. Questo perché potrebbe aumenterebbe di molto il ritardo nello sviluppo del progetto rispetto al previsto.
		\end{itemize}

	\subsubsection{PROC003 Produzione documenti}
	Questo processo rimane attivo in tutta la durata dello sviluppo del prodotto perché ha il compito di produrre dei documenti che riportino le scelte effettuate, gli strumenti utilizzati e le modifiche attuate nell'intera durata del progetto.

		\paragraph{Funzioni}
		Il processo comprende il ciclo di vita di ogni documento, che è spiegato in dettaglio nelle \NdP, e prevede:

		\begin{itemize}
			\item Redazione
			\item Verifica
			\item Approvazione
		\end{itemize}

		\paragraph{Obiettivi}

		\begin{itemize}
			\item \textbf{QPR008 Rispetto delle fasi del ciclo di vita}: il ciclo di vita di ogni documento deve essere rispettato in ogni sua fase e questo deve venir pubblicato secondo le scadenze prestabilite.
		\end{itemize}

	\subsubsection{PROC004 Verifica}
	Il processo, attivo in tutta la durata del progetto, ha il compito di valutare la correttezza dei prodotti dati in input, stabilire se presentano errori
	e se sono sufficientemente di qualità.

		\paragraph{Funzioni}

		\begin{itemize}
			\item \textbf{Verificare le funzionalità dei prodotti}: i prodotti devono saper soddisfare i requisiti richiesti. Nella fase di verifica viene accertato che questo effettivamente avvenga.
			\item \textbf{Verificare la corretta esecuzione dei processi}: i processi possiedono un ciclo di vita che è suddiviso in fasi, per la buona esecuzione del processo queste devono essere eseguite nell'ordine corretto restituendo gli output attesi.
			\item \textbf{Verificare che siano rispettate le \NdP}: viene stabilito un nostro Way of Working che dobbiamo rispettare per tutta la durata del progetto. Anche la verifica di queste norme deve essere effettuata.
		\end{itemize}

		\paragraph{Metriche}

		\begin{itemize}
			\item MPR009 Frequenza controllo prodotti
		\end{itemize}

		\paragraph{Obiettivi}

		\begin{itemize}
			\item \textbf{QPR009 Effettuare una verifica costante}: la fase di verifica è sempre attiva perché ogni singolo prodotto deve essere testato e controllato ogni volta che viene modificato o prima dell'arrivo di una \gloss{milestone}.
			\item \textbf{QPR010 Rispetto delle fasi di verifica}: la fase di verifica deve essere eseguita in iterativamente senza cambiare in modo sostanziale nel corso del progetto. Un mutamento della fase di verifica renderebbe incoerenti i risultati delle precedenti iterazioni, impedendo di vedere se è presente un miglioramento della qualità o meno.
		\end{itemize}

    \subsubsection{PROC005 Progettazione}\label{proc005}
    Tale processo si attiva successivamente al processo di analisi e si occupa di restituire la progettazione completa dei prodotti che verranno infine eseguiti. \par
    La progettazione è fondamentale per tracciare il percorso di esecuzione del processo di codifica, onde per cui deve essere eseguito con criterio e cautela.

        \paragraph{Funzioni}
        \begin{itemize}
            \item \textbf{Progettare i moduli prima di codificarli}: specialmente alle prime esperienze di progetti come quello che stiamo svolgendo ora, è facile cadere nella tentazione di codificare un \gloss{modulo} prima di averlo progettato. In questo modo però, spesso non si realizza un modulo nell'ottica di renderlo estensibile, mantenibile e disaccoppiato dagli altri moduli. La progettazione serve invece ad evitare questi difetti in un prodotto, per questo è meglio svolgerla prima della codifica.
            \item \textbf{Applicazione ponderata dei design pattern}: i \gloss{design patter} sono molto utili ai fini della progettazione per ottenere più facilmente le caratteristiche descritte al punto precedente. La loro applicazione però rischia di aggiungere eccessiva complessità al prodotto, per questo un design patter deve essere applicato solo se aiuta nella realizzazione del progetto, non perché un alto numero di design pattern equivale ad una buona progettazione. Dunque il loro utilizzo deve essere ragionato e non eccessivo.
            \item \textbf{Documentazione e tracciamento della progettazione}: un prodotto, per essere mantenibile, deve essere documentato anche nella sua struttura. Dunque anche la progettazione deve essere documentata e tracciata attraverso i \MUd\ e \MSd.
        \end{itemize}

        \paragraph{Metriche}
        \begin{itemize}
            \item MPR010 Moduli codificati prima della progettazione
            \item MPR011 Numero di design pattern applicati
        \end{itemize}

        \paragraph{Obiettivi}
        \begin{itemize}
            \item \textbf{QPR011 Progettare prima di codificare}: ogni modulo dovrebbe essere progettato in relazione agli altri del progetto prima di essere codificato.
            \item \textbf{QPR012 Usare un numero limitato di design pattern}: usare design pattern solo nel caso in cui questi semplifichino la struttura del progetto.
        \end{itemize}

    \subsubsection{PROC006 Codifica}\label{proc006}
    Successivamente alla progettazione avviene la realizzazione del prodotto finale che verrà rilasciato. Il processo di codifica, insieme a quello di verifica, restituisce il prodotto che verrà usato e mantenuto. La codifica di ogni modulo deve trattarsi della ``traduzione'' in codice della progettazione.

        \paragraph{Funzioni}
        \begin{itemize}
            \item \textbf{Produrre codice comprensibile}: il prodotto rilasciato in futuro potrà essere usato da altre persone, per questo il codice deve poter essere facilmente comprensibile.
            \item \textbf{Produrre codice mantenibile}: le componenti del prodotto nel tempo potranno risultare obsolete, per questo le nuove tecnologie devono poter essere applicate all'interno del prodotto.
            \item \textbf{Produrre codice estensibile}: il prodotto deve poter concedere la possibilità di ricevere nuove funzionalità in modo semplice e veloce.
            \item \textbf{Produrre codice corretto}: il prodotto, al momento del rilascio, non deve riscontrare anomalie ed essere eseguito correttamente.
        \end{itemize}

        \paragraph{Metriche}
        \begin{itemize}
            \item MPR012 Moduli non testati
        \end{itemize}

        \paragraph{Obiettivi}
        \begin{itemize}
            \item \textbf{QPR013 I moduli possiedono suite di test}: i moduli codificati devono possedere una suite di test per essere verificati in tutte le loro parti e in correlazione tra di loro.
            \item \textbf{QPR014 Rispetto delle norme di codifica}: il codice, per essere meglio comprensibile e corretto deve essere prodotto seguendo delle norme. Il rispetto di tali norme viene verificato nelle \NdPd.
        \end{itemize}

\subsection{Tabella qualità di processo}
Le tabelle indicano gli obiettivi di qualità discussi nelle precedenti sezioni che ogni processo deve possedere.

Ogni obiettivo di qualità è indicato con:

\begin{itemize}
	%\item \textbf{Obiettivo}: indica il proprio codice identificativo secondo quanto esplicato nella sezione \S\ref{nomenclatura}.
	\item \textbf{Obiettivo}: indica il proprio codice identificativo come spiegato nella sezione \S\ref{nomenclatura}.
	\item \textbf{Metrica}: la metrica utilizzata per valutare l'obiettivo di qualità assegnatole denominato sempre secondo quanto dichiarato in \S\ref{nomenclatura}. Nel caso non fosse possibile associare una metrica ad un obiettivo di qualità, questa non verrà indicata.
	\item \textbf{Valore desiderato}: il valore che vogliamo ottenere per soddisfare appieno l'obiettivo di qualità. È possibile dare un valore quantificabile solo tramite una metrica, quindi questo campo è nullo nel caso in cui quest'ultima sia assente.
	\item \textbf{Descrizione}: descrizione generale dell'obiettivo di qualità prefisso.
\end{itemize}

\newcommand{\grigiodesc}{gray!15}

\begin{table}[H]
	{\def\arraystretch{1.4}
	\begin{tabularx}{\textwidth}{YYY}
		\rowcolor{white}
		\multicolumn{3}{>{\hsize=\dimexpr3\hsize+4\tabcolsep}Y}{\textbf{PROC001 Pianificazione del progetto, organizzazione e struttura}} \\
		\rowcolor{gray!30}
		\textbf{Obiettivo} &
		\textbf{Metrica} &
		\textbf{Valore desiderato}\\\toprule

		\rowcolor{white}
		QPR001 Rispetto dei periodi della pianificazione & MPR001 Varianza della pianificazione & 96 ore\\
		\rowcolor{\grigiodesc} \multicolumn{3}{>{\hsize=\dimexpr3\hsize+4\tabcolsep}X}{\textbf{Descrizione}: le scadenze orarie per ogni ruolo riportate nel \PdP\ considerano anche un tempo di \gloss{slack} per eventuali ritardi. Queste sono state stabilite all'inizio del progetto, perciò molto probabilmente possono subire variazioni in corso d'opera, l'importante è che queste variazioni non siano eccessive. Considerando che ogni membro del gruppo deve svolgere un numero fisso di ore, è naturale che togliendo ore au un ruolo se ne aggiungano ad un altro o viceversa, per questo una variazione totale di 96 ore (4 giorni), non ci nè sembra eccessiva nè restringente.} \\

		\hline \rowcolor{white}
		QPR002 Variazione del budget & MPR002 Varianza dei costi & \EUR{0 - 200}\\
		\rowcolor{\grigiodesc} \multicolumn{3}{>{\hsize=\dimexpr3\hsize+4\tabcolsep}X}{\textbf{Descrizione}: ogni ruolo che copriamo possiede una tariffa oraria. Può accadere che nel corso del progetto sia richiesta una quantità di lavoro diversa da quella preventivamente conteggiata. Si è previsto di poter tollerare fino a un massimo di \EUR{200} di differenza.} \\
		\hline \rowcolor{white}
		QPR003 Rispetto delle fasi del ciclo di vita & MPR003 Aderenza agli standard & Livello di maturità: 3 Valutazione attributi: L\\
		\rowcolor{\grigiodesc} \multicolumn{3}{>{\hsize=\dimexpr3\hsize+4\tabcolsep}X}{\textbf{Descrizione}: dai sondaggi è emerso che la maggior parte delle aziende italiane che si sono adeguate allo standard ISO/IEC 15504 ha raggiunto un livello di maturità pari a 3. Ci prefiggiamo quindi di raggiungere anche noi un tale livello di maturità soddisfacendo i vari attributi in almeno il 75\% della loro interezza.}\\
		\hline \rowcolor{white}
		QPR004 Versionamento & MPR004 Frequenza commit nella repository & 25+\\
		\rowcolor{\grigiodesc} \multicolumn{3}{>{\hsize=\dimexpr3\hsize+4\tabcolsep}X}{\textbf{Descrizione}: i commit di una repository permettono di tenere una miglior traccia delle modifiche e di accedere facilmente all'ultima versione del prodotto.
		In base a sondaggi presenti in Internet, abbiamo constatato che un giusto numero di commit da effettuare in media durante una settimana produttiva è almeno 25.}\\
	\end{tabularx}}
	\caption{Obiettivi di qualità per il PROC001}
\end{table}

%\textsl{}\mydoublerule{\linewidth}{0pt}{2pt}

\begin{table}[H]
	{\def\arraystretch{1.4}
	\begin{tabularx}{\textwidth}{YYY}
		\rowcolor{white}
		\multicolumn{3}{>{\hsize=\dimexpr3\hsize+4\tabcolsep}Y}{\textbf{PROC002 Analisi}}\\
		\rowcolor{gray!30}
		\textbf{Obiettivo} &
		\textbf{Metrica} &
		\textbf{Valore desiderato}\\\toprule
		\rowcolor{white}
		QPR005 Soddisfacimento dei requisiti obbligatori & MPR005 Requisiti obbligatori non soddisfatti & 0\\
		\rowcolor{\grigiodesc} \multicolumn{3}{>{\hsize=\dimexpr3\hsize+4\tabcolsep}X}{\textbf{Descrizione}: i requisiti obbligatori devono essere tutti soddisfatti entro la consegna finale del progetto.}\\
		\hline \rowcolor{white}
		QPR006 Soddisfacimento dei requisiti opzionali e desiderabili & MPR006 Requisiti desiderabili non soddisfatti MPR007 Requisiti opzionali non soddisfatti & 0 - (n-2)\\
		\rowcolor{\grigiodesc} \multicolumn{3}{>{\hsize=\dimexpr3\hsize+4\tabcolsep}X}{\textbf{Descrizione}: i requisiti opzionali e desiderabili, non essendo
			obbligatori possono non essere svolti all'interno del progetto. Dato che però offrono un valore aggiunto, miriamo a soddisfarne almeno due.}\\
		\hline \rowcolor{white}
		QPR007 Verificarsi dei rischi previsti & MPR008 Rischi non previsti avvenuti & 0\\
		\rowcolor{\grigiodesc} \multicolumn{3}{>{\hsize=\dimexpr3\hsize+4\tabcolsep}X}{\textbf{Descrizione}: il verificarsi di imprevisti può accadere nel corso del progetto, ma devono poter essere rilevabili prima che succeda. Per questo si prevede che non si presenti nessun altro rischio al di fuori di quelli previsti.}\\
	\end{tabularx}}
	\caption{Obiettivi di qualità per il PROC002}
\end{table}

\mydoublerule{\linewidth}{0pt}{2pt}

\begin{table}[H]
	{\def\arraystretch{1.5}
	\begin{tabularx}{\textwidth}{YYY}
		\rowcolor{white}
		\multicolumn{3}{>{\hsize=\dimexpr3\hsize+4\tabcolsep}Y}{\textbf{PROC003 Produzione documenti}}\\
		\rowcolor{gray!30}
		\textbf{Obiettivo} &
		\textbf{Metrica} &
		\textbf{Valore desiderato}\\\toprule
		\rowcolor{white}
		QPR008 Rispetto delle fasi del ciclo di vita & - & -\\
		\rowcolor{\grigiodesc} \multicolumn{3}{>{\hsize=\dimexpr3\hsize+4\tabcolsep}X}{\textbf{Descrizione}: ogni documento deve attraversare determinate fasi del ciclo di vita. È compito del \Ver~accertarsi che tali fasi vengano rispettate.}\\
	\end{tabularx}}
	\caption{Obiettivi di qualità per il PROC003}
\end{table}

%\mydoublerule{\linewidth}{0pt}{2pt}

\begin{table}[H]
	{\def\arraystretch{1.5}
	\begin{tabularx}{\textwidth}{YYY}
		\rowcolor{white}
		\multicolumn{3}{>{\hsize=\dimexpr3\hsize+4\tabcolsep}Y}{\textbf{PROC004 Verifica}}\\
		\rowcolor{gray!30}
		\textbf{Obiettivo} &
		\textbf{Metrica} &
		\textbf{Valore desiderato}\\\toprule
		\rowcolor{white}
		QPR009 Effettuare una verifica costante & MPR009 Frequenza controllo prodotti & 5 modifiche\\
		\rowcolor{\grigiodesc} \multicolumn{3}{>{\hsize=\dimexpr3\hsize+4\tabcolsep}X}{\textbf{Descrizione}: il \Ver\ deve controllare i prodotti in modo frequente, in modo tale che siano corretti nella forma, contenuto e funzionalità. Si considera 5 un numero sufficiente di modifiche apportate a un documento tale da necessitare di una verifica.}\\
		%Teoricamente non solo..
		%, per i prodotti software invece i commit effettuati.}
		\hline \rowcolor{white}
		QPR010 Rispetto delle fasi di verifica & - & -\\
		\rowcolor{\grigiodesc} \multicolumn{3}{>{\hsize=\dimexpr3\hsize+4\tabcolsep}X}{\textbf{Descrizione}: è compito del \Ver~assicurarsi che la fase di verifica venga eseguita nel modo corretto per disporre sempre di risultati di verifica attendibili ed analizzabili.}\\
	\end{tabularx}}
	\caption{Obiettivi di qualità per il PROC004}
\end{table}

\mydoublerule{\linewidth}{0pt}{2pt}

\begin{table}[H]
    {\def\arraystretch{1.4}
        \begin{tabularx}{\textwidth}{YYY}
            \rowcolor{white}
            \multicolumn{3}{>{\hsize=\dimexpr3\hsize+4\tabcolsep}Y}{\textbf{PROC005 Progettazione}} \\
            \rowcolor{gray!30}
            \textbf{Obiettivo} &
            \textbf{Metrica} &
            \textbf{Valore desiderato}\\\toprule

            \rowcolor{white}
            QPR011 Progettare prima di codificare & MPR010 Moduli codificati prima della progettazione & 0\% - 25\%\\
            \rowcolor{\grigiodesc} \multicolumn{3}{>{\hsize=\dimexpr3\hsize+4\tabcolsep}X}{\textbf{Descrizione}: i moduli devono essere progettati prima della loro codifica. Purtroppo questo è difficilmente possibile nel periodo che antecede la fase di progettazione e che richiede il rilascio  di un \gloss{PoC}, per questo la soglia di tolleranza per moduli codificati prima di essere progettati è al 0\% - 25\%.} \\

            \hline \rowcolor{white}
            QPR012 Usare un numero limitato di design pattern & MPR011 Numero di design pattern applicati & 10 - 15\\
            \rowcolor{\grigiodesc} \multicolumn{3}{>{\hsize=\dimexpr3\hsize+4\tabcolsep}X}{\textbf{Descrizione}: data la dimensione del progetto, dai 10 ai 15 design pattern applicati ci sembrano un numero sufficiente per ottenere un codice di qualità non troppo complesso.}\\
        \end{tabularx}}
    \caption{Obiettivi di qualità per il PROC005}
\end{table}

\mydoublerule{\linewidth}{0pt}{2pt}

\begin{table}[H]
    {\def\arraystretch{1.4}
        \begin{tabularx}{\textwidth}{YYY}
            \rowcolor{white}
            \multicolumn{3}{>{\hsize=\dimexpr3\hsize+4\tabcolsep}Y}{\textbf{PROC006 Codifica}} \\
            \rowcolor{gray!30}
            \textbf{Obiettivo} &
            \textbf{Metrica} &
            \textbf{Valore desiderato}\\\toprule

            \rowcolor{white}
            QPR013 I moduli possiedono suite di test & MPR012 Moduli non testati & 0\\
            \rowcolor{\grigiodesc} \multicolumn{3}{>{\hsize=\dimexpr3\hsize+4\tabcolsep}X}{\textbf{Descrizione}: ogni modulo deve essere testato per verificarne l'integrità e la correttezza.} \\

            \hline \rowcolor{white}
            QPR014 Rispetto delle norme di codifica & - & -\\
            \rowcolor{\grigiodesc} \multicolumn{3}{>{\hsize=\dimexpr3\hsize+4\tabcolsep}X}{\textbf{Descrizione}: all'interno del processo di codifica bisogna seguire delle norme per produrre del codice di qualità. Tali norme sono meglio indicate nelle \NdPd.}\\
    \end{tabularx}}
    \caption{Obiettivi di qualità per il PROC006}
\end{table}
