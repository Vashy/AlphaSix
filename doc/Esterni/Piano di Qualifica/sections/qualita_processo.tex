\section{Qualità di processo}\label{QualitaProcesso}

\subsection{Scopo}
La qualità di un prodotto è fortemente influenzata dal processo utilizzato nell'arco di creazione del prodotto stesso: da un tubo sporco non può uscire acqua pulita.

Per questo è necessario operare con un buon ciclo di vita per i processi, che devono essere verificati e valutati. A questo scopo viene seguito lo schema del ciclo di Deming e dell'ISO 15504 descritti all'Appendice §A.

\subsection{Nomenclatura metriche ed obiettivi di qualità}
La nomenclatura degli obiettivi di qualità, delle metriche ed il loro funzionamento è spiegato in dettaglio nel \Doc{\NdPv}. In questa sezione gli obiettivi e le metriche svengono sinteticamente descritte:

\begin{itemize}
	\item \textbf{Obiettivi}: 
	
	\begin{center}
	\texttt{QPD[ID] [Nome]}
	\end{center} 

	Le prime tre lettere indicano "Qualità dei documenti", l'ID è un numero progressivo di tre cifre e il nome descrive brevemente il processo;
	\item \textbf{Metriche}:
	
	\begin{center}
		\texttt{MPD[ID] [Nome]}
	\end{center}
	
	Le prime tre lettere indicano "Metrica dei documenti", l'ID è un numero progressivo di tre cifre e il nome descrive brevemente la metrica;
\end{itemize}

\subsection{Processi}
I processi saranno elencati nel seguente modo:

\begin{center}
	\texttt{PROC[ID] [Nome]}
\end{center}

\begin{itemize}
	\item \textbf{ID}: un numero incrementale per indicare in modo univoco il processo.
	\item \textbf{Nome}: una breve frase per indicare la funzione del processo.
\end{itemize}

Per ogni processo sono elencate le sue funzioni principali, gli aspetti da misurare per ottenere la qualità desiderata e le metriche utilizzate.
Gli obiettivi di qualità elencati, quando è possibile, sono affiancati da una particolare metrica. 

	\subsubsection{PROC001 Pianificazione del progetto, organizzazione e struttura}
	Tale macro-processo ha lo scopo di pianificare il lavoro da svolgere per soddisfare i requisiti richiesti dal progetto. È in questo processo che viene messo in atto il \gloss{way of working} del team di sviluppo che ha un'importanza particolare perché i suoi output vanno a condizionare l'esito della qualità dell'intero progetto.
	
		\paragraph*{Funzioni}
	
		\begin{itemize}
			\item \textbf{Sviluppare sotto-processi}: i vari obiettivi devono poter essere associati ad azioni ben precise, ognuna delle quali appartiene ad un sotto-processo.
			\item \textbf{Suddividere i compiti}: è compito dell'\Amm\ assegnare i vari compiti ai vari ruoli del team.
			\item \textbf{Candelarizzare i documenti}: altro compito dell'\Amm\ è quello di stabilire delle \gloss{baseline} durante il progetto. Sarà poi compito dei vari membri del team organizzare i loro impegni per rispettare tali scadenze.
			\item \textbf{Formazione personale}: l'inesperienza del team richiede di un periodo di formazione personale più lungo del normale. Questo periodo deve essere contato dall'\Amm\ all'interno della calendarizzazione.
			\item \textbf{Standard}: vengono scelti gli standard da seguire.
			\item \textbf{Budget}: è necessario conoscere le proprie risorse in termini di tempo $\frac{\text{costo}}{\text{persona}}$ in modo tale da restare
				il più fedeli possibile al preventivo stilato.
		\end{itemize}
	
		\paragraph*{Metriche}
		
		\begin{itemize}
			\item MPR001 Varianza della programmazione
			\item MPR002 Varianza dei costi
			\item MPR003 Aderenza agli standard
			\item MPR004 Frequenza commit nella repository
		\end{itemize}
	
		\paragraph*{Obiettivi}
		
		\begin{itemize} 
			\item \textbf{QPR001 Rispetto delle fasi dell'organigramma}: all'interno dell'organigramma presente alla sezione (inserire sezione organigramma) del \Doc{\PdPv}, sono presenti le date di scadenza delle varie attività. L'obiettivo è quello di rispettarle il più possibile.
			\item \textbf{QPR002 Variazione del budget}: le risorse messe a disposizione all'inizio del progetto devono potersi mantenere tali in tutta la sua durata.
			\item \textbf{QPR003 Rispetto delle fasi del ciclo di vita}: ogni processo deve rispettare le fasi del ciclo di Deming.
			\item \textbf{QPR004 Versionamento}: durante ogni processo, i prodotti che vengono realizzati devono possedere un numero di versione in modo da tener traccia delle modifiche. In questo modo è possibile sapere le cause di certi comportamenti dovuti ad un mutamento del prodotto in modo molto rapido.  
		\end{itemize}
	
	\subsubsection{PROC002 Analisi}
	Il processo di analisi serve per comprendere le richieste del progetto che consistono in: requisiti, tecnologie da utilizzare e risorse da spendere.
	Insieme all'analisi dei requisiti, questo processo prevede anche l'analisi dei rischi presente alla sezione (sezione dell'analisi dei rischi) nel \Doc{\PdPv}.
	
	Completata questa attività il team di sviluppo, oltre ad aver studiato i vari requisiti, deve poter dire se il progetto risulta fattibile per le sue possibilità in termini di competenze e risorse.
	
	Tale processo non deve essere rendicontato per il preventivo a finire.
	
		\paragraph*{Funzioni}
		
		\begin{itemize}
			\item \textbf{Individuare i requisiti}: il capitolato che viene presentato dal cliente può possedere con un lessico più discorsivo che tecnico dovuto all'inesperienza del cliente nel settore.
				\`E necessario dunque effettuare un lavoro di "traduzione" per individuare i requisiti e classificarli in

			\begin{itemize}
				\item Obbligatori
				\item Desiderabili  
				\item Opzionali
			\end{itemize}
			\item \textbf{Individuare le tecnologie da utilizzare}: in base al tipo di progetto le tecnologie da utilizzare migliori per realizzarlo possono cambiare da progetto a progetto.
			\item \textbf{Comprendere la quantità di risorse richieste}: sviluppare un preventivo è fondamentale, sia per il cliente che vuole sapere il costo del progetto, che per il fornitore che deve sapere se il progetto per lui è fattibile.
			\item \textbf{Individuare i rischi}: lungo il corso del progetto possono accadere degli imprevisti, perciò è necessario poterli prevedere ed organizzare fin da subito una reazione ad essi.
			\item \textbf{Stabilire la fattibilità del progetto}: dopo aver fatto l'analisi dei requisiti, dei rischi e dei costi bisogna sapere se iniziare lo sviluppo del progetto o meno.
		\end{itemize}

		\paragraph*{Metriche}

		\begin{itemize}
			\item \textbf{MPR005 Requisiti obbligatori non soddisfatti}
			\item \textbf{MPR006 Requisiti desiderabili non soddisfatti}
			\item \textbf{MPR007 Requisiti opzionali non soddisfatti}
			\item \textbf{MPR008 Rischi non previsti avvenuti}
		\end{itemize}
		I risultati di tutte queste metriche possono essere pubblicate ed analizzate solo a progetto compiuto, prima avrebbe poca utilità. 
	
		\paragraph*{Obiettivi}
		
		\begin{itemize}
			\item \textbf{QPR005 Soddisfacimento dei requisiti obbligatori}: tutti i requisiti obbligatori devono essere soddisfatti a fine progetto.
			\item \textbf{QPR006 Soddisfacimento dei requisiti opzionali e desiderabili}: i requisiti opzionali e desiderabili possono non essere rispettati
				solo se non avanzano più risorse in termini di $ \frac{\text{tempo}}{\text{persona}}$ prima della consegna del progetto. I requisiti inoltre
				devono essere stabiliti in modo corretto: hanno l'obbligo di rispecchiare quanto chiesto dal cliente e non sovrapporsi tra loro.
			\item \textbf{QPR007 Verificarsi dei rischi previsti}: lo sviluppo del progetto deve essere sicuro e non presentare imprevisti la cui soluzione non è stata prevista, perché potenzialmente aumenterebbe di molto il ritardo nello sviluppo del progetto rispetto al rischio previsto.
		\end{itemize}
	
	\subsubsection{PROC003 Produzione documenti}
	Questo processo rimane attivo in tutta la durata del ciclo di vita del prodotto perché ha il compito di produrre dei documenti che riportino le scelte effettuate, gli strumenti utilizzati e le modifiche attuate nell'intero corso del progetto.
	
		\paragraph*{Funzioni}
		Il processo comprende il ciclo di vita di ogni documento, che è spiegato in dettaglio nelle \Doc{\NdPv}, prevede:
		
		\begin{itemize}
			\item Redazione
			\item Verifica
			\item Approvazione
		\end{itemize}
	
		\paragraph*{Obiettivi}
		
		\begin{itemize}
			\item \textbf{QPR008 Rispetto delle fasi del ciclo di vita}: il ciclo di vita di ogni documento deve essere rispettato in ogni sua fase e questo deve venir pubblicato secondo le scadenze stabilite. Le sue fasi sono:
			
			\begin{itemize}
				\item Redazione
				\item Verifica
				\item Approvazione
			\end{itemize}
		\end{itemize}

	\subsubsection{PROC004 Verifica}
	Il processo, attivo in tutta la durata del progetto, ha il compito di valutare la correttezza dei prodotti dati in input e stabilire se presentano errori
	e se sono sufficientemente di qualità. Il processo di verifica è descritto più nel dettaglio alla sezione §4.2 del documento \mbox{\Doc{\NdPv}}.
	
		\paragraph*{Funzioni}
		
		\begin{itemize}
			\item \textbf{Verificare le funzionalità dei prodotti}: i prodotti devono saper soddisfare i requisiti richiesti. Nella fase di verifica viene accertato che questo effettivamente avvenga.
			\item \textbf{Verificare la corretta esecuzione dei processi}: i processi possiedono un ciclo di vita che è suddiviso in fasi, per la buona esecuzione del processo queste devono essere eseguite nell'ordine corretto restituendo gli output attesi.
			\item \textbf{Verificare che siano rispettate le \NdP}: il team di sviluppo stabilisce un suo \gloss{way of working} che deve rispettare in tutta la durata del progetto. Anche la verifica di queste norme deve essere effettuata.
		\end{itemize}
		
		\paragraph*{Metriche}
		
		\begin{itemize}
			\item MPR009 Frequenza controllo prodotti
		\end{itemize}
		
		\paragraph*{Obiettivi}
		
		\begin{itemize}
			\item \textbf{QPR009 Effettuare una verifica costante}: la fase di verifica è sempre attiva perché ogni singolo prodotto deve essere testato e controllato ogni volta che viene modificato o prima della scadenza di una \gloss{milestone}.
			%\item \textbf{QPR010 Rispetto delle fasi di verifica}: la fase di verifica deve essere eseguita in modo iterativo senza cambiare in modo sostanziale nel corso del progetto. Un mutamento della fase di verifica renderebbe incoerenti i risultati delle precedenti iterazioni, impedendo di vedere se è presente un miglioramento della qualità o meno.
		\end{itemize}

\subsection{Tabella qualità di processo}
Le tabelle indicano gli obiettivi di qualità che ogni processo deve possedere.

Ogni obiettivo di qualità è indicato con:

\begin{itemize}
	\item \textbf{Obiettivo}: viene indicato l'obiettivo di qualità col suo codice identificativo e nome;
	\item \textbf{Metrica}: la metrica utilizzata per valutare l'obiettivo di qualità assegnatole. Nel caso non fosse possibile associare una metrica ad un obiettivo di qualità questa non verrà indicata;
	\item \textbf{Valore desiderato}: il valore che si vuole ottenere attraverso la metrica indicata per soddisfare a pieno l'obiettivo di qualità;
	\item \textbf{Descrizione}: descrizione generale dell'obiettivo di qualità e della metrica.
\end{itemize}

\newcommand{\grigiodesc}{gray!15}

\begin{table}[H]
	{\def\arraystretch{1.4}
	\begin{tabularx}{\textwidth}{YYY}
		\rowcolor{white}
		\multicolumn{3}{>{\hsize=\dimexpr3\hsize+4\tabcolsep}Y}{\textbf{PROC001 Pianificazione del progetto, organizzazione e struttura}} \\
		\rowcolor{gray!30}
		\textbf{Obiettivo} &
		\textbf{Metrica} &
		\textbf{Valore desiderato}\\\toprule
		\rowcolor{white}

		QPR001 Rispetto delle fasi dell'organigramma & MPR001 Varianza della programmazione & 0-2 giorni\\
		\rowcolor{\grigiodesc} \multicolumn{3}{>{\hsize=\dimexpr3\hsize+4\tabcolsep}X}{\textbf{Descrizione}: le scadenze date dall'organigramma possono possedere un ritardo massimo di due giorni, per le scadenza infatti viene considerato anche un tempo di \gloss{slack} per eventuali ritardi. Il valore desiderato indica la media del numero di giorni di ritardo nella chiusura di una issue durante tutto il macro periodo.} \\

		\hline \rowcolor{white}
		QPR002 Variazione del budget & MPR002 Varianza dei costi & \EUR{0-200}\\
		\rowcolor{\grigiodesc} \multicolumn{3}{>{\hsize=\dimexpr3\hsize+4\tabcolsep}X}{\textbf{Descrizione}: ogni ruolo nel team di sviluppo possiede una tariffa oraria. Può accadere che nel corso del progetto sia richiesto più lavoro di quello conteggiato nel preventivo, queste ore in più devono essere rendicontate.} \\
		\hline \rowcolor{white}
		QPR003 Rispetto delle fasi del ciclo di vita & MPR003 Aderenza agli standard & Livello di maturità: 3 Valutazione attributi: L\\
		\rowcolor{\grigiodesc} \multicolumn{3}{>{\hsize=\dimexpr3\hsize+4\tabcolsep}X}{\textbf{Descrizione}: dai sondaggi è emerso che la maggior parte delle aziende italiane che si sono adeguate allo standard ISO/IEC 15504 ha raggiunto un livello di maturità pari a 3. Il team di sviluppo si prefigge di raggiungere anch'esso un tale livello di maturità soddisfacendo i vari attributi di almeno il 75\% della loro interezza.}\\
		\hline \rowcolor{white}
		QPR004 Versionamento & MPR004 Frequenza commit nella repository & 25\\
		\rowcolor{\grigiodesc} \multicolumn{3}{>{\hsize=\dimexpr3\hsize+4\tabcolsep}X}{\textbf{Descrizione}: il frequente commit in una repository permette di tenere una miglior traccia delle modifiche e di accedere all'ultima versione del progetto a tutti i membri del team di sviluppo. Nel valore desiderato è indicato il numero minimo di commit da effettuare in media ogni settimana lavorativa durante il un macro periodo.}\\
	\end{tabularx}}
	\caption{Obiettivi di qualità per il PROC001}
\end{table}

\mydoublerule{\linewidth}{0pt}{2pt}

\begin{table}[H]
	{\def\arraystretch{1.4}
	\begin{tabularx}{\textwidth}{YYY}
		\rowcolor{white}
		\multicolumn{3}{>{\hsize=\dimexpr3\hsize+4\tabcolsep}Y}{\textbf{PROC002 Analisi}}\\
		\rowcolor{gray!30}
		\textbf{Obiettivo} &
		\textbf{Metrica} &
		\textbf{Valore desiderato}\\\toprule
		\rowcolor{white}
		QPR005 Soddisfacimento dei requisiti obbligatori & MPR005 Requisiti obbligatori non soddisfatti & 0\\
		\rowcolor{\grigiodesc} \multicolumn{3}{>{\hsize=\dimexpr3\hsize+4\tabcolsep}X}{\textbf{Descrizione}: i requisiti obbligatori devono essere tutti soddisfatti alla consegna finale del progetto.}\\
		\hline \rowcolor{white}
		QPR006 Soddisfacimento dei requisiti opzionali e desiderabili & MPR006 Requisiti desiderabili non soddisfatti MPR007 Requisiti opzionali non soddisfatti & 0-(n-2)\\
		\rowcolor{\grigiodesc} \multicolumn{3}{>{\hsize=\dimexpr3\hsize+4\tabcolsep}X}{\textbf{Descrizione}: i requisiti opzionali e desiderabili, non essendo
			obbligatori possono non essere svolti all'interno del progetto, dato che però offrono un valore aggiunto, il team di sviluppo vuole soddisfarne almeno due.}\\
		\hline \rowcolor{white}
		QPR007 Verificarsi dei rischi previsti & MPR008 Rischi non previsti avvenuti & 0\\
		\rowcolor{\grigiodesc} \multicolumn{3}{>{\hsize=\dimexpr3\hsize+4\tabcolsep}X}{\textbf{Descrizione}: il verificarsi di imprevisti può accadere nel corso
			del progetto, ma devono essere tutti già previsti dal team di sviluppo in modo tale da intervenire tempestivamente con una soluzione.}\\
	\end{tabularx}}
	\caption{Obiettivi di qualità per il PROC002}
\end{table}

\mydoublerule{\linewidth}{0pt}{2pt}

\begin{table}[H]
	{\def\arraystretch{1.5}
	\begin{tabularx}{\textwidth}{YYY}
		\rowcolor{white}
		\multicolumn{3}{>{\hsize=\dimexpr3\hsize+4\tabcolsep}Y}{\textbf{PROC003 Produzione documenti}}\\
		\rowcolor{gray!30}
		\textbf{Obiettivo} &
		\textbf{Metrica} &
		\textbf{Valore desiderato}\\\toprule
		\rowcolor{white}
		QPR008 Rispetto delle fasi del ciclo di vita & - & -\\
		\rowcolor{\grigiodesc} \multicolumn{3}{>{\hsize=\dimexpr3\hsize+4\tabcolsep}X}{\textbf{Descrizione}: ogni documento deve attraversare determinate fasi del ciclo di vita, è compito del \Ver~accertarsi che tali fasi vengano rispettate.}\\
	\end{tabularx}}
	\caption{Obiettivi di qualità per il PROC003}
\end{table}

\mydoublerule{\linewidth}{0pt}{2pt}

\begin{table}[H]
	{\def\arraystretch{1.5}
	\begin{tabularx}{\textwidth}{YYY}
		\rowcolor{white}
		\multicolumn{3}{>{\hsize=\dimexpr3\hsize+4\tabcolsep}Y}{\textbf{PROC004 Verifica}}\\
		\rowcolor{gray!30}
		\textbf{Obiettivo} &
		\textbf{Metrica} &
		\textbf{Valore desiderato}\\\toprule
		\rowcolor{white}
		QPR009 Effettuare una verifica costante & MPR009 Frequenza controllo prodotti & 5 modifiche\\
		\rowcolor{\grigiodesc} \multicolumn{3}{>{\hsize=\dimexpr3\hsize+4\tabcolsep}X}{\textbf{Descrizione}: il \Ver\ deve controllare i prodotti in modo frequente, in modo tale che siano corretti nella forma, contenuto e funzionalità. Per quanto riguarda i documenti le modifiche sono le varie versioni del documento riportate nel diario, per i prodotti software invece i commit effettuati.}\\
		\hline \rowcolor{white}
%		QPR010 Rispetto delle fasi di verifica & - & -\\
%		\rowcolor{\grigiodesc} \multicolumn{3}{>{\hsize=\dimexpr3\hsize+4\tabcolsep}X}{\textbf{Descrizione}: è compito del \Ver~assicurarsi che la fase di verifica venga eseguita nel modo corretto per disporre sempre di risultati di verifica attendibili ed analizzabili.}\\
	\end{tabularx}}
	\caption{Obiettivi di qualità per il PROC003}
\end{table}

\mydoublerule{\linewidth}{0pt}{2pt}