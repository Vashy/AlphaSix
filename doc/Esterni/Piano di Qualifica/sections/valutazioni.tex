\newpage
\section{Valutazioni per il miglioramento}	\label{valutazioni per il miglioramento}

	Questo paragrafo vuole elencare i problemi riscontrati nel corso del progetto evidenziati dalle nostre varie considerazioni e dai risultati riportati all'appendice \S\ref{ResocontoAttivitaVerifica}.

	Per ogni problema verrà considerata una soluzione di miglioramento da applicare dalla versione attuale del documento alle successive.

    \subsection{Primo periodo (RR)}

	\subsubsection{Valutazioni sull'organizzazione}
		\begin{itemize}
			\item \textbf{Problema riscontrato}: la difficoltà maggiore è stata quella di entrare nell'ottica del progetto abituandoci ai cambi di ruolo e ai compiti da svolgere coordinandoci tra di noi.
			\item \textbf{Soluzione proposta}: per ovviare a questo i membri si sono parlati e si è organizzato il lavoro dopo aver studiato il capitolato e i documenti indicati nei riferimenti normativi. Ognuno di noi ha avuto modo di coprire ogni ruolo attivo fino ad ora, perciò sarà meno impegnativo in futuro rispettare la rotazione dei ruoli citata nel \PdPd.
			\item \textbf{Problema riscontrato}: le issue create richiedevano dei compiti che si sovrapponevano tra loro, rischiando di effettuare più volte un lavoro inutilmente, oppure, non venendo assegnate, più \gloss{componenti} risolvevano la stessa issue.
			\item \textbf{Soluzione proposta}: l'\Amm\ si impegna a creare issue più precise e circoscritte, evitando di dimenticarsi di inserire un assegnatario.
		\end{itemize}

	%TODO Effettuare una valutazione sul consuntivo di periodo

	\subsubsection{Valutazione dei ruoli}

		\paragraph{Responsabile}
			\begin{itemize}
				\item \textbf{Problema riscontrato}: la difficoltà più sentita da chi ha ricoperto questo ruolo è stata la suddivisione del lavoro in maniera equa ed omogenea capendo anche i punti di forza e di debolezza di ciascun membro.
				\item \textbf{Soluzione proposta}: per evitare che questo problema persistesse, in corso del progetto è stato usato lo strumento di \gloss{issue tracking system} di GitHub per assegnare i compiti in maniera incrementale e omogenea.
			\end{itemize}

		\paragraph{Amministratore}
			\begin{itemize}
				\item \textbf{Problema riscontrato}: la difficoltà maggiore è stata quella di non avere una base di partenza per quanto riguarda la redazione delle \NdP\ e sulla granularità delle informazioni contenute al suo interno.
				\item \textbf{Soluzione proposta}: all'inizio e nel corso della revisione dei requisiti sono state fatte delle discussioni per normare gli aspetti più importanti e per avere una salda base di partenza per redarle in maniera incrementale nel corso del progetto e senza modificare quello che è stato scritto in precedenza.
			\end{itemize}

		\paragraph{Analista}
			\begin{itemize}
				\item \textbf{Problema riscontrato}: la principale difficoltà è stata la stesura dell'\AdR\ in quanto i contenuti di questo documento sono nuovi e per noi di difficile comprensione. In particolar modo l'individuazione dei corretti \gloss{casi d'uso} del progetto.
				\item \textbf{Soluzione proposta}: anche in questo caso, dopo aver studiato autonomamente l'argomento, abbiamo fatto delle riunioni per poter redarre i paragrafi di maggiore importanza come in particolare quello riguardante le tecnologie e i casi d'uso.
			\end{itemize}

		\paragraph{Verificatore}
			\begin{itemize}
				\item \textbf{Problema riscontrato}: la verifica è avvenuta in maniera non costante all'inizio e questo ha provocato una mole di documenti da verificare più ampia del previsto.
				\item \textbf{Soluzione proposta}: una pianificazione migliore del lavoro da svolgere ha aiutato in corso d'opera a evitare che questo succedesse nuovamente e si continuerà ad usare nello sviluppo del progetto per evitare che si possa ripresentare.
			\end{itemize}

	\subsubsection{Valutazione sugli strumenti}

		\paragraph{\LaTeX}
			\begin{itemize}
				\item \textbf{Problema riscontrato}: la necessità iniziale di avere dei \gloss{template} su cui poter lavorare è stato uno dei problemi iniziali che ha necessitato grande attenzione in quanto non tutti i membri sapevano usare {\LaTeX} allo stesso livello.
				\item \textbf{Soluzione proposta}: inizialmente la creazione dei template è andata a essere definita insieme alle norme più importanti per poi continuare la loro costruzione in maniera incrementale.
			\end{itemize}

		\paragraph{Git}
			\begin{itemize}
				\item \textbf{Problema riscontrato}: una difficoltà riscontrata, anche se raramente, è stata quella dei conflitti durante i commit sulla \gloss{repository} in quanto questa è utilizzata da più persone.
				\item \textbf{Soluzione proposta}: tramite coordinazione e azioni varie fornite da Git, i conflitti si sono presentati di rado. Questo anche perché ciascun membro ha sempre lavorato su file separati non sovrapponendo il proprio lavoro con quello degli altri. Un ulteriore miglioramento consiste nel tener costantemente monitorata lo stato della repository mentre si lavora al progetto, in modo tale da effettuare un aggiornamento ogni qual volta avviene un \gloss{push} da un'altro di noi.
			\end{itemize}

	\subsubsection{Integrità di prodotti e strumenti}
		\begin{itemize}
			\item \textbf{Problema riscontrato}: nel corso del progetto non sono state rispettate tutte le \NdP~prestabilite o nell'aggiornamento della repository sono stati inseriti problemi inattesi.
			\item \textbf{Soluzione proposta}: prima di effettuare una modifica nella repository è tassativo controllare che non si presentino problemi a qualcuno di noi. Dunque, oltre a dover avere ben chiaro il contenuto delle \NdP, è necessario avvisare tempestivamente chi ha introdotto l'errore nella repository oppure nel prodotto testato o utilizzato in fase di sviluppo. La maggior parte di questi errori dovrebbero essere segnalati dal \Ver, ma è possibile anche che le segnalazioni arrivino da chi ricopre altri ruoli.
		\end{itemize}

    \subsubsection{Applicazione dei miglioramenti dopo la RR}
    Durante il secondo periodo del progetto abbiamo cercato di applicare le soluzioni proposte nei precedenti paragrafi, e in questo riportiamo gli esisti di tali applicazioni.

    \begin{itemize}
        \item \textbf{Organizzazione}: dopo il primo periodo ogni membro del gruppo ha compreso meglio i compiti di ogni ruolo e in quale si trova più a suo agio. In questo modo è stato più semplice assegnare i ruoli seguendo le preferenze di ognuno, tenendo sempre in mente che ogni membro del gruppo deve ricoprire ogni ruolo. Anche l'esperienza accumulata con l'issue tracking system ha permesso una miglior assegnazione di ruoli.
        \item \textbf{Ruoli}:
            \begin{itemize}
                \item \textbf{Responsabile}: conoscendo ora meglio i membri del gruppo, il \Res\ riesce ad assegnare meglio i compiti, avendo anche più esperienza con l'issue tracking system.
                \item \textbf{Amministratore}: essendo già in possesso di un documento per le norme, l'Amministratore è ora più agevolato nello svolgere i suoi compiti.
                \item \textbf{Analista}: essendo l'argomento dei casi d'uso particolarmente ostico l'idea di trovarsi in gruppo per redarli si è rivelata corretta.
                \item \textbf{Verificatore}: indicando con anticipo le date o le condizioni per effettuare verifica, la pianificazione di questa fase si è rivelata meno difficoltosa; grazie anche al supporto di strumenti che effettuano verifiche in automatico.
            \end{itemize}
        \item \textbf{Strumenti}:
            \begin{itemize}
                \item \textbf{\LaTeX}: l'utilizzo per tutto il corso del progetto di questo strumento ha portato ad assorbire tutti i problemi precedentemente riscontrati.
                \item \textbf{Git}: l'utilizzo maggiore di \gloss{branch}, ha dato una più solidità alla repository, causandone meno conflitti.
            \end{itemize}
        \item \textbf{Prodotti e strumenti}: l'uso più tassativo delle norme e dei controlli nei prodotti prima di inserire un cambiamento nella repository si è rilevata una scelta \gloss{efficiente}.
    \end{itemize}

    \subsection{Secondo periodo (RP)}\label{valutazioni per il miglioramento:RP}

    \subsubsection{Valutazioni sull'organizzazione}
        \begin{itemize}
            \item \textbf{Problema riscontrato}: avendo avuto meno tempo da dedicare al progetto a causa della sessione degli esami, ci siamo trovati a comprimere molti compiti in un periodo ristretto e a doverli svolgere in modo celere.
            \item \textbf{Soluzione proposta}: avere ben chiari i giorni di disponibilità fino alla fine del progetto, per effettuare una pianificazione che riscontri sempre meno variazioni.
            \item \textbf{Problema riscontrato}: l'accelerazione dello svolgimento dei compiti ha portato i vari membri del gruppo a non conoscere in modo molto chiaro tutte le componenti del progetto, in quanto il lavoro è stato più parallelizzato.
            \item \textbf{Soluzione proposta}: oltre ad una miglior organizzazione, ci si fa spiegare dal membro del gruppo, che ha effettuato una modifica nei prodotti, in cosa questa consiste.
        \end{itemize}

    \subsubsection{Valutazione dei ruoli}

        \paragraph{Analista}
            \begin{itemize}
                \item \textbf{Problema riscontrato}: la natura particolare del progetto ha richiesto più volte la modifica dei casi d'uso anche nel secondo periodo, quando l'attività di analisi dei requisiti doveva solo essere ultimata.
                \item \textbf{Soluzione proposta}: per avere un'idea più chiara dei casi d'uso e, più in generale, dei requisiti del progetto, serve avere uno scambio di opinioni più frequente, sia col committente, che con l'azienda proponente.
            \end{itemize}

        \paragraph{Progettista}
            \begin{itemize}
                \item \textbf{Problema riscontrato}: le ore assegnategli sono risultate essere eccessive per il ruolo che ha avuto in questo periodo di lavoro, a discapito di altri ruoli che si sono rivelati essere più presenti del previsto.
                \item \textbf{Soluzione proposta}: purtroppo, essendo la prima esperienza di un progetto simile, è difficile stabilire con precisione quanto verrà ricoperto un ruolo fino alla fine del progetto. Ora però è più chiaro l'andamento che questo può avere. Ad esempio il Progettista nel prossimo periodo (RQ) sarà una delle figure principali per il progetto.
            \end{itemize}

        \paragraph{Programmatore}
            \begin{itemize}
                \item \textbf{Problema riscontrato}: si è presentata qualche difficoltà nel modificare del codice scritto da un altro membro del gruppo, questo perché non tutte le norme di codifica sono state rispettate fin da subito.
                \item \textbf{Soluzione proposta}: perseguire in modo più preciso e costante le norme di codifica, dovrebbe rendere il codice più fruibile a tutti i membri del gruppo.
            \end{itemize}

        \paragraph{Verificatore}
            \begin{itemize}
                \item \textbf{Problema riscontrato}: nei periodi in cui è stato più impellente rilasciare i prodotti, la verifica è avvenuta in modo meno frequente.
                \item \textbf{Soluzione proposta}: inserire, in modo più stringete, il numero o le date per le fasi di verifica obbligatorie che non possono essere prorogate.
            \end{itemize}

    \subsubsection{Valutazione sugli strumenti}

        \paragraph{Docker}
            \begin{itemize}
                \item \textbf{problema riscontrato}: conoscere una nuova tecnologia richiede tempo, e nel nostro caso con \gloss{Docker} abbiamo impiegato più del previsto a conoscerne le caratteristiche per noi di interesse.
                \item \textbf{Soluzione proposta}: individuare le parti di interesse di una tecnologia, che verranno utilizzate, per poi dividersele tra i vari membri del gruppo. Successivamente, in una riunione, riportare quanto imparato.
            \end{itemize}


    \subsection{Terzo periodo (RQ)}\label{valutazioni per il miglioramento:RQ}

    \subsubsection{Valutazioni sull'organizzazione}
        \begin{itemize}
            \item \textbf{Problema riscontrato}: la scarsa conoscenza dei design pattern ha rallentato inizialmente il lavoro pianificato.
            \item \textbf{Soluzione proposta}: maggior studio individuale dei design pattern.
        \end{itemize}

    \subsubsection{Valutazione dei ruoli}

        \paragraph{Analista}
            \begin{itemize}
                \item \textbf{Problema riscontrato}: è stato necessario modificare alcuni requisiti e aggiungere alcuni diagrammi dei casi d'uso, attività non previste dalla pianificazione.
                \item \textbf{Soluzione proposta}: abbiamo aggiunto delle ore necessarie alla correzione dei requisiti e dei casi d'uso.
            \end{itemize}

        \paragraph{Progettista}
            \begin{itemize}
                \item \textbf{Problema riscontrato}: l'eccessivo zelo del gruppo nell'applicare i design pattern al progetto ha portato a una complessità eccessiva nella codifica.
                \item \textbf{Soluzione proposta}: a seguito del riscontro della PB, abbiamo ritenuto necessario rivedere la progettazione di alcune componenti del sistema in modo da semplificare l'architettura.
            \end{itemize}

        \paragraph{Programmatore}
            \begin{itemize}
                \item \textbf{Problema riscontrato}:
                \item \textbf{Soluzione proposta}:
            \end{itemize}

        \paragraph{Verificatore}
            \begin{itemize}
                \item \textbf{Problema riscontrato}:
                \item \textbf{Soluzione proposta}:
            \end{itemize}

    \subsubsection{Valutazione sugli strumenti}

        \paragraph{Rancher}
            \begin{itemize}
                \item \textbf{problema riscontrato}:
                \item \textbf{Soluzione proposta}: 
            \end{itemize}



	%\subsection{Valutazione sulla distribuzione delle risorse}

	%	\subsubsection{Prospetto economico}
	%		\begin{itemize}
	%			\item \textbf{Problema riscontrato}:
	%			\item \textbf{Soluzione proposta}:
	%		\end{itemize}

%	\subsection{Cambiamenti a seguito delle revisioni}
%		\subsubsection{Revisione dei requisiti}
%		\subsubsection{Revisione dei requisiti}
%		...
