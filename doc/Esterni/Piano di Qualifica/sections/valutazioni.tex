\newpage
\section{Valutazioni per il miglioramento}	\label{valutazioni per il miglioramento}

	Questo paragrafo vuole elencare i problemi riscontrati nel corso del progetto evidenziati dalle nostre varie considerazioni e dai risultati riportati all'appendice \S\ref{B}.

	Per ogni problema verrà considerata una soluzione di miglioramento da applicare dalla versione attuale del documento alle successive.

	\subsection{Valutazioni sull'organizzazione}
		\begin{itemize}
			\item \textbf{Problema riscontrato}: la difficoltà maggiore è stata quella di entrare nell'ottica del progetto abituandoci ai cambi di ruolo e ai compiti da svolgere coordinandoci tra di noi.
			\item \textbf{Soluzione proposta}: per ovviare a questo i membri si sono parlati e si è organizzato il lavoro dopo aver studiato il capitolato e i documenti indicati nei riferimenti normativi. Ognuno di noi ha avuto modo di coprire ogni ruolo attivo fino ad ora, perciò sarà meno impegnativo in futuro rispettare la rotazione dei ruoli citata nel \PdPd.
			\item \textbf{Problema riscontrato}: le issue create richiedevano dei compiti che si sovrapponevano tra loro, rischiando di effettuare più volte un lavoro inutilmente, oppure, non venendo assegnate, più \gloss{componenti} risolvevano la stessa issue.
			\item \textbf{Soluzione proposta}: l'\Amm\ si impegna a creare issue più precise e circoscritte, evitando di dimenticarsi inserire un assegnatario.
		\end{itemize}

	%TODO Effettuare una valutazione sul consuntivo di periodo

	\subsection{Valutazione dei ruoli}

		\subsubsection{Responsabile}
			\begin{itemize}
				\item \textbf{Problema riscontrato}: la difficoltà più sentita da chi ha ricoperto questo ruolo è stata la suddivisione del lavoro in maniera equa ed omogenea capendo anche i punti di forza e di debolezza di ciascun membro.
				\item \textbf{Soluzione proposta}: per evitare che questo problema persistesse, in corso del progetto è stato usato lo strumento di \gloss{issue tracking system} di \gloss{GitHub} per assegnare i compiti in maniera incrementale e omogenea.
			\end{itemize}

		\subsubsection{Amministratore}
			\begin{itemize}
				\item \textbf{Problema riscontrato}: la difficoltà maggiore è stata quella di non avere una base di partenza per quanto riguarda la redazione delle \NdP\ e sulla granularità delle informazioni contenute al suo interno.
				\item \textbf{Soluzione proposta}: all'inizio e nel corso della revisione dei requisiti sono state fatte delle discussioni per normare gli aspetti più importanti e per avere una salda base di partenza per redarle in maniera incrementale nel corso del progetto e senza modificare quello che è stato scritto in precedenza.
			\end{itemize}

		\subsubsection{Analista}
			\begin{itemize}
				\item \textbf{Problema riscontrato}: la principale difficoltà è stata la stesura dell'\AdR\ in quanto i contenuti di questo documento sono nuovi e per noi di difficile comprensione. In particolar modo l'individuazione dei corretti \gloss{casi d'uso} del progetto.
				\item \textbf{Soluzione proposta}: anche in questo caso, dopo aver studiato autonomamente l'argomento, abbiamo fatto delle riunioni per poter redarre i paragrafi di maggiore importanza come in particolare quello riguardante le tecnologie e i casi d'uso.
			\end{itemize}

		\subsubsection{Verificatore}
			\begin{itemize}
				\item \textbf{Problema riscontrato}: la verifica è avvenuta in maniera non costante all'inizio e questo ha provocato una mole di documenti da verificare più ampia del previsto.
				\item \textbf{Soluzione proposta}: una pianificazione migliore del lavoro da svolgere ha aiutato in corso d'opera a evitare che questo succedesse nuovamente e si continuerà ad usare nello sviluppo del progetto per evitare che si possa ripresentare.
			\end{itemize}

	\subsection{Valutazione sugli strumenti}

		\subsubsection{\LaTeX}
			\begin{itemize}
				\item \textbf{Problema riscontrato}: la necessità iniziale di avere dei \gloss{template} su cui poter lavorare è stato uno dei problemi iniziali che ha necessitato grande attenzione in quanto non tutti i membri sapevano usare {\LaTeX} allo stesso livello.
				\item \textbf{Soluzione proposta}: inizialmente la creazione dei template è andata a essere definita insieme alle norme più importanti per poi continuare la loro costruzione in maniera incrementale.
			\end{itemize}
		
		\subsubsection{Git}
			\begin{itemize}
				\item \textbf{Problema riscontrato}: una difficoltà riscontrata, anche se raramente, è stata quella dei conflitti durante i commit sulla repository in quanto questa è utilizzata da più persone.
				\item \textbf{Soluzione proposta}: tramite coordinazione e azioni varie fornite da Git, i conflitti si sono presentati di rado. Questo anche perché ciascun membro ha sempre lavorato su file separati non sovrapponendo il proprio lavoro con quello degli altri. Un ulteriore miglioramento consiste nel tener costantemente monitorata lo stato della repository mentre si lavora al progetto, in modo tale da effettuare un aggiornamento ogni qual volta avviene un \gloss{push} da un'altro di noi. 
			\end{itemize}
		
		\subsection{Integrità di prodotti e strumenti}
			\begin{itemize}
				\item \textbf{Problema riscontrato}: nel corso del progetto non sono state rispettate tutte le norme di progetto prestabilite o nell'aggiornamento della repository sono stati inseriti problemi inattesi.
				\item \textbf{Soluzione proposta}: prima di effettuare una modifica nella repository è tassativo controllare che non si presentino problemi a qualcuno di noi. Dunque, oltre a dover avere ben chiaro il contenuto delle \NdP, è necessario avvisare tempestivamente chi ha introdotto l'errore nella repository oppure nel prodotto testato o utilizzato in fase di sviluppo. La maggior parte di questi errori dovrebbero essere segnalati dal \Ver, ma è possibile anche che le segnalazioni arrivino da chi ricopre altri ruoli.
			\end{itemize}
		
	%\subsection{Valutazione sulla distribuzione delle risorse}
		
	%	\subsubsection{Prospetto economico}
	%		\begin{itemize}
	%			\item \textbf{Problema riscontrato}:
	%			\item \textbf{Soluzione proposta}:
	%		\end{itemize}
		
%	\subsection{Cambiamenti a seguito delle revisioni}
%		\subsubsection{Revisione dei requisiti}
%		\subsubsection{Revisione dei requisiti}
%		...