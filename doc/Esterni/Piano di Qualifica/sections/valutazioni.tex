\newpage
\section{Valutazioni per il miglioramento}

	Questo paragrafo ha come obiettivo quello di aiutare a valutare come il gruppo AlphaSix ha lavorato internamente fino a questo momento. 

	\subsection{Valutazioni sull'organizzazione}
		\begin{itemize}
			\item \textbf{Problema riscontrato}: la difficoltà maggiore è stata quella di entrare nell'ottica del progetto abituandosi ai cambi di ruolo e ai compiti da svolgere in coordinazione con gli altri membri del gruppo.
			\item \textbf{Soluzione proposta}: per ovviare a questo si è parlato e si è organizzato il lavoro dopo aver studiato il capitolato e i documenti indicati nei riferimenti normativi.
		\end{itemize}
	
	\subsection{Valutazioni sui ruoli}
	
		\subsubsection{Responsabile}
			\begin{itemize}
				\item \textbf{Problema riscontrato}: la difficoltà più sentita da chi ha fatto questo ruolo è stata la suddivisione del lavoro in maniera equa ed omogenea capendo anche gli interessi di ciascun membro.
				\item \textbf{Soluzione proposta}: per evitare che questo problema persistesse, in corso del progetto è stato usato lo strumento di issue tracking system di GitHub per assegnare i compiti in maniera incrementale e omogenea.
			\end{itemize}

		\subsubsection{Amministratore}
			\begin{itemize}
				\item \textbf{Problema riscontrato}: la difficoltà maggiore è stata quella di non avere una base di partenza per quanto riguarda la redazione delle norme di progetto e sulla granularità delle informazioni contenute all'interno.
				\item \textbf{Soluzione proposta}: all'inizio e nel durante del periodo precedente alla consegna dei documenti sono stati fatti brainstorming per normare gli aspetti più importanti e per avere una salda base di partenza per poterle redare in maniera incrementale nel corso del progetto e senza modificare quello che è stato scritto in precedenza.
			\end{itemize}

		\subsubsection{Analista}
			\begin{itemize}
				\item \textbf{Problema riscontrato}: la principale difficoltà è stata la stesura dell'Analisi dei Requisiti in quanto i contenuti di questo documento sono nuovi e non sono stati mai visti prima dai componenti del gruppo.
				\item \textbf{Soluzione proposta}: anche in questo caso, dopo aver studiato autonomamente l'argomento, il gruppo ha fatto brainstorming per poter redarre i paragrafi di maggiore importanza come in particolare quello riguardante i casi d'uso.
			\end{itemize}

		\subsubsection{Verificatore}
			\begin{itemize}
				\item \textbf{Problema riscontrato}: la verifica è avvenuta in maniera non costante all'inizio e questo ha provocato che la mole di documenti da verificare in seguito fosse più ampia del dovuto.
				\item \textbf{Soluzione proposta}: una pianificazione migliore del lavoro da svolgere ha aiutato in corso d'opera a evitare che questo succedesse nuovamente e si continuerà ad usare nello sviluppo del progetto per evitare che si possa ripresentare.
			\end{itemize}

	\subsection{Valutazione sugli strumenti}

		\subsubsection{\LaTeX}
			\begin{itemize}
				\item \textbf{Problema riscontrato}: la necessità iniziale di avere dei template su cui poter lavorare è stato uno dei problemi iniziali che ha necessitato di più attenzione in quanto non tutti i membri sapevano usare {\LaTeX} allo stesso livello.
				\item \textbf{Soluzione proposta}: inizialmente la creazione dei template è andata a essere definita insieme alle norme più importanti per poi continuare la loro costruzione in maniera incrementale.
			\end{itemize}
		
		\subsubsection{Git}
			\begin{itemize}
				\item \textbf{Problema riscontrato}: una difficoltà riscontrata raramente è stata quella dei conflitti durante i commit sulla repository in quanto questa è utilizzata da più persone.
				\item \textbf{Soluzione proposta}: tramite coordinazione e azioni come pull e stash i conflitti si sono presentati in modo raro, questo anche perchè ciascun membro ha sempre lavorato su file separati non sovrapponendo il proprio lavoro con quello di altri.
			\end{itemize}
		
	%\subsection{Valutazione sulla distribuzione delle risorse}
		
	%	\subsubsection{Prospetto economico}
	%		\begin{itemize}
	%			\item \textbf{Problema riscontrato}:
	%			\item \textbf{Soluzione proposta}:
	%		\end{itemize}
		
%	\subsection{Cambiamenti a seguito delle revisioni}
%		\subsubsection{Revisione dei requisiti}
%		\subsubsection{Revisione dei requisiti}
%		...