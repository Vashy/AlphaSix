\newpage

\section{Mitigazione delle variazioni orarie}	\label{C}
Alla fine della fase di revisione dei requisiti ci siamo accorti di alcune variazioni orarie nel consuntivo rispetto a quanto preventivato.
Non si tratta di grossi scostamenti, perché il gruppo è formato da sei persone e le ore a preventivo sono state calcolate in modo da poter coprire tutte le attività previste dal progetto.\\
Per fare in modo che le variazioni orarie vengano mitigate useremo con maggior frequenza gli strumenti già impiegati per la coordinazione del gruppo in modo da stabilire con maggior precisione a che punto delle attività siamo arrivati.\\
Terremo sotto controllo le ore dedicate da ciascun membro a ciascun ruolo, in modo da evitare che si verifichino eccessi o difetti rispetto a quanto preventivato.

    \subsection{Strumenti e mitigazione}
	Gli strumenti in uso per tenere sotto controllo l'andamento delle attività e le ore impiegate sono gli stessi usati per il coordinamento del gruppo:
	\begin{itemize}
	    \item \textbf{GitHub}: lo usiamo per tenere traccia di quali attività sono da svolgere e da quale membro è stata completata ciascuna attività.
	    \item \textbf{Slack}: lo usiamo per coordinarci, discutere e capire quanto lavoro è servito per il completamento delle attività.
	\end{itemize}
	
	\subsection{Mitigazione dei costi}
	Con l'uso degli strumenti citati cercheremo di restare, in ogni momento del progetto in cui verrà effettuata una verifica delle ore impiegate da ciascun membro, in pari o al di sotto del limite preventivato.\\
	Nella prima fase del progetto, pur con delle variazioni orarie, siamo riusciti a rimanere al di sotto dei costi posti in preventivo.\\
	Aumentando il numero dei controlli delle ore impiegate tra una revisione e l'altra potremo ottenere risultati migliori.

	\subsection{Conclusioni}
	Per mitigare le variazioni verificatesi valuteremo altri strumenti per il controllo più accurato delle ore impiegate da ciascun membro per ciascuna attività.
	Questo sarà utile specialmente per le fasi successive, perché sarà necessario verificare di soddisfare le aspettative del proponente e al contempo risolvere le eventuali mancanze riscontrate nella prima fase.\\
	Per le variazioni verificatesi nel primo periodo, ogni membro con delle carenze orarie in un ruolo cercherà di apportare un maggior contributo in quel ruolo nel secondo periodo, in modo da restare in pari con le ore dedicate al progetto.
	