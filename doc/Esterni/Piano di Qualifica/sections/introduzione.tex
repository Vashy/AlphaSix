\section{Introduzione}\label{introduzione}

    \subsection{Glossario e documenti esterni}
Al fine di rendere il documento più chiaro possibile, i termini che possono assumere un significato ambiguo o i riferimenti a documenti esterni
avranno delle diciture convenzionali:

\begin{itemize}
    \item \textbf{D}: indica che il termine si riferisce al titolo di un particolare documento (ad esempio \Doc{\PdPv});
    \item \textbf{G}: indica che il termine si riferisce ad una voce riportata nel \Doc{\Glv} (ad esempio \gloss{Redmine}).
\end{itemize}


    \subsection{Premessa}
    Il \gloss{documento} che segue verrà prodotto incrementalmente al presentarsi di esigenze di perseguimento della \gloss{qualità}.
    Per questo motivo, non è da considerare al pari di un documento completo (e.g. la parte relativa ai non ci sarà fino
    al presentarsi della necessità di testare).

    \subsection{Scopo del documento}
    Lo scopo di questo documento è riportare formalmente tutte le \gloss{norme} che il team \gruppo\ rispetterà per perseguire
    la qualità di prodotto, sia per quanto riguarda il prodotto in s\'e, sia per quanto riguarda i \gloss{processi}
    utilizzati per costruirlo. Per fare ciò, saranno effettuate continuamente verifiche atte a impedire a qualsiasi tipo di malformità o errore
    di rimanere presente per troppo tempo, rendendo più facile la manutenzione.

    \subsection{Scopo del prodotto}

%%| Ex Norme di Progetto |%%
% Il prodotto che \gruppo\ si incarica di realizzare è Butterfly: un \gloss{tool} di supporto alle figure di	sviluppo di aziende di software
% (non solamente quella committente). Questo applicativo permette di incanalare le notifiche dei vari strumenti utilizzati nel percorso di
% \gloss{CI/CD} (come \gloss{Redmine}, \gloss{GitLab}, ecc.) di un software e, tramite un \gloss{Broker} (\gloss{Apache Kafka} in questo caso),
% spedirli alla persona interessata tramite canale di comunicazione preferito scelto da quest’ultimo (email, \gloss{Telegram}, \gloss{Slack}, ecc).

% \vspace{1cm}

%%| Ex Analisi dei Requisiti |%%
Lo scopo del \gloss{prodotto} è creare un \gloss{applicativo} per poter gestire i messaggi o le segnalazioni provenienti da diversi prodotti per la realizzazione di software,
come \gloss{Redmine}, \gloss{GitLab} e opzionalmente \gloss{SonarQube}, attraverso un \gloss{Broker} che possa incanalare questi messaggi e distribuirli a strumenti come
\gloss{Telegram}, e-mail e opzionalmente \gloss{Slack}.\par
Il software dovrà inoltre essere in grado di riconoscere il \gloss{Topic} dei messaggi in input per poterli inviare in determinati canali a cui i
destinatari dovranno iscriversi.\par
\`E anche richiesto di creare un canale specifico per gestire le particolari esigenze dell'azienda. Dovrà essere in grado, attraverso la lettura di
particolari	\gloss{metadati}, di reindirizzare i messaggi ricevuti al destinatario più appropriato.

% \vspace{1cm}

%%| Ex Piano di Qualifica |%%
% Il prodotto finale consiste in uno strumento in grado di ricevere messaggi o segnalazioni da vari tipi di servizi per la produzione software chiamati
% \gloss{producer} (e.g. \gloss{GitLab}, \gloss{Redmine} e \gloss{SonarQube}), per poterli poi incanalare verso altri servizi chiamati \gloss{Consumer}
% atti a notificare gli sviluppatori (e.g. \gloss{Slack}, \gloss{Telegram} e Email).\par    
% L'applicazione sarà inoltre capace di organizzare le segnalazioni suddividendole per topic a cui i vari utenti dovranno iscriversi per esserne notificati.
% Nel caso in cui il destinatario dovesse segnalare di non essere disponibile, l'applicativo deve reindirizzare il messaggio verso la persona di competenza
% più prossima. 

% \vspace{1cm}

%%| Ex Piano di Progetto |%%
% Il prodotto che \gruppo\ si incarica di realizzare è Butterfly: un tool di supporto alle figure di sviluppo in aziende che producono software (non
% solamente quella del committente).
% Questo applicativo permette di incanalare le notifiche dei vari strumenti utilizzati nel percorso di \gloss{CI} e \gloss{CD} (come Redmine,
% GitLab, ecc.) di un software e, tramite un \gloss{broker} (\gloss{Apache Kafka} in questo caso), spedirli alla persona interessata tramite
% il canale di comunicazione preferito scelto da quest'ultimo (email, Telegram, Slack, ecc.).


    \subsection{Riferimenti}
	
	\subsubsection{Normativi}
    \begin{itemize}
    	\item \textit{\NdPv}\ped{\tiny{D}}
    	\item \gloss{Capitolato} d'appalto C1:\\ \url{https://www.math.unipd.it/~tullio/IS-1/2018/Progetto/C1.pdf}
    \end{itemize}
    
    \subsubsection{Informativi}\label{riferimenti informativi}
    \begin{itemize}
    	\item Presentazione capitolato C1:\\ \url{https://www.math.unipd.it/~tullio/IS-1/2018/Progetto/C1p.pdf}
    	\item Slide del corso di Ingegneria del Software:
    	
    	\begin{itemize}
    		\item Qualità di prodotto:\\ \url{https://www.math.unipd.it/~tullio/IS-1/2018/Dispense/L13.pdf}
    		\item Qualità di processo:\\ \url{https://www.math.unipd.it/~tullio/IS-1/2018/Dispense/L14.pdf}
    		\item Verifica e validazione:\\
    		\url{https://www.math.unipd.it/~tullio/IS-1/2018/Dispense/L16.pdf}
    	\end{itemize}
    	\item \gloss{The Twelve-Factor App}:
    	\url{https://12factor.net/it/}
    	\item Standard ISO/IEC 9126:
    	\url{https://it.wikipedia.org/wiki/ISO/IEC_9126}
    	\item Ciclo di Deming:
    	\url{https://it.wikipedia.org/wiki/Ciclo_di_Deming}
    	\item Modello a V:
    	\url{https://en.wikipedia.org/wiki/V-Model}
    	\item Slide del corso Tecnologie Open Source del professor Bertazzo Nicola sul software testing:\\
    	\url{https://elearning.unipd.it/math/pluginfile.php/43469/mod_resource/content/4/7-Testing.pdf}
    \end{itemize}