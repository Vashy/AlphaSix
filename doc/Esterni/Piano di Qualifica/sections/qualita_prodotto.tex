
\section{Qualità di prodotto}\label{qualità di prodotto}

\subsection{Scopo}
Abbiamo scelto di prendere come modello gli standard ISO 9126 e 90003:2004 individuando alcune caratteristiche a cui i prodotti dovrebbero puntare per raggiungere la qualità desiderata.

\subsection{Nomenclatura metriche ed obiettivi di qualità}	\label{nomenclaturaprodotti}
La nomenclatura degli obiettivi di qualità, delle metriche ed il loro funzionamento è spiegato in dettaglio nelle \NdP. In questa sezione gli obiettivi e le metriche vengono sinteticamente descritte:

	\begin{itemize}

		\item \textbf{Metriche}:
		
		\begin{center}
			\texttt{M[Prodotto][ID] [Nome]}
		\end{center}
	
		\begin{itemize}
			\item \textbf{Prodotto}: può essere:
			\begin{itemize}
				\item \textbf{PD}: indica "Metrica del prodotto documento".
				\item \textbf{PS}: indica "Metrica del prodotto software".
			\end{itemize}
			\item \textbf{ID}: numero progressivo di tre cifre.
			\item \textbf{Nome}: descrive brevemente la metrica.
		\end{itemize}

		\item \textbf{Obiettivi}: 
		
		\begin{center}
			\texttt{Q[Prodotto][ID] [Nome]}
		\end{center} 
		
		\begin{itemize}
			\item \textbf{Prodotto}: può essere:
			\begin{itemize}
				\item \textbf{PD}: indica "Qualità del prodotto documento".
				\item \textbf{PS}: indica "Qualità del prodotto software".
			\end{itemize}
			\item \textbf{ID}: numero progressivo di tre cifre.
			\item \textbf{Nome}: descrive brevemente il processo.
		\end{itemize}
		
		
	\end{itemize}


\subsection{Prodotti} 
Per prodotti è inteso tutto ciò che è concretamente utilizzabile, consultabile o eseguibile.
Nel contesto di questo progetto i prodotti sono documenti e software.

	\subsubsection{Documenti}
	I documenti rilasciati devono presentarsi leggibili e comprensibili già ad una prima lettura, con contenuti che rispecchino le premesse del documento.

		\paragraph*{Metriche} 
		\begin{itemize}
			\item MPD001 \gloss{Indice di Gulpease}
			\item MPD002 Correttezza ortografica
		\end{itemize} 
		
		\paragraph*{Obiettivi} 		
		Le caratteristiche di un documento che vengono analizzate sono:
		
		\begin{itemize}
			\item \textbf{QPD001 Leggibilità del testo}: i documenti devono poter essere letti in modo fluido, evitando perciò periodo troppo lunghi o un alto numero di subordinate.
			\item \textbf{QPD002 Correttezza ortografica}: non devono essere presenti errori ortografici.
%			\item \textbf{QPD002 Correttezza contenuti}: i contenuti devono rispecchiare ciò che è scritto nei titoli dei paragrafi e devono dare valore aggiunto al documento.
			\item \textbf{QPD003 Organizzazione del documento}: i contenuti devono essere inseriti nelle sezioni e nei documenti appropriati.
%			\item \textbf{QPD004 Manutenibilità del documento}: i documenti devono possedere un diario delle modifiche prontamente aggiornato per agevolarne la manutenibilità.
		\end{itemize}

	\subsubsection{Software} \label{prodottosw}
	Il software che realizziamo vogliamo che sia di qualità.
	Per fare ciò, utilizziamo i dati che ci rende disponibile SonarQube e le norme dello standard PEP 8\footnote{Vedere in \S\ref{riferimenti informativi}} per l'analisi statica del codice.
	Essi ci consentono di definire delle semplici metriche con dei grandi obiettivi da raggiungere nel corso del tempo.

		\paragraph*{Metriche} 
		\begin{itemize}
			\item MPS001 Presenza di bug
			\item MPS002 Presenza di vulnerabilità
			\item MPS003 Presenza di code smell
			\item MPS004 Duplicazione del codice
            \item MPS005 File senza intestazione
            \item MPS006 Righe non formattate
            \item MPS007 Nomi di variabili, metodi e classi non normati
            \item MPS008 Righe non indentate
            \item MPS009 Stringhe non normate
            \item MPS010 Righe troppo lunghe
            \item MPS011 Metodi troppo lunghi
            \item MPS012 Metodi con troppi parametri
            \item MPS013 Metodi con troppa complessità ciclomatica
            \item MPS014 Classi che implementano classi concrete
            \item MPS015 Commenti non normati
            \item MPS016 Metodi non documentati
		\end{itemize} 

		\paragraph*{Obiettivi} 	
		L'intero nostro prodotto sofwtare mira a raggiungere:
		\begin{itemize}
			\item \textbf{QPS001 Assenza di bug}: la totale assenza di bug è essenziale per la buona riuscita del rilascio del nostro software.
			\item \textbf{QPS002 Assenza di vulnerabilità}: miriamo a rimuovere qualunque tipo di vulnerabilità che possa compromettere la sicurezza del nostro prodotto.
			\item \textbf{QPS003 Assenza di \gloss{code smell}}: vogliamo che il nostro codice non abbia code smell, al fine di migliorarne la struttura e abbassarne la complessità, in modo che esso ci risulti facilmente leggibile.
			\item \textbf{QPS004 Minima duplicazione del codice}: tendiamo a raggiungere la minor quantità di codice duplicato possibile, poiché grande quantità di codice ripetuto abbassa la qualità di esso, ma non sempre si riesce ad ottenere del codice completamente non ridondante.
            \item \textbf{QPS005 Rispetto delle norme di PEP 8}: per produrre codice di qualità è necessario seguire delle norme per uniformare tutti i file e per renderli comprensibili ed efficienti. Tali norme è meglio che provengano da uno standard creato da persone con molta più esperienza di noi, per questo abbiamo scelto di seguire lo standard PEP 8. 
		\end{itemize}


\subsection{Tabelle qualità di prodotto} \label{tabellequalitaprodotto}
Le tabelle indicano gli obiettivi di qualità che ogni prodotto deve possedere.

Ogni obiettivo di qualità è indicato con:

\begin{itemize}
	\item \textbf{Obiettivo}: viene indicato il codice dell'obiettivo di qualità secondo quanto descritto nella sezione \S\ref{nomenclaturaprodotti}.
	\item \textbf{Metrica}: la metrica utilizzata per valutare l'obiettivo di qualità assegnatole con identificativo secondo quanto scritto in \S\ref{nomenclaturaprodotti}. Nel caso in cui non fosse possibile associare una metrica ad un obiettivo di qualità questa non verrà indicata.
	\item \textbf{Valore desiderato}: il valore che si vuole ottenere attraverso la metrica indicata per soddisfare appieno l'obiettivo di qualità, qualora una metrica sia presente.
	\item \textbf{Descrizione}: descrizione generale dell'obiettivo di qualità.
\end{itemize}



\begin{table}[H]
	{\def\arraystretch{1.5}
	\begin{tabularx}{\textwidth}{YYY}
		\rowcolor{white}
		\multicolumn{3}{>{\hsize=\dimexpr3\hsize+4\tabcolsep}Y}{\textbf{Documenti}} \\
		\rowcolor{gray!30}
		\textbf{Obiettivo} &
		\textbf{Metrica} &
		\textbf{Valore desiderato}\\
		\toprule\rowcolor{white}
		QPD001 Leggibilità del testo & MPD001 Indice Gulpease & 50 - 60\\
		\rowcolor{gray!15}
		\multicolumn{3}{>{\hsize=\dimexpr3\hsize+4\tabcolsep}X}{\textbf{Descrizione}: quanto il testo è leggibile e comprensibile a livello sintattico lo stabilisce l'indice Gulpease e una verifica da parte del \Ver. In particolare è stato scelto l'intervallo 50-60 poiché vogliamo che i nostri documenti siano facilmente leggibili, ma anche con un lessico adeguato e più ricercato del solito.} \\
		\hline \rowcolor{white}
		QPD002 Correttezza ortografica & MPD002 Correttezza ortografica & 0\\
		\rowcolor{gray!15}
		\multicolumn{3}{>{\hsize=\dimexpr3\hsize+4\tabcolsep}X}{\textbf{Descrizione}: il testo non deve presentare alcun errore ortografico.} \\
%		\hline \rowcolor{white}
%		QPD002 Correttezza contenuti & - & -\\
%		\rowcolor{gray!15}
%		\multicolumn{3}{>{\hsize=\dimexpr3\hsize+4\tabcolsep}X}{\textbf{Descrizione}: la correttezza dei contenuti viene controllata attraverso più letture del documento dal \Ver.} \\
		\rowcolor{gray!15}
		\hline \rowcolor{white}
		QPD003 Organizzazione del documento & - & -\\
		\rowcolor{gray!15}
		\multicolumn{3}{>{\hsize=\dimexpr3\hsize+4\tabcolsep}X}{\textbf{Descrizione}: l'organizzazione dei documenti e la loro struttura logica viene controllata dal \Ver~e dall'\Amm.} \\
		\hline %\rowcolor{white}
%		QPD004 Manutenibilità del documento & - & -\\
%		\rowcolor{gray!15}
%		\multicolumn{3}{>{\hsize=\dimexpr3\hsize+4\tabcolsep}X}{\textbf{Descrizione}: ad ogni modifica di un documento viene indicata al suo interno la modifica attuata con data e versione del documento.} \\
	\end{tabularx}}
\caption{Obiettivi di qualità per i documenti}
\end{table}

\mydoublerule{\linewidth}{0pt}{2pt}


\begin{table}[H]
	{\def\arraystretch{1.5}
	\begin{tabularx}{\textwidth}{YYY}
		\rowcolor{white}
		\multicolumn{3}{>{\hsize=\dimexpr3\hsize+4\tabcolsep}Y}{\textbf{Software}} \\
		\rowcolor{gray!30}
		\textbf{Obiettivo} &
		\textbf{Metrica} &
		\textbf{Valore desiderato}\\

		\toprule\rowcolor{white}
		QPS001 Assenza di bug & MPS001 Presenza di bug & 0\\
		\rowcolor{gray!15}
		\multicolumn{3}{>{\hsize=\dimexpr3\hsize+4\tabcolsep}X}{\textbf{Descrizione}: il codice non deve presentare bug al momento del rilascio per evitare problemi nella sua esecuzione, per questo il suo numero di bug deve essere nullo al termine del progetto.} \\
		
		\hline \rowcolor{white}
		QPS002 Assenza di vulnerabilità & MPS002 Presenza di vulnerabilità & 0\\
		\rowcolor{gray!15}
		\multicolumn{3}{>{\hsize=\dimexpr3\hsize+4\tabcolsep}X}{\textbf{Descrizione}: il codice non deve possedere vulnerabilità per quanto riguarda la sicurezza o la manutenibilità entro il termine del progetto.} \\
		
		\rowcolor{gray!15}
		\hline \rowcolor{white}
		QPS003 Assenza di code smell & MPS003 Presenza di code smell & 0\\
		\rowcolor{gray!15}
		\multicolumn{3}{>{\hsize=\dimexpr3\hsize+4\tabcolsep}X}{\textbf{Descrizione}: per mantenere il codice leggibile, il numero di code smell deve essere nullo al termine del progetto.} \\
		
		\hline \rowcolor{white}
		QPS004 Minima duplicazione del codice & MPS004 Duplicazione del codice & 0 - 20\%\\
		\rowcolor{gray!15}
		\multicolumn{3}{>{\hsize=\dimexpr3\hsize+4\tabcolsep}X}{\textbf{Descrizione}: una percentuale tollerabile di linee di codice duplicato non deve eccedere del 20\% rispetto al numero totale.} \\

		\hline \rowcolor{white}
		QPS005 Assenza di file senza intestazione & MPS005 File senza intestazione & 0 \\
		\rowcolor{gray!15}
		\multicolumn{3}{>{\hsize=\dimexpr3\hsize+4\tabcolsep}X}{\textbf{Descrizione}: per comprendere ad una prima occhiata i compiti di un file, qusto deve possedere un'intestazione scritta in una certa forma. Dunque il numero di file senza la corretta intestazione deve essere nullo al termine del progetto.} \\

		\hline \rowcolor{white}
		QPS006 Assenza di righe non formattate & MPS006 Righe non formattate & 0 \\
		\rowcolor{gray!15}
		\multicolumn{3}{>{\hsize=\dimexpr3\hsize+4\tabcolsep}X}{\textbf{Descrizione}: nel momento in cui le righe di codice sono formattate secondo le \NdP, questo risulta essere più leggibile, perciò  il numero di righe non formattate deve essere nullo al termine del progetto.} \\

		\hline \rowcolor{white}
        QPS005 Rispetto delle norme di PEP 8 & MPS007 Nomi di variabili, metodi e classi non normati & 0
        \\
        \rowcolor{gray!15}
        \multicolumn{3}{>{\hsize=\dimexpr3\hsize+4\tabcolsep}X}{\textbf{Descrizione}: i nomi delle variabili, classi e metodi devono risultare il più parlanti possibile, per questo sono presenti delle norme per la loro dichiarazione. Il numero di nomi di variabili, metodi e classi non normati deve essere nullo al termine del progetto.}\\
        
        \hline \rowcolor{white}
        QPS005 Rispetto delle norme di PEP 8 & MPS008 Righe non indentate & 0
        \\
        \rowcolor{gray!15}
        \multicolumn{3}{>{\hsize=\dimexpr3\hsize+4\tabcolsep}X}{\textbf{Descrizione}: la lunghezza delle tabulazioni nel codice deve essere sempre uguale per favorirne la leggibilità, per questo il numero di righe non indentate deve essere nullo al termine del progetto.}\\
        
        \hline
	\end{tabularx}}
\caption{Obiettivi di qualità per il software (1)}
\end{table}

\mydoublerule{\linewidth}{0pt}{2pt}

\begin{table}[H]
    {\def\arraystretch{1.5}
        \begin{tabularx}{\textwidth}{YYY}
            \rowcolor{white}
            \multicolumn{3}{>{\hsize=\dimexpr3\hsize+4\tabcolsep}Y}{\textbf{Software}} \\
            \rowcolor{gray!30}
            \textbf{Obiettivo} &
            \textbf{Metrica} &
            \textbf{Valore desiderato}\\
            
            \toprule\rowcolor{white}
            QPS005 Rispetto delle norme di PEP 8 & MPS009 Stringhe non normate & 0
            \\
            \rowcolor{gray!15}
            \multicolumn{3}{>{\hsize=\dimexpr3\hsize+4\tabcolsep}X}{\textbf{Descrizione}: per dichiarare le stringhe in Python esistono molti modi, usarne uno solo aumenta la leggibilità del codice. Il numero di stringhe non normate deve essere nullo al termine del progetto.}\\

            \hline\rowcolor{white}
            QPS005 Rispetto delle norme di PEP 8 & MPS010 Righe troppo lunghe & 0
            \\
            \rowcolor{gray!15}
            \multicolumn{3}{>{\hsize=\dimexpr3\hsize+4\tabcolsep}X}{\textbf{Descrizione}: scrivere righe di una lunghezza eccessiva obbliga il lettore a scorrere in orizzontale la finestra che mostra il codice, dunque è stato normato di limitare a 79 caratteri la lunghezza di una linea di codice. Il numero di righe troppo lunghe deve essere nullo al termine del progetto.}\\

            \hline\rowcolor{white}
            QPS005 Rispetto delle norme di PEP 8 & MPS011 Metodi troppo lunghi & 0
            \\
            \rowcolor{gray!15}
            \multicolumn{3}{>{\hsize=\dimexpr3\hsize+4\tabcolsep}X}{\textbf{Descrizione}: perché un metodo sia facilmente testabile questo deve avere il minor numero di compiti, un modo per limitarne il numero è stabilire una lunghezza massima di righe (50) dei metodi. Il numero di metodi troppo lunghi deve essere nullo al termine del progetto.}\\
            
            \hline\rowcolor{white}
            QPS005 Rispetto delle norme di PEP 8 & MPS012 Metodi con troppi parametri & 0
            \\
            \rowcolor{gray!15}
            \multicolumn{3}{>{\hsize=\dimexpr3\hsize+4\tabcolsep}X}{\textbf{Descrizione}: anche un numero alto di parametri in un metodi è sintomo che questo esegua troppe attività; per questo il numero di parametri non dovrebbe essere superiore di 6. Il numero di metodi troppo lunghi deve essere nullo al termine del progetto.}\\

            \hline\rowcolor{white}
            QPS005 Rispetto delle norme di PEP 8 & MPS013 Metodi con troppa complessità ciclomatica & 0
            \\
            \rowcolor{gray!15}
            \multicolumn{3}{>{\hsize=\dimexpr3\hsize+4\tabcolsep}X}{\textbf{Descrizione}: se un metodo possiede troppi cammini che può compiere una variabile al suo interno, allora vuol dire che questo è difficile da testare e, in alcuni casi, da eseguire. Il numero di metodi che ha più di 3 cicli annidati deve essere nullo al termine del progetto.}\\

            \hline\rowcolor{white}
            QPS005 Rispetto delle norme di PEP 8 & MPS014 Classi che implementano classi concrete & 0
            \\
            \rowcolor{gray!15}
            \multicolumn{3}{>{\hsize=\dimexpr3\hsize+4\tabcolsep}X}{\textbf{Descrizione}: l'ereditarietà è da evitare, per impedire dipendenza tra le classi, per questo il numero di classi che implementano una o più classi concrete deve essere nullo al termine del progetto.}\\
            
            \hline
        \end{tabularx}}
    \caption{Obiettivi di qualità per il software (2)}
\end{table}

\mydoublerule{\linewidth}{0pt}{2pt}

\begin{table}[H]
    {\def\arraystretch{1.5}
        \begin{tabularx}{\textwidth}{YYY}
            \rowcolor{white}
            \multicolumn{3}{>{\hsize=\dimexpr3\hsize+4\tabcolsep}Y}{\textbf{Software}} \\
            \rowcolor{gray!30}
            \textbf{Obiettivo} &
            \textbf{Metrica} &
            \textbf{Valore desiderato}\\
            
            \toprule \rowcolor{white}
            QPS005 Rispetto delle norme di PEP 8 & MPS015 Commenti non normati & 0
            \\
            \rowcolor{gray!15}
            \multicolumn{3}{>{\hsize=\dimexpr3\hsize+4\tabcolsep}X}{\textbf{Descrizione}: i commenti, per essere utili devono essere coerenti col codice che commentano, sintetici e ben strutturati; nel nostro caso devono possedere un spazio prima dell'inizio ed iniziare con la lettera maiuscola. Il numero di classi che implementano una o più classi concrete deve essere nullo al termine del progetto.}\\
            
            \hline\rowcolor{white}
            QPS005 Rispetto delle norme di PEP 8 & MPS016 Metodi non documentati & 0
            \\
            \rowcolor{gray!15}
            \multicolumn{3}{>{\hsize=\dimexpr3\hsize+4\tabcolsep}X}{\textbf{Descrizione}: come per i file, anche per i metodi una documentazione è necessaria per capire subito il compito di un metodo, per questo il numero di classi che implementano una o più classi concrete deve essere nullo al termine del progetto.}\\
            
            \hline
        \end{tabularx}}
    \caption{Obiettivi di qualità per il software (3)}
\end{table}

\mydoublerule{\linewidth}{0pt}{2pt}