\section{Qualità di prodotto}



\subsection{Scopo}
\gruppo\ prende come modello gli standard ISO 9126 e 90003:2004 individuando alcune caratteristiche a cui i prodotti dovrebbero puntare per raggiungere la qualità desiderata.

\subsection{Nomenclatura metriche ed obiettivi di qualità}	\label{nomenclaturaprodotti}
La nomenclatura degli obiettivi di qualità, delle metriche ed il loro funzionamento è spiegato in dettaglio nelle \NdP. In questa sezione gli obiettivi e le metriche vengono sinteticamente descritte:

	\begin{itemize}
		\item \textbf{Obiettivi}: 
		
		\begin{center}
			\texttt{QPD[ID] [Nome]}
		\end{center} 
		
		\begin{itemize}
			\item \textbf{QPD}: indica "Qualità dei documenti".
			\item \textbf{ID}: numero progressivo di tre cifre.
			\item \textbf{Nome}: descrive brevemente il processo.
		\end{itemize}
		
		\item \textbf{Metriche}:
		
		\begin{center}
			\texttt{MPD[ID] [Nome]}
		\end{center}
	
		\begin{itemize}
			\item \textbf{MPD}: indica "Metrica dei documenti".
			\item \textbf{ID}: numero progressivo di tre cifre.
			\item \textbf{Nome}: descrive brevemente la metrica.
		\end{itemize}
		
	\end{itemize}


\subsection{Prodotti} 
Per prodotti è inteso tutto ciò che è concretamente utilizzabile, consultabile o eseguibile. Nel contesto di questo progetto i prodotti sono documenti e software.

	\subsubsection{Documenti}
	I documenti rilasciati devono presentarsi leggibili e comprensibili già ad una prima lettura, con contenuti che rispecchino le premesse del documento.

		\paragraph*{Metriche} 
		\begin{itemize}
			\item MPD001 \gloss{Indice di Gulpease}
			\item MPD002 Correttezza ortografica
		\end{itemize} 
		
		\paragraph*{Obiettivi} 		
		Le caratteristiche di un documento che vengono analizzate sono:
		
		\begin{itemize}
			\item \textbf{QPD001 Leggibilità del testo}: i documenti devono poter essere letti in modo fluido, evitando perciò periodo troppo lunghi o un alto numero di subordinate.
			\item \textbf{QPD002 Correttezza ortografica}: non devono essere presenti errori ortografici.
%			\item \textbf{QPD002 Correttezza contenuti}: i contenuti devono rispecchiare ciò che è scritto nei titoli dei paragrafi e devono dare valore aggiunto al documento.
			\item \textbf{QPD003 Organizzazione del documento}: i contenuti devono essere inseriti nelle sezioni e nei documenti appropriati.
%			\item \textbf{QPD004 Manutenibilità del documento}: i documenti devono possedere un diario delle modifiche prontamente aggiornato per agevolarne la manutenibilità.
		\end{itemize}

\subsection{Tabella qualità di prodotto}
Le tabelle indicano gli obiettivi di qualità che ogni prodotto deve possedere.

Ogni obiettivo di qualità è indicato con:

\begin{itemize}
	\item \textbf{Obiettivo}: viene indicato il codice dell'obiettivo di qualità secondo quanto descritto nella sezione \S\ref{nomenclaturaprodotti}.
	\item \textbf{Metrica}: la metrica utilizzata per valutare l'obiettivo di qualità assegnatole con identificativo secondo quanto scritto in \S\ref{nomenclaturaprodotti}. Nel caso in cui non fosse possibile associare una metrica ad un obiettivo di qualità questa non verrà indicata.
	\item \textbf{Valore desiderato}: il valore che si vuole ottenere attraverso la metrica indicata per soddisfare appieno l'obiettivo di qualità, qualora una metrica sia presente.
	\item \textbf{Descrizione}: descrizione generale dell'obiettivo di qualità.
\end{itemize}



\begin{table}[H]
	{\def\arraystretch{1.5}
	\begin{tabularx}{\textwidth}{YYY}
		\rowcolor{white}
		\multicolumn{3}{>{\hsize=\dimexpr3\hsize+4\tabcolsep}Y}{\textbf{Documenti}} \\
		\rowcolor{gray!30}
		\textbf{Obiettivo} &
		\textbf{Metrica} &
		\textbf{Valore desiderato}\\
		\toprule\rowcolor{white}
		QPD001 Leggibilità del testo & MPD001 Indice Gulpease & 50 - 60\\
		\rowcolor{gray!15}
		\multicolumn{3}{>{\hsize=\dimexpr3\hsize+4\tabcolsep}X}{\textbf{Descrizione}: quanto il testo è leggibile e comprensibile a livello sintattico lo stabilisce l'indice Gulpease e una verifica da parte del \Ver. In particolare è stato scelto l'intervallo 50-60 poiché \gruppo\ vuole che i propri documenti siano facilmente leggibili, ma anche con un lessico adeguato e più ricercato del solito.} \\
		\hline \rowcolor{white}
		QPD002 Correttezza ortografica & MPD002 Correttezza ortografica & 0\\
		\rowcolor{gray!15}
		\multicolumn{3}{>{\hsize=\dimexpr3\hsize+4\tabcolsep}X}{\textbf{Descrizione}: il testo non deve presentare alcun errore ortografico.} \\
%		\hline \rowcolor{white}
%		QPD002 Correttezza contenuti & - & -\\
%		\rowcolor{gray!15}
%		\multicolumn{3}{>{\hsize=\dimexpr3\hsize+4\tabcolsep}X}{\textbf{Descrizione}: la correttezza dei contenuti viene controllata attraverso più letture del documento dal \Ver.} \\
		\rowcolor{gray!15}
		\hline \rowcolor{white}
		QPD003 Organizzazione del documento & - & -\\
		\rowcolor{gray!15}
		\multicolumn{3}{>{\hsize=\dimexpr3\hsize+4\tabcolsep}X}{\textbf{Descrizione}: l'organizzazione dei documenti e la loro struttura logica viene controllata dal \Ver~e dall'\Amm.} \\
		\hline %\rowcolor{white}
%		QPD004 Manutenibilità del documento & - & -\\
%		\rowcolor{gray!15}
%		\multicolumn{3}{>{\hsize=\dimexpr3\hsize+4\tabcolsep}X}{\textbf{Descrizione}: ad ogni modifica di un documento viene indicata al suo interno la modifica attuata con data e versione del documento.} \\
	\end{tabularx}}
\caption{Obiettivi di qualità per i Documenti}
\end{table}

%\mydoublerule{\linewidth}{0pt}{2pt}