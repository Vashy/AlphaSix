\section{Qualità di prodotto}



\subsection{Scopo}
\gruppo\ prende come modello gli standard ISO 9126 e 90003:2004 individuando alcune caratteristiche a cui i prodotti dovrebbero puntare per raggiungere la qualità desiderata.

\subsection{Nomenclatura metriche ed obiettivi di qualità}	\label{nomenclaturaprodotti}
La nomenclatura degli obiettivi di qualità, delle metriche ed il loro funzionamento è spiegato in dettaglio nelle \NdP. In questa sezione gli obiettivi e le metriche vengono sinteticamente descritte:

	\begin{itemize}
		\item \textbf{Obiettivi}: 
		
		\begin{center}
			\texttt{QPD[ID] [Nome]}
		\end{center} 
		
		\begin{itemize}
			\item \textbf{QPD}: indica "Qualità dei documenti".
			\item \textbf{ID}: numero progressivo di tre cifre.
			\item \textbf{Nome}: descrive brevemente il processo.
		\end{itemize}
		
		\item \textbf{Metriche}:
		
		\begin{center}
			\texttt{MPD[ID] [Nome]}
		\end{center}
	
		\begin{itemize}
			\item \textbf{MPD}: indica "Metrica dei documenti".
			\item \textbf{ID}: numero progressivo di tre cifre.
			\item \textbf{Nome}: descrive brevemente la metrica.
		\end{itemize}
		
	\end{itemize}


\subsection{Prodotti} 
Per prodotti è inteso tutto ciò che è concretamente utilizzabile, consultabile o eseguibile. Nel contesto di questo progetto i prodotti sono documenti e software.

	\subsubsection{Documenti}
	I documenti rilasciati devono presentarsi leggibili e comprensibili già ad una prima lettura, con contenuti che rispecchino le premesse del documento.

		\paragraph*{Metriche} 
		\begin{itemize}
			\item MPD001 \gloss{Indice di Gulpease}
			\item MPD002 Correttezza ortografica
		\end{itemize} 
		
		\paragraph*{Obiettivi} 		
		Le caratteristiche di un documento che vengono analizzate sono:
		
		\begin{itemize}
			\item \textbf{QPD001 Leggibilità del testo}: i documenti devono poter essere letti in modo fluido, evitando perciò periodo troppo lunghi o un alto numero di subordinate.
			\item \textbf{QPD002 Correttezza ortografica}: non devono essere presenti errori ortografici.
%			\item \textbf{QPD002 Correttezza contenuti}: i contenuti devono rispecchiare ciò che è scritto nei titoli dei paragrafi e devono dare valore aggiunto al documento.
			\item \textbf{QPD003 Organizzazione del documento}: i contenuti devono essere inseriti nelle sezioni e nei documenti appropriati.
%			\item \textbf{QPD004 Manutenibilità del documento}: i documenti devono possedere un diario delle modifiche prontamente aggiornato per agevolarne la manutenibilità.
		\end{itemize}
	

\subsection{The Twelve-Factor App}
\gloss{The Twelve-Factor App} è una serie di dodici regole destinate a chi vuole sviluppare \gloss{software-as-a-service} (SaaS).

I suoi principi sono:

\begin{enumerate}
	\item \textbf{\gloss{Codebase}}: deve essere presente una sola codebase versionata da un \gloss{Version Control System} (VCS) come \gloss{GitLAb} da cui
	possono derivare diversi \gloss{deploy}.
	\item \textbf{Dipendenze}: le librerie usate dal codice devono essere presenti nella directory della singola applicazione e non attive a livello di \gloss{sistema}. In questo modo l'applicazione è il meno dipendente possibile dal sistema di esecuzione.
	\item \textbf{Configurazione}: i parametri di configurazione dell'applicazione devono essere completamente separati dalla sua implementazione.
	\item \textbf{Backing Service}: l'applicazione non deve far distinzione tra funzionalità uguali usate in locale o remoto.
	\item \textbf{Build, release, esecuzione}: bisogna separare in modo netto la fase di build, quella di deploy e quella esecuzione, usando \gloss{tool} differenti e diverse repository per salvare i risultati delle varie fasi.
	\item \textbf{Processi}: l'esecuzione dell'applicazione deve essere vista come l'insieme di uno o più processi che restituiscono un risultato. Questi sono di tipo \gloss{stateless}.
	\item \textbf{\gloss{Binding} delle Porte}: l'applicazione è completamente contenuta in se stessa e non fa affidamento ad un altro servizio nell'ambiente di esecuzione. Effettua invece il binding delle porte diventando un servizio per le richieste esterne.
	\item \textbf{Concorrenza}: sviluppare i processi in modo tale che possano lavorare su un sistema decentralizzato.
	\item \textbf{Rilasciabilità}: i processi dell'applicazione devono poter essere avviati e fermati quando se ne ha bisogno senza passaggi bruschi.
	\item \textbf{Parità tra sviluppo e produzione}: deve esserci meno differenza possibile tra lo stato di sviluppo e quello di produzione. Questo si ottiene facendo un rilascio continuo del prodotto.
	\item \textbf{Log}: l'applicazione dovrebbe poter offrire un sistema di login.
	\item \textbf{Processi di Amministrazione}: porre attenzione a quei processi che devono essere eseguiti una tantum dagli sviluppatori ad esempio. Questi processi devono poter essere accessibili solo ad alcuni e indicati in una specifica release.
\end{enumerate}

In accordo col cliente \II, non tutti i punti di Twelve-Factor App possono essere rispettati. In particolare non è richiesto per \gruppo\ il soddisfacimento del punto 12, mentre il punto 11 è lasciato al fornitore come opzionale.
%TODO: aggiungere motivo

%\begin{itemize} 
%	\item \textbf{11}: 
%	%dato che l'applicazione verrà eseguita nella rete interna dell'azienda, non sarà necessaria una fase di autenticazione dell'utente.
%\end{itemize}

\subsection{Tabella qualità di prodotto}
Le tabelle indicano gli obiettivi di qualità che ogni prodotto deve possedere.

Ogni obiettivo di qualità è indicato con:

\begin{itemize}
	\item \textbf{Obiettivo}: viene indicato il codice dell'obiettivo di qualità secondo quanto descritto nella sezione \S\ref{nomenclaturaprodotti}.
	\item \textbf{Metrica}: la metrica utilizzata per valutare l'obiettivo di qualità assegnatole con identificativo secondo quanto scritto in \S\ref{nomenclaturaprodotti}. Nel caso in cui non fosse possibile associare una metrica ad un obiettivo di qualità questa non verrà indicata.
	\item \textbf{Valore desiderato}: il valore che si vuole ottenere attraverso la metrica indicata per soddisfare appieno l'obiettivo di qualità, qualora una metrica sia presente.
	\item \textbf{Descrizione}: descrizione generale dell'obiettivo di qualità.
\end{itemize}



\begin{table}[H]
	{\def\arraystretch{1.5}
	\begin{tabularx}{\textwidth}{YYY}
		\rowcolor{white}
		\multicolumn{3}{>{\hsize=\dimexpr3\hsize+4\tabcolsep}Y}{\textbf{Documenti}} \\
		\rowcolor{gray!30}
		\textbf{Obiettivo} &
		\textbf{Metrica} &
		\textbf{Valore desiderato}\\
		\toprule\rowcolor{white}
		QPD001 Leggibilità del testo & MPD001 Indice Gulpease & 50 - 60\\
		\rowcolor{gray!15}
		\multicolumn{3}{>{\hsize=\dimexpr3\hsize+4\tabcolsep}X}{\textbf{Descrizione}: quanto il testo è leggibile e comprensibile a livello sintattico lo stabilisce l'indice Gulpease e una verifica da parte del \Ver. In particolare è stato scelto l'intervallo 50-60 poiché \gruppo\ vuole che i propri documenti siano facilmente leggibili, ma anche con un lessico adeguato e più ricercato del solito.} \\
		\hline \rowcolor{white}
		QPD002 Correttezza ortografica & MPD002 Correttezza ortografica & 0\\
		\rowcolor{gray!15}
		\multicolumn{3}{>{\hsize=\dimexpr3\hsize+4\tabcolsep}X}{\textbf{Descrizione}: il testo non deve presentare alcun errore ortografico.} \\
%		\hline \rowcolor{white}
%		QPD002 Correttezza contenuti & - & -\\
%		\rowcolor{gray!15}
%		\multicolumn{3}{>{\hsize=\dimexpr3\hsize+4\tabcolsep}X}{\textbf{Descrizione}: la correttezza dei contenuti viene controllata attraverso più letture del documento dal \Ver.} \\
		\rowcolor{gray!15}
		\hline \rowcolor{white}
		QPD003 Organizzazione del documento & - & -\\
		\rowcolor{gray!15}
		\multicolumn{3}{>{\hsize=\dimexpr3\hsize+4\tabcolsep}X}{\textbf{Descrizione}: l'organizzazione dei documenti e la loro struttura logica viene controllata dal \Ver~e dall'\Amm.} \\
		\hline %\rowcolor{white}
%		QPD004 Manutenibilità del documento & - & -\\
%		\rowcolor{gray!15}
%		\multicolumn{3}{>{\hsize=\dimexpr3\hsize+4\tabcolsep}X}{\textbf{Descrizione}: ad ogni modifica di un documento viene indicata al suo interno la modifica attuata con data e versione del documento.} \\
	\end{tabularx}}
\caption{Obiettivi di qualità per i Documenti}
\end{table}

%\mydoublerule{\linewidth}{0pt}{2pt}