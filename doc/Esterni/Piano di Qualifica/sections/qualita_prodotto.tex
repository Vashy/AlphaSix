
\section{Qualità di prodotto}\label{qualità di prodotto}

\subsection{Scopo}
Abbiamo scelto di prendere come modello gli standard ISO 9126 e 90003:2004 individuando alcune caratteristiche a cui i prodotti dovrebbero puntare per raggiungere la qualità desiderata.

\subsection{Nomenclatura metriche ed obiettivi di qualità}	\label{nomenclaturaprodotti}
La nomenclatura degli obiettivi di qualità, delle metriche ed il loro funzionamento è spiegato in dettaglio nelle \NdP. In questa sezione gli obiettivi e le metriche vengono sinteticamente descritte:

	\begin{itemize}

		\item \textbf{Metriche}:
		
		\begin{center}
			\texttt{M[Prodotto][ID] [Nome]}
		\end{center}
	
		\begin{itemize}
			\item \textbf{Prodotto}: può essere:
			\begin{itemize}
				\item \textbf{PD}: indica "Metrica del prodotto documento".
				\item \textbf{PS}: indica "Metrica del prodotto software".
			\end{itemize}
			\item \textbf{ID}: numero progressivo di tre cifre.
			\item \textbf{Nome}: descrive brevemente la metrica.
		\end{itemize}

		\item \textbf{Obiettivi}: 
		
		\begin{center}
			\texttt{Q[Prodotto][ID] [Nome]}
		\end{center} 
		
		\begin{itemize}
			\item \textbf{Prodotto}: può essere:
			\begin{itemize}
				\item \textbf{PD}: indica "Qualità del prodotto documento".
				\item \textbf{PS}: indica "Qualità del prodotto software".
			\end{itemize}
			\item \textbf{ID}: numero progressivo di tre cifre.
			\item \textbf{Nome}: descrive brevemente il processo.
		\end{itemize}
		
		
	\end{itemize}


\subsection{Prodotti} 
Per prodotti è inteso tutto ciò che è concretamente utilizzabile, consultabile o eseguibile.
Nel contesto di questo progetto i prodotti sono documenti e software.

	\subsubsection{Documenti}
	I documenti rilasciati devono presentarsi leggibili e comprensibili già ad una prima lettura, con contenuti che rispecchino le premesse del documento.

		\paragraph*{Metriche} 
		\begin{itemize}
			\item MPD001 \gloss{Indice di Gulpease}
			\item MPD002 Correttezza ortografica
		\end{itemize} 
		
		\paragraph*{Obiettivi} 		
		Le caratteristiche di un documento che vengono analizzate sono:
		
		\begin{itemize}
			\item \textbf{QPD001 Leggibilità del testo}: i documenti devono poter essere letti in modo fluido, evitando perciò periodo troppo lunghi o un alto numero di subordinate.
			\item \textbf{QPD002 Correttezza ortografica}: non devono essere presenti errori ortografici.
%			\item \textbf{QPD002 Correttezza contenuti}: i contenuti devono rispecchiare ciò che è scritto nei titoli dei paragrafi e devono dare valore aggiunto al documento.
			\item \textbf{QPD003 Organizzazione del documento}: i contenuti devono essere inseriti nelle sezioni e nei documenti appropriati.
%			\item \textbf{QPD004 Manutenibilità del documento}: i documenti devono possedere un diario delle modifiche prontamente aggiornato per agevolarne la manutenibilità.
		\end{itemize}

	\subsubsection{Software} \label{prodottosw}
	Il software che realizziamo vogliamo che sia di qualità.
	Per fare ciò, utilizziamo i dati che ci rende disponibile SonarQube, per l'analisi statica del codice.
	Esso ci consente di definire delle semplici metriche con dei grandi obiettivi da raggiungere nel corso del tempo.

		\paragraph*{Metriche} 
		\begin{itemize}
			\item MPS001 Presenza di bug
			\item MPS002 Presenza di vulnerabilità
			\item MPS003 Presenza di code smell
			\item MPS004 Duplicazione del codice
		\end{itemize} 

		\paragraph*{Obiettivi} 	
		L'intero nostro prodotto sofwtare mira a raggiungere:
		\begin{itemize}
			\item \textbf{QPS001 Assenza di bug}: la totale assenza di bug è essenziale per la buona riuscita del rilascio del nostro software.
			\item \textbf{QPS002 Assenza di vulnerabilità}: miriamo a rimuovere qualunque tipo di vulnerabilità che possa compromettere la sicurezza del nostro prodotto.
			\item \textbf{QPS003 Assenza di \gloss{code smell}}: vogliamo che il nostro codice non abbia code smell, al fine di migliorarne la struttura e abbassarne la complessità, in modo che esso ci risulti facilmente leggibile.
			\item \textbf{QPS004 Minima duplicazione del codice}: tendiamo a raggiungere la minor quantità di codice duplicato possibile, poichè grande quantità di codice ripetuto abbassa la qualità di esso, ma non sempre si riesce ad ottenere del codice completamente non ridondante.
		\end{itemize}


\subsection{Tabella qualità di prodotto}
Le tabelle indicano gli obiettivi di qualità che ogni prodotto deve possedere.

Ogni obiettivo di qualità è indicato con:

\begin{itemize}
	\item \textbf{Obiettivo}: viene indicato il codice dell'obiettivo di qualità secondo quanto descritto nella sezione \S\ref{nomenclaturaprodotti}.
	\item \textbf{Metrica}: la metrica utilizzata per valutare l'obiettivo di qualità assegnatole con identificativo secondo quanto scritto in \S\ref{nomenclaturaprodotti}. Nel caso in cui non fosse possibile associare una metrica ad un obiettivo di qualità questa non verrà indicata.
	\item \textbf{Valore desiderato}: il valore che si vuole ottenere attraverso la metrica indicata per soddisfare appieno l'obiettivo di qualità, qualora una metrica sia presente.
	\item \textbf{Descrizione}: descrizione generale dell'obiettivo di qualità.
\end{itemize}



\begin{table}[H]
	{\def\arraystretch{1.5}
	\begin{tabularx}{\textwidth}{YYY}
		\rowcolor{white}
		\multicolumn{3}{>{\hsize=\dimexpr3\hsize+4\tabcolsep}Y}{\textbf{Documenti}} \\
		\rowcolor{gray!30}
		\textbf{Obiettivo} &
		\textbf{Metrica} &
		\textbf{Valore desiderato}\\
		\toprule\rowcolor{white}
		QPD001 Leggibilità del testo & MPD001 Indice Gulpease & 50 - 60\\
		\rowcolor{gray!15}
		\multicolumn{3}{>{\hsize=\dimexpr3\hsize+4\tabcolsep}X}{\textbf{Descrizione}: quanto il testo è leggibile e comprensibile a livello sintattico lo stabilisce l'indice Gulpease e una verifica da parte del \Ver. In particolare è stato scelto l'intervallo 50-60 poiché vogliamo che i nostri documenti siano facilmente leggibili, ma anche con un lessico adeguato e più ricercato del solito.} \\
		\hline \rowcolor{white}
		QPD002 Correttezza ortografica & MPD002 Correttezza ortografica & 0\\
		\rowcolor{gray!15}
		\multicolumn{3}{>{\hsize=\dimexpr3\hsize+4\tabcolsep}X}{\textbf{Descrizione}: il testo non deve presentare alcun errore ortografico.} \\
%		\hline \rowcolor{white}
%		QPD002 Correttezza contenuti & - & -\\
%		\rowcolor{gray!15}
%		\multicolumn{3}{>{\hsize=\dimexpr3\hsize+4\tabcolsep}X}{\textbf{Descrizione}: la correttezza dei contenuti viene controllata attraverso più letture del documento dal \Ver.} \\
		\rowcolor{gray!15}
		\hline \rowcolor{white}
		QPD003 Organizzazione del documento & - & -\\
		\rowcolor{gray!15}
		\multicolumn{3}{>{\hsize=\dimexpr3\hsize+4\tabcolsep}X}{\textbf{Descrizione}: l'organizzazione dei documenti e la loro struttura logica viene controllata dal \Ver~e dall'\Amm.} \\
		\hline %\rowcolor{white}
%		QPD004 Manutenibilità del documento & - & -\\
%		\rowcolor{gray!15}
%		\multicolumn{3}{>{\hsize=\dimexpr3\hsize+4\tabcolsep}X}{\textbf{Descrizione}: ad ogni modifica di un documento viene indicata al suo interno la modifica attuata con data e versione del documento.} \\
	\end{tabularx}}
\caption{Obiettivi di qualità per i documenti}
\end{table}

\mydoublerule{\linewidth}{0pt}{2pt}


\begin{table}[H]
	{\def\arraystretch{1.5}
	\begin{tabularx}{\textwidth}{YYY}
		\rowcolor{white}
		\multicolumn{3}{>{\hsize=\dimexpr3\hsize+4\tabcolsep}Y}{\textbf{Software}} \\
		\rowcolor{gray!30}
		\textbf{Obiettivo} &
		\textbf{Metrica} &
		\textbf{Valore desiderato}\\

		\toprule\rowcolor{white}
		QPS001 Assenza di bug & MPS001 Presenza di bug & 0\\
		\rowcolor{gray!15}
		\multicolumn{3}{>{\hsize=\dimexpr3\hsize+4\tabcolsep}X}{\textbf{Descrizione}: il numero di bug deve essere nullo al termine del progetto.} \\
		
		\hline \rowcolor{white}
		QPS002 Assenza di vulnerabilità & MPS002 Presenza di vulnerabilità & 0\\
		\rowcolor{gray!15}
		\multicolumn{3}{>{\hsize=\dimexpr3\hsize+4\tabcolsep}X}{\textbf{Descrizione}: il numero di vulnerabilità deve essere nullo al termine del progetto.} \\
		
		\rowcolor{gray!15}
		\hline \rowcolor{white}
		QPS003 Assenza dicode smell & MPS003 Presenza di code smell & 0\\
		\rowcolor{gray!15}
		\multicolumn{3}{>{\hsize=\dimexpr3\hsize+4\tabcolsep}X}{\textbf{Descrizione}: il numero di code smell deve essere nullo al termine del progetto.} \\
		
		\hline \rowcolor{white}
		QPS004 Minima duplicazione del codice & MPS004 Duplicazione del codice & 0 - 20\%\\
		\rowcolor{gray!15}
		\multicolumn{3}{>{\hsize=\dimexpr3\hsize+4\tabcolsep}X}{\textbf{Descrizione}: una percentuale tollerabile di linee di codice duplicato non deve eccedere del 20\% rispetto al numero totale.} \\

		\hline 
	\end{tabularx}}
\caption{Obiettivi di qualità per il software}
\end{table}

%\mydoublerule{\linewidth}{0pt}{2pt}