\newpage

\section{Resoconto delle attività di verifica} \label{ResocontoAttivitaVerifica}
In questa sezione viene riportata una sintesi conclusiva dei risultati ottenuti dalle fasi di verifica effettuate nei vari periodi del progetto attraverso le metriche indicate in tale documento. Essi possono coincidere o meno con i valori desiderati da AlphaSix e, nel secondo caso in particolar modo, saranno oggetto di valutazioni per il miglioramento descritte all'appendice \S\ref{mitigazione variazioni}.

I test che vengono introdotti nel corso del progetto non vanno a sostituire i test precedentemente sviluppati. Ogni test, appena è creato viene eseguito periodicamente fino alla fine del progetto, per evitare che i nuovi cambiamenti possano reintrodurre errori precedentemente risolti.

	\subsection{Primo periodo (RR)}
	Periodo individuato come \AdR\ che dura circa un mese di tempo e un quarto del tempo complessivo per realizzare il progetto.

    \subsubsection{Riassunto delle attività di verifica}
    L'attività di verifica si è rivelata per noi più faticosa del previsto. I motivi sono molteplici:
    	\begin{itemize}
    		\item Nel primo periodo perché non avevamo sufficiente esperienza per effettuare una verifica sistematica e perché la nostra attenzione si è interamente concentrata sull'organizzazione dei ruoli e la loro funzione
    		\item Nel secondo periodo invece la verifica è stata più sistematica e meno impegnativa perché gli elaborati venivano consegnati in orario, ma comunque la correzione si è rivelata onerosa
    	\end{itemize}
    

    Come indicato nelle \NdPd, l'attività di verifica è stata effettuata inizialmente attraverso Walkthrought e successivamente attraverso Inspection. Essendo ancora alle prime armi all'inizio del progetto, abbiamo effettuato la verifica secondo Walkthrough per una buona parte del periodo di analisi dei requisiti realizzata insieme. Dopodiché siamo passati ad adottare Inspection in quanto una ricerca dispersiva non era utile alla parallelizzazione dei compiti. 
    Nel momento in cui sarà necessario verificare prodotti software, riteniamo opportuno effettuare il passaggio da Walkthrough a Inspection in tempi più rapidi di come avvenuto finora.
    
    \subsubsection{Risultati delle verifiche tramite analisi}
    Nelle tabelle seguenti vengono riportati i risultati ottenuti applicando le metriche in correlazione all'obiettivo scelto. 
    In ogni tabella sono indicati i prodotti o i processi sottoposti alle metriche e i risultati ottenuti sono classificati in:
    
    \begin{itemize}
    	\item \textbf{Soddisfacente}: il risultato è quello atteso.
    	\item \textbf{Poco soddisfacente}: il risultano non è quello atteso, ma gli è vicino.
    	\item \textbf{Insoddisfacente}: il risultato non è per niente quello atteso.
    \end{itemize} 

    \paragraph{Documenti}

    \begin{table}[H]
    	{\def\arraystretch{1.5}
   		\begin{tabularx}{\textwidth}{YYY}
   			\rowcolor{white}
   			\textbf{QPD001 Leggibilità del testo} & \textbf{MPD001 Indice Gulpease} & \textbf{50 -60} \\
			\hline
   			\rowcolor{gray!30}
   			\textbf{Prodotto/processo testato} & \textbf{Risultato ottenuto} & \textbf{Valutazione} \\
   			\toprule
   			\rowcolor{white} 	\NdPd & 54.57 & Soddisfacente \\
   			\rowcolor{\grigiodesc} 		\SdFd & 53.7 & Soddisfacente \\
   			\rowcolor{white} 	\PdPd & 51.75 & Soddisfacente \\
   			\rowcolor{\grigiodesc} 	\PdQd & 53.29 & Soddisfacente \\
   			\rowcolor{white} \AdRd & 55.62 & Soddisfacente \\
   			\toprule %\rowcolor{gray!30}
   			\multicolumn{3}{>{\hsize=\dimexpr3\hsize+4\tabcolsep}X}{\textbf{Nota}: tutti i documenti soddisfano pienamente l'obiettivo di qualifica indicato.} \\ \bottomrule
   		\end{tabularx}}
   	\caption{Risultati di MPD001 Indice Gulpease}
    \end{table}

	\mydoublerule{\linewidth}{0pt}{2pt}

	\begin{table}[H]
		{\def\arraystretch{1.5}
		\begin{tabularx}{\textwidth}{YYY}
			\rowcolor{white}
			\textbf{QPD002 Correttezza ortografica} & \textbf{MPD002 Correttezza
				ortografica} & \textbf{0} \\
			\hline
			\rowcolor{gray!30}
			\textbf{Prodotto/processo testato} & \textbf{Risultato ottenuto} & \textbf{Valutazione} \\
			\toprule
			\rowcolor{white} \NdPd & 0 & Soddisfacente \\
			\rowcolor{\grigiodesc} \SdFd & 0 & Soddisfacente \\
			\rowcolor{white} \PdPd & 0 & Soddisfacente \\
			\rowcolor{\grigiodesc} \PdQd & 0 & Soddisfacente \\
			\rowcolor{white} \AdRd & 0 & Soddisfacente \\
			\bottomrule
			\multicolumn{3}{>{\hsize=\dimexpr3\hsize+4\tabcolsep}X}{\textbf{Nota}: tutti i documenti soddisfano pienamente l'obiettivo di qualifica indicato.} \\
		\end{tabularx}}
	\caption{Risultati di MPD002 Correttezza
		ortografica}
	\end{table}

	\paragraph{Processi}

		
		\begin{table}[H]
			{\def\arraystretch{1.5}
				\begin{tabularx}{\textwidth}{YYY}
					\rowcolor{white}
					\textbf{QPR001 Rispetto dei periodi della pianificazione} & \textbf{MPR001 Varianza della pianificazione} & \textbf{0-2 Giorni} \\
					\hline
					\rowcolor{gray!30}
					\textbf{Prodotto/processo testato} & \textbf{Risultato ottenuto} & \textbf{Valutazione} \\
					\toprule
					\rowcolor{white} \PdP & 1.23 & Soddisfacente \\
					\toprule
					\multicolumn{3}{>{\hsize=\dimexpr3\hsize+4\tabcolsep}X}{\textbf{Nota}: non tutte le scadenze sono state rispettate, ma i ritardi sono rientrati nei valori tollerati. Il valore ottenuto è la media sul totale di ritardi.} \\
			\end{tabularx}}
			\caption{Risultati di MPR001 Varianza della pianificazione}
		\end{table}
	
		% \vspace{5pt}
		\mydoublerule{\linewidth}{0pt}{2pt}
		\vspace{20pt}
	
		\begin{table}[H]
			{\def\arraystretch{1.5}
				\begin{tabularx}{\textwidth}{YYY}
					\rowcolor{white}
					\textbf{QPR002 Variazione del budget} & \textbf{MPR002 Varianza dei costi} & \textbf{\EUR{0-200}} \\
					\hline
					\rowcolor{gray!30}
					\textbf{Prodotto/processo testato} & \textbf{Risultato ottenuto} & \textbf{Valutazione} \\
					\toprule\rowcolor{white}
					Differenza consuntivo e consuntivo di periodo & \euro -15,00 & Soddisfacente \\ 
					\bottomrule
					\multicolumn{3}{>{\hsize=\dimexpr3\hsize+4\tabcolsep}X}{\textbf{Nota}: il valore desiderato indicato precedentemente sembra essere fin troppo permissivo. Il dettaglio del risultato ottenuto è indicato nel consuntivo di periodo.} \\
			\end{tabularx}}
			\caption{Risultati di MPR002 Varianza dei costi}
		\end{table}
	
		\mydoublerule{\linewidth}{0pt}{2pt}
		\vspace{20pt}
	
		\begin{table}[H]
			{\def\arraystretch{1.5}
				\begin{tabularx}{\textwidth}{YYY}
					\rowcolor{white}
					\textbf{QPR003 Rispetto delle fasi del ciclo di vita} & \textbf{MPR003 Aderenza agli standard} & \textbf{Livello di maturità: 3 Valutazione attributi: L} \\
					\hline
					\rowcolor{gray!30}
					\textbf{Prodotto/processo testato} & \textbf{Risultato ottenuto} & \textbf{Valutazione} \\
					\toprule\rowcolor{white}
					PROC001 Pianificazione del progetto, organizzazione e struttura & Livello di maturità: 2 Valutazione attributi: L & Poco soddisfacente \\
					\bottomrule
					\multicolumn{3}{>{\hsize=\dimexpr3\hsize+4\tabcolsep}X}{\textbf{Nota}: essendo il primo periodo è prevedibile che il livello di maturità desiderato non sia ancora quello sperato. Si prevede un miglioramento per le successive revisioni.} \\
			\end{tabularx}}
			\caption{Risultati di MPR003 Aderenza agli standard}
		\end{table}
	
		%\mydoublerule{\linewidth}{0pt}{2pt}
		\vspace{20pt}
		
		\begin{table}[H]
			{\def\arraystretch{1.5}
				\begin{tabularx}{\textwidth}{YYY}
					\rowcolor{white}
					\textbf{QPR004 Versionamento} & \textbf{MPR004 Frequenza commit nella repository} & \textbf{25} \\
					\hline
					\rowcolor{gray!30}
					\textbf{Prodotto/processo testato} & \textbf{Risultato ottenuto} & \textbf{Valutazione} \\
					\toprule\rowcolor{white}
					Repository & 31.25 & Soddisfacente \\
					\bottomrule
					\multicolumn{3}{>{\hsize=\dimexpr3\hsize+4\tabcolsep}X}{\textbf{Nota}: il numero di commit effettuati risulta ottimo.} \\
			\end{tabularx}}
			\caption{Risultati di MPR004 Frequenza commit nella repository}
		\end{table}
		
		\mydoublerule{\linewidth}{0pt}{2pt}
		\vspace{20pt}
		
		\begin{table}[H]
			{\def\arraystretch{1.5}
				\begin{tabularx}{\textwidth}{YYY}
					\rowcolor{white}
					\textbf{QPR009 Effettuare una verifica costante} & \textbf{MPR009 Frequenza controllo prodotti} & \textbf{5 Modifiche} \\
					\hline
					\rowcolor{gray!30}
					\textbf{Prodotto/processo testato} & \textbf{Risultato ottenuto} & \textbf{Valutazione} \\
					\toprule\rowcolor{white}
					Documenti & 7.2 & Poco soddisfacente \\
					\bottomrule
					\multicolumn{3}{>{\hsize=\dimexpr3\hsize+4\tabcolsep}X}{\textbf{Nota}: il valore ottenuto non soddisfa il risultato atteso perché nella prima parte del macro periodo \gruppo\ ha prestato molta attenzione alla formazione personale a discapito del tempo che poteva essere dedicato alla verifica dei prodotti.} \\
			\end{tabularx}}
			\caption{Risultati di MPR009 Frequenza controllo prodotti}
		\end{table}
	
	%\mydoublerule{\linewidth}{0pt}{2pt}
	
	\subsection{Secondo periodo (RP)}
	Periodo individuato come Revisione Progettuale la quale dura all'incirca un mese ed quarto dell'intera durata del progetto.
	
	\subsubsection{Riassunto delle attività di verifica}
	
	\subsubsection{Risultati delle verifiche tramite analisi}
