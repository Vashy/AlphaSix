\newpage
\section{Introduzione} \label{Introduzione}
	
	\subsection{Scopo del documento}
	Questo \gloss{documento} ha l'intento di specificare la \gloss{pianificazione} e l'approccio che \gruppo\ adotterà per portare a termine il \gloss{progetto} Butterfly.
	All'interno vengono illustrate le strategie, le suddivisioni dei compiti, l'utilizzo delle \gloss{risorse}, la gestione dei rischi e le attività secondo le quali il team di sviluppo ha intenzione di lavorare.

    \subsection{Scopo del prodotto}

%%| Ex Norme di Progetto |%%
% Il prodotto che \gruppo\ si incarica di realizzare è Butterfly: un \gloss{tool} di supporto alle figure di	sviluppo di aziende di software
% (non solamente quella committente). Questo applicativo permette di incanalare le notifiche dei vari strumenti utilizzati nel percorso di
% \gloss{CI/CD} (come \gloss{Redmine}, \gloss{GitLab}, ecc.) di un software e, tramite un \gloss{Broker} (\gloss{Apache Kafka} in questo caso),
% spedirli alla persona interessata tramite canale di comunicazione preferito scelto da quest’ultimo (email, \gloss{Telegram}, \gloss{Slack}, ecc).

% \vspace{1cm}

%%| Ex Analisi dei Requisiti |%%
Lo scopo del \gloss{prodotto} è creare un \gloss{applicativo} per poter gestire i messaggi o le segnalazioni provenienti da diversi prodotti per la realizzazione di software,
come \gloss{Redmine}, \gloss{GitLab} e opzionalmente \gloss{SonarQube}, attraverso un \gloss{Broker} che possa incanalare questi messaggi e distribuirli a strumenti come
\gloss{Telegram}, e-mail e opzionalmente \gloss{Slack}.\par
Il software dovrà inoltre essere in grado di riconoscere il \gloss{Topic} dei messaggi in input per poterli inviare in determinati canali a cui i
destinatari dovranno iscriversi.\par
\`E anche richiesto di creare un canale specifico per gestire le particolari esigenze dell'azienda. Dovrà essere in grado, attraverso la lettura di
particolari	\gloss{metadati}, di reindirizzare i messaggi ricevuti al destinatario più appropriato.

% \vspace{1cm}

%%| Ex Piano di Qualifica |%%
% Il prodotto finale consiste in uno strumento in grado di ricevere messaggi o segnalazioni da vari tipi di servizi per la produzione software chiamati
% \gloss{Producer} (e.g. \gloss{GitLab}, \gloss{Redmine} e \gloss{SonarQube}), per poterli poi incanalare verso altri servizi chiamati \gloss{Consumer}
% atti a notificare gli sviluppatori (e.g. \gloss{Slack}, \gloss{Telegram} e Email).\par    
% L'applicazione sarà inoltre capace di organizzare le segnalazioni suddividendole per topic a cui i vari utenti dovranno iscriversi per esserne notificati.
% Nel caso in cui il destinatario dovesse segnalare di non essere disponibile, l'applicativo deve reindirizzare il messaggio verso la persona di competenza
% più prossima. 

% \vspace{1cm}

%%| Ex Piano di Progetto |%%
% Il prodotto che \gruppo\ si incarica di realizzare è Butterfly: un tool di supporto alle figure di sviluppo in aziende che producono software (non
% solamente quella del committente).
% Questo applicativo permette di incanalare le notifiche dei vari strumenti utilizzati nel percorso di \gloss{CI} e \gloss{CD} (come Redmine,
% GitLab, ecc.) di un software e, tramite un \gloss{broker} (\gloss{Apache Kafka} in questo caso), spedirli alla persona interessata tramite
% il canale di comunicazione preferito scelto da quest'ultimo (email, Telegram, Slack, ecc.).


	\subsection{Glossario e documenti esterni}\label{GlossarioDocumentiEsterni}
Al fine di rendere il documento più chiaro possibile, i termini che possono assumere un significato ambiguo o i riferimenti a documenti esterni
avranno delle diciture convenzionali:

\begin{itemize}
    \item \textbf{D}: indica che il termine si riferisce al nome di un particolare documento (ad esempio \PdPd).
    \item \textbf{G}: indica che il termine si riferisce ad una voce riportata nel \textit{Glossario v3.0.0\ped{D}} (ad esempio \gloss{Redmine}).
\end{itemize}

	\subsection{Riferimenti}
		\subsubsection{Riferimenti Normativi}
			\begin{itemize}
				\item \NdPd
				\item \gloss{Capitolato} d'appalto C1:\\
				\url{https://www.math.unipd.it/~tullio/IS-1/2018/Progetto/C1.pdf}
				\item Vincoli di \gloss{organigramma} e specifiche economiche\\
				\url{https://www.math.unipd.it/~tullio/IS-1/2018/Progetto/RO.html}
				\item \gloss{The Twelve-Factor App}, norme per lo sviluppo di un prodotto software consigliate dall'azienda.\\
				\url{https://12factor.net/}
			\end{itemize}
		
		\subsubsection{Riferimenti Informativi}\label{rifinfo}
			\begin{itemize}
				\item Software Engineering - Ian Sommerville - 10 th Edition (2016)
				\item Slide dell’insegnamento Ingegneria del Software\\
				\url{http://www.math.unipd.it/~tullio/IS-1/2018/}
				\item I sistemi per la gestione dei rischi (presentazione rilasciata dalla Bocconi per la gestione dei rischi).\\
				\url{https://www2.deloitte.com/content/dam/Deloitte/it/Documents/risk/Board\%20Academy\%20Corso\%20C6\%2020\%20dic\%202012\%20SDA\%20Bocconi.pdf}
				\item Fonte Figura \ref{fig:modello_incrementale}:\\
				\url{https://it.wikipedia.org/wiki/Modello_incrementale}
			\end{itemize}
		
	\subsection{Scadenze}\label{Scadenze}
	\gruppo\ ha deciso di rispettare le scadenze indicate dal professor Vardanega, riportate di seguito:
	\begin{itemize}
		\item \textbf{Revisione dei Requisiti}: 21-01-2019
		\item \textbf{Revisione di Progetto}: 15-03-2019
		\item \textbf{Revisione di Qualifica}: 19-04-2019
		\item \textbf{Revisione di Accettazione}: 17-05-2019.
	\end{itemize}
	
	\subsection{Modello di sviluppo} % Usare modello di sviluppo come termine al posto di Ciclo di vita in questo contesto. Vedere #26
	Data la natura del progetto, composto da più parti modulari e con un basso valore di accoppiamento, si è scelto di adottare un \gloss{modello di
	sviluppo} ibrido tra quello a \gloss{componenti} e quello incrementale.
	Essi si adattano particolarmente bene a questo tipo di progetto, in quanto:
	\begin{itemize}
		\item Il modello incrementale prevede ripetizioni identificate come cicli di incremento che verranno ripetute fino a quando il prodotto non arriverà a soddisfare i \gloss{requisiti} richiesti dal cliente
		\item Il modello a componenti è basato sul riuso di unità software che possono avere diverse dimensioni:
		\begin{itemize}
			\item \textbf{System reuse}: un intero \gloss{sistema}, composto da più applicazioni, può essere riusato come parte di un sistema di tanti sistemi % TODO: rivedere la frase
			\item \textbf{Application reuse}: un'applicazione può essere riusata incorporandola in altri sistemi senza apportare cambiamenti, 
				oppure configurandola
			\item \textbf{Component reuse}: i \gloss{componenti} di un'applicazione, che possono essere da sotto-sistemi a singoli oggetti, risiedono
				in un \gloss{cloud} o in server privati e possono essere accessibili tramite \gloss{Application Programming Interface} (API)
			\item \textbf{Object and function reuse}: componenti software che implementano una singola funzione o una classe oggetto. Si 
				possono riusare collegandole con lo sviluppo di nuovo codice. Molte di queste sono liberamente disponibili. 
		\end{itemize}
		Oppure, nel caso in cui le componenti siano così specifiche da essere troppo costoso adattarle ad una nuova situazione,
		è possibile fare "concept reuse", ovvero riusare le idee che stanno alla base del componente (e.g. riusare un \gloss{way of working} o un algoritmo). \par
		In particolare, i benefici che si possono trarre dal riuso sono:
		\begin{itemize}
			\item \textbf{Costo complessivo di sviluppo più basso}: perché il numero di componenti software che devono essere progettati, implementati e validati è minore.
			\item \textbf{Sviluppo accelerato}
			\item \textbf{Aumento dell'affidabilità}: un software che è stato provato e testato in altri sistemi risulta più affidabile di un software appena implementato.
			Buona parte dei suoi difetti di progettazione e implementazione dovrebbero già esser stati individuati e corretti.
			%\item  Ridotto rischio di processo, vero specialmente per grandi componenti software riusate come sottosistemi. È un fattore importante per il Project Manager perché riduce il margine di errore nella stima dei costi di un progetto.
			\item \textbf{Conformità con gli standard}: alcuni standard
			%, come gli interface standard, 
			possono essere implementati come set di componenti riusabili.
		\end{itemize}
		%e prevede che venga riutilizzata una base per lo sviluppo dei vari pezzi che formano il progetto, fra loro indipendenti
	\end{itemize}
	Inizialmente si possono spendere le risorse nella realizzazione di una base di partenza per le componenti, che verrà successivamente sviluppata per ciascun requisito richiesto, rappresentando il nucleo del prodotto finale.
	A tale \gloss{milestone} si potranno integrare le funzionalità secondarie richieste dal cliente insieme ai possibili requisiti impliciti desiderabili presenti nel capitolato. In base alla pianificazione svolta, le risorse disponibili saranno ridistribuite in modo da garantire lo sviluppo completo del prodotto.
	L'immagine che segue rappresenta il modello incrementale e come il progetto viene composto da componenti sviluppati ciascuno secondo cicli con fasi ben definite.
	\begin{figure}[H]
		\centering
		\includegraphics[scale=0.5]{img/modello_incrementale.png}
		\caption{Rappresentazione del modello incrementale\protect\footnotemark}
		\label{fig:modello_incrementale}
	\end{figure}

	\footnotetext{Fonte in \S\ref{rifinfo}}
	