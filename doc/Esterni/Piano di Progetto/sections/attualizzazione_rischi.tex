
\newpage
\section{Attualizzazione dei rischi} \label{AttualizzazioneDeiRischi}
    Per dar senso ai rischi posti in \S\ref{AnalisiDeiRischi}, forniamo un resoconto di quanti di questi si sono effettivamente verificati nei periodi di analisi dei requisiti e di progettazione della base tecnologica.\\
    Poiché i rischi cambiano dinamicamente durante lo svolgimento del progetto, questo appendice verrà aggiornato a intervalli regolari come stabilito nelle \NdPd.
    I rischi che riportiamo sono quanto abbiamo riscontrato nel portare a compimento le attività del progetto e possono differire da quelli presi precedentemente in analisi.\\

	\subsection{Classificazione}
    La classificazione dei rischi avverrà in modo simile a quella usata per l'analisi, riportando in questo caso anche la data in cui è stato riscontrato il rischio.
    Riportiamo per comodità del lettore la classificazione adottata per l'attualizzazione dei rischi.

	A ciascun rischio viene assegnato un codice identificativo in modo da essere univoco e facilmente riconoscibile.

	Questo codice è:

	\begin{center}
		\texttt{[Tipologia][ID]-[Gravità][Probabilità][Classe]:[Data]}
	\end{center}

	composto da:

	\begin{itemize}
		\item \textbf{Tipologia}:
			\begin{itemize}
				\item \textbf{O}: organizzativo.
				\item \textbf{P}: personale.
				\item \textbf{R}: requisiti.
				\item \textbf{S}: strumentale.
				\item \textbf{T}: tecnologico.
			\end{itemize}

		\item \textbf{ID}: numero progressivo di tre cifre che inizia da uno (001 - 999).
		\item \textbf{Gravità}:
			\begin{itemize}
				\item \textbf{0}: accettabile.
				\item \textbf{1}: tollerabile.
				\item \textbf{2}: inaccettabile.
			\end{itemize}

		\item \textbf{Probabilità}:
			\begin{itemize}
				\item \textbf{0}: bassa.
				\item \textbf{1}: media.
				\item \textbf{2}: alta.
			\end{itemize}

		\item \textbf{Classe}: ci si riferisce ai livelli di rischio individuati dalla matrice in Figura \ref{fig:rischi}
			\begin{itemize}
				\item \textbf{0}: basso (verde).
				\item \textbf{1}: medio (arancione).
				\item \textbf{2}: alto (rosso).
			\end{itemize}

		\item \textbf{Data}: data in cui si è verificato il rischio, nel formato riportato nelle \NdPd.
	\end{itemize}

	Ad esempio, con P001-021:2018-11-10 ci riferiamo al primo rischio di tipo personale, di gravità accettabile, probabilità alta e con un valore di classe medio, verificatosi in data 2018-11-10.

	\subsection{Lista rischi riscontrati}

	Per elencare i rischi usiamo una struttura tabellare dove indichiamo nella prima riga il codice identificativo e il nome di ciascun rischio. Mentre nelle righe successive indichiamo la dua descrizione, le strategie per rilevarlo e le eventuali contromisure e mitigazioni.\par
    
    Data la dinamicità dei rischi, le tabelle hanno un'alta probabilità di essere modificate.

	\subsubsection{Periodo precedente alla RR}

	\begin{table}[H]
		\begin{risktable}{\columnwidth}{m{4cm}m{11cm}}
			\thead{P002-122:2018-11-16} &
			Impreparazione del team a livello gestionale \\
			\rowcolor{\tablegray}
			\multicolumn{2}{X}{
				\textbf{Descrizione}: all'inizio dello svolgimento delle attività di progetto la conoscenza dei ruoli era insufficiente.
			}\\
			\multicolumn{2}{X}{
				\textbf{Strategia adottata}: ogni membro ha studiato lo scopo dei vari ruoli e le relative attività da svolgere.
			}\\
		\end{risktable}
		\caption{Specifica rischio P002-122:2018-11-16}
	\end{table}
	
	\mydoublerule{\linewidth}{0pt}{2pt}
	
	\begin{table}[H]
		\begin{risktable}{\columnwidth}{m{4cm}m{11cm}}
			\thead{P003-122:2019-01-13} &
			Verifica e approvazione errata di documenti \\
			\rowcolor{\tablegray}
			\multicolumn{2}{X}{
				\textbf{Descrizione}: si sono verificate delle incorrettezze durante la verifica e l'approvazione dei documenti.
			}\\
			\multicolumn{2}{X}{
				\textbf{Strategia adottata}: le verifiche vanno effettuate con più attenzione. È necessario dedicare più tempo alle attività di verifica.
			}\\
		\end{risktable}
		\caption{Specifica rischio P003-122:2019-01-13}
	\end{table}

	In seguito poniamo i rischi analizzati la cui priorità è cambiata nel corso del progetto:
	\begin{itemize}
		\item \textbf{P005-011}: Intesa parziale tra i membri del team di sviluppo.
	\end{itemize}


	\subsubsection{Periodo tra RR ed RP}

	\begin{table}[H]
		\begin{risktable}{\columnwidth}{m{4cm}m{11cm}}
			\thead{S001-100:2019-01-18} &
			Problematiche hardware \\

			\rowcolor{\tablegray}
			\multicolumn{2}{X}{
				\textbf{Descrizione}: uno dei pc destinato alla presentazione non funzionava al momento occorrente.
			}\\

			\multicolumn{2}{X}{
				\textbf{Strategia adottata}: abbiamo usato il pc di un altro membro del gruppo.
			}\\
		\end{risktable}
		\caption{Specifica rischio S001-100:2019-01-18}
	\end{table}

	\mydoublerule{\linewidth}{0pt}{2pt}

	\begin{table}[H]
		\begin{risktable}{\columnwidth}{m{4cm}m{11cm}}
			\thead{O002-111:2019-02-09} &
			Mancanza di comunicazione con l'azienda \\

			\rowcolor{\tablegray}
			\multicolumn{2}{X}{
				\textbf{Descrizione}: la mancanza di una comunicazione abbastanza approfondita con la proponente ci ha portato a dover rivedere nel dettaglio i casi d'uso dell'\AdRd.
			}\\

			\multicolumn{2}{X}{
				\textbf{Strategia adottata}: abbiamo previsto più momenti con cui comunicare con l'azienda, in modo da evitare ulteriori ritardi.
			}\\
		\end{risktable}
		\caption{Specifica rischio O002-111:2019-02-09}
	\end{table}

	%\mydoublerule{\linewidth}{0pt}{2pt}

	\begin{table}[H]

		\begin{risktable}{\columnwidth}{m{4cm}m{11cm}}
			\thead{P007-122:2019-02-12} &
			\gloss{Pianificazione} errata dei tempi \\
			\rowcolor{\tablegray}
			\multicolumn{2}{X}{
				\textbf{Descrizione}: l'impegno universitario particolarmente oneroso durante le sessioni d'esame e i periodi di malattia da parte dei membri del gruppo hanno causato ritardi nella programmazione. Questo ha portato a un ritardo non previsto nello svolgimento delle attività.
			}\\

			\multicolumn{2}{X}{
				\textbf{Strategia adottata}: abbiamo usato gli strumenti di pianificazione e coordinamento quali \gloss{GitHub} e Slack per gestire il lavoro da ultimare.
			}\\
		\end{risktable}
		\caption{Specifica rischio P007-122:2019-02-12}
	\end{table}

	\mydoublerule{\linewidth}{0pt}{2pt}

    \begin{table}[H]
		\begin{risktable}{\columnwidth}{m{4cm}m{11cm}}
			\thead{T001-100:2019-02-23} &
			Problematiche software \\

			\rowcolor{\tablegray}
			\multicolumn{2}{X}{
				\textbf{Descrizione}: i software installati per il progetto danno problemi al corretto funzionamento dei nostri pc. Nel nostro caso Docker causava il continuo riavvio di un pc.
			}\\

			\multicolumn{2}{X}{
				\textbf{Strategia adottata}: bloccare o terminare il software prima che questo causi complicazioni.
			}\\
		\end{risktable}
		\caption{Specifica rischio T001-100:2019-02-23}
	\end{table}

    %\mydoublerule{\linewidth}{0pt}{2pt}
	\vspace{1cm}

	In seguito poniamo i rischi analizzati la cui priorità è cambiata nel corso del progetto:
	\begin{itemize}
	    \item \textbf{R001-111}: Interpretazione errata dei requisiti: aggiunta o modifica di requisiti in corso di sviluppo.
	\end{itemize}

	\subsubsection{Periodo tra RP ed RQ}
	
	\begin{table}[H]
		\begin{risktable}{\columnwidth}{m{4cm}m{11cm}}
			\thead{P001-111:2019-04-02} &
			Inesperienza del team a livello tecnico \\
			
			\rowcolor{\tablegray}
			\multicolumn{2}{X}{
				\textbf{Descrizione}: l'inesperienza nell'utilizzo di design pattern visti in maniera teorica a lezione ha portato il gruppo ad utilizzarli in maniera parlzialmente erronea e questo si è dimostrato durante la presentazione della Product Baseline con il \RC. 
			}\\
			
			\multicolumn{2}{X}{
				\textbf{Strategia adottata}: è stato necessario modificare l'applicazione dei design pattern in maniera tale che questi siano totalmente conformi al progetto senza che ne complichino la progettazione e l'implementazione.
			}\\
		\end{risktable}
		\caption{Specifica rischio T001-100:2019-04-02}
	\end{table}

\mydoublerule{\linewidth}{0pt}{2pt}
\vspace{1cm}
	
	In seguito poniamo i rischi analizzati la cui priorità è cambiata nel corso del progetto:
	\begin{itemize}
		\item \textbf{P002-111}: Impreparazione del team a livello gestionale.
		\item \textbf{P006-122}: Cattiva amministrazione delle risorse.
	\end{itemize}
	
	
	%\subsubsection{Periodo successivo alla RQ}
	
	%In seguito poniamo i rischi analizzati la cui priorità è cambiata nel corso del progetto:
	%\begin{itemize}
	%	\item lorem ipsum
	%\end{itemize}