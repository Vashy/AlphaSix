%\newpage

\section{Consuntivo di periodo}	\label{consuntivo}
Nella fase di verifica al termine del periodo dell'analisi dei requisiti è stato redatto un consuntivo di periodo al fine di verificare se il numero di ore assegnate ad ogni ruolo è stato rispettato o meno. 



	\subsection{Analisi dei requisiti}
	Vengono riportate in seguito le ore di lavoro effettive relative al periodo di analisi dei requisiti.
	
	\begin{table}[H]
		\begin{detailtable}{\columnwidth}{m{3cm}YYYYYYY}
			\thead{Membro} & 
			\thead{Re} &
			\thead{Am} &
			\thead{An} &
			\thead{Pj} &
			\thead{Pr} &
			\thead{Ve} &
			\thead{Totale}\\\toprule\rowcolor{\tablegray}
			Ciprian Voinea & 11 (+3) & & 7 (-2) & & & 5 (-2) & 23 (-1)\\
			Laura Cameran & & 8 & 9 & & & 7 & 24\\\rowcolor{\tablegray}
			Matteo Marchiori & 7 (-1) & & 9 & & & 8 (+1) & 24\\
			Nicola Carlesso & 8 & & 10 (+1) & & & 7 & 25 (+1)\\\rowcolor{\tablegray} 
			Samuele Gardin & & 7 (-1) & 9 & & & 7 & 23 (-1)\\ 
			Timoty Graziero & & 7 (-1) & 8 (-1) & & & 9 (+2) & 24\\\bottomrule
		\end{detailtable}
		\caption{Ore consuntivate nel periodo di analisi dei requisiti}
	\end{table}
	
	Viene riportato in seguito il consuntivo relativo al periodo di analisi dei requisiti:
	
	\begin{table}[H]
		\begin{detailtable}{\columnwidth}{m{3cm}YY}
			\thead{Ruolo} & 
			\thead{Totale ore} &
			\thead{Costo in \euro}\\\toprule\rowcolor{\tablegray}
			Responsabile & 26 (+2) & 780,00 (+60,00)\\
			Amministratore & 22 (-2) & 440,00 (-40,00)\\\rowcolor{\tablegray}
			Analista & 52 (-2) & 1300,00 (-50,00)\\
			Progettista & & \\\rowcolor{\tablegray}
			Programmatore & &\\
			Verificatore & 43 (+1) & 645,00 (+15,00)\\\rowcolor{\tablegray}
			\textbf{Totale} & \textbf{143 (-1)} & \textbf{3165,00 (-15,00)} \\\bottomrule
		\end{detailtable}
		\caption{Consuntivo del periodo di analisi dei requisiti}
	\end{table}
	
	\subsubsection{Conclusioni}
	Non si ritiene opportuno aggiungere osservazioni in merito ai discostamenti di orario rispetto a quanto preventivato.\\
	Questo primo consuntivo di periodo è a noi di grande utilità poiché è possibile osservare in modo oggettivo l'andamento del lavoro.\\
	% Inoltre il preventivo iniziale viene presentato con questa prima versione del documento al proponente.
	
	\newpage
	
	\subsection{Progettazione della base tecnologica}
	Vengono riportate in seguito le ore di lavoro effettive relative al secondo periodo, denominato progettazione della base tecnologica.
	
	\begin{table}[H]
		\begin{detailtable}{\columnwidth}{m{3cm}YYYYYYY}
			\thead{Membro} & 
			\thead{Re} &
			\thead{Am} &
			\thead{An} &
			\thead{Pj} &
			\thead{Pr} &
			\thead{Ve} &
			\thead{Totale}\\\toprule\rowcolor{\tablegray}
			\CV &    & 6      &    & 10      & 14      & 17      & 46 \\
			\LC & 7  & 	      & 7  & 6 (-1)  & 6       & 17 (-2) & 43 (-3)\\\rowcolor{\tablegray}
			\MM & 	 & 	      &    & 10      & 26 (-2) & 6       & 42 (-2)\\
			\NC & 	 & 7      &    & 10      & 20      & 13 (+1) & 50 (+1)\\\rowcolor{\tablegray} 
			\SG & 8  &        &    & 5 (-2)  & 21      & 6       & 40 (-2)\\ 
			\TG & 10 & 7      & 7  & 10 (+2) & 15 (+1) & 7       & 56 (+3)\\\bottomrule
		\end{detailtable}
		\caption{Ore consuntivate nel periodo di progettazione della base tecnologica}
	\end{table}

	Viene riportato in seguito il consuntivo relativo al periodo di progettazione della base tecnologica:
	
	\begin{table}[H]
		\begin{detailtable}{\columnwidth}{m{3cm}YY}
			\thead{Ruolo} & 
			\thead{Totale ore} &
			\thead{Costo in \euro}\\\toprule\rowcolor{\tablegray}
			Responsabile & 25 & 660,00 \\
			Amministratore & 21 & 400,00 \\\rowcolor{\tablegray}
			Analista & 14 & 350,00\\
			Progettista & 51 (-1) & 1122,00 (-22,00) \\\rowcolor{\tablegray}
			Programmatore & 102 (-1) & 1530,00 (-15,00) \\
			Verificatore & 66 (-1) & 990,00 (-15,00) 
			\\\rowcolor{\tablegray}
			\textbf{Totale} & \textbf{269 (-3)} & \textbf{5052,00 (-52,00)} \\\bottomrule
		\end{detailtable}
		\caption{Consuntivo del periodo di progettazione della base tecnologica}
	\end{table}
	
	
	\subsubsection{Conclusioni}
	I discostamenti visibili da quest'ultimo consuntivo riguarda principalmente una variazione di ore tra le persone.
	Ciò è dovuto al fatto che nel periodo di sessione esami, i componenti del gruppo più impegnati hanno dedicato meno tempo al progetto, mentre gli studenti con meno esami hanno potuto proseguire più a lungo con il lavoro per il progetto. 
	\newpage
	
	\subsubsection{Preventivo a finire}
		La seguente tabella mostra il preventivo a finire.
		Lo scopo è quello di mettere in risalto la differenza tra l'ammontare preventivato e l'effettivo consuntivo.
		Se il valore del consuntivo di una certa attività non è presente si fa riferimento al valore preventivato.

		\begin{table}[H]
			\begin{detailtable}{\columnwidth}{YYY}
				\thead{Ruolo} & 
				\thead{Preventivo in \euro\ (Ore)} &
				\thead{Consuntivo in \euro\ (Ore)}\\\toprule\rowcolor{\tablegray}
				Progettazione della base tecnologica & 5179,00 (280) & 5052,00 (269) \\
				Progettazione di dettaglio e codifica & 4674,00
			    (254) & - \\\rowcolor{\tablegray}
				Validazione e collaudo & 1770,00 (93)
				& - \\
				\textbf{Totale} & \textbf{11623,00 (627)} & \textbf{11496,00 (616)} \\
			\end{detailtable}
			\caption{Preventivo a finire}
		\end{table}
	
	\subsubsection{Pianificazione futura}
	Per il periodo di progettazione di dettaglio e codifica si prevede un andamento delle ore più coeso tra i vari componenti. Si prevede però un possibile incremento di ore per il ruolo di \Ver, vista la mole maggiore di documenti e software da controllare, e un possibile decremento di ore per il ruolo di \Prog\ rispetto a quanto pianificato.