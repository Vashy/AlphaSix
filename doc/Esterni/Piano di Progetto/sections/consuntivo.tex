%\newpage

\section{Consuntivo di periodo}	\label{consuntivo}
Nell'attività di verifica al termine del periodo dell'analisi dei requisiti è stato redatto un consuntivo di periodo al fine di verificare se il numero di ore assegnate ad ogni ruolo è stato rispettato o meno.



	\subsection{Analisi dei requisiti}\label{consuntivoAnalisiDeiRequisiti}
	Vengono riportate in seguito le ore di lavoro effettive relative al periodo di analisi dei requisiti.

	\begin{table}[H]
		\begin{detailtable}{\columnwidth}{m{3cm}YYYYYYY}
			\thead{Membro} &
			\thead{Re} &
			\thead{Am} &
			\thead{An} &
			\thead{Pj} &
			\thead{Pr} &
			\thead{Ve} &
			\thead{Totale}\\\toprule\rowcolor{\tablegray}
			Ciprian Voinea & 11 (+3) & & 7 (-2) & & & 5 (-2) & 23 (-1)\\
			Laura Cameran & & 8 & 9 & & & 7 & 24\\\rowcolor{\tablegray}
			Matteo Marchiori & 7 (-1) & & 9 & & & 8 (+1) & 24\\
			Nicola Carlesso & 8 & & 10 (+1) & & & 7 & 25 (+1)\\\rowcolor{\tablegray}
			Samuele Gardin & & 7 (-1) & 9 & & & 7 & 23 (-1)\\
			Timoty Graziero & & 7 (-1) & 8 (-1) & & & 9 (+2) & 24\\\bottomrule
		\end{detailtable}
		\caption{Ore consuntivate nel periodo di analisi dei requisiti}
	\end{table}

	Viene riportato in seguito il consuntivo relativo al periodo di analisi dei requisiti:

	\begin{table}[H]
		\begin{detailtable}{\columnwidth}{m{3cm}YY}
			\thead{Ruolo} &
			\thead{Totale ore} &
			\thead{Costo in \euro}\\\toprule\rowcolor{\tablegray}
			Responsabile & 26 (+2) & 780,00 (+60,00)\\
			Amministratore & 22 (-2) & 440,00 (-40,00)\\\rowcolor{\tablegray}
			Analista & 52 (-2) & 1300,00 (-50,00)\\
			Progettista & & \\\rowcolor{\tablegray}
			Programmatore & &\\
			Verificatore & 43 (+1) & 645,00 (+15,00)\\\rowcolor{\tablegray}
			\textbf{Totale} & \textbf{143 (-1)} & \textbf{3165,00 (-15,00)} \\\bottomrule
		\end{detailtable}
		\caption{Consuntivo del periodo di analisi dei requisiti}
	\end{table}

	\subsubsection{Conclusioni}
	Pur con delle variazioni orarie, siamo riusciti a rimanere al di sotto dei costi posti in preventivo.\\
    Ci siamo resi conto che aumentando il numero dei controlli delle ore impiegate tra una revisione e l'altra potremo ottenere risultati migliori.\\
	Questo primo consuntivo di periodo è a noi di grande utilità poiché è possibile osservare in modo oggettivo l'andamento del lavoro.\\
	% Inoltre il preventivo iniziale viene presentato con questa prima versione del documento al proponente.

	\newpage

	\subsection{Progettazione della base tecnologica}\label{consuntivoProgettazioneBaseTecnologica}
	Vengono riportate in seguito le ore di lavoro effettive relative al secondo periodo, denominato progettazione della base tecnologica.

	\begin{table}[H]
		\begin{detailtable}{\columnwidth}{m{3cm}YYYYYYY}
			\thead{Membro} &
			\thead{Re} &
			\thead{Am} &
			\thead{An} &
			\thead{Pj} &
			\thead{Pr} &
			\thead{Ve} &
			\thead{Totale}\\\toprule\rowcolor{\tablegray}
			\CV &  & 6 &  & 10 & 13 & 17 & 46\\
			\LC & 7 & & 7 & 7 & 3 (-3) & 17 (-2) & 41(-5)\\\rowcolor{\tablegray}
			\MM & & & & 10 & 28 & 6 & 44\\
			\NC & & 6 (-1) & & 10 & 20 & 12 & 48 (-1)\\\rowcolor{\tablegray}
			\SG & 8 & & & 7 & 19 (-2) & 6 & 40 (-2)\\
			\TG & 10 & 9 (+2) & 7 & 12 (+4) & 14 & 7 & 59 (+6)\\\bottomrule
		\end{detailtable}
		\caption{Ore consuntivate nel periodo di progettazione della base tecnologica}
	\end{table}

	Viene riportato in seguito il consuntivo relativo al periodo di progettazione della base tecnologica:

	\begin{table}[H]
		\begin{detailtable}{\columnwidth}{m{3cm}YY}
			\thead{Ruolo} &
			\thead{Totale ore} &
			\thead{Costo in \euro}\\\toprule\rowcolor{\tablegray}
			Responsabile & 25 & 750,00 \\
			Amministratore & 21 (+1) & 420,00 (+20,00)\\\rowcolor{\tablegray}
			Analista & 14 & 350,00 \\
			Progettista & 53 (+1) & 1166,00 (+22,00) \\\rowcolor{\tablegray}
			Programmatore & 100 (-2) & 1500,00 (-30,00) \\
			Verificatore & 65 (-2) & 975,00 (-30,00)
			\\\rowcolor{\tablegray}
			\textbf{Totale} & \textbf{278 (-2)} & \textbf{5161,00 (-18,00)} \\\bottomrule
		\end{detailtable}
		\caption{Consuntivo del periodo di progettazione della base tecnologica}
	\end{table}


	\subsubsection{Conclusioni}
    In questo periodo abbiamo effettuato due variazioni significative.\\
	La prima riguarda la variazione delle ore assegnate al \Progr\ e al \Prog, per poter consegnare i prodotti richiesti.
    La seconda, la variazione di ore tra le persone. Ciò è dovuto al fatto che nel periodo di sessione esami, i componenti del gruppo più impegnati hanno dedicato meno tempo al progetto, mentre gli studenti con meno esami hanno potuto proseguire più a lungo con il lavoro per il progetto.

	\newpage

	\subsection{Progettazione di dettaglio e codifica}\label{consuntivoProgettazioneDettaglioCodifica}
	Vengono riportate in seguito le ore di lavoro effettive relative al terzo periodo, denominato progettazione di dettaglio e codifica.

	\begin{table}[H]
		\begin{detailtable}{\columnwidth}{m{3cm}YYYYYYY}
			\thead{Membro} &
			\thead{Re} &
			\thead{Am} &
			\thead{An} &
			\thead{Pj} &
			\thead{Pr} &
			\thead{Ve} &
			\thead{Totale}\\\toprule\rowcolor{\tablegray}
			\CV &   & 6 & 3 (+2) & 9 (+2) & 16 (-4) & 14 & 48\\
			\LC & 7 &   & 2 & 10 & 21 (+4) &    & 40 (+4)\\\rowcolor{\tablegray}
			\MM & 8 (+1) & 4 (-2) &   & 8  &    & 18 & 38 (-1)\\
			\NC & 8 &   &   & 5 (-2)  & 11 & 22 (+1) & 46 (-1)\\\rowcolor{\tablegray}
			\SG &   & 8 &   & 16 &    & 18 (+2) & 42 (+2)\\
			\TG &   &   &   & 14 & 13 (-4) & 8 (-3) & 36 (-7)\\\bottomrule
		\end{detailtable}
		\caption{Ore consuntivate nel periodo di progettazione di dettaglio e codifica}
	\end{table}

	Viene riportato in seguito il consuntivo relativo al periodo di progettazione di dettaglio e codifica:

	\begin{table}[H]
		\begin{detailtable}{\columnwidth}{m{3cm}YY}
			\thead{Ruolo} &
			\thead{Totale ore} &
			\thead{Costo in \euro}\\\toprule\rowcolor{\tablegray}
			Responsabile & 23 (+1) & 690,00 (+30,00) \\
			Amministratore & 18 (-2) & 360,00 (-40,00)\\\rowcolor{\tablegray}
			Analista & 5 (+2) & 125,00 (+50,00) \\
			Progettista & 62 & 1364,00 \\\rowcolor{\tablegray}
			Programmatore & 61 (-4) & 915,00 (-60,00) \\
			Verificatore & 81 (+1) & 1215,00 (+15,00)
			\\\rowcolor{\tablegray}
			\textbf{Totale} & \textbf{250 (-2)} & \textbf{4669,00 (-5,00)} \\\bottomrule
		\end{detailtable}
		\caption{Consuntivo del periodo di progettazione di dettaglio e codifica}
	\end{table}

	\subsubsection{Ragioni degli scostamenti}
	I discostamenti che si sono verificati in questo periodo sono dovuti a impegni imprevisti di alcuni membri del gruppo, ad esempio i colloqui con aziende.

	\subsubsection{Conclusioni}
    Per quanto riguarda i ruoli di questo terzo periodo abbiamo variato l'assegnazione delle ore in maniera da diminuire quelle assegnate al \Progr\ e al \Ver\ in favore a quelle dell'\Ana. 
    Questa modifica non ha portato a variazioni al preventivo.
	Malgrado la correzione della product baseline non si sono verificate variazioni orarie significative.
	Per questo motivo non prevediamo un particolare piano di mitigazione per il futuro, facendoci ben pensare che anche quella dell'ultimo periodo sia molto verosimile.

	\newpage

    \subsection{Validazione e collaudo}\label{consuntivoValidazioneCollaudo}
    Vengono riportate in seguito le ore di lavoro effettive relative al quarto periodo, denominato Validazione e collaudo.
    
    \begin{table}[H]
        \begin{detailtable}{\columnwidth}{m{3cm}YYYYYYY}
            \thead{Membro} &
            \thead{Re} &
            \thead{Am} &
            \thead{An} &
            \thead{Pj} &
            \thead{Pr} &
            \thead{Ve} &
            \thead{Totale}\\\toprule\rowcolor{\tablegray}
            \CV & 4 (-1) &   &   &   &   & 6 (+1) & 10\\
            \LC &   & 5 &   &   &   & 18 (+2) & 23 (+2)\\\rowcolor{\tablegray}
            \MM & 3 & 5 &   &   & 2 (+2) & 12 (-2) & 22\\
            \NC &   & 5 &   &   &   & 5 (+2) & 10 (+2)\\\rowcolor{\tablegray}
            \SG & 7 (-2) &   &   &   &   & 15 (+1) & 22 (-1)\\
            \TG & 2 (-1) &   &   &   &   & 8 (+2) & 10 (+1)\\\bottomrule
        \end{detailtable}
        \caption{Ore consuntivate nel periodo di Validazione e collaudo}
    \end{table}
    
    Viene riportato in seguito il consuntivo relativo al periodo di Validazione e collaudo:
    
    \begin{table}[H]
        \begin{detailtable}{\columnwidth}{m{3cm}YY}
            \thead{Ruolo} &
            \thead{Totale ore} &
            \thead{Costo in \euro}\\\toprule\rowcolor{\tablegray}
            Responsabile & 16 (-4) & 480,00 (-120,00) \\
            Amministratore & 15 & 300,00\\\rowcolor{\tablegray}
            Analista &   & \\
            Progettista &   &  \\\rowcolor{\tablegray}
            Programmatore & 2 (+2) & 30,00 (+30,00)\\
            Verificatore & 64 (+6) & 960,00 (+90,00)
            \\\rowcolor{\tablegray}
            \textbf{Totale} & \textbf{97 (+4)} & \textbf{1770,00} \\\bottomrule
        \end{detailtable}
        \caption{Consuntivo del periodo di Validazione e collaudo}
    \end{table}
    
    \subsubsection{Ragioni degli scostamenti}
    %TODO
    
    \subsubsection{Conclusioni}
    %TODO
    
    \newpage

	\subsubsection{Preventivo a finire}\label{PreventivoAFinire}
		La seguente tabella mostra il preventivo a finire. Lo
		scopo è quello di mettere in risalto la differenza tra
		l'ammontare preventivato e l'effettivo consuntivo. Se il valore del consuntivo di una certa attività non è presente si fa riferimento al valore preventivato.

		\begin{table}[H]
			\begin{detailtable}{\columnwidth}{YYY}
				\thead{Ruolo} &
				\thead{Preventivo in \euro\ (Ore)} &
				\thead{Consuntivo in \euro\ (Ore)}\\\toprule\rowcolor{\tablegray}
				Progettazione della base tecnologica & 5179,00 (280) & 5161,00 (278) \\
				Progettazione di dettaglio e codifica & 4674,00
			    (252) & 4669,00 (250) \\\rowcolor{\tablegray}
				Validazione e collaudo & 1770,00 (93)
				& 1770,00 (97) \\
				\textbf{Totale} & \textbf{11623,00 (627)} & \textbf{11600,00 (625)} \\
                \bottomrule
			\end{detailtable}
			\caption{Preventivo a finire}
		\end{table}

	\subsubsection{Considerazioni finali}
    % TODO
    % PRENDERE SPUNTO
    % Alla fine di ogni fase di revisione ci siamo accorti che sono quasi sempre presenti delle variazioni orarie nel consuntivo rispetto a quanto preventivato.
    % Può trattarsi di grandi o piccoli scostamenti, dovuti spesso a dei ritardi accumulati nei periodi immediatamente successivi alla consegna dei prodotti. In ogni caso il gruppo è formato da sei persone e le ore a preventivo sono state calcolate in modo da poter coprire tutte le attività previste dal progetto.\\
    % Per fare in modo che le variazioni orarie vengano mitigate cerchiamo di usare sempre con maggior frequenza gli strumenti già impiegati per la coordinazione del gruppo in modo da stabilire con maggior precisione a che punto delle attività siamo arrivati.\\
    % Si cerca di tenere sotto controllo le ore dedicate da ciascun membro a ciascun ruolo, in modo da evitare che si verifichino eccessi o difetti rispetto a quanto preventivato.
    % Per mitigare le variazioni verificatesi valuteremo altri strumenti per il controllo più accurato delle ore impiegate da ciascun membro per ciascuna attività.
    % Questo sarà utile specialmente per le fasi successive, perché sarà necessario verificare di soddisfare le aspettative del proponente e al contempo risolvere le eventuali mancanze riscontrate nel primo periodo.\\
    % Per le variazioni verificatesi nel primo e secondo periodo, ogni membro con delle carenze orarie in un ruolo cercherà di apportare un maggior contributo in quel ruolo nel secondo periodo, in modo da restare in pari con le ore dedicate al progetto.
