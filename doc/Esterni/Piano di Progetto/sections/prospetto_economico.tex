\newpage
\section{Prospetto economico} \label{ProspettoEconomico}
	La sezione riporta il prospetto economico dettagliato rispettante la suddivisione del lavoro stabilita.

	\subsection{Analisi dei requisiti}\label{Analisi dei Requisiti}
		Il prospetto economico relativo al periodo di Analisi dei requisiti è il seguente.
		
		\begin{table}[H]
			\begin{detailtable}{\columnwidth}{m{3cm}YY}
				\thead{Ruolo} & 
				\thead{Totale ore} &
				\thead{Costo in \euro}\\\toprule\rowcolor{\tablegray}
				Responsabile & 24 & 720,00\\
				Amministratore & 24 & 480,00\\\rowcolor{\tablegray}
				Analista & 54 & 1350,00\\
				Progettista & - & - \\\rowcolor{\tablegray}
				Programmatore & - & - \\
				Verificatore & 42 & 630,00\\\rowcolor{\tablegray}
				\textbf{Totale} & \textbf{144} & \textbf{3180,00}\\\bottomrule
			\end{detailtable}
			\caption{Prospetto economico del periodo di Analisi dei requisiti}
		\end{table}

	\subsection{Progettazione della base tecnologica}\label{Progettazione base tecnologica}
		Il prospetto economico relativo al periodo di Progettazione della base tecnologica è il seguente.
		
		\begin{table}[H]
			\begin{detailtable}{\columnwidth}{m{3cm}YY}
				\thead{Ruolo} & 
				\thead{Totale ore} &
				\thead{Costo in \euro}\\\toprule\rowcolor{\tablegray}
				Responsabile & 25 & 750,00\\
				Amministratore & 20 & 400,00\\\rowcolor{\tablegray}
				Analista & 14 & 350,00\\
				Progettista & 52 & 1144,00\\\rowcolor{\tablegray}
				Programmatore & 102 & 1530,00\\
				Verificatore & 67 & 1005,00\\\rowcolor{\tablegray}
				\textbf{Totale} & \textbf{280} & \textbf{5179,00}\\\bottomrule
			\end{detailtable}
			\caption{Prospetto economico del periodo di Progettazione della base tecnologica}
		\end{table}
		
	\newpage
	
	\subsection{Progettazione di dettaglio e codifica}\label{Progettazione di dettaglio e codifica}
		Il prospetto economico relativo al periodo di Progettazione di dettaglio e codifica è il seguente.
		
		\begin{table}[H]
			\begin{detailtable}{\columnwidth}{m{3cm}YY}
				\thead{Ruolo} & 
				\thead{Totale ore} &
				\thead{Costo in \euro}\\\toprule\rowcolor{\tablegray}
				Responsabile & 22 & 660,00\\
				Amministratore & 20 & 400,00\\\rowcolor{\tablegray}
				Analista & - & - \\
				Progettista & 62 & 1364,00\\\rowcolor{\tablegray}
				Programmatore & 69 & 1035,00\\
				Verificatore & 81 & 1215,00\\\rowcolor{\tablegray}
				\textbf{Totale} & \textbf{254} & \textbf{4674,00}\\\bottomrule
			\end{detailtable}
			\caption{Prospetto economico del periodo di Progettazione di dettaglio e codifica}
		\end{table}
		
	\subsection{Validazione e collaudo}\label{Validazione e collaudo}
	Il prospetto economico relativo al periodo di Validazione e collaudo è il seguente.
	
		\begin{table}[H]
			\begin{detailtable}{\columnwidth}{m{3cm}YY}
				\thead{Ruolo} & 
				\thead{Totale ore} &
				\thead{Costo in \euro}\\\toprule\rowcolor{\tablegray}
				Responsabile & 20 & 600,00\\
				Amministratore & 15 & 300,00\\\rowcolor{\tablegray}
				Analista & - & - \\
				Progettista & - & - \\\rowcolor{\tablegray}
				Programmatore & - & - \\
				Verificatore & 58 & 870,00\\\rowcolor{\tablegray}
				\textbf{Totale} & \textbf{93} & \textbf{1770,00}\\\bottomrule
			\end{detailtable}
			\caption{Prospetto economico del periodo di Validazione e collaudo}
		\end{table}
		
	\newpage

	\subsection{Totale}
		\subsubsection{Totale del prospetto economico rendicontato}
		Viene in seguito riportato il prospetto economico riguardante le ore preventivate a carico del committente, quindi dei periodi di
		\hyperref[Progettazione base tecnologica]{Progettazione della base tecnologica},
		\hyperref[Progettazione di dettaglio e codifica]{Progettazione di dettaglio e codifica},
		\hyperref[Validazione e collaudo]{Validazione e collaudo}.

		\begin{table}[H]
			\begin{detailtable}{\columnwidth}{m{3cm}YY}
				\thead{Ruolo} & 
				\thead{Totale ore} &
				\thead{Costo in \euro}\\\toprule\rowcolor{\tablegray}
				Responsabile & 67 & 2010,00\\
				Amministratore & 55 & 1100,00\\\rowcolor{\tablegray}
				Analista & 14 & 350,00\\
				Progettista & 114 & 2508,00\\\rowcolor{\tablegray}
				Programmatore & 171 & 2565,00\\
				Verificatore & 206 & 3090,00\\\rowcolor{\tablegray}
				\textbf{Totale} & \textbf{627} & \textbf{11623,00}\\\bottomrule
			\end{detailtable}
			\caption{Prospetto economico rendicontato}
		\end{table}

		\subsubsection{Totale del prospetto economico con investimento}
		Viene in seguito riportato il prospetto economico riguardante le ore totali di lavoro, inclusive delle ore di investimento.
	
		\begin{table}[H]
			\begin{detailtable}{\columnwidth}{m{3cm}YY}
				\thead{Ruolo} & 
				\thead{Totale ore} &
				\thead{Costo in \euro}\\\toprule\rowcolor{\tablegray}
				Responsabile & 115 & 3450,00\\
				Amministratore & 91 & 1820,00\\\rowcolor{\tablegray}
				Analista & 80 & 2000,00\\
				Progettista & 138 & 3036,00\\\rowcolor{\tablegray}
				Programmatore & 183 & 2745,00\\
				Verificatore & 260 & 3900,00\\\rowcolor{\tablegray}
				\textbf{Totale} & \textbf{867} & \textbf{16951,00}\\\bottomrule
			\end{detailtable}
			\caption{Prospetto economico rendicontato e di investimento}
		\end{table}

	\subsubsection{Conclusioni}
	Da quanto si può evincere dalle ultime due tabelle, la differenza dei due totali non è trascurabile.
	La motivazione risiede nel fatto che non possedevamo inizialmente le conoscenze adeguate
	e l'esperienza minima per svolgere il progetto in modo lineare. Di conseguenza, le ore di formazione personale
	per ogni ruolo aumentano il totale delle ore di lavoro da un 10\% per il \Progr\ fino ad un 570\% per l'\Ana,
	il quale deve svolgere la maggior parte del suo lavoro durante la fase di analisi che non è rendicontata.
