\newpage
\section{Descrizione generale}\label{DescrizioneGenerale}

%	\subsection{Intento del prodotto}    
%    L'obiettivo del prodotto è di fornire uno strumento standard per la gestione delle notifiche generate da software utilizzati nella \gloss{CI/CD}.
%    Per ottenere questo verrà usato un \gloss{pattern Publisher / Subscriber}, di modo da contribuire alla scalabilità del \gloss{sistema} e ottenere una miglior suddivisione logica dei \gloss{componenti}.
	
	\subsection{Funzioni del prodotto}
	
    Le funzioni offerte dal prodotto sono:
    \begin{itemize}
        \item Ricezione delle segnalazioni provenienti da Redmine e GitLab
        \item Gestione dei Topic tramite un Broker
        \item Inoltro dei messaggi verso Telegram ed e-mail
        \item Personalizzazione da parte dell'utente finale di impostazioni riguardanti le segnalazioni interessate
	\end{itemize}

	\subsection{Caratteristiche degli utenti}
    
    Gli utenti che useranno il prodotto saranno team di sviluppatori software che lavorano abitualmente usando gli strumenti per realizzare \gloss{CI/CD}.
    Tramite Butterfly, essi potranno configurare la \gloss{pipeline} per lo sviluppo del proprio progetto, in modo da ottenere le notifiche in tempo reale direttamente sulle piattaforme di messaggistica selezionate.
    Nel caso in cui la persona non sia reperibile, verrà effettuata una segnalazione in un apposito calendario sul quale il sistema farà riferimento per l'inoltro alla seconda persona più appropriata a riceverla all'interno del contesto lavorativo.
	
%	\subsection{Vincoli del progetto}
%	
%		\subsubsection{Requisiti minimi}
%            \begin{itemize}
%                \item Ogni componente del software rispetterà, per quanto applicabile, i fattori esposti nel documento \gloss{The Twelve-Factor App}.
%                \item Ogni componente sarà istanziabile in un container \gloss{docker}.
%                \item Vengono esposti \gloss{API Rest} ed eventuali altri protocolli dei componenti per l'uso dell'applicativo.
%                \item vengono rilasciati tutti i test necessari a garantire la qualità del software sviluppato.
%            \end{itemize}
%	
%		\subsubsection{Requisiti opzionali}
%		    \begin{itemize}
%                \item Viene usato il linguaggio Java, Python oppure Node.js per lo sviluppo dei componenti applicativi.
%                \item Viene usato Apache Kafka come Broker.
%            \end{itemize}