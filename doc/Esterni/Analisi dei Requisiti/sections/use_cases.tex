\clearpage
\section{Casi d'uso}\label{CasiDUso}
Questa sezione elenca le funzionalità offerte da \progetto\ descritte attraverso il linguaggio \gloss{UML}.
\progetto\ può essere visto come l'insieme di più sottosistemi che verranno in seguito elencati e che sono stati descritti in modo molto generale anche attraverso la Figura \ref{fig:butterfly}.\\
Questa mostra in maniera un po' più chiarificante la suddivisione e ne facilita l'analisi per la stesura dei casi d'uso.
Abbiamo quindi come attori non soltanto le applicazioni che mandano messaggi al sistema, ma anche componenti quali i vari Producer e Consumer che interagiscono col Gestore Personale. \\
All'interno dei casi d'uso il termine ``sistema'' è inteso come l'intera applicazione \progetto, mentre ``piattaforma di messaggistica'' fa sempre riferimento a Telegram o Email e il termine ``persona di fiducia'' indica la persona a cui inoltrare le proprie notifiche nei giorni in cui non si è disponibili.
	
	\subsection{Attori}
	\begin{itemize}
		\item Redmine
		\item GitLab
		\item Producer Redmine
		\item Producer GitLab
		\item Utente non acceduto
		\item Utente non registrato
		\item Utente (inteso come acceduto e registrato nel sistema)
		\item Consumer Telegram
		\item Consumer Email
		\item Telegram
		\item Email
	\end{itemize}

	\subsection{Sistemi e sottosistemi}
	Per indicare i vari sottosistemi nei codici dei casi d'uso sono allegati delle sigle per rendere più chiara la posizione del caso d'uso all'interno di \progetto.
	\begin{itemize}
		%\item \textbf{B}: \progetto.
		\item \textbf{PR}: Producer Redmine.
		\item \textbf{PG}: Producer GitLab.
		\item \textbf{GP}: Gestore Personale.
		\item \textbf{CT}: Consumer telegram.
		\item \textbf{CE}: Consumer Email.
		\item \textbf{BT}: bot Telegram.
		\item \textbf{SE}: server Email.
	\end{itemize}

	\clearpage

	\subsection{Elenco casi d'uso}

%TODO: da ricordarsi: se qualcuno è offline, c'è la possibilità che il messaggio venga perso.

\newcounter{uccount}
\newcounter{subuccount}[uccount]
\newcounter{subsubuccount}[subuccount]
\newcounter{subsubsubuccount}[subsubuccount]

%\stepcounter{uccount}
%
%\subsubsection{UC\theuccount-PR - Redmine invia segnalazione al Producer Redmine}
    \begin{figure}[H]
		\centering
		\includegraphics[width=0.5\textwidth]{img/casi_d'uso/UC\theuccount.png}\\
		\caption{UC1-PR - Redmine invia segnalazione al Producer Redmine}
	\end{figure}
\begin{itemize}
	\item \textbf{Codice}: UC\theuccount-PR.
	\item \textbf{Titolo}: Redmine invia segnalazione al Producer Redmine.
	\item \textbf{Attori primari}: Redmine.
	\item \textbf{Descrizione}: Redmine invia a \progetto\ una segnalazione.
	\item \textbf{Precondizione}: su Redmine viene eseguita un'operazione che scaturisce una
	segnalazione da inviare a \progetto.
	\item \textbf{Postcondizione}: il Producer Redmine riceve la segnalazione da Redmine.
	\item \textbf{Scenario principale}:
	\begin{enumerate}
		\item Viene eseguita un'operazione in Redmine da far scaturire l'invio di una segnalazione
		\item Redmine procede all'invio della segnalazione al Producer Redmine
        \item La segnalazione viene ricevuta dal Producer Redmine
	\end{enumerate}

\end{itemize}

\stepcounter{subuccount}

\subsubsection{UC\theuccount.\thesubuccount-PR - Redmine segnala apertura issue al Producer Redmine}
%    \begin{figure}[H]
%		\centering
%		\includegraphics[width=0.5\textwidth]{img/casi_d'uso/UC2.png}\\
%		\caption{UC\theuccount-PR - Redmine segnala apertura issue al Producer Redmine}
%	\end{figure}
\begin{itemize}
	\item \textbf{Codice}: UC\theuccount.\thesubuccount-PR.
	\item \textbf{Titolo}: Redmine segnala apertura issue al Producer Redmine.
	\item \textbf{Attori primari}: Redmine.
	\item \textbf{Descrizione}: Redmine segnala a \progetto\ l'apertura di una nuova issue tramite webhook.

	L'apertura di una issue in un particolare progetto su Redmine contiene i seguenti campi di interesse:
	\begin{itemize}
		\item Status
		\item Tracker
		\item Subject
		\item Priority e opzionalmente:
		\begin{itemize}
			\item Description
			\item Assignee
		\end{itemize}
	\end{itemize}
	Il campo status conterrà sempre al suo interno ``opened''.
	\item \textbf{Precondizione}: Viene aperta una issue su Redmine da
	segnalare a \progetto.
	\item \textbf{Postcondizione}: il Producer Redmine riceve la segnalazione di apertura issue da Redmine.
	\item \textbf{Scenario principale}:
	\begin{enumerate}
		\item Viene aperta una nuova issue su Redmine compilando i campi indicati
		\item Redmine procede all'invio della segnalazione di issue al Producer Redmine
        \item Il Producer Redmine riceve la segnalazione di apertura issue
	\end{enumerate}

\end{itemize}

\stepcounter{subuccount}

\subsubsection{UC\theuccount.\thesubuccount-PR - Redmine segnala la modifica di una issue al Producer Redmine}
%	\begin{figure}[H]
%		\centering
%		\includegraphics[width=0.5\textwidth]{img/casi_d'uso/UC3.png}\\
%		\caption{UC\theuccount-PR - Redmine segnala la modifica di una issue al Producer Redmine}
%	\end{figure}
\begin{itemize}
	\item \textbf{Codice}: UC\theuccount\thesubuccount-PR.
	\item \textbf{Titolo}: Redmine segnala la modifica di una issue al Producer Redmine.
	\item \textbf{Attori primari}: Redmine.
	\item \textbf{Descrizione}: quando una issue viene modificata, l'invio di tale segnalazione
	avviene da parte di Redmine tramite webhook.
	I campi di interesse sono gli stessi dell'apertura di una issue, con la differenza che necessariamente il campo status contiene ora ``updated''.
	\item \textbf{Precondizione}: Viene modificata una issue già aperta su un
	progetto di Redmine da segnalare a \progetto.
	\item \textbf{Postcondizione}: il Producer Redmine riceve la segnalazione di modifica issue da Redmine.
	\item \textbf{Scenario principale}:
	\begin{enumerate}
		\item Viene modificata una issue già esistente su Redmine
		\item Redmine procede all'invio della segnalazione di modifica issue al Producer Redmine
        \item Il Producer Redmine riceve la segnalazione di modifica issue
	\end{enumerate}

\end{itemize}

\stepcounter{subuccount}

\subsubsection{UC\theuccount.\thesubuccount-PR - Redmine segnala il commento di una issue al Producer Redmine}
%	\begin{figure}[H]
%		\centering
%		\includegraphics[width=0.5\textwidth]{img/casi_d'uso/UC3.png}\\
%		\caption{UC\theuccount-PR - Redmine segnala la modifica di una issue al Producer Redmine}
%	\end{figure}
\begin{itemize}
	\item \textbf{Codice}: UC\theuccount.\thesubuccount-PR.
	\item \textbf{Titolo}: Redmine segnala il commento di una issue al Producer Redmine.
	\item \textbf{Attori primari}: Redmine.
	\item \textbf{Descrizione}: quando una issue viene commentata, l'invio di tale segnalazione
	avviene da parte di Redmine tramite webhook.
	I campi di interesse sono gli stessi dell'apertura di una issue, con la differenza che necessariamente il campo status contiene ora ``updated''.
	\item \textbf{Precondizione}: Viene modificata una issue già aperta su un
	progetto di Redmine da segnalare a \progetto.
	\item \textbf{Postcondizione}: il Producer Redmine riceve la segnalazione del commento di una issue da Redmine.
	\item \textbf{Scenario principale}:
	\begin{enumerate}
		\item Viene modificata una issue già esistente su Redmine
		\item Redmine procede all'invio della segnalazione di modifica issue al Producer Redmine
	\end{enumerate}

\end{itemize}


\stepcounter{uccount}
%\clearpage
\subsubsection{UC\theuccount-PR - Redmine invia segnalazione al Producer Redmine}
%    \begin{figure}[H]
%		\centering
%		\includegraphics[width=0.5\textwidth]{img/casi_d'uso/UC2.png}\\
%		\caption{UC\theuccount-PR - Redmine segnala apertura issue al Producer Redmine}
%	\end{figure}
	\begin{itemize}
		\item \textbf{Codice}: UC\theuccount-PR.
		\item \textbf{Titolo}: Redmine invia segnalazione al Producer Redmine.
		\item \textbf{Attori primari}: Redmine.
		\item \textbf{Descrizione}: Redmine invia a \progetto\ una segnalazione.
		\item \textbf{Precondizione}: su Redmine viene eseguita una operazione che scaturisce una
		segnalazione da inviare a \progetto.
		\item \textbf{Postcondizione}: il Producer Redmine riceve la segnalazione da Redmine.
		\item \textbf{Scenario principale}: 
		\begin{enumerate}
			\item Viene eseguita un'operazione
			\item Redmine procede all'invio della segnalazione al Producer Redmine
		\end{enumerate}
		
	\end{itemize}

	\stepcounter{subuccount}
	
	\subsubsection{UC\theuccount.\thesubuccount-PR - Redmine segnala apertura issue al Producer Redmine}
	%    \begin{figure}[H]
	%		\centering
	%		\includegraphics[width=0.5\textwidth]{img/casi_d'uso/UC2.png}\\
	%		\caption{UC\theuccount-PR - Redmine segnala apertura issue al Producer Redmine}
	%	\end{figure}
	\begin{itemize}
		\item \textbf{Codice}: UC\theuccount-PR.
		\item \textbf{Titolo}: Redmine segnala apertura issue al Producer Redmine.
		\item \textbf{Attori primari}: Redmine.
		\item \textbf{Descrizione}: Redmine segnala a \progetto\ l'apertura di una nuova issue tramite webhook.
		
		L'apertura di una issue in un particolare progetto su Redmine contiene i seguenti campi di interesse:
		\begin{itemize}
			\item Status
			\item Tracker
			\item Subject
			\item Priority e opzionalmente:
			\begin{itemize}
				\item Description
				\item Assignee
			\end{itemize}
		\end{itemize}
		Il campo status conterrà sempre al suo interno ``opened''.
		\item \textbf{Precondizione}: Viene aperta una issue su Redmine da
		segnalare a \progetto.
		\item \textbf{Postcondizione}: il Producer Redmine riceve la segnalazione da Redmine.
		\item \textbf{Scenario principale}: 
		\begin{enumerate}
			\item Viene aperta una nuova issue su Redmine
			\item Redmine procede all'invio della segnalazione di issue al Producer Redmine
		\end{enumerate}
		
	\end{itemize}

\stepcounter{uccount}
%\clearpage
	\subsubsection{UC\theuccount-PR - Redmine segnala la modifica di una issue al Producer Redmine}
%	\begin{figure}[H]
%		\centering
%		\includegraphics[width=0.5\textwidth]{img/casi_d'uso/UC3.png}\\
%		\caption{UC\theuccount-PR - Redmine segnala la modifica di una issue al Producer Redmine}
%	\end{figure}
	\begin{itemize}
		\item \textbf{Codice}: UC\theuccount-PR.
		\item \textbf{Titolo}: Redmine segnala la modifica di una issue al Producer Redmine.
		\item \textbf{Attori primari}: Redmine.
		\item \textbf{Descrizione}: quando una issue viene modificata, l'invio di tale segnalazione
			avviene da parte di Redmine tramite webhook.
		\item \textbf{Precondizione}: Viene modificata una issue già aperta su un
		progetto di Redmine da segnalare a \progetto.
		\item \textbf{Postcondizione}: il Producer Redmine riceve la segnalazione da Redmine.
		\item \textbf{Scenario principale}: 
		\begin{enumerate}
			\item Viene modificata una issue già esistente su Redmine
			\item Redmine procede all'invio della segnalazione di modifica issue al Producer Redmine
		\end{enumerate}
		
	\end{itemize}

\stepcounter{uccount}
%\clearpage
\subsubsection{UC\theuccount-GP - Producer GitLab invia messaggio al Gestore Personale}
	\begin{figure}[H]
		\centering
		\includegraphics[width=0.7\textwidth]{img/casi_d'uso/UC7.png}\\
		\caption{UC\theuccount-GP - Producer GitLab invia messaggio al Gestore Personale}
	\end{figure}
	\begin{itemize}
		\item \textbf{Codice}: UC\theuccount-GP.
		\item \textbf{Titolo}: Producer GitLab invia messaggio al Gestore Personale.
		\item \textbf{Attori primari}: Producer GitLab.
		\item \textbf{Descrizione}: il Producer GitLab, dopo aver ricevuto una segnalazione da GitLab, elabora un messaggio da inviare al Gestore Personale.
		\item \textbf{Precondizione}: il Producer GitLab ha ricevuto una segnalazione da GitLab.
		\item \textbf{Postcondizione}: il Producer GitLab ha inviato al Gestore Personale il messaggio  \newline elaborato.
		\item \textbf{Scenario principale}: 
		\begin{enumerate}
			\item Il Producer GitLab riceve una segnalazione da GitLab
			\item Il Producer GitLab prepara il messaggio in modo che venga inserito correttamente nel Gestore Personale
			\item Il Poducer GitLab invia il messaggio al Gestore Personale
		\end{enumerate}
		
	\end{itemize}
	\newpage
	\stepcounter{subuccount}
	\subsubsection{UC\theuccount.\thesubuccount-GP - Producer GitLab invia uno o più messaggi di commit al Gestore Personale}
		
		\begin{itemize}
			\item \textbf{Codice}: UC\theuccount.\thesubuccount-GP.
			\item \textbf{Titolo}: Producer GitLab invia uno o più messaggi di commit al Gestore Personale.
			\item \textbf{Attori primari}: Producer GitLab.
			\item \textbf{Descrizione}: il Producer GitLab, dopo aver ricevuto una segnalazione di push da  \newline GitLab, elabora un messaggio particolare per i commit che verrà catalogato sotto il Topic "commits".
			Il messaggio elaborato conterrà i campi:
			\begin{itemize}
				\item Project
				\item Topic
				\item Message
			\end{itemize}
			\item \textbf{Precondizione}: il Producer GitLab ha ricevuto una segnalazione da GitLab.
			\item \textbf{Postcondizione}: il Producer GitLab ha inviato uno o più messaggi elaborati di commit.
			\item \textbf{Scenario principale}: 
			\begin{enumerate}
				\item Il Producer GitLab riceve la segnalazione di uno o più commit da GitLab
				\item Il Producer GitLab prepara i messaggi in modo che vengano inseriti correttamente nel Gestore Personale
				\item Il Producer GitLab invia i messaggi di
				commit al Gestore Personale
			\end{enumerate}
			
		\end{itemize}
		
	\stepcounter{subuccount}
	\subsubsection{UC\theuccount.\thesubuccount-GP -  Producer GitLab invia messaggio di issue al Gestore Personale}
		\begin{figure}[H]
			\centering
			\includegraphics[width=0.8\textwidth]{img/casi_d'uso/UC7_2.png}\\
			\caption{UC\theuccount.\thesubuccount-GP -  Producer GitLab invia messaggio di issue al Gestore Personale}
		\end{figure}
		\begin{itemize}
			\newpage
			\item \textbf{Codice}: UC\theuccount.\thesubuccount-GP.
			\item \textbf{Titolo}:  Producer GitLab invia messaggio di issue al Gestore Personale.
			\item \textbf{Attori primari}: Producer GitLab.
			\item \textbf{Descrizione}: il Producer GitLab, dopo aver ricevuto una segnalazione di issue da GitLab, controlla se la issue è appena stata creata o si tratta della modifica di una issue preesistente. Il messaggio, una volta elaborato, conterrà i campi:
			\begin{itemize}
				\item Project
				\item Topic
				\item Subject e opzionalmente:
				\begin{itemize}
					\item Description
					\item Due Date
					\item Milestone
					\item Assignee
				\end{itemize}
			\end{itemize}
			\item \textbf{Precondizione}: il Producer GitLab ha ricevuto una segnalazione da GitLab.
			\item \textbf{Postcondizione}: il Producer GitLab ha inviato al Gestore Personale il messaggio  \newline elaborato.
			\item \textbf{Scenario principale}: 
			\begin{enumerate}
				\item Il Producer GitLab riceve la segnalazione di issue da GitLab
				\item Il Producer GitLab prepara il messaggio di issue in modo che venga inserito  \newline correttamente nel Gestore Personale
				\item Il Producer GitLab invia il messaggio di issue al Gestore Personale
			\end{enumerate}
			
		\end{itemize}
		
		\stepcounter{subsubuccount}
		\subsubsection{UC\theuccount.\thesubuccount.\thesubsubuccount-GP - Producer GitLab invia messaggio di una nuova issue al Gestore Personale}
			
			\begin{itemize}
				\item \textbf{Codice}: UC\theuccount.\thesubuccount.\thesubsubuccount-GP.
				\item \textbf{Titolo}: Producer GitLab invia messaggio di una nuova issue al Gestore Personale.
				\item \textbf{Attori primari}: Producer GitLab.
				\item \textbf{Descrizione}: il Producer GitLab, dopo aver ricevuto una segnalazione di una nuova issue da GitLab, elabora il messaggio da inoltrare al Gestore Personale.
				% che conterrà i campi:
				% \begin{itemize}
				% 	\item Project
				% 	\item Topic
				% 	\item Subject e opzionalmente:
				% 	\begin{itemize}
				% 		\item Description
				% 		\item Due Date
				% 		\item Milestone
				% 		\item Assignee
				% 	\end{itemize}
				% \end{itemize}
				\item \textbf{Precondizione}: il Producer GitLab ha ricevuto una segnalazione da GitLab.
				\item \textbf{Postcondizione}: il Producer GitLab ha inviato al Gestore Personale il messaggio   \newline elaborato di nuova issue.
				\item \textbf{Scenario principale}: 
				\begin{enumerate}
					\item Il Producer GitLab riceve la segnalazione di una nuova issue da GitLab
					\item Il Producer GitLab prepara il messaggio di una nuova issue in modo che venga inserito correttamente nel Gestore Personale
					\item Il Producer GitLab invia il messaggio di una nuova issue al Gestore Personale
				\end{enumerate}
				
			\end{itemize}
		
		\stepcounter{subsubuccount}
		\subsubsection{UC\theuccount.\thesubuccount.\thesubsubuccount-GP - Producer GitLab invia messaggio di modifica di una issue al Gestore Personale}
			
			\begin{itemize}
				\item \textbf{Codice}: UC\theuccount.\thesubuccount.\thesubsubuccount-GP.
				\item \textbf{Titolo}: Producer GitLab invia messaggio di modifica di una issue al Gestore Personale.
				\item \textbf{Attori primari}: Producer GitLab.
				\item \textbf{Descrizione}: il Producer GitLab, dopo aver ricevuto una segnalazione di modifica di una issue da GitLab, controlla se sono stati modificati i campi ``label'' o ``title''.
				In caso positivo, viene inviato un messaggio elaborato al Gestore Personale.
				% , il quale conterrà:
				% \begin{itemize}
				% 	\item Project
				% 	\item Topic
				% 	\item Subject e opzionalmente:
				% 	\begin{itemize}
				% 		\item Description
				% 		\item Due Date
				% 		\item Milestone
				% 		\item Assignee
				% 	\end{itemize}
				% \end{itemize}
				\item \textbf{Precondizione}: il Producer GitLab ha ricevuto una segnalazione da GitLab.
				\item \textbf{Postcondizione}: il Producer GitLab ha inviato al Gestore Personale il messaggio   \newline elaborato di modifica issue.
				\item \textbf{Scenario principale}: 
				\begin{enumerate}
					\item Il Producer GitLab riceve la segnalazione di modifica issue da GitLab
					\item Il Producer GitLab prepara il messaggio di modifica issue in modo che venga inserito correttamente nel Gestore Personale
					\item Il Poducer GitLab invia il messaggio di modifica issue al Gestore Personale
				\end{enumerate}
				\item \textbf{Estensioni}: 
				\begin{enumerate}
					\item Se ci sono dei messaggi non validi, questi vengono scartati [UC\theuccount.\thesubuccount.3-GP].
				\end{enumerate}
			\end{itemize}
		
		\stepcounter{subsubuccount}
		\subsubsection{UC\theuccount.\thesubuccount.\thesubsubuccount-GP - Producer GitLab scarta i messaggi non validi}
			
			\begin{itemize}
				\item \textbf{Codice}: UC\theuccount.\thesubuccount.\thesubsubuccount-GP.
				\item \textbf{Titolo}: Producer GitLab scarta i messaggi non validi.
				\item \textbf{Attori primari}: Producer GitLab.
				\item \textbf{Descrizione}: il Producer GitLab, dopo aver ricevuto un messaggio da GitLab, controlla
                se il messaggio contiene una segnalazione che \progetto\ sa analizzare. In caso negativo, il messaggio viene scartato.
                \item \textbf{Precondizione}: il Producer GitLab ha ricevuto un messaggio da GitLab.
                \item \textbf{Postcondizione}: il Producer GitLab ha scartato il messaggio.
                \item \textbf{Scenario principale}: 
                \begin{enumerate}
                    \item Il Producer GitLab riceve un messaggio
                    \item Il Producer GitLab vede che il messaggio contiene una segnalazione che \progetto\ non sa analizzare
                    \item Il Producer GitLab scarta il messaggio non valido.
                \end{enumerate}
            \end{itemize}

\stepcounter{uccount}
%\clearpage
	\subsubsection{UC\theuccount-PG - Gitlab segnala la modifica di una issue al Producer Gitlab}
%	\begin{figure}[H]
%		\centering
%		\includegraphics[width=0.5\textwidth]{img/casi_d'uso/UC5.png}\\
%		\caption{UC\theuccount-PG - Gitlab segnala la modifica di una issue al Producer Gitlab}
%	\end{figure}
	\begin{itemize}
		\item \textbf{Codice}: UC\theuccount-PG.
		\item \textbf{Titolo}: Gitlab segnala la modifica di una issue al Producer Gitlab.
		\item \textbf{Attori primari}: GitLab.
		\item \textbf{Descrizione}: GitLab segnala la modifica di una issue esistente tramite webhook a
		\newline \progetto.
		I campi di interesse sono gli stessi dell'apertura issue, con la differenza che il campo action contiene ``udpate''.
		\item \textbf{Precondizione}: Viene modificata una issue già aperta su un
		progetto di GitLab da segnalare a \progetto.
		\item \textbf{Postcondizione}: il Producer GitLab riceve la segnalazione da GitLab.
		\item \textbf{Scenario principale}: 
		\begin{enumerate}
			\item Viene modificata una issue già esistente su GitLab
			\item GitLab procede all'invio della segnalazione di modifica issue al Producer GitLab
		\end{enumerate}
		
	\end{itemize}

\stepcounter{uccount}
%\clearpage
\subsubsection{UC\theuccount-PG - GitLab segnala un evento di push a Producer GitLab}
	\begin{figure}[H]
		\centering
		\includegraphics[width=0.7\textwidth]{img/casi_d'uso/UC6.png}\\
		\caption{UC\theuccount-PG - GitLab segnala un evento di push a Producer GitLab}
	\end{figure}
	\begin{itemize}
		\item \textbf{Codice}: UC\theuccount-PG.
		\item \textbf{Titolo}: GitLab segnala un evento di push a Producer GitLab.
		\item \textbf{Attori primari}: GitLab.
		\item \textbf{Descrizione}: GitLab segnala un evento di push tramite webhook. L'evento di	push può essere composto da uno o più commit.
		\item \textbf{Precondizione}: Viene effettuato un push su GitLab da segnalare a \progetto.
		\item \textbf{Postcondizione}: il Producer GitLab riceve la segnalazione da GitLab.
		\item \textbf{Scenario principale}: 
		\begin{enumerate}
			\item Viene effettuato un push in GitLab
			\item GitLab procede all'invio della segnalazione di push al Producer GitLab
		\end{enumerate}
		
	\end{itemize}

\stepcounter{uccount}
%\clearpage
\subsubsection{UC\theuccount-BT - Consumer Telegram inoltra il messaggio finale al bot Telegram}
%	\begin{figure}[H]
%		\centering
%		\includegraphics[width=0.8\textwidth]{img/casi_d'uso/UC11.png}\\
%		\caption{UC\theuccount-BT - Consumer Telegram inoltra il messaggio finale al bot Telegram}
%	\end{figure}
	\begin{itemize}
		\item \textbf{Codice}: UC\theuccount-BT.
		\item \textbf{Titolo}: Consumer Telegram inoltra il messaggio finale al bot Telegram.
		\item \textbf{Attori primari}: Consumer Telegram.
		\item \textbf{Descrizione}: il Consumer Telegram inoltra il messaggio finale al bot Telegram, il quale notifica il destinatario finale attraverso Telegram.
		\item \textbf{Precondizione}: il Consumer Telegram ha ricevuto almeno un messaggio.
		\item \textbf{Postcondizione}: il bot Telegram ha ricevuto il messaggio finale.
		\item \textbf{Scenario principale}: 
		\begin{enumerate}
			\item Il Consumer Telegram riceve un messaggio dal Gestore Personale
			\item Il Consumer Telegram inoltra il messaggio finale al bot Telegram
            \item Il bot Telegram riceve il messaggio dal Consumer Telegram
		\end{enumerate}
		
	\end{itemize}

\stepcounter{uccount}
%\clearpage
\subsubsection{UC\theuccount-GP - Producer GitLab invia messaggio al Gestore Personale}
	\begin{figure}[H]
		\centering
		\includegraphics[width=0.7\textwidth]{img/casi_d'uso/UC8.png}\\
		\caption{UC\theuccount-GP - Producer GitLab invia messaggio al Gestore Personale}
	\end{figure}
	\begin{itemize}
		\item \textbf{Codice}: UC\theuccount-GP.
		\item \textbf{Titolo}: Producer GitLab invia messaggio al Gestore Personale.
		\item \textbf{Attori primari}: Producer GitLab.
		\item \textbf{Descrizione}: il Producer GitLab, dopo aver ricevuto una segnalazione da GitLab, elabora un messaggio da inviare al Gestore Personale.
		\item \textbf{Precondizione}: il Producer GitLab ha ricevuto una segnalazione da GitLab.
		\item \textbf{Postcondizione}: il Producer GitLab ha inviato al Gestore Personale il messaggio elaborato.
		\item \textbf{Scenario principale}: 
		\begin{enumerate}
			\item il Producer GitLab riceve una segnalazione da GitLab
			\item il Producer GitLab prepara il messaggio in modo che venga inserito correttamente nel Gestore Personale
			\item il Poducer GitLab invia il messaggio al Gestore Personale
		\end{enumerate}
		
	\end{itemize}
	
	\stepcounter{subuccount}
	\paragraph{UC\theuccount.\thesubuccount-GP - Producer GitLab invia uno o più messaggi di commit al Gestore Personale}
		
		\begin{itemize}
			\item \textbf{Codice}: UC\theuccount.\thesubuccount-GP.
			\item \textbf{Titolo}: Producer GitLab invia uno o più messaggi di commit al Gestore Personale.
			\item \textbf{Attori primari}: Producer GitLab.
			\item \textbf{Descrizione}: il Producer GitLab, dopo aver ricevuto una segnalazione di push da GitLab,	elabora un messaggio particolare per i commit che verrà catalogato sotto il Topic "commits".
			Il messaggio elaborato conterrà i campi:
			\begin{itemize}
				\item Project
				\item Topic
				\item Message
			\end{itemize}
			\item \textbf{Precondizione}: il Producer GitLab ha ricevuto una segnalazione da GitLab.
			\item \textbf{Postcondizione}: il Producer GitLab ha inviato uno o più messaggi elaborati di commit.
			\item \textbf{Scenario principale}: 
			\begin{enumerate}
				\item Il Producer GitLab riceve la segnalazione di uno o più commit da GitLab
				\item Il Producer GitLab prepara i messaggi in modo che vengano inseriti correttamente nel Gestore Personale
				\item Il Poducer GitLab invia i messaggi di
				commit al Gestore Personale
				\item Producer GitLab invia uno o più messaggi
			 	di commit al Gestore Personale
			\end{enumerate}
			
		\end{itemize}
		
	\stepcounter{subuccount}
	\paragraph{UC\theuccount.\thesubuccount-GP -  Producer GitLab invia messaggio di issue al Gestore Personale}
		\begin{figure}[H]
			\centering
			\includegraphics[width=0.8\textwidth]{img/casi_d'uso/UC8_2.png}\\
			\caption{UC\theuccount.\thesubuccount-GP -  Producer GitLab invia messaggio di issue al Gestore Personale}
		\end{figure}
		\begin{itemize}
			\item \textbf{Codice}: UC\theuccount.\thesubuccount-GP.
			\item \textbf{Titolo}:  Producer GitLab invia messaggio di issue al Gestore Personale.
			\item \textbf{Attori primari}: Producer GitLab.
			\item \textbf{Descrizione}: il Producer GitLab, dopo aver ricevuto una segnalazione di issue da GitLab, controlla se la issue è appena stata creata o si tratta della modifica di una issue preesistente. Il messaggio, una volta elaborato, conterrà i campi:
			\begin{itemize}
				\item Project
				\item Topic
				\item Subject e opzionalmente:
				\begin{itemize}
					\item Description
					\item Due Date
					\item Milestone
					\item Assignee
				\end{itemize}
			\end{itemize}
			\item \textbf{Precondizione}: il Producer GitLab ha ricevuto una segnalazione da GitLab.
			\item \textbf{Postcondizione}: il Producer GitLab ha inviato al Gestore Personale il messaggio elaborato.
			\item \textbf{Scenario principale}: 
			\begin{enumerate}
				\item Il Producer GitLab riceve la segnalazione di issue da GitLab
				\item Il Producer GitLab prepara il messaggio di issue in modo che venga inseriti correttamente nel Gestore Personale
				\item Il Poducer GitLab invia il messaggio di issue al Gestore Personale
				\item Producer GitLab invia il messaggio di issue al Gestore Personale
			\end{enumerate}
			
		\end{itemize}
		
		\stepcounter{subsubuccount}
		\subparagraph{UC\theuccount.\thesubuccount.\thesubsubuccount-GP - Producer GitLab invia messaggio di una nuova issue al Gestore Personale}
			
			\begin{itemize}
				\item \textbf{Codice}: UC\theuccount.\thesubuccount.\thesubsubuccount-GP.
				\item \textbf{Titolo}: Producer GitLab invia messaggio di una nuova issue al Gestore Personale.
				\item \textbf{Attori primari}: Producer GitLab.
				\item \textbf{Descrizione}: il Producer GitLab, dopo aver ricevuto una segnalazione di una nuova issue da GitLab, elabora il messaggio che conterrà i campi:
				\begin{itemize}
					\item Project
					\item Topic
					\item Subject e opzionalmente:
					\begin{itemize}
						\item Description
						\item Due Date
						\item Milestone
						\item Assignee
					\end{itemize}
				\end{itemize}
				\item \textbf{Precondizione}: il Producer GitLab ha ricevuto una segnalazione da GitLab.
				\item \textbf{Postcondizione}: il Producer GitLab ha inviato al Gestore Personale il messaggio elaborato di nuova issue.
				\item \textbf{Scenario principale}: 
				\begin{enumerate}
					\item Il Producer GitLab riceve la segnalazione di una nuova issue da GitLab
					\item Il Producer GitLab prepara il messaggio di una nuova issue in modo che venga inseriti correttamente nel Gestore Personale
					\item Il Poducer GitLab invia il messaggio di una nuova issue al Gestore Personale
				\end{enumerate}
				
			\end{itemize}
		
		\stepcounter{subsubuccount}
		\subparagraph{UC\theuccount.\thesubuccount.\thesubsubuccount-GP - Producer GitLab invia messaggio di modifica di una issue al Gestore Personale}
			
			\begin{itemize}
				\item \textbf{Codice}: UC\theuccount.\thesubuccount.\thesubsubuccount-GP.
				\item \textbf{Titolo}: Producer GitLab invia messaggio di modifica di una issue al Gestore Personale.
				\item \textbf{Attori primari}: Producer GitLab.
				\item \textbf{Descrizione}: il Producer GitLab, dopo aver ricevuto una segnalazione di modifica di una issue da GitLab, controlla se sono stati modificati i campi Label o Title.
				In caso positivo, viene inviato un messaggio elaborato al Gestore Personale, il quale conterrà:
				\begin{itemize}
					\item Project
					\item Topic
					\item Subject e opzionalmente:
					\begin{itemize}
						\item Description
						\item Due Date
						\item Milestone
						\item Assignee
					\end{itemize}
				\end{itemize}
				\item \textbf{Precondizione}: il Producer GitLab ha ricevuto una segnalazione da GitLab.
				\item \textbf{Postcondizione}: il Producer GitLab ha inviato al Gestore Personale il messaggio elaborato di modifica issue.
				\item \textbf{Scenario principale}: 
				\begin{enumerate}
					\item Il Producer GitLab riceve la segnalazione di modifica issue da GitLab
					\item Il Producer GitLab prepara il messaggio di modifica issue in modo che venga inseriti correttamente nel Gestore Personale
					\item Il Poducer GitLab invia il messaggio di modifica issue al Gestore Personale.
				\end{enumerate}
				\item \textbf{Estensioni}: 
				\begin{enumerate}
					\item Se ci sono dei messaggi non validi, questi vengono scartati [UCUC\theuccount.\thesubuccount.3-GP].
				\end{enumerate}
			\end{itemize}
		
		\stepcounter{subsubuccount}
		\subparagraph{UC\theuccount.\thesubuccount.\thesubsubuccount-GP - Producer GitLab scarta i messaggi non validi}
			
			\begin{itemize}
				\item \textbf{Codice}: UC\theuccount.\thesubuccount.\thesubsubuccount-GP.
				\item \textbf{Titolo}: Producer GitLab scarta i messaggi non validi.
				\item \textbf{Attori primari}: Producer GitLab.
				\item \textbf{Descrizione}: il Producer GitLab, dopo aver ricevuto una segnalazione di una modifica issue da GitLab, controlla
				se sono state modificati i campi Label o Title. In caso negativo, il messaggio viene scartato.
				\item \textbf{Precondizione}: il Producer GitLab ha ricevuto una segnalazione da GitLab.
				\item \textbf{Postcondizione}: il Producer GitLab ha scartato il messaggio.
				\item \textbf{Scenario principale}: 
				\begin{enumerate}
					\item Il Producer GitLab riceve la segnalazione di modifica issue da GitLab
					\item Il Producer GitLab vede che non sono stati modificati i campi Label o Title
					\item Il Producer GitLab scarta i messaggi non validi.
				\end{enumerate}
			\end{itemize}

\stepcounter{uccount}
%\clearpage
\subsubsection{UC\theuccount-CT - Gestore Personale invia il messaggio finale al Consumer Telegram}
%	\begin{figure}[H]
%		\centering
%		\includegraphics[width=0.6\textwidth]{img/casi_d'uso/UC9.png}\\
%		\caption{UC\theuccount-CT - Gestore Personale invia il messaggio finale al Consumer Telegram}
%	\end{figure}
	\begin{itemize}
		\item \textbf{Codice}: UC\theuccount-CT.
		\item \textbf{Titolo}: Gestore Personale invia il messaggio finale al Consumer Telegram.
		\item \textbf{Attori primari}: Gestore Personale.
		\item \textbf{Descrizione}: il Gestore Personale, dopo aver ricevuto il messaggio elaborato
		dai Producer Redmine o GitLab, controlla i messaggi sullo specifico Topic "commits", se ce ne
		sono vengono analizzati in base alla	lista di keyword. Se gli utenti iscritti a quella determinata
		keyword sono disponibili e se vogliono ricevere il messaggio tramite Telegram viene preparato
		il messaggio finale da inviare e inviato al Consumer Telegram. Altrimenti valuta il campo Topic del
		messaggio e controlla chi è iscritto a quel Topic, se gli utenti iscritti a quel Topic sono disponibili e
		se vogliono ricevere il messaggio tramite Telegram. Se tutte queste condizioni sono verificate, viene preparato
		il messaggio finale da inviare e inviato al Consumer Telegram.\\
		Il messaggio finale, una volta elaborato, conterrà i campi:
		\begin{itemize}
			\item Id della chat del destinatario
			\item Applicazione di provenienza
			\item Ora di invio
			\item Tipo di segnalazione(commit o issue)
			\item Project
			\item Topic
			\item Subject e opzionalmente
		 	\begin{itemize}
				\item Description
				\item Due date
				\item Milestone
				\item Assignee
			\end{itemize}
		\end{itemize}
		\item \textbf{Precondizione}: il Gestore Personale ha ricevuto il messaggio elaborato dai Producer Redmine o GitLab.
		\item \textbf{Postcondizione}: Il Gestore Personale ha inviato il messaggio finale al Consumer Telegram.
		\item \textbf{Scenario principale}: 
		\begin{enumerate}
			\item ll Gestore Personale riceve un messaggio dal Producer Redmine o dal Producer GitLab
			\item Il Gestore Personale valuta quali utenti sono iscritti al Topic del messaggio ricevuto, se sono disponibili e se vogliono ricevere il messaggio tramite Telegram
			\item Gestore Personale procede all'invio del messaggio finale al Consumer Telegram
		\end{enumerate}
		
	\end{itemize}

\stepcounter{uccount}
%\clearpage
\subsubsection{UC\theuccount-CE - Gestore Personale invia il messaggio finale al Consumer Email}
%	\begin{figure}[H]
%		\centering
%		\includegraphics[width=0.6\textwidth]{img/casi_d'uso/UC10.png}\\
%		\caption{UC\theuccount-CE - Gestore Personale invia il messaggio finale al Consumer Email}
%	\end{figure}
	\begin{itemize}
		\item \textbf{Codice}: UC\theuccount-CE.
		\item \textbf{Titolo}: Gestore Personale invia il messaggio finale al Consumer Email.
		\item \textbf{Attori primari}: Gestore Personale.
		\item \textbf{Descrizione}: il Gestore Personale, dopo aver ricevuto il messaggio elaborato dal
		Producer Redmine o GitLab, controlla i messaggi sullo specifico Topic "commits", se ce ne sono
		vengono analizzati in base alla	lista di keyword. Se gli utenti iscritti a quella determinata
		keyword sono disponibili e se vogliono ricevere il messaggio tramite Email viene preparato il
		messaggio finale da inviare e inviato al Consumer Email. Altrimenti valuta il campo Topic del
		messaggio e controlla se qualcuno è iscritto a quel Topic, se gli utenti iscritti a quel Topic sono
		disponibili e se vogliono ricevere il messaggio tramite Email. Se tutte queste condizioni sono
		verificate, viene preparato il messaggio finale da inviare e inviato al Consumer Email.\\
		Il messaggio finale, una volta elaborato, conterrà i campi:
		\begin{itemize}
			\item Email del destinatario
			\item Applicazione di provenienza
			\item Ora di invio
			\item Tipo di segnalazione(commit, issue)
			\item Project
			\item Topic
			\item Subject e opzionalmente
		 	\begin{itemize}
				\item Description
				\item Due date
				\item Milestone
				\item Assignee
			\end{itemize}
		\end{itemize}
		\item \textbf{Precondizione}: il Gestore Personale ha ricevuto il messaggio elaborato dal Producer Redmine o GitLab.
		\item \textbf{Postcondizione}: Il Gestore Personale ha inviato il messaggio finale al Consumer Email.
		\item \textbf{Scenario principale}: 
		\begin{enumerate}
			\item ll Gestore Personale riceve un messaggio dal Producer Redmine o dal Producer GitLab
			\item Il Gestore Personale valuta quali utenti sono iscritti al Topic del messaggio ricevuto, se sono disponibili e se vogliono ricevere il messaggio tramite Email
			\item Gestore Personale procede all'invio del messaggio finale al Consumer Email
		\end{enumerate}
		
	\end{itemize}

\stepcounter{uccount}
%\clearpage
\subsubsection{UC\theuccount-BT - Consumer Telegram inoltra il messaggio finale al bot Telegram}
	\begin{figure}[H]
		\centering
		\includegraphics[width=0.7\textwidth]{img/casi_d'uso/UC11.png}\\
		\caption{UC\theuccount-BT - Consumer Telegram inoltra il messaggio finale al bot Telegram}
	\end{figure}
	\begin{itemize}
		\item \textbf{Codice}: UC\theuccount-BT.
		\item \textbf{Titolo}: Consumer Telegram inoltra il messaggio finale al bot Telegram.
		\item \textbf{Attori primari}: Consumer Telegram.
		\item \textbf{Descrizione}: il Consumer Telegram inoltra il messaggio finale al bot Telegram, il quale notifica il destinatario finale attraverso Telegram.
		\item \textbf{Precondizione}: il Consumer Telegram ha ricevuto almeno un messaggio.
		\item \textbf{Postcondizione}: il bot Telegram ha ricevuto il messaggio finale con successo.
		\item \textbf{Scenario principale}: 
		\begin{enumerate}
			\item Il Consumer telegram riceve un messaggio dal Gestore Personale
			\item Il Consumer Telegram inoltra il messaggio finale al bot Telegram
		\end{enumerate}
		
	\end{itemize}

\stepcounter{uccount}
%\clearpage
\input{casi_d'uso/UC12}

\stepcounter{uccount}
%\clearpage
\subsubsection{UC\theuccount-GP - L'utente rimuove se stesso dal sistema}

\begin{itemize}
	\item \textbf{Codice}: UC\theuccount-GP.
	\item \textbf{Titolo}: l'utente rimuove se stesso dal sistema.
	\item \textbf{Attori primari}: utente.
	\item \textbf{Descrizione}:  l’utente inserisce il proprio identificativo per rimuoversi dal sistema.
	\item \textbf{Precondizione}: un utente già presente nel sistema deve essere rimosso.
	\item \textbf{Postcondizione}: l'utente non risulta più acceduto al sistema che non riconosce più l'utente ormai rimosso.
	\item \textbf{Scenario principale}:
	\begin{enumerate}
		\item L'utente selezionato attraverso il contatto Telegram o Email viene rimosso dal sistema.
		\item L'utente viene automaticamente fatto uscire dall'applicazione.
		%TODO: aggiungere un caso include perchè se l'utente è rimosso allora => l'utente viene fatto uscire dal sistema?
	\end{enumerate}
\end{itemize}

\stepcounter{uccount}
%\clearpage
\subsubsection{UC\theuccount-GP - Rimozione utente}
		\begin{figure}[H]
			\centering
				\includegraphics[width=\columnwidth]{img/casi_d'uso/UC15.png}\\
			\caption{UC\theuccount-GP - Rimozione utente}
		\end{figure}
	\begin{itemize}
		\item \textbf{Codice}: UC\theuccount-GP.
		\item \textbf{Titolo}: rimozione utente.
		\item \textbf{Attori primari}: utente.
		\item \textbf{Descrizione}: l'utente rimuove l'utente selezionato dal sistema.
		\item \textbf{Precondizione}: un utente già presente nel sistema deve essere rimosso.
		\item \textbf{Postcondizione}: un utente viene rimosso dal sistema.
		\item \textbf{Scenario principale}:
		\begin{enumerate}
			\item L'utente rimuove l'utente selezionato
		\end{enumerate}
\end{itemize}
	
	\stepcounter{subuccount}
	\subsubsection{UC\theuccount.\thesubuccount-GP - Rimozione utente dal sistema}
		\begin{figure}[H]
			\centering
			\includegraphics[width=0.6\columnwidth]{img/casi_d'uso/UC15_1.png}\\
			\caption{UC\theuccount.\thesubuccount-GP - Rimozione utente dal sistema}
		\end{figure}
		\begin{itemize}
			\item \textbf{Codice}: UC\theuccount.\thesubuccount-GP.
			\item \textbf{Titolo}: rimozione utente dal sistema.
			\item \textbf{Attori primari}: utente.
			\item \textbf{Descrizione}: un utente, acceduto al sistema, rimuove un utente presente nel sistema tramite l'inserimento del suo contatto Email o Telegram. Questo utente può essere anche se stesso.
			%il contatto Email o Telegram dell'utente da rimuovere è presente nel sistema, per cui la rimozione avviene con successo.
			\item \textbf{Precondizione}: un utente già presente nel sistema deve essere rimosso.
			\item \textbf{Postcondizione}: un utente con il contatto Email o Telegram inserito viene rimosso dal sistema.
			\item \textbf{Scenario principale}:
			\begin{enumerate}
				\item L'utente inserisce ciò che è richiesto dal sistema
				\item L'utente conferma l'invio dei dati
				\item L'utente da rimuovere è stato rimosso dal sistema
			\end{enumerate}
			\item \textbf{Estensioni}:
			\begin{itemize}
				\item Errore contatto non presente nel sistema [UC\theuccount.2-GP]
			\end{itemize}
		\end{itemize}
			
			\stepcounter{subsubuccount}
			\subsubsection{UC\theuccount.\thesubuccount.\thesubsubuccount-GP - Inserimento contatto Email}
				
				\begin{itemize}
					\item \textbf{Codice}: UC\theuccount.\thesubuccount.\thesubsubuccount-GP.
					\item \textbf{Titolo}: inserimento contatto Email.
					\item \textbf{Attori primari}: utente.
					\item \textbf{Descrizione}: l'utente ha aggiunto il contatto Email relativo all'utente che vuole rimuovere. L'inserimento di questo capo non avviene se viene invece inserito il contatto Telegram.
					\item \textbf{Precondizione}: un utente già presente nel sistema deve essere rimosso.
					\item \textbf{Postcondizione}: il contatto dell'utente da rimuovere Email è stato inserito.
					\item \textbf{Scenario principale}:
					\begin{enumerate}
						\item L'utente inserisce il contatto Email dell'utente da rimuovere
					\end{enumerate}
				\end{itemize}
			
			\stepcounter{subsubuccount}
			\subsubsection{UC\theuccount.\thesubuccount.\thesubsubuccount-GP - Inserimento contatto Telegram}
				
				\begin{itemize}
					\item \textbf{Codice}: UC\theuccount.\thesubuccount.\thesubsubuccount-GP.
					\item \textbf{Titolo}: inserimento contatto Telegram.
					\item \textbf{Attori primari}: utente.
					\item \textbf{Descrizione}: l'utente ha aggiunto il contatto Telegram relativo all'utente che vuole \newline rimuovere. L'inserimento di questo capo non avviene se viene invece inserito il contatto Email.
					\item \textbf{Precondizione}: un utente già presente nel sistema deve essere rimosso.
					\item \textbf{Postcondizione}: il contatto Telegram dell'utente da rimuovere è stato inserito.
					\item \textbf{Scenario principale}:
					\begin{enumerate}
						\item L'utente inserisce il contatto Telegram del utente da rimuovere.
					\end{enumerate}
				\end{itemize}
			
			\stepcounter{subuccount}
			\subsubsection{UC\theuccount.\thesubuccount-GP - Errore contatto non presente nel sistema}
					
				\begin{itemize}
					\item \textbf{Codice}: UC\theuccount.\thesubuccount-GP.
					\item \textbf{Titolo}: errore contatto non presente nel sistema.
					\item \textbf{Attori primari}: utente.
					\item \textbf{Descrizione}: l’utente viene avvisato che il contatto inserito non è presente nel sistema.
					\item \textbf{Precondizione}: un utente già presente nel sistema deve essere rimosso.
					\item \textbf{Postcondizione}: il sistema comunica all’utente che utilizza il sistema l’errore e nessun utente viene rimosso.
					\item \textbf{Scenario principale}:
					\begin{enumerate}
						\item L'utente selezionato attraverso il contatto Telegram o Email non viene rimosso perché non presente nel sistema.
					\end{enumerate}
				\end{itemize}

\stepcounter{uccount}
%\clearpage
\subsubsection{UC\theuccount-GP - Modifica utente}
		\begin{figure}[H]
			\centering
				\includegraphics[width=0.9\textwidth]{img/casi_d'uso/UC\theuccount.png}\\
			\caption{UC\theuccount-GP - Modifica utente}
		\end{figure}
	\begin{itemize}
		\item \textbf{Codice}: UC\theuccount-GP.
		\item \textbf{Titolo}: modifica utente.
		\item \textbf{Attori primari}: utente.
		\item \textbf{Descrizione}: l’utente vuole modificare le proprie informazioni.
		\item \textbf{Precondizione}: l'utente vuole modificare i propri dati.
		\item \textbf{Postcondizione}: i campi sono stati modificati correttamente.
		\item \textbf{Scenario principale}:
		\begin{enumerate}
			\item L'utente modifica i propri dati
		\end{enumerate}
	\end{itemize}


	\stepcounter{subuccount}

		\subsubsection{UC\theuccount.\thesubuccount-GP - Modifica utente del sistema}
			\begin{figure}[H]
				\centering
				\includegraphics[width=0.6\columnwidth]{img/casi_d'uso/UC\theuccount_\thesubuccount.png}\\
				\caption{UC\theuccount.\thesubuccount-GP - Modifica di un utente del sistema}
			\end{figure}
			\begin{itemize}
				\item \textbf{Codice}: UC\theuccount.\thesubuccount-GP.
				\item \textbf{Titolo}: modifica di un utente del sistema.
				\item \textbf{Attori primari}: utente.
				\item \textbf{Descrizione}: l'utente modifica i propri campi.
				\item \textbf{Precondizione}: l'utente vuole modificare i propri dati.
				\item \textbf{Postcondizione}: l'utente è stato modificato.
				\item \textbf{Scenario principale}:
				\begin{enumerate}
					\item L'utente viene modificato
				\end{enumerate}
				\item \textbf{Estensioni}:
				\begin{itemize}
					\item Errore, nuovi dati dell'utente già esistenti [UC\theuccount.2-GP]
				\end{itemize}
			\end{itemize}

			\stepcounter{subsubuccount}
			\subsubsection{UC\theuccount.\thesubuccount.\thesubsubuccount-GP - Inserimento del nuovo nome}

				\begin{itemize}
					\item \textbf{Codice}: UC\theuccount.\thesubuccount.\thesubsubuccount-GP.
					\item \textbf{Titolo}: inserimento del nuovo nome.
					\item \textbf{Attori primari}: utente.
					\item \textbf{Descrizione}: l'utente aggiunge il nuovo nome.
					\item \textbf{Precondizione}: l'utente vuole modificare i propri dati.
					\item \textbf{Postcondizione}: il nome è stato inserito.
					\item \textbf{Scenario principale}:
					\begin{enumerate}
						\item L'utente inserisce il nuovo nome
					\end{enumerate}
				\end{itemize}

			\stepcounter{subsubuccount}
			\subsubsection{UC\theuccount.\thesubuccount.\thesubsubuccount-GP - Inserimento del nuovo cognome}

				\begin{itemize}
					\item \textbf{Codice}: UC\theuccount.\thesubuccount.\thesubsubuccount-GP.
					\item \textbf{Titolo}: inserimento del nuovo cognome.
					\item \textbf{Attori primari}: utente.
					\item \textbf{Descrizione}: l'utente aggiunge il nuovo cognome
					\item \textbf{Precondizione}: l'utente vuole modificare i propri dati.
					\item \textbf{Postcondizione}: il cognome è stato inserito.
					\item \textbf{Scenario principale}:
					\begin{enumerate}
						\item L'utente inserisce il nuovo cognome
					\end{enumerate}
				\end{itemize}

			\stepcounter{subsubuccount}
			\subsubsection{UC\theuccount.\thesubuccount.\thesubsubuccount-GP - Inserimento del nuovo contatto Email}

				\begin{itemize}
					\item \textbf{Codice}: UC\theuccount.\thesubuccount.\thesubsubuccount-GP.
					\item \textbf{Titolo}: inserimento del nuovo contatto Email.
					\item \textbf{Attori primari}: utente.
					\item \textbf{Descrizione}: l'utente aggiunge il nuovo contatto Email.
					\item \textbf{Precondizione}: l'utente vuole modificare i propri dati.
					\item \textbf{Postcondizione}: il contatto Email è stato inserito.
					\item \textbf{Scenario principale}:
					\begin{enumerate}
						\item L'utente inserisce il nuovo contatto Email
					\end{enumerate}
				\end{itemize}

			\stepcounter{subsubuccount}
			\subsubsection{UC\theuccount.\thesubuccount.\thesubsubuccount-GP - Inserimento del nuovo contatto Telegram}

				\begin{itemize}
					\item \textbf{Codice}: UC\theuccount.\thesubuccount.\thesubsubuccount-GP.
					\item \textbf{Titolo}: inserimento del nuovo contatto Telegram.
					\item \textbf{Attori primari}: utente.
					\item \textbf{Descrizione}: l'utente aggiunge il nuovo contatto Telegram.
					\item \textbf{Precondizione}: l'utente vuole modificare i propri dati.
					\item \textbf{Postcondizione}: il contatto Telegram è stato inserito.
					\item \textbf{Scenario principale}:
					\begin{enumerate}
						\item L'utente inserisce il nuovo contatto Telegram
					\end{enumerate}
				\end{itemize}

		\stepcounter{subuccount}
		\subsubsection{UC\theuccount.\thesubuccount-GP - Errore, nuovi dati dell'utente già esistenti}

		\begin{itemize}
			\item \textbf{Codice}: UC\theuccount.\thesubuccount-GP.
			\item \textbf{Titolo}: errore, nuovi dati dell'utente già esistenti.
			\item \textbf{Attori primari}: utente.
			\item \textbf{Descrizione}: i nuovi dati dell'utente da modificare che sono stati inseriti sono già
			presenti nel sistema, ovvero i nuovi campi corrispondono a quelli di un utente già esistente.
			In particolare il contatto Telegram o Email, perchè una persona può avere lo stesso nome e cognome di un
			altro, ma non la stessa mail e nemmeno lo stesso identificativo Telegram.
			\item \textbf{Precondizione}: l'utente vuole modificare un utente esistente nel sistema.
			\item \textbf{Postcondizione}: l'utente non è stato modificato e viene visualizzato un messaggio di errore.
			\item \textbf{Scenario principale}:
			\begin{enumerate}
				\item L'utente inserisce i campi richiesti dal sistema per la modifica di un utente
				\item L'utente visualizza un messaggio di errore
			\end{enumerate}
		\end{itemize}

%	\stepcounter{subuccount}
%	\subsubsection{UC\theuccount.\thesubuccount-GP - Errore identificativo inesistente}
%
%		\begin{itemize}
%			\item \textbf{Codice}: UC\theuccount.\thesubuccount-GP.
%			\item \textbf{Titolo}: errore identificativo inesistente.
%			\item \textbf{Attori primari}: utente.
%			\item \textbf{Descrizione}:  l’utente inserisce l'identificativo dell'utente di cui vuole modificare le informazioni, ma viene avvisato che ha inserito un'identificativo errato perchè esso non è presente all'interno del sistema.
%			\item \textbf{Precondizione}: l'utente vuole modificare un utente già presente.
%			\item \textbf{Postcondizione}: il sistema comunica all’utilizzatore l’errore.
%			\item \textbf{Scenario principale}:
%			\begin{enumerate}
%				\item L'utente inserisce un identificativo errato
%				\item Il sistema comunica all’utilizzatore l’errore
%			\end{enumerate}
%		\end{itemize}


\stepcounter{uccount}
%\clearpage
\subsubsection{UC\theuccount-GP - Rimozione utente dal sistema}
		\begin{figure}[H]
			\centering
				\includegraphics[width=\columnwidth]{img/casi_d'uso/UC15.png}\\
			\caption{UC\theuccount-GP - Rimozione utente dal sistema}
		\end{figure}
	\begin{itemize}
		\item \textbf{Codice}: UC\theuccount-GP.
		\item \textbf{Titolo}: rimozione utente dal sistema.
		\item \textbf{Attori primari}: utente.
		\item \textbf{Descrizione}: l'utente rimuove l'utente selezionato dal sistema.
		\item \textbf{Precondizione}: un utente già presente nel sistema deve essere rimosso.
		\item \textbf{Postcondizione}: un utente viene rimosso dal sistema.
		\item \textbf{Scenario principale}:
		\begin{enumerate}
			\item L'utente rimuove l'utente selezionato
		\end{enumerate}
\end{itemize}
	
	\stepcounter{subuccount}
	\subsubsection{UC\theuccount.\thesubuccount-GP - Rimozione avvenuta}
		\begin{figure}[H]
			\centering
			\includegraphics[width=0.6\columnwidth]{img/casi_d'uso/UC15_1.png}\\
			\caption{UC\theuccount.\thesubuccount-GP - Rimozione avvenuta}
		\end{figure}
		\begin{itemize}
			\item \textbf{Codice}: UC\theuccount.\thesubuccount-GP.
			\item \textbf{Titolo}: rimozione avvenuta.
			\item \textbf{Attori primari}: utente.
			\item \textbf{Descrizione}: un utente, acceduto al sistema, rimuove un altro utente presente nel sistema tramite l'inserimento del suo contatto Email o Telegram.
			%il contatto Email o Telegram dell'utente da rimuovere è presente nel sistema, per cui la rimozione avviene con successo.
			\item \textbf{Precondizione}: un utente già presente nel sistema deve essere rimosso.
			\item \textbf{Postcondizione}: un utente con il contatto Email o Telegram inserito viene rimosso dal sistema.
			\item \textbf{Scenario principale}:
			\begin{enumerate}
				\item L'utente inserisce ciò che è richiesto dal sistema
				\item L'utente conferma l'invio dei dati
				\item L'utente da rimuovere è stato rimosso dal sistema
			\end{enumerate}
			\item \textbf{Estensioni}:
			\begin{itemize}
				\item Errore contatto non presente nel sistema [UC\theuccount.2-GP]
				\item L'utente vuole rimuovere se stesso [UC\theuccount.3-GP]
			\end{itemize}
		\end{itemize}
			
			\stepcounter{subsubuccount}
			\subsubsection{UC\theuccount.\thesubuccount.\thesubsubuccount-GP - Inserimento contatto Email}
				
				\begin{itemize}
					\item \textbf{Codice}: UC\theuccount.\thesubuccount.\thesubsubuccount-GP.
					\item \textbf{Titolo}: inserimento contatto Email.
					\item \textbf{Attori primari}: utente.
					\item \textbf{Descrizione}: l'utente ha aggiunto il contatto Email relativo all'utente che vuole rimuovere.
					\item \textbf{Precondizione}: un utente già presente nel sistema deve essere rimosso.
					\item \textbf{Postcondizione}: il contatto dell'utente da rimuovere Email è stato inserito.
					\item \textbf{Scenario principale}:
					\begin{enumerate}
						\item L'utente inserisce il contatto Email dell'utente da rimuovere
					\end{enumerate}
				\end{itemize}
			
			\stepcounter{subsubuccount}
			\subsubsection{UC\theuccount.\thesubuccount.\thesubsubuccount-GP - Inserimento contatto Telegram}
				
				\begin{itemize}
					\item \textbf{Codice}: UC\theuccount.\thesubuccount.\thesubsubuccount-GP.
					\item \textbf{Titolo}: inserimento contatto Telegram.
					\item \textbf{Attori primari}: utente.
					\item \textbf{Descrizione}: l'utente ha aggiunto il contatto Telegram relativo all'utente che vuole \newline rimuovere.
					\item \textbf{Precondizione}: un utente già presente nel sistema deve essere rimosso.
					\item \textbf{Postcondizione}: il contatto Telegram dell'utente da rimuovere è stato inserito.
					\item \textbf{Scenario principale}:
					\begin{enumerate}
						\item L'utente inserisce il contatto Telegram del utente da rimuovere.
					\end{enumerate}
				\end{itemize}
			
			\stepcounter{subuccount}
			\subsubsection{UC\theuccount.\thesubuccount-GP - Errore contatto non presente nel sistema}
					
				\begin{itemize}
					\item \textbf{Codice}: UC\theuccount.\thesubuccount-GP.
					\item \textbf{Titolo}: errore contatto non presente nel sistema.
					\item \textbf{Attori primari}: utente.
					\item \textbf{Descrizione}: l’utente viene avvisato che il contatto inserito non è presente nel sistema.
					\item \textbf{Precondizione}: un utente già presente nel sistema deve essere rimosso.
					\item \textbf{Postcondizione}: il sistema comunica all’utente che utilizza il sistema l’errore e nessun utente viene rimosso.
					\item \textbf{Scenario principale}:
					\begin{enumerate}
						\item L'utente selezionato attraverso il contatto Telegram o Email non viene rimosso perché non presente nel sistema.
					\end{enumerate}
				\end{itemize}
			
			\stepcounter{subuccount}
			\subsubsection{UC\theuccount.\thesubuccount-GP - L'utente rimuove se stesso dal sistema}
			
			\begin{itemize}
				\item \textbf{Codice}: UC\theuccount.\thesubuccount-GP.
				\item \textbf{Titolo}: l'utente rimuove se stesso dal sistema.
				\item \textbf{Attori primari}: utente.
				\item \textbf{Descrizione}:  l’utente inserisce il proprio identificativo per rimuoversi dal sistema.
				\item \textbf{Precondizione}: un utente già presente nel sistema deve essere rimosso.
				\item \textbf{Postcondizione}: l'utente non risulta più acceduto al sistema che non riconosce più l'utente ormai rimosso.
				\item \textbf{Scenario principale}:
				\begin{enumerate}
					\item L'utente selezionato attraverso il contatto Telegram o Email viene rimosso dal sistema.
					\item L'utente viene automaticamente fatto uscire dall'applicazione.
					%TODO: aggiungere un caso include perchè se l'utente è rimosso allora => l'utente viene fatto uscire dal sistema?
				\end{enumerate}
			\end{itemize}

\stepcounter{uccount}
%\clearpage
\subsubsection{UC\theuccount-GP - Modifica utente}
		\begin{figure}[H]
			\centering
				\includegraphics[width=0.8\textwidth]{img/casi_d'uso/UC17.png}\\
			\caption{UC\theuccount-GP - Modifica utente}
		\end{figure}
	\begin{itemize}
		\item \textbf{Codice}: UC\theuccount-GP.
		\item \textbf{Titolo}: modifica utente.
		\item \textbf{Attori primari}: utente.
		\item \textbf{Descrizione}: l’utente vuole modificare le informazioni relative a un altro utente, o di se stesso.
		\item \textbf{Precondizione}: l'utente vuole modificare i dati di un utente già presente nel sistema.
		\item \textbf{Postcondizione}: i campi dell'utente sono stati modificati correttamente.
		\item \textbf{Scenario Principale}:
		\begin{enumerate}
			\item L'utente modifica i dati relativi di un utente
		\end{enumerate}
	\end{itemize}
	
	\stepcounter{subuccount}
	\paragraph{UC\theuccount.\thesubuccount-GP - Selezione ID utente}
		\begin{figure}[H]
			\centering
			\includegraphics[width=0.8\textwidth]{img/casi_d'uso/UC17_1.png}\\
			\caption{UC\theuccount.\thesubuccount-GP - Selezione ID utente}
		\end{figure}
		\begin{itemize}
			\item \textbf{Codice}: UC\theuccount.\thesubuccount-GP.
			\item \textbf{Titolo}: selezione ID utente.
			\item \textbf{Attori primari}: utente.
			\item \textbf{Descrizione}: l'utente aggiunge l'ID dell'utente che vuole modificare.
			\item \textbf{Precondizione}: l'utente vuole modificare un utente già presente.
			\item \textbf{Postcondizione}: l'ID utente è stato inserito.
			\item \textbf{Scenario Principale}:
			\begin{enumerate}
				\item L'utente procede con l'inserimento dell'ID dell'utente da modificare
			\end{enumerate}
			\item \textbf{Estensioni}:
			\begin{itemize}
				\item Errore ID utente inesistente [UC\theuccount.2-GP]
			\end{itemize}
		\end{itemize}
		
		\stepcounter{subsubuccount}
		\subparagraph{UC\theuccount.\thesubuccount.\thesubsubuccount-GP - Modifica utente avvenuta con successo}
			\begin{figure}[H]
				\centering
				\includegraphics[width=0.5\columnwidth]{img/casi_d'uso/UC17_1_1.png}\\
				\caption{UC\theuccount.\thesubuccount.\thesubsubuccount-GP - Modifica utente avvenuta con successo}
			\end{figure}
			\begin{itemize}
				\item \textbf{Codice}: UC\theuccount.\thesubuccount.\thesubsubuccount-GP.
				\item \textbf{Titolo}: modifica utente avvenuta con successo.
				\item \textbf{Attori primari}: utente.
				\item \textbf{Descrizione}: l'ID utente è presente nel sistema e ne vengono modificati i relativi campi con successo.
				\item \textbf{Precondizione}: l'utente vuole modificare un utente già presente.
				\item \textbf{Postcondizione}: l'utente è stato modificato con successo.
				\item \textbf{Scenario Principale}:
				\begin{enumerate}
					\item L'utente viene modificato con successo
				\end{enumerate}
				\item \textbf{Estensioni}:
				\begin{itemize}
					\item Errore, nuovi dati dell'utente già esistenti [UC\theuccount.\thesubuccount.2-GP]
				\end{itemize}
			\end{itemize}
			
			\stepcounter{subsubsubuccount}
			\subsubparagraph{UC\theuccount.\thesubuccount.\thesubsubuccount.\thesubsubsubuccount-GP - Inserimento del nuovo nome}
				
				\begin{itemize}
					\item \textbf{Codice}: UC\theuccount.\thesubuccount.\thesubsubuccount.\thesubsubsubuccount-GP.
					\item \textbf{Titolo}: inserimento del nuovo nome.
					\item \textbf{Attori primari}: utente.
					\item \textbf{Descrizione}: l'utente aggiunge il nuovo nome relativo all'ID utente inserito che vuole modificare.
					\item \textbf{Precondizione}: l'utente vuole modificare un utente già presente.
					\item \textbf{Postcondizione}: il nome è stato inserito.
					\item \textbf{Scenario Principale}:
					\begin{enumerate}
						\item L'utente inserisce il nuovo nome dell'utente che vuole modificare
					\end{enumerate}
				\end{itemize}
			
			\stepcounter{subsubsubuccount}
			\subsubparagraph{UC\theuccount.\thesubuccount.\thesubsubuccount.\thesubsubsubuccount-GP - Inserimento del nuovo cognome}
				
				\begin{itemize}
					\item \textbf{Codice}: UC\theuccount.\thesubuccount.\thesubsubuccount.\thesubsubsubuccount-GP.
					\item \textbf{Titolo}: inserimento del nuovo cognome.
					\item \textbf{Attori primari}: utente.
					\item \textbf{Descrizione}: l'utente aggiunge il nuovo cognome relativo all'ID utente inserito che vuole modificare.
					\item \textbf{Precondizione}: l'utente vuole modificare un utente già presente.
					\item \textbf{Postcondizione}: il cognome è stato inserito.
					\item \textbf{Scenario Principale}:
					\begin{enumerate}
						\item L'utente inserisce il nuovo cognome dell'utente che vuole modificare.
					\end{enumerate}
				\end{itemize}
			
			\stepcounter{subsubsubuccount}
			\subsubparagraph{UC\theuccount.\thesubuccount.\thesubsubuccount.\thesubsubsubuccount-GP - Inserimento del nuovo contatto Email}
				
				\begin{itemize}
					\item \textbf{Codice}: UC\theuccount.\thesubuccount.\thesubsubuccount.\thesubsubsubuccount-GP.
					\item \textbf{Titolo}: inserimento del nuovo contatto Email.
					\item \textbf{Attori primari}: utente.
					\item \textbf{Descrizione}: l'utente aggiunge il nuovo contatto Email relativo all'ID utente inserito che vuole modificare.
					\item \textbf{Precondizione}: l'utente vuole modificare un utente già presente.
					\item \textbf{Postcondizione}: il contatto Email è stato inserito.
					\item \textbf{Scenario Principale}:
					\begin{enumerate}
						\item L'utente inserisce il nuovo contatto Email dell'utente che vuole modificare
					\end{enumerate}
				\end{itemize}
			
			\stepcounter{subsubsubuccount}
			\subsubparagraph{UC\theuccount.\thesubuccount.\thesubsubuccount.\thesubsubsubuccount-GP - Inserimento del nuovo contatto Telegram}
				
				\begin{itemize}
					\item \textbf{Codice}: UC\theuccount.\thesubuccount.\thesubsubuccount.\thesubsubsubuccount-GP.
					\item \textbf{Titolo}: inserimento del nuovo contatto Telegram.
					\item \textbf{Attori primari}: utente.
					\item \textbf{Descrizione}: l'utente aggiunge il nuovo contatto Telegram relativo all'ID utente inserito che vuole modificare.
					\item \textbf{Precondizione}: l'utente vuole modificare un utente già presente.
					\item \textbf{Postcondizione}: il contatto Telegram è stato inserito.
					\item \textbf{Scenario Principale}:
					\begin{enumerate}
						\item L'utente inserisce il nuovo contatto Telegram dell'utente che vuole modificare
					\end{enumerate}
				\end{itemize}
			
		\stepcounter{subsubuccount}
		\subparagraph{UC\theuccount.\thesubuccount.\thesubsubuccount-GP - Errore, nuovi dati dell'utente già esistenti}
		
		\begin{itemize}
			\item \textbf{Codice}: UC\theuccount.\thesubuccount.\thesubsubuccount-GP.
			\item \textbf{Titolo}: errore, nuovi dati dell'utente già esistenti.
			\item \textbf{Attori primari}: utente.
			\item \textbf{Descrizione}: i nuovi dati dell'utente da modificare che sono stati inseriti sono già presenti nel sistema.
			\item \textbf{Precondizione}: l'utente vuole modificare un utente già presente.
			\item \textbf{Postcondizione}: l'utente non è stato modificato.
			\item \textbf{Scenario Principale}:
			\begin{enumerate}
				\item L'utente non viene modificato perché i nuovi campi corrispondono a quelli di un utente già esistente
			\end{enumerate}
		\end{itemize}
	\stepcounter{subuccount}
	\paragraph{UC\theuccount.\thesubuccount-GP - Errore ID utente inesistente}
		
		\begin{itemize}
			\item \textbf{Codice}: UC\theuccount.\thesubuccount-GP.
			\item \textbf{Titolo}: errore ID utente inesistente.
			\item \textbf{Attori primari}: utente.
			\item \textbf{Descrizione}:  l’utente viene avvisato che ha inserito un'ID utente errato.
			\item \textbf{Precondizione}: l'utente vuole modificare un utente già presente.
			\item \textbf{Postcondizione}: il sistema comunica all’utilizzatore l’errore.
			\item \textbf{Scenario Principale}:
			\begin{enumerate}
				\item L'utente ha inserito un ID utente errato e il sistema comunica all’utilizzatore l’errore.
			\end{enumerate}
		\end{itemize}

\stepcounter{uccount}
%\clearpage
\subsubsection{UC\theuccount-GP - Aggiunta preferenze}
		\begin{figure}[H]
			\centering
				\includegraphics[width=1\textwidth]{img/casi_d'uso/UC18.png}\\
			\caption{UC\theuccount-GP - Aggiunta preferenze}
		\end{figure}
	\begin{itemize}
		\item \textbf{Codice}: UC\theuccount-GP.
		\item \textbf{Titolo}: aggiunta preferenze.
		\item \textbf{Attori primari}: utente.
		\item \textbf{Descrizione}: l’utente, date le varie opzioni per configurare Butterfly, aggiunge una
		preferenza tra Topic, giorni di calendario, piattaforma di messaggistica (Telegram o e-mail)	preferita e la persona di fiducia che lo può sostituire.
		\item \textbf{Precondizione}: l’utente ha acceduto con le sue credenziali corrette nel sistema e non ha già selezionato tutte le preferenze possibili proposte da \progetto.
		\item \textbf{Postcondizione}: la nuova configurazione contiene una o più preferenze in aggiunta rispetto a quella precedente.
		\item \textbf{Scenario Principale}:
		\begin{enumerate}
			\item L'utente aggiunge di una o più preferenze
		\end{enumerate}
	\end{itemize}

	\stepcounter{subuccount}
	\paragraph{UC\theuccount.\thesubuccount-GP - Iscrizione Topic}

		\begin{itemize}
			\item \textbf{Codice}: UC\theuccount.\thesubuccount-GP.
			\item \textbf{Titolo}: iscrizione Topic.
			\item \textbf{Attori primari}: utente.
			\item \textbf{Descrizione}: data la lista di Topic presenti, l’utente ne seleziona uno o	più a cui è interessato, ricevendone una notifica. I Topic sono divisi per categoria e	comprendono etichette, progetto a cui sono legate e l'applicazione di provenienza: Redmine o GitLab.
			\item \textbf{Precondizione}: l’utente ha acceduto correttamente nel sistema e non ha già selezionato tutti i Topic possibili proposti da \progetto.
			\item \textbf{Postcondizione}: il numero di Topic a cui è interessato l’utente è aumentato.
			\item \textbf{Scenario Principale}:
			\begin{enumerate}
				\item L'utente procede all'iscrizione di uno o più Topic
			\end{enumerate}
		\end{itemize}
	
	\stepcounter{subuccount}
	\paragraph{UC\theuccount.\thesubuccount-GP - Aggiunta dei giorni di indisponibilità nel calendario}
		
		\begin{itemize}
			\item \textbf{Codice}: UC\theuccount.\thesubuccount-GP.
			\item \textbf{Titolo}: aggiunta dei giorni di indisponibilità nel calendario.
			\item \textbf{Attori primari}: utente.
			\item \textbf{Descrizione}: dato il calendario lavorativo, l’utente aggiunge uno o più giorni in cui non è reperibile e non vuole ricevere notifiche.
			\item \textbf{Precondizione}: l’utente ha acceduto correttamente nel sistema e non ha già selezionato tutti i giorni di indisponibilità.
			\item \textbf{Postcondizione}: il numero di giorni in cui l’utente non si rende disponibile è aumentato.
			\item \textbf{Scenario Principale}:
			\begin{enumerate}
				\item L'utente procede all'inserimento di uno o più giorni di indisponibilità
			\end{enumerate}
		\end{itemize}
	
	\stepcounter{subuccount}
	\paragraph{UC\theuccount.\thesubuccount-GP - Aggiunta della piattaforma di messaggistica preferita}
		
		\begin{itemize}
			\item \textbf{Codice}: UC\theuccount.\thesubuccount-GP.
			\item \textbf{Titolo}: aggiunta della piattaforma di messaggistica preferita.
			\item \textbf{Attori primari}: utente.
			\item \textbf{Descrizione}: l’utente aggiunge la sua preferenza tra Telegram e Email dove vuole ricevere le notifiche.
			\item \textbf{Precondizione}: l’utente ha acceduto correttamente nel sistema e non ha già selezionato tutte le piattaforme di messaggistica possibili proposte da \progetto.
			\item \textbf{Postcondizione}: il numero di piattaforme di messaggistica selezionate dall’utente è aumentato.
			\item \textbf{Scenario Principale}:
			\begin{enumerate}
				\item L'utente procede all'aggiunta di una o più piattaforme di messaggistica
			\end{enumerate}
		\end{itemize}
	
	\stepcounter{subuccount}
	\paragraph{UC\theuccount.\thesubuccount-GP - Aggiunta persona di fiducia}
		
		\begin{itemize}
			\item \textbf{Codice}: UC\theuccount.\thesubuccount-GP.
			\item \textbf{Titolo}: aggiunta persona di fiducia.
			\item \textbf{Attori primari}: utente.
			\item \textbf{Descrizione}: l’utente aggiunge l'utente legato a un ID di sua preferenza a cui inoltrare i messaggi in caso di indisponibilità.
			\item \textbf{Precondizione}: l’utente ha acceduto con le sue credenziali corrette nel sistema e non ha già selezionato la persona a cui inoltrare le notifiche.
			\item \textbf{Postcondizione}: la preferenza viene aggiunta correttamente.
			\item \textbf{Scenario Principale}:
			\begin{enumerate}
				\item L'utente procede all'aggiunta della sua persona di fiducia
			\end{enumerate}
			\item \textbf{Estensioni}:
			\begin{enumerate}
				\item Errore ID persona di fiducia inesistente [UC\theuccount.5-GP]
			\end{enumerate}
		\end{itemize}
	
	\stepcounter{subuccount}
	\paragraph{UC\theuccount.\thesubuccount-GP - Errore ID persona di fiducia inesistente}
		
		\begin{itemize}
			\item \textbf{Codice}: UC\theuccount.\thesubuccount-GP.
			\item \textbf{Titolo}: errore ID persona di fiducia inesistente.
			\item \textbf{Attori primari}: utente.
			\item \textbf{Descrizione}: l’utente viene avvisato che ha inserito un ID utente errato.
			\item \textbf{Precondizione}: l’utente ha acceduto con le sue credenziali corrette nel sistema e non ha già selezionato la persona a cui inoltrare le notifiche.
			\item \textbf{Postcondizione}: il sistema comunica all’utilizzatore l’errore di preferenza.
			\item \textbf{Scenario Principale}:
			\begin{enumerate}
				\item L'utente procede all'aggiunta della sua persona di fiducia ma questa non esiste e visualizza l'errore
			\end{enumerate}
		\end{itemize}
	
	\stepcounter{subuccount}
	\paragraph{UC\theuccount.\thesubuccount-GP - Aggiunta keyword per i push di GitLab}
		
		\begin{itemize}
			\item \textbf{Codice}: UC\theuccount.\thesubuccount-GP.
			\item \textbf{Titolo}: aggiunta keyword per i push di GitLab.
			\item \textbf{Attori primari}: utente.
			\item \textbf{Descrizione}: l’utente aggiunge le keyword che vuole che siano contenute nei messaggi di commit dei push di cui vuole ricevere la notifica.
			\item \textbf{Precondizione}: l’utente ha acceduto con le sue credenziali corrette nel sistema.
			\item \textbf{Postcondizione}: nelle nuove configurazioni dell'utente selezionato sono presenti una o più nuove keyword per ricevere notifiche da push di GitLab.
			\item \textbf{Scenario Principale}:
			\begin{enumerate}
				\item L'utente procede all'aggiunta di una o più nuove keyword
			\end{enumerate}
			\item \textbf{Estensioni}:
			\begin{enumerate}
				\item Errore keyword già esistente [UC\theuccount.7-GP]
			\end{enumerate}
		\end{itemize}
	
	\stepcounter{subuccount}
	\paragraph{UC\theuccount.\thesubuccount-GP - Errore keyword già esistente}
	
	\begin{itemize}
		\item \textbf{Codice}: UC\theuccount.\thesubuccount-GP.
		\item \textbf{Titolo}: errore keyword già esistente.
		\item \textbf{Attori primari}: utente.
		\item \textbf{Descrizione}: la keyword che vuole aggiungere l'utente è già registrata nel sistema.
		\item \textbf{Precondizione}:  l’utente ha acceduto con le sue credenziali corrette nel sistema.
		\item \textbf{Postcondizione}: il sistema comunica all’utilizzatore l’errore della keyword.
		\item \textbf{Scenario Principale}:
		\begin{enumerate}
			\item L'utente procede all'aggiunta della keyword, ma questa è già presente e visualizza l'errore
		\end{enumerate}
	\end{itemize}

\stepcounter{uccount}
%\clearpage
\subsubsection{UC\theuccount-GP - Rimozione preferenze}
		\begin{figure}[H]
			\centering
				\includegraphics[width=\textwidth]{img/casi_d'uso/UC19.png}\\
			\caption{UC\theuccount-GP - Rimozione preferenze}
		\end{figure}
	\begin{itemize}
		\item \textbf{Codice}: UC\theuccount-GP.
		\item \textbf{Titolo}: rimozione preferenze.
		\item \textbf{Attori primari}: utente.
		\item \textbf{Descrizione}: l’utente, dopo aver selezionato delle preferenze dalle opzioni di configurazione, ne rimuove una o più. Le preferenze consistono in Topic, date di calendario, piattaforma di messaggistica (Telegram e email) e persona di fiducia che lo può sostituire.
		\item \textbf{Precondizione}: l’utente ha eseguito l'accesso nel sistema ed è presente almeno	una preferenza selezionata tra quelle proposte da Butterfly.
		\item \textbf{Postcondizione}: la nuova configurazione contiene una o più preferenze in meno rispetto	a quella precedente.
		\item \textbf{Scenario Principale}:
		\begin{enumerate}
			\item L'utente procede alla rimozione di una o più preferenze
		\end{enumerate}
	\end{itemize}

	\stepcounter{subuccount}
	\paragraph{UC\theuccount.\thesubuccount-GP - Disiscrizione Topic}
		
		\begin{itemize}
			\item \textbf{Codice}: UC\theuccount.\thesubuccount-GP.
			\item \textbf{Titolo}: disiscrizione Topic.
			\item \textbf{Attori primari}: utente.
			\item \textbf{Descrizione}: l’utente si disiscrive da uno o più Topic dai quali prima riceveva delle notifiche.
			\item \textbf{Precondizione}: l’utente ha acceduto correttamente nel sistema e non ha già selezionato tutti i Topic possibili proposti da \progetto.
			\item \textbf{Postcondizione}: il numero di Topic a cui è iscritto l’utente è diminuito.
			\item \textbf{Scenario Principale}:
			\begin{enumerate}
				\item L'utente procede alla disiscrizione di uno o più Topic
			\end{enumerate}
		\end{itemize}
	
	\stepcounter{subuccount}
	\paragraph{UC\theuccount.\thesubuccount-GP - Rimozione di uno o più giorni di irreperibilità nel calendario}
	
	\begin{itemize}
		\item \textbf{Codice}: UC\theuccount.\thesubuccount-GP.
		\item \textbf{Titolo}: rimozione di uno o più giorni di irreperibilità nel calendario.
		\item \textbf{Attori primari}: utente.
		\item \textbf{Descrizione}: l’utente rimuove i giorni di calendario in cui precedentemente	non era reperibile, tornando disponibile.
		\item \textbf{Precondizione}: l’utente ha acceduto correttamente nel sistema ed è presente almeno un giorno di calendario selezionato tra quelli proposti da \progetto.
		\item \textbf{Postcondizione}: il numero di giorni di calendario in cui l’utente non è reperibile è diminuito.
		\item \textbf{Scenario Principale}:
		\begin{enumerate}
			\item L'utente procede alla rimozione di uno o più giorni di irreperibilità
		\end{enumerate}
	\end{itemize}
	
	\stepcounter{subuccount}
	\paragraph{UC\theuccount.\thesubuccount-GP - Rimozione preferenza piattaforma di messaggistica}
	
	\begin{itemize}
		\item \textbf{Codice}: UC\theuccount.\thesubuccount-GP.
		\item \textbf{Titolo}: rimozione preferenza piattaforma di messaggistica.
		\item \textbf{Attori primari}: utente.
		\item \textbf{Descrizione}: l’utente rimuove una o più preferenze tra Telegram e Email dalle	quali non vuole più ricevere notifiche tramite \progetto.
		\item \textbf{Precondizione}: l’utente ha acceduto correttamente nel sistema ed è presente almeno una piattaforma di messaggistica selezionata tra quelle proposte da \progetto.
		\item \textbf{Postcondizione}: il numero di piattaforme di messaggistica da cui l’utente vuole ricevere notifiche è diminuito.
		\item \textbf{Scenario Principale}:
		\begin{enumerate}
			\item L'utente procede alla rimozione di una o più piattaforme di messaggistica
		\end{enumerate}
	\end{itemize}
	
	\stepcounter{subuccount}
	\paragraph{UC\theuccount.\thesubuccount-GP - Rimozione persona di fiducia}
	
	\begin{itemize}
		\item \textbf{Codice}: UC\theuccount.\thesubuccount-GP.
		\item \textbf{Titolo}: aggiunta persona di fiducia.
		\item \textbf{Attori primari}: utente.
		\item \textbf{Descrizione}:  l’utente rimuove l'utente legato a un ID di sua preferenza a cui inoltrare i messaggi in caso di indisponibilità.
		\item \textbf{Precondizione}: l’utente ha eseguito l'accesso nel sistema ed è presente almeno uno user con l'ID selezionato tra quelle proposte da \progetto.
		\item \textbf{Postcondizione}: la preferenza viene rimossa correttamente.
		\item \textbf{Scenario Principale}:
		\begin{enumerate}
			\item L'utente procede alla rimozione della sua persona di fiducia
		\end{enumerate}
		\item \textbf{Estensioni}:
		\begin{enumerate}
			\item Errore ID persona di fiducia inesistente [UC\theuccount.5-GP].
		\end{enumerate}
	\end{itemize}
	
	\stepcounter{subuccount}
	\paragraph{UC\theuccount.\thesubuccount-GP - Errore ID persona di fiducia inesistente}
	
	\begin{itemize}
		\item \textbf{Codice}: UC\theuccount.\thesubuccount-GP.
		\item \textbf{Titolo}: errore ID persona di fiducia inesistente.
		\item \textbf{Attori primari}: utente.
		\item \textbf{Descrizione}: l’utente viene avvisato che ha inserito un ID utente errato.
		\item \textbf{Precondizione}: l’utente ha acceduto con le sue credenziali corrette nel sistema e non ha già selezionato la persona a cui inoltrare le notifiche.
		\item \textbf{Postcondizione}: il sistema comunica all’utilizzatore l’errore di preferenza.
		\item \textbf{Scenario Principale}:
		\begin{enumerate}
			\item L'utente procede alla rimozione della sua persona di fiducia, questa però non esiste e
			visualizza l'errore
		\end{enumerate}
	\end{itemize}

	\stepcounter{subuccount}
	\paragraph{UC\theuccount.\thesubuccount-GP - Rimozione con successo di keyword per i push di GitLab}
	
	\begin{itemize}
		\item \textbf{Codice}: UC\theuccount.\thesubuccount-GP.
		\item \textbf{Titolo}: rimozione con successo di keyword per i push di GitLab.
		\item \textbf{Attori primari}: utente.
		\item \textbf{Descrizione}: l’utente seleziona e rimuove una o più keyword già presente nel sistema per non ricevere la notifica di push in
		cui i messaggi di commit contengono la keyword rimossa.
		\item \textbf{Precondizione}:  l’utente ha acceduto con le sue credenziali corrette nel sistema.
		\item \textbf{Postcondizione}: nelle nuove configurazioni dell'utente selezionato sono state rimosse delle keyword precedentemente presenti.
		\item \textbf{Scenario Principale}:
		\begin{enumerate}
			\item L'utente rimuove con successo una o più keyword per cui non vuole iù ricevere messaggi di push
		\end{enumerate}
		\item \textbf{Estensioni}:
		\begin{enumerate}
			\item Errore keyword da rimuovere non presente [UC\theuccount.7-GP]
		\end{enumerate}
	\end{itemize}
	
	\stepcounter{subuccount}
	\paragraph{UC\theuccount.\thesubuccount-GP - Errore keyword da rimuovere non presente}
	
	\begin{itemize}
		\item \textbf{Codice}: UC\theuccount.\thesubuccount-GP.
		\item \textbf{Titolo}: errore keyword da rimuovere non presente.
		\item \textbf{Attori primari}: utente.
		\item \textbf{Descrizione}: la keyword che l'utente intende rimuovere non è registrata nel sistema.
		\item \textbf{Precondizione}: l’utente ha acceduto con le sue credenziali corrette nel sistema.
		\item \textbf{Postcondizione}: viene visualizzato un messaggio d'errore con indicato che la keyword	selezionata che non è presente nel sistema.
		\item \textbf{Scenario Principale}:
		\begin{enumerate}
			\item L'utente procede alla rimozione di una keyword che non è presente nel sistema e visualizza l'errore
		\end{enumerate}
	\end{itemize}
	

\clearpage