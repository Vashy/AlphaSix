\newpage
\section{Introduzione}

	\subsection{Glossario e documenti esterni}
Al fine di rendere il documento più chiaro possibile, i termini che possono assumere un significato ambiguo o i riferimenti a documenti esterni
avranno delle diciture convenzionali:

\begin{itemize}
    \item \textbf{D}: indica che il termine si riferisce al titolo di un particolare documento (ad esempio \Doc{\PdPv});
    \item \textbf{G}: indica che il termine si riferisce ad una voce riportata nel \Doc{\Glv} (ad esempio \gloss{Redmine}).
\end{itemize}


	\subsection{Scopo documento}
	Tale \gloss{documento} ha l'obiettivo di esporre e analizzare i \gloss{requisiti} espliciti e impliciti per la realizzazione del progetto \progetto\ (C1) proposto dall'azienda \II.

	Il documento vuole fungere da base per la fase di progettazione del software in modo che essa sia conforme alle richieste fatte dall'azienda proponente.

	%\subsection{Scopo del prodotto}

%%| Ex Norme di Progetto |%%
% Il prodotto che \gruppo\ si incarica di realizzare è Butterfly: un \gloss{tool} di supporto alle figure di	sviluppo di aziende di software
% (non solamente quella committente). Questo applicativo permette di incanalare le notifiche dei vari strumenti utilizzati nel percorso di
% \gloss{CI/CD} (come \gloss{Redmine}, \gloss{GitLab}, ecc.) di un software e, tramite un \gloss{Broker} (\gloss{Apache Kafka} in questo caso),
% spedirli alla persona interessata tramite canale di comunicazione preferito scelto da quest’ultimo (email, \gloss{Telegram}, \gloss{Slack}, ecc).

% \vspace{1cm}

%%| Ex Analisi dei Requisiti |%%
Lo scopo del \gloss{prodotto} è creare un \gloss{applicativo} per poter gestire i messaggi o le segnalazioni provenienti da diversi prodotti per la realizzazione di software,
come \gloss{Redmine}, \gloss{GitLab} e opzionalmente \gloss{SonarQube}, attraverso un \gloss{Broker} che possa incanalare questi messaggi e distribuirli a strumenti come
\gloss{Telegram}, e-mail e opzionalmente \gloss{Slack}.\par
Il software dovrà inoltre essere in grado di riconoscere il \gloss{Topic} dei messaggi in input per poterli inviare in determinati canali a cui i
destinatari dovranno iscriversi.\par
\`E anche richiesto di creare un canale specifico per gestire le particolari esigenze dell'azienda. Dovrà essere in grado, attraverso la lettura di
particolari	\gloss{metadati}, di reindirizzare i messaggi ricevuti al destinatario più appropriato.

% \vspace{1cm}

%%| Ex Piano di Qualifica |%%
% Il prodotto finale consiste in uno strumento in grado di ricevere messaggi o segnalazioni da vari tipi di servizi per la produzione software chiamati
% \gloss{producer} (e.g. \gloss{GitLab}, \gloss{Redmine} e \gloss{SonarQube}), per poterli poi incanalare verso altri servizi chiamati \gloss{Consumer}
% atti a notificare gli sviluppatori (e.g. \gloss{Slack}, \gloss{Telegram} e Email).\par    
% L'applicazione sarà inoltre capace di organizzare le segnalazioni suddividendole per topic a cui i vari utenti dovranno iscriversi per esserne notificati.
% Nel caso in cui il destinatario dovesse segnalare di non essere disponibile, l'applicativo deve reindirizzare il messaggio verso la persona di competenza
% più prossima. 

% \vspace{1cm}

%%| Ex Piano di Progetto |%%
% Il prodotto che \gruppo\ si incarica di realizzare è Butterfly: un tool di supporto alle figure di sviluppo in aziende che producono software (non
% solamente quella del committente).
% Questo applicativo permette di incanalare le notifiche dei vari strumenti utilizzati nel percorso di \gloss{CI} e \gloss{CD} (come Redmine,
% GitLab, ecc.) di un software e, tramite un \gloss{broker} (\gloss{Apache Kafka} in questo caso), spedirli alla persona interessata tramite
% il canale di comunicazione preferito scelto da quest'ultimo (email, Telegram, Slack, ecc.).

    
    \subsection{Scopo del prodotto}
	Lo scopo del \gloss{prodotto} è creare un \gloss{applicativo} per poter gestire i messaggi o le segnalazioni provenienti da diversi prodotti per la realizzazione di software, come \gloss{Redmine}, \gloss{GitLab} e opzionalmente \gloss{SonarQube}, attraverso un \gloss{Broker} che possa incanalare questi messaggi e distribuirli a strumenti come	\gloss{Telegram}, e-mail e opzionalmente \gloss{Slack}.\par
	Per ciascuna di queste tecnologie viene creato un \gloss{Producer} e un \gloss{Consumer} associato per l'effettiva ricezione e invio delle varie segnalazioni.
	Il software dovrà inoltre essere in grado di riconoscere il \gloss{Topic} dei messaggi in input per poterli inviare in determinati canali a cui i
	destinatari dovranno iscriversi.\par
	Dovrà essere in grado, attraverso la lettura di	particolari	\gloss{metadati}, di reindirizzare i messaggi ricevuti al destinatario più appropriato. 
	\`E anche richiesto di creare un canale specifico per gestire le particolari esigenze dell'azienda. 
	Questo viene chiamato Gestore Personale e permette agli utenti di poter personalizzare su quale piattaforma di messaggistica e da quali Topic specifici ricevere i messaggi.

	\subsection{Riferimenti}

	\subsubsection{Normativi}
	\begin{itemize}
		\item \Doc{\NdPv}
		\item \gloss{Capitolato} d'appalto C1\\
		\url{https://www.math.unipd.it/~tullio/IS-1/2018/Progetto/C1.pdf}
	\end{itemize}

	\subsubsection{Informativi} \label{sec:RiferimentiInformativi}
	\begin{itemize}
		\item Presentazione capitolato C1\\
		\url{https://www.math.unipd.it/~tullio/IS-1/2018/Progetto/C1p.pdf}
		\item \gloss{Slide} del corso di Ingegneria del Software
		\begin{itemize}
			\item Analisi dei requisiti\\
			\url{https://www.math.unipd.it/~tullio/IS-1/2018/Dispense/L08.pdf}
			\item Diagrammi dei \gloss{casi d'uso}\\
			\url{https://www.math.unipd.it/~tullio/IS-1/2018/Dispense/E05b.pdf}
		\end{itemize}
		\item \gloss{Webhook} di Redmine\\
		\url{http://www.redmine.org/projects/redmine/wiki/Hooks}
		\item Webhook di GitLab\\
		\url{https://docs.gitlab.com/ee/user/project/integrations/Webhooks.html}
		\item Bot di Telegram\\
		\url{https://core.telegram.org/bots}
	\end{itemize}