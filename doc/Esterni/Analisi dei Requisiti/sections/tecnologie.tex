%\newpage
\section{Tecnologie e scelte relative al progetto}
	L'obiettivo di questo paragrafo è descrivere in maniera specifica come \progetto\ andrà a interfacciarsi con le tecnologie.\\
	Per ciascuna di queste è previsto un \gloss{microservizio}, il quale avrà come scopo quello di fare da tramite tra lo strumento che genera il messaggio e quello che lo riceve.\\
	Questo pattern si chiama \gloss{Publisher / Subscriber} e utilizza uno strumento intermedio detto Broker per lo smistamento dei messaggi e la gestione dei flussi.

	\begin{figure}[H]
		\centering
		\includegraphics[width=\textwidth]{img/butterfly.png}\\
		\caption{Visione generale del sistema \progetto}
		\label{fig:butterfly}
	\end{figure}
	L'immagine precedente rappresenta una suddivisione del sistema in quattro sezioni principali:
	\begin{itemize}
		\item \progetto\ (arancione)
		\item Producer (azzurro)
		\item Broker (rosso)
		\item Consumer (verde)
	\end{itemize}
	Questa scomposizione permette di analizzare più approfonditamente i vari sottosistemi del prodotto in rapporto alle tecnologie esterne a Butterfly (GitLab, Redmine, Telegram ed e-mail), ai microservizi interni come i Producer / Consumer associati e il Broker.
	
	\subsection{Producer}
	
		Per ciascuno degli strumenti che invia messaggi verso il sistema è necessario creare una componente applicativa di tipo Producer che li riceva e li elabori inoltrandoli successivamente verso il Broker.
		Le tecnologie dalle quali vengono ricevuti i messaggi sono elencate di seguito in ordine di priorità in base a quanto richiesto dal \gloss{committente}.
		\begin{itemize}
			\item Redmine
			\item GitLab
			\item SonarQube
		\end{itemize}
		\gruppo\ ha scelto di sviluppare i Producer per le prime due applicazioni in base alle priorità suggerite da \II\ nel capitolato, lasciando l'applicativo relativo a SonarQube come opzionale.
		
		Ciascun messaggio ricevuto da queste tecnologie verrà analizzato dal Producer associato che ne ricercherà al suo interno keyword che ne indicano il Topic.
		Queste sono preimpostate da una persona incaricata dall'azienda che le inserisce nel database.
		
		\subsubsection{Redmine}
		Ciascuna istanza di Redmine permette l'utilizzo di webhook\footnote{Riferirsi alla voce ``Webhook di Redmine'' alla sezione \S\ref{sec:RiferimentiInformativi}} che inviano segnalazioni al Producer alla modifica del progetto.
		Queste vengono ricevute dal server tramite un microservizio che resta in ascolto, aggiornando in base a quello che riceve i dati presenti sul gestore personale e inoltrando le notifiche ai Consumer interessati.
		In particolare, le modifiche relative al progetto prese in considerazione sono:
		\begin{itemize}
			\item Apertura issue
		\end{itemize} 
		
		\subsubsection{GitLab}
		Ciascuna istanza di GitLab, online o in un server locale interno dell'azienda, mette a disposizione la configurazione di webhook\footnote{Riferirsi alla voce ``Webhook di GitLab'' alla sezione \S\ref{sec:RiferimentiInformativi}} che, alla modifica della \gloss{repository}, manda un messaggio con le informazioni dei cambiamenti, inseriti nella repository, a un microservizio capace di aggiornare i dati presenti nel gestore personale e, come per Redmine, inoltrare le notifiche ai Consumer interessati.
		In particolare, le modifiche relative alla repository prese in considerazione sono:
		\begin{itemize}
			\item Commit
			\item Apertura \gloss{issue}
			\item Chiusura issue
		\end{itemize}
		Le segnalazioni sono generate da commit contenenti \gloss{keyword} preimpostate.
		
%		\subsubsection{SonarQube}
%		Come GitLab, anche SonarQube prevede l'utilizzo di Webhooks\footnote{\url{https://docs.sonarqube.org/latest/project-administration/Webhooks/}} che dopo ciascuna build comunica il risultato e le informazioni relative ad essa ad un microservizio capace di aggiornare i dati presenti nel gestore personale e, anche in questo caso, inoltrare le notifiche ai customer interessati.
		
	\subsection{Broker}
	
		Il ruolo del Broker è quello smistare i messaggi in base ai Topic con cui questi sono contrassegnati verso i vari microservizi con cui l'utente ha deciso che gli venga inoltrata la notifica.
		L'azienda consiglia di utilizzare come Broker per i messaggi \gloss{Apache Kafka}.
	
		\subsubsection{Apache Kafka}
		\gloss{Apache Kafka} è un software \gloss{open source} che permette la lettura e la scrittura di messaggi su stream di dati.
		Questi messaggi arrivano dai microservizi Producer che ricevono le notifiche di applicazioni di terze parti e mandandole verso il Broker. Questo le elabora analizzandone il contenuto e contrassegnandole con Topic che verranno utilizzati per l'inoltro ai Consumer e successivamente agli utenti finali, i quali possono abbonarsi a più Topic e gestendo le proprie preferenze.		
	
	\subsection{Consumer}
		Come per gli strumenti precedentemente elencati, anche per quelli su cui il messaggio andrà ad essere inoltrato, è necessaria la creazione di un microservizio in grado di fare da tramite tra Broker e \gloss{client} dello strumento.
		Le tecnologie alle quali vengono inoltrati i messaggi sono elencate di seguito in ordine di priorità in base a quanto richiesto dal committente.
			
		\subsubsection{Telegram}
		Telegram permette l'interazione in maniera automatica con gli utenti tramite \gloss{bot} che possono essere configurati per mandare messaggi ricevuti da strumenti di terze parti.
		Il Broker manda i messaggi ad un microservizio che si occupa della trasmissione effettiva al bot di Telegram.
		%https://core.telegram.org/bots
		
		\subsubsection{E-mail}
		Per inoltrare le e-mail agli utenti finali è necessario un microservizio che sfrutta un server di posta in modo tale da poter ricevere i messaggi dal Broker e poi mandarli all'indirizzo specificato.
		%https://realpython.com/python-send-email/
		
%		\subsubsection{Slack}
%		È possibile utilizzare le API di Slack per poter mandare push notifications ai canali o alle persone interessate specificando il nome del canale (o lo username della persona), testo del messaggio, e lo username che verrà mostrato per il mittente.
%		%https://www.confluent.io/blog/real-time-syslog-processing-with-apache-kafka-and-ksql-part-2-event-driven-alerting-with-slack/
	
	\subsection{Gestore personale}
	Il gestore personale è quel \gloss{componente} che permette agli utenti di impostare le proprie preferenze per il canale di inoltro dei messaggi ed i Topic a cui ci si iscrive, interfacciandosi dunque con l'intero sistema.
	È previsto come un'applicazione web installata in un server interno all'azienda accessibile dai dipendenti interessati, i quali potranno quindi modificare, perciò con la possibilità di aggiungere e rimuovere, impostazioni come Topic a cui si è iscritti e modalità di ricezione dei messaggi.
	
	\subsection{Container Software}
	
		Un container software simula questo ambiente virtuale dove è possibile testare e mantenere le proprie applicazioni, permettendo di aumentare l'efficienza riducendone i costi e simulando l'esecuzione di sistema operativo su una macchina con \gloss{risorse} condivise.
		
		\subsection{Differenza tra container e macchina virtuale}
		A differenza delle \gloss{macchine virtuali}, dove lo stato dell'ambiente viene salvato su disco occupando memoria, i container si adattano in maniera più performante all'applicativo richiesto, in quanto il loro scopo è quello di massimizzare la quantità delle applicazioni in esecuzione riducendo al minimo il numero delle macchine per eseguirla.
		Sono quindi più leggeri, occupando meno memoria su disco e impiegando meno risorse. Purtroppo al termine della loro esecuzione non viene salvato lo stato, a meno che non sia esplicitamente richiesto.
		
		\subsubsection{Docker e Dockerfile}
		L'azienda consiglia di utilizzare \gloss{Docker} per la semplicità e perché si adatta all'architettura a microservizi.
		La configurazione avverrà tramite un \gloss{dockerfile} in cui verranno specificate informazioni come sistema operativo, script di avvio, numero di istanze ed altri parametri specifici.

	\subsection{Utenti}
		\subsubsection{Amministratore}
		\subsubsection{Utente finale}
		
	\subsection{Topic}