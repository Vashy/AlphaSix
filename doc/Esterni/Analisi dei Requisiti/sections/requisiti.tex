\newpage
\section{Requisiti}
Ad ogni requisito viene assegnato il codice identificativo univoco:
	\begin{center}
		\texttt{R[Numero][Tipo][Priorità]}
	\end{center}
	in cui ogni parte ha un significato preciso:
	\begin{itemize}
		\item \textbf{R}: requisito.
		\item \textbf{Numero}: numero progressivo che segue la struttura dei documenti.
		\item \textbf{Tipo}: la la tipologia di requisito che può essere di:
		\begin{itemize}
			\item \textbf{F}: funzionalità.
			\item \textbf{Q}: \gloss{qualità}.
			\item \textbf{V}: vincolo.
		\end{itemize}
		\item \textbf{Priorità}: indica il grado di urgenza di un requisito di essere soddisfatto, come:
		\begin{itemize}
			\item \textbf{0}: opzionale.
			\item \textbf{1}: desiderabile.
			\item \textbf{2}: obbligatorio.
		\end{itemize}
	\end{itemize}


	Esempio: \texttt{R2Q1} indica il secondo requisito di qualità ed è desiderabile.

	%I requisiti di seguito riportati sono elencati in modo tale da seguire la struttura dei documenti. Ovvero, si possono trovare raggruppati i requisiti derivanti dello stesso tipo, ad esempio solo i requisiti di funzionalità dal sesto al diciannovesimo derivano dai casi d'uso.
	%TODO: rifare id requisiti

%inserire il fatto che una persona può aggiungere il proprio nickname o altro da interfaccia, come requisito opzionale	...

% Generazione automatica dei numeri
%\newcounter{vaZ} % valore
%\newcommand{\incrZ}{\addtocounter{vaZ}{+1}} % Comando per l'aumento automatico del counter vaZ
%\newcommand{\addZNumber}[0]{\incrZ\thevaZ} % Comando per generare
%
%\newcommand{\Freq}[3]{R\addZNumber F#1 & #2 & #3 \\}
%\newcommand{\Fsubreq}[3]{R\thevaZ F#1 & #2 & #3 \\} % Se c'è bisogno di sottocasi

% Comando requisito generico
\newcommand{\req}[3]{%
#1 & #2 & #3 \\
}

	%COMANDI PER REQ DI FUNZIONALITÀ
	% Generazione automatica dei numeri
	\newcounter{vaF} % valore
	\newcounter{secF}[vaF] % per il secondo livello del requisito
	\newcounter{thF}[secF] % terzo livello

	\newcommand{\ReqF}[3]{\stepcounter{vaF}R\thevaF F#1 & #2 & #3 \\} % Primo livello
	\newcommand{\subReqF}[3]{\stepcounter{secF}R\thevaF.\thesecF F#1 & #2 & #3 \\} % Secondo livello
	\newcommand{\subsubReqF}[3]{\stepcounter{thF}R\thevaF.\thesecF.\thethF F#1 & #2 & #3 \\} % Terzo livello

	\newcounter{tableCounter} % Per automatizzare conteggio tabelle

	\subsection{Requisiti di funzionalità}\label{RequisitiFunzionalità}
	
	\stepcounter{tableCounter}
	\begin{table}[H]
		\begin{paddedtablex}[1.7]{\textwidth}{cXc}%0 opz  2 obb
			\textbf{Codice} & \textbf{Requisito} & \textbf{Fonte} \\\toprule

			% \stepcounter{vaF} % Per allineare i requisiti ai casi d'uso

			% da casi d'uso
			\ReqF{2}{Redmine deve poter inviare la segnalazione di apertura issue al Producer Redmine}{Interno UC1-PR}
			\ReqF{2}{Redmine deve poter inviare la segnalazione di modifica issue al Producer Redmine}{Interno UC2-PR}
			\ReqF{2}{GitLab deve essere in grado di segnalare l'apertura di issue al Producer GitLab}{Interno UC3-PG}
			\ReqF{2}{GitLab deve essere in grado di segnalare la modifica issue al Producer GitLab}{Interno UC4-PG}
			\ReqF{2}{GitLab deve poter segnalare un evento di push al Producer GitLab}{Interno UC5-PG}
			\ReqF{2}{Il Producer Redmine deve essere in grado di inviare un messaggio al Gestore Personale}{Interno UC6-GP}
				\subReqF{2}{Il Producer Redmine deve essere in grado di inviare un messaggio di apertura issue al Gestore Personale}{Interno UC6.1-GP}
				\subReqF{2}{Il Producer Redmine deve essere in grado di inviare un messaggio di modifica issue al Gestore Personale}{Interno UC6.2-GP}
			\ReqF{1}{Il Producer Redmine deve essere in grado di scartare i messaggi non validi}{Interno UC6.3-GP}
			\ReqF{2}{Il Producer GitLab deve essere in grado di inviare un messaggio al Gestore Personale}{Interno UC7-GP}
				\subReqF{2}{Il Producer GitLab deve essere in grado di inviare messaggi di commit al Gestore Personale}{Interno UC7.1-GP}
				\subReqF{2}{Il Producer GitLab deve essere in grado di inviare un messaggio di issue al Gestore Personale}{Interno UC7.2-GP}
					\subsubReqF{2}{Il Producer GitLab deve essere in grado di inviare un messaggio di nuova issue al Gestore Personale}{Interno UC7.2.1-GP}
					\subsubReqF{2}{Il Producer GitLab deve essere in grado di inviare un messaggio di modifica issue al Gestore Personale}{Interno UC7.2.2-GP}
			\ReqF{1}{Il Producer GitLab deve essere in grado di scartare i messaggi di issue non validi}{Interno UC7.2.3-GP}
			\ReqF{2}{Il Gestore Personale deve poter inviare il messaggio finale al Consumer Telegram}{Interno UC8-CT}

			\bottomrule\\
		\end{paddedtablex}
		\caption{Elenco dei requisiti di funzionalità (\thetableCounter)}
	\end{table}

	\stepcounter{tableCounter}
	\begin{table}[H]
		\begin{paddedtablex}[1.7]{\textwidth}{cXc}%0 opz  2 obb
			\textbf{Codice} & \textbf{Requisito} & \textbf{Fonte} \\\toprule

			\ReqF{2}{Il Gestore Personale deve poter inviare il messaggio finale al Consumer Email}{Interno UC9-CE}
			\ReqF{2}{Il Consumer Telegram deve poter inoltrare il messaggio finale al bot Telegram}{Interno UC10-BT}
			\ReqF{2}{Il Consumer Email deve poter inoltrare il messaggio finale al server Email}{Interno UC11-SE}
			% \stepcounter{vaF} e usare \subReqF per rientrare di un livello
			\ReqF{0}{L'utente può eseguire l'accesso al Gestore Personale}{Interno UC12.1-GP}
				\subReqF{0}{L'utente può inserire il proprio identificativo all'interno del sistema}{Interno UC12.1.1}
			\ReqF{0}{\progetto\ fa apparire un messaggio di errore se il tentativo di accesso non è andato a buon fine}{Interno UC12.1.2}
			\ReqF{0}{L'utente acceduto deve poter uscire dal sistema}{Interno UC13-GP}
			\ReqF{0}{L'utente acceduto deve poter aggiungere un nuovo utente}{Interno UC14-GP}
				\subReqF{0}{L'utente acceduto deve poter inserire il nome dell'utente da aggiungere}{Interno UC14.1.1-GP}
				\subReqF{0}{L'utente acceduto deve poter inserire il cognome dell'utente da aggiungere}{Interno UC14.1.2-GP}
				\subReqF{0}{L'utente acceduto deve poter inserire il contatto Email dell'utente da aggiungere}{Interno UC14.1.3-GP}
					\subsubReqF{0}{L'utente acceduto deve poter visualizzare un messaggio di errore se il contatto Email non è univoco}{Interno UC14.2-GP}
				\subReqF{0}{L'utente acceduto deve poter inserire il contatto Telegram dell'utente da aggiungere}{Interno UC14.1.4-GP}
					\subsubReqF{0}{L'utente acceduto deve poter visualizzare un messaggio di errore se il contatto Telegram non è univoco}{Interno UC14.2-GP}
			\ReqF{0}{L'utente acceduto deve poter rimuovere un utente presente nel sistema}{Interno UC15-GP}
			% \stepcounter{secF}
				\subReqF{0}{L'utente acceduto deve poter inserire il contatto Email dell'utente da rimuovere}{Interno UC15.1.1-GP}
				\subReqF{0}{L'utente acceduto deve poter inserire il contatto Telegram dell'utente da rimuovere}{Interno UC15.1.2-GP}
				\subReqF{0}{L'utente acceduto deve poter visualizzare un messaggio di errore se l'identificativo non è presente nel sistema}{Interno UC15.2-GP}

			\bottomrule\\
		\end{paddedtablex}
		\caption{Elenco dei requisiti di funzionalità (\thetableCounter)}
	\end{table}

	\stepcounter{tableCounter}
	\begin{table}[H]
		\begin{paddedtablex}[1.7]{\textwidth}{cXc}%0 opz  2 obb
			\textbf{Codice} & \textbf{Requisito} & \textbf{Fonte} \\\toprule

				\subReqF{0}{L'utente acceduto deve potersi rimuovere dal sistema}{Interno UC15.3-GP}
			\ReqF{0}{L'utente acceduto deve poter modificare le informazioni relative a un utente}{Interno UC16-GP}
				\subReqF{0}{L'utente acceduto deve poter selezionare l'identificativo dell'utente da modificare}{Interno UC16.1-GP}
					\subsubReqF{0}{L'utente acceduto deve poter visualizzare un messaggio di errore se l'identificativo non è riconosciuto dal sistema}{Interno UC16.2-GP}
				\subReqF{0}{L'utente acceduto deve poter scegliere il nuovo nome dell'utente da modificare}{Interno UC16.1.1.1-GP}
				\subReqF{0}{L'utente acceduto deve poter scegliere il nuovo cognome dell'utente da modificare}{Interno UC16.1.1.2-GP}
				\subReqF{0}{L'utente acceduto deve poter scegliere il nuovo contatto Email dell'utente da modificare}{Interno UC16.1.1.3-GP}
					\subsubReqF{0}{L'utente acceduto deve poter visualizzare un messaggio di errore se il contatto Email è già presente}{Interno UC16.1.2-GP}				
				\subReqF{0}{L'utente acceduto deve poter scegliere il nuovo contatto Telegram dell'utente da modificare}{Interno UC16.1.1.4-GP}
					\subsubReqF{0}{L'utente acceduto deve poter visualizzare un messaggio di errore se il contatto Telegram è già presente}{Interno UC16.1.2-GP}

			%\stepcounter{vaF}
			\ReqF{0}{L'utente acceduto deve poter aggiungere le proprie preferenze nel sistema}{Interno UC17-GP}
				\subReqF{0}{L'utente acceduto deve poter aggiungere nuovi Topic di iscrizione}{Interno UC17.1-GP}
				\subReqF{0}{L'utente acceduto deve poter aggungere nuovi giorni di indisponibilità nel calendario}{Interno UC17.2-GP}
				\subReqF{0}{L'utente acceduto deve poter aggungere una nuova piattaforma di messaggistica preferita}{Interno UC17.3-GP}
				\subReqF{0}{L'utente acceduto deve poter aggungere la propria persona di fiducia}{Interno UC17.4-GP}
					\subsubReqF{0}{L'utente acceduto deve poter visualizzare un messaggio di errore se il contatto Email della persona di fiducia non è presente nel sistema}{Interno UC17.5-GP}

			\bottomrule\\
		\end{paddedtablex}
		\caption{Elenco dei requisiti di funzionalità (\thetableCounter)}
	\end{table}

	\stepcounter{tableCounter}
	\begin{table}[H]
		\begin{paddedtablex}[1.7]{\textwidth}{cXc}%0 opz  2 obb
			\textbf{Codice} & \textbf{Requisito} & \textbf{Fonte} \\\toprule

					\subsubReqF{0}{L'utente acceduto deve poter visualizzare un messaggio di errore se il contatto Telegram della persona di fiducia non è presente nel sistema}{Interno UC17.5-GP}
				\subReqF{0}{L'utente acceduto deve poter aggiungere nuove keyword di interesse per i messaggi di commit di GitLab}{Interno UC17.6-GP}
					\subsubReqF{0}{L'utente acceduto deve poter visualizzare un messaggio di errore se la keyword inserita era già nella sua lista}{Interno UC17.7-GP}

			%\stepcounter{vaF}
			\ReqF{0}{L'utente acceduto deve poter rimuovere le proprie preferenze dal sistema}{Interno UC18-GP}
				\subReqF{0}{L'utente acceduto deve poter rimuovere i Topic a cui è iscritto}{Interno UC18.1-GP}
				\subReqF{0}{L'utente acceduto deve poter rimuovere giorni di indisponibilità nel calendario}{Interno UC18.2-GP}
				\subReqF{0}{L'utente acceduto deve poter rimuovere una piattaforma di messaggistica preferita}{Interno UC18.3-GP}
				\subReqF{0}{L'utente acceduto deve poter rimuovere la propria persona di fiducia}{Interno UC18.4-GP}
					\subsubReqF{0}{L'utente acceduto deve poter visualizzare un messaggio di errore se il contatto Email della persona di fiducia non è presente nel sistema}{Interno UC18.5-GP}
					\subsubReqF{0}{L'utente acceduto deve poter visualizzare un messaggio di errore se il contatto Telegram della persona di fiducia non è presente nel sistema}{Interno UC18.5-GP}
				\subReqF{0}{L'utente acceduto deve poter rimuovere keyword di interesse per i messaggi di commit di GitLab}{Interno UC18.6-GP}
					\subsubReqF{0}{L'utente acceduto deve poter visualizzare un messaggio di errore se la keyword da rimuovere è assente dalla sua lista}{Interno UC18.7-GP}

			\bottomrule
		\end{paddedtablex}
		\caption{Elenco dei requisiti di funzionalità (\thetableCounter)}
	\end{table}
			% \Finitsecondreq{0}{L'utente può eseguire l'accesso al gestore personale}{Interno UC5.1}
			% \Finitthirdreq{0}{L'utente può inserire il proprio username che lo identifica all'interno di \progetto}{Interno UC5.1.1}
			% \Fsecondreq{0}{\progetto\ fa apparire un messaggio di errore se il tentativo di accesso non è andato a buon fine}{Interno UC5.2}
			% \req{R\addFNumber.1.1F0}{L'utente può iscriversi a un Topic}{Interno UC6.1.1}
			% \req{R\thevaF.1.2F0}{L'utente può aggiungere nel gestore personale i giorni in cui non è reperibile}{Interno UC6.1.2}
			% \req{R\thevaF.1.3F0}{L'utente può scegliere la piattaforma di messaggistica, tra Telegram ed e-mail, in cui ricevere le notifiche}{Interno UC6.1.3}
			% \req{R\thevaF.2.1F0}{L'utente può disiscriversi da un Topic}{Interno UC6.2.1}
			% \req{R\thevaF.2.2F0}{L'utente può rimuovere i giorni in cui aveva selezionato precedentemente di non essere reperibile}{Interno UC6.2.2}
			% \req{R\thevaF.2.3F0}{L'utente può togliere le proprie preferenze sulle piattaforme di messaggistica da cui ricevere notifiche inviate attraverso \progetto}{Interno UC6.2.3}


	\stepcounter{tableCounter}
	\begin{table}[H]
		\begin{paddedtablex}[1.7]{\textwidth}{cXc}
			\textbf{Codice} & \textbf{Requisito} & \textbf{Fonte} \\\toprule

			% da capitolato
			% stesso di "L'applicativo Producer/Consumer deve essere in grado di reindirizzare i messaggi verso la persona più appropriata"
			\ReqF{2}{Le componenti Consumer devono essere in grado di inviare i messaggi provenienti da un Topic verso il corretto destinatario}{Capitolato}
			\ReqF{2}{Le componenti Consumer devono essere in grado di abbonarsi ai Topic scelti}{Capitolato}
			\ReqF{2}{Le segnalazioni devono poter essere gestite in maniera automatica e personalizzabile}{Capitolato}
			\ReqF{2}{Nel sistema deve essere presente un Broker che istanzia e gestisce le segnalazioni organizzandole per Topic}{Capitolato}
			\ReqF{2}{Le componenti Producer devono riuscire a pubblicare le segnalazioni recuperate sotto forma di messaggi secondo i Topic corretti}{Capitolato}
			\ReqF{2}{Le componenti devono esporre delle \gloss{API Rest} per le interazioni con le altre componenti}{Capitolato}

			\bottomrule
		\end{paddedtablex}
		\caption{Elenco dei requisiti di funzionalità (\thetableCounter)}
	\end{table}

	%COMANDI PER REQUISITI DI Qualità
	% Generazione automatica dei numeri
	\newcounter{vaQ} % valore
	\newcounter{secQ}[vaQ]
	\newcounter{thQ}[secQ] % terzo livello

	\newcommand{\ReqQ}[3]{\stepcounter{vaQ}R\thevaQ Q#1 & #2 & #3 \\}
	\newcommand{\subReqQ}[3]{\stepcounter{secQ}R\thevaQ.\thesecQ Q#1 & #2 & #3 \\}
	\newcommand{\subsubReqQ}[3]{\stepcounter{thQ}R\thevaQ.\thesecQ.\thethQ Q#1 & #2 & #3 \\} % Terzo livello

	%\addtocounter{vaQ}{1} % inizia da 1 il contatore
	% \newcommand{\incrQ}{\addtocounter{vaQ}{+1}} % Comando per l'aumento automatico del counter vaZ
	% \newcommand{\addQNumber}{\incrQ\thevaQ} % Comando per generare il valore incrementato di uno rispetto a prima

	% \newcounter{secondQ} % per il secondo livello del requisito
	% \addtocounter{secondQ}{1}
	% \newcommand{\secIncrQ}{\addtocounter{secondQ}{+1}} % Comando per l'aumento automatico del counter per il secondo livello
	% \newcommand{\addSecQNumber}{\secIncrQ\thesecondQ} % Comando per generare il valore incrementato di uno rispetto a prima
	% \newcommand{\resetQCounter}{\setcounter{secondQ}{1}}
	% \newcommand{\decSecQ}{\resetQCounter\thesecondQ}


	% \newcommand{\Qreq}[3]{R\addQNumber Q#1 & #2 & #3 \\} % Nuovo requisito, maggiore del precedente
	% \newcommand{\Qsubreq}[3]{R\thevaQ Q#1 & #2 & #3 \\} % Requisito diverso ma con stesso numero progressivo
	% \newcommand{\Qsecondreq}[3]{R\thevaQ.\addSecQNumber Q#1 & #2 & #3 \\}
	% \newcommand{\Qinitsecondreq}[3]{R\thevaQ.\decSecQ Q#1 & #2 & #3 \\}


	\setcounter{tableCounter}{1}
	% NB: molti requisiti di qualità sono stati tolti perchè non sono veri requisiti del sistema, riguardano noi e il nostro modo di fare, non il sistema
	\subsection{Requisiti di qualità}\label{RequisitiQualità}

	\begin{table}[H]
		\begin{paddedtablex}[1.7]{\textwidth}{cXc}
			\textbf{Codice} & \textbf{Requisito} & \textbf{Fonte} \\
			\toprule
			% presi dal PdQ
			\ReqQ{1}{È stabilito un numero massimo di giorni di ritardo per la chiusura di una issue}{Interno QPR001}
			\ReqQ{1}{L'\gloss{indice di Gulpease} di ogni documento deve rientrare all'interno di un intervallo stabilito}{Interno QPD001}
			%\Qreq{1}{I costi previsti dalla \gloss{pianificazione} non dovrebbero variare più di quanto stabilito}{Interno QPR002}
			\ReqQ{1}{Una frequenza minima di commit devono essere effettuati in una settimana}{Interno QPR004}
			\ReqQ{1}{Un numero stabilito di requisiti desiderabili deve essere soddisfatto}{Interno QPR006}
			\ReqQ{1}{Nessun rischio non verificato precedentemente deve accadere nel corso del progetto}{Interno QPR007}
			\ReqQ{1}{Ogni documento deve attraversare tutte le fasi previste dal suo \gloss{ciclo di vita}}{Interno QPR008}
			\ReqQ{1}{Viene stabilito il numero massimo di modifiche che può ricevere un prodotto prima di essere verificato}{Interno QPR009}
			\ReqQ{2}{Tutte le \gloss{norme} inserite nelle \NdP\ devono essere rispettate}{Interno}
			\ReqQ{2}{Tutti i vincoli presenti nel \PdQ\ devono essere rispettati}{Interno}
			\ReqQ{2}{Le applicazioni sviluppate devono rispettare i fattori trattati in The Twelve-Factor App segnati nel \PdQd}{Capitolato \Doc{VE\_2018-12-12}}
			\ReqQ{2}{È necessario presentare il \gloss{bug} reporting per ogni componente}{Capitolato}

			\bottomrule
		\end{paddedtablex}
		\caption{Elenco dei requisiti di qualità (\thetableCounter)}
	\end{table}



	\begin{table}[H]
		\begin{paddedtablex}[1.7]{\textwidth}{cXc}
			\textbf{Codice} & \textbf{Requisito} & \textbf{Fonte} \\\toprule

			\ReqQ{2}{Deve essere redatta la documentazione sulle scelte progettuali effettuate}{Capitolato}
			%\req{R\thevaQ.1Q2}{Ogni scelta descritta nella documentazione deve essere correlata dalle relative motivazioni}{Capitolato}
			\subReqQ{2}{Ogni scelta descritta nella documentazione deve essere correlata dalle relative motivazioni}{Capitolato}
			\ReqQ{2}{È necessario testare ogni prodotto software considerando ogni sistema di riferimento e interazione tra le sue parti, perciò con test d'unità, d'integrazione e di sistema}{Capitolato}
			%\req{R\thevaQ.1Q2}{È necessario fornire test unitari per ogni componente applicativo}{Capitolato}
			\subReqQ{2}{È necessario fornire test d'unità per ogni componente applicativo}{Capitolato}
			\subReqQ{2}{È necessario fornire test d'integrazione per ogni componente applicativo}{Capitolato}
			\subReqQ{2}{È necessario testare interamente il sistema con test di sistema}{Capitolato}
			\ReqQ{1}{Per ogni problema aperto documentato, si allegano delle soluzioni da attuare in futuro}{Capitolato}
			\subReqQ{2}{Deve essere redatta una documentazione su eventuali problemi riscontrati rimasti ancora aperti al termine del progetto}{Capitolato}
			\ReqQ{2}{È necessario presentare un file \gloss{README} per ogni componente}{Capitolato}
			%\req{R\thevaQ.1Q2}{I file README delle componenti applicative devono contenere la documentazione delle \gloss{API} esposte dal servizio}{Capitolato}
			\subReqQ{2}{I file README delle componenti applicative devono contenere la documentazione delle \gloss{API} esposte dal servizio}{Capitolato}
			\subReqQ{2}{I file README delle componenti applicative devono contenere le istruzioni per il loro utilizzo}{Capitolato}
			\subReqQ{2}{È necessario presentare un file README per il Dockerfile}{Capitolato}
				\subsubReqQ{2}{Il file README per il Dockerfile deve contenere le istruzioni per l'avvio}{Capitolato}
				\subsubReqQ{2}{Il file README per il Dockerfile deve contenere la documentazione delle configurazioni custom scelte}{Capitolato}

			\bottomrule \\
		\end{paddedtablex}
		\caption{Elenco dei requisiti di qualità (\thetableCounter)}
	\end{table}

	%COMANDI PER REQ DI VINCOLO
	% Generazione automatica dei numeri

	\newcounter{vaV} % valore
	\newcounter{secV}[vaV]

	\newcommand{\ReqV}[3]{\stepcounter{vaV}R\thevaV V#1 & #2 & #3 \\}
	\newcommand{\subReqV}[3]{\stepcounter{secV}R\thevaV.\thesecV V#1 & #2 & #3 \\}

	% \newcommand{\incrV}{\addtocounter{vaV}{+1}} % Comando per l'aumento automatico del counter vaZ
	% \newcommand{\addVNumber}{\incrV\thevaV} % Comando per generare il valore incrementato di uno rispetto a prima
	
	% \newcounter{secondV} % per il secondo livello del requisito
	% \addtocounter{secondV}{1}
	% \newcommand{\secIncrV}{\addtocounter{secondV}{+1}} % Comando per l'aumento automatico del counter per il secondo livello
	% \newcommand{\addSecVNumber}{\secIncrV\thesecondV} % Comando per generare il valore incrementato di uno rispetto a prima
	% \newcommand{\resetVCounter}{\setcounter{secondV}{1}}
	% \newcommand{\decSecV}{\resetVCounter\thesecondV}
	
	
	% \newcommand{\Vreq}[3]{R\addVNumber V#1 & #2 & #3 \\} % Nuovo requisito, maggiore del precedente
	% \newcommand{\Vsubreq}[3]{R\thevaV V#1 & #2 & #3 \\} % Requisito diverso ma con stesso numero progressivo
	% \newcommand{\Vsecondreq}[3]{R\thevaV.\addSecVNumber V#1 & #2 & #3 \\}
	% \newcommand{\Vinitsecondreq}[3]{R\thevaV.\decSecV V#1 & #2 & #3 \\}
	

	%TODO: aggiungere versione di Slack
	\subsection{Requisiti di vincolo}\label{RequisitiVincolo}

	\begin{table}[H]
		\begin{paddedtablex}[1.7]{\textwidth}{cXc} %\rowcolors{1}{\tablegray}{\lightgray}
			\textbf{Codice} & \textbf{Requisito} & \textbf{Fonte} \\\toprule

			\ReqV{2}{Devono essere sviluppati due componenti applicativi Producer tra \redmine, \gitlab\ e SonarQube 6.7}{Capitolato}
			%\req{R\thevaV.1V0}
			\subReqV{0}{È possibile avere un terzo componente applicativo Producer oltre ai due obbligatori}{Capitolato}
			\ReqV{2}{Devono essere sviluppati due componenti applicativi Consumer tra \telegram, Email e Slack}{Capitolato}
			%\req{R\thevaV.1V0}
			\subReqV{0}{È possibile avere un terzo componente applicativo Consumer oltre ai due obbligatori}{Capitolato}
			\ReqV{2}{\docker\ deve essere la tecnologia di riferimento per l'istanziazione di tutte le componenti}{Capitolato}
			%\req{R\thevaV.1V2}
			\subReqV{2}{È necessario presentare un Dockerfile per ogni componente}{Capitolato}
			\ReqV{1}{Per lo sviluppo dei componenti applicativi è possibile usare come linguaggio \gloss{Java} 8 o una versione più recente, \gloss{\python} o \gloss{Node.js} 10.15.1}{Capitolato}
			\ReqV{1}{È possibile utilizzare \kafka\ come Broker}{Capitolato}
			\ReqV{0}{L'interfaccia è sviluppata utilizzando \html\ e \css}{Interno}
			\ReqV{0}{Il database viene sviluppato utilizzando \mongodb}{Interno}
			\ReqV{0}{Il server web viene sviluppato utilizzando \python\ con la libreria \gloss{CherryPy}}{Interno}
			\ReqV{0}{Gli URL dell'interfaccia web devono rispettare lo standard REST}{Interno}
			\ReqV{0}{Ciascun componente Docker viene istanziato tramite un file \dockercompose}{Interno}

			\bottomrule\\
		\end{paddedtablex}
		\caption{Elenco dei requisiti di vincolo}
	\end{table}



	\subsection{Tracciamento}\label{Tracciamento}

		\subsubsection{Tracciamento fonte-requisito}

		% \begin{table}[H]
		% 	\centering
		% 	{\def\arraystretch{1.4}
		% 	\begin{oldtabularx}{\textwidth}{YY}
		% 		\textbf{Fonte} & \textbf{Requisito} \\
		% 		\toprule
		% 		\cellcolor{white} & R7F2 \\
		% 		\cellcolor{white} & R8F2 \\
		% 		\cellcolor{white} & R9F2 \\
		% 		\cellcolor{white} & R10F2 \\
		% 		\cellcolor{white} & R11F2 \\
		% 		% \cmidrule{2-2}
		% 		\cellcolor{white} & R14Q2 \\
		% 		\cellcolor{white} & R15Q2 \\
		% 		\cellcolor{white} & R16Q2 \\
		% 		\cellcolor{white} & R17Q2 \\
		% 		\cellcolor{white} & R18Q2 \\
		% 		\cellcolor{white} & R18.1Q2 \\
		% 		\cellcolor{white} & R18.2Q2 \\
		% 		\cellcolor{white} \multirow{-13}{*}{Capitolato} & R18.3Q2 \\
		% 		\bottomrule\\
		% 	\end{oldtabularx}}
		% 	\caption{Elenco dei requisiti del capitolato (1)}
		% \end{table}

		\begin{table}[H]
			\centering
			{\def\arraystretch{1.6}
			\begin{oldtabularx}{0.7\textwidth}{YY}
				\textbf{Fonte} & \textbf{Requisito} \\
				\toprule
				\multirow{22}{*}{Capitolato}
				& R22F2 \\\cline{2-2}
				& R23F2 \\\cline{2-2}
				& R24F2 \\\cline{2-2}
				& R25F2 \\\cline{2-2}
				& R26F2 \\\cline{2-2}
				& R27F2 \\\cline{2-2}
				% \cmidrule{2-2}
				& R10Q2 \\\cline{2-2}
				& R11Q2 \\\cline{2-2}
				& R12Q2 \\\cline{2-2}
				& R12.1Q2 \\\cline{2-2}
				& R13Q2 \\\cline{2-2}
				& R13.1Q2 \\\cline{2-2}
				& R13.2Q2 \\\cline{2-2}
				& R13.3Q2 \\\cline{2-2}
				& R14Q1 \\\cline{2-2}
				& R14.1Q2 \\\cline{2-2}
				& R15Q2 \\\cline{2-2}
				& R15.1Q2 \\\cline{2-2}
				& R15.2Q2 \\\cline{2-2}
				& R15.3Q2 \\\cline{2-2}
				& R15.3.1Q2 \\\cline{2-2}
				& R15.3.2Q2 \\\bottomrule
			\end{oldtabularx}}
			\caption{Elenco dei requisiti del capitolato (1)}
		\end{table}


		\begin{table}[H]
			\centering
			{\def\arraystretch{1.6}
			\begin{oldtabularx}{0.7\textwidth}{YY}
				\textbf{Fonte} & \textbf{Requisito} \\
				\toprule
				\multirow{8}{*}{Capitolato}
				& R1V2 \\\cline{2-2}
				& R1.1V0 \\\cline{2-2}
				& R2V2 \\\cline{2-2}
				& R2.1V0 \\\cline{2-2}
				& R3V2 \\\cline{2-2}
				& R3.1V2 \\\cline{2-2}
				& R4V1 \\\cline{2-2}
				& R5V1 \\\bottomrule
			\end{oldtabularx}}
			\caption{Elenco dei requisiti del capitolato (2)}
		\end{table}

		% \begin{table}[H]
		% 	\centering
		% 	{\def\arraystretch{1.5}
		% 		\begin{tabularx}{\textwidth}{YY}
		% 			\textbf{Fonte} & \textbf{Requisito} \\
		% 			\toprule
		% 			\cellcolor{white} & R11Q1 \\
		% 			\cellcolor{white} & R12Q2 \\
		% 			\cellcolor{white} \multirow{-2}{*}{Interno} & R13Q2 \\
		% 			\bottomrule
		% 		\end{tabularx}}
		% 	\caption{Elenco dei requisiti interni}
		% \end{table}

		% \begin{table}[H]
		% 	\centering
		% 	{\def\arraystretch{1.6}
		% 	\begin{oldtabularx}{0.7\textwidth}{YY}
		% 		\textbf{Fonte} & \textbf{Requisito} \\
		% 		\toprule
		% 		\multirow{3}{*}{Capitolato}
		% 		& R11Q1 \\\cline{2-2}
		% 		& R12Q2 \\\cline{2-2}
		% 		& R13Q3 \\\bottomrule
		% 	\end{oldtabularx}}
		% 	\caption{Elenco dei requisiti interni}
		% \end{table}

		\begin{table}[H]
			\centering
			\rowcolors{2}{white}{\tablegray}
			{\def\arraystretch{1.5}
			\begin{tabularx}{0.7\textwidth}{YY}
				\textbf{Fonte} & \textbf{Requisito} \\
				\toprule
				UC1-PR & R1F2 \\
				UC2-PR & R2F2 \\
				UC3-PG & R3F2 \\
				UC4-PG & R4F2 \\
				UC5-PG & R5F2 \\
				UC6-GP & R6F2 \\
				UC6.1-GP & R6.1F2 \\
				UC6.2-GP & R6.2F2 \\
				UC6.3-GP & R7F1 \\
				UC7-GP & R8F2 \\
				UC7.1-GP & R8.1F2 \\
				UC7.2-GP & R8.2F2 \\
				UC7.2.1-GP & R8.2.1F2 \\
				UC7.2.2-GP & R8.2.2F2 \\
				UC7.2.3-GP & R9F1 \\
				UC8-CT & R10F2 \\
				UC9-CE & R11F2 \\
				\bottomrule \\
			\end{tabularx}}
			\caption{Elenco dei requisiti per i casi d'uso (1)}
		\end{table}


		\begin{table}[H]
			\centering
			%\rowcolors{2}{white}{\tablegray}
			{\def\arraystretch{1.5}
			\begin{oldtabularx}{0.7\textwidth}{YY}
				\textbf{Fonte} & \textbf{Requisito} \\
				\toprule
				\rowcolor{\tablegray} UC10-BT & R12F2 \\
				UC11-SE & R13F2 \\
				\rowcolor{\tablegray} UC12.1-GP & R14F0 \\
				UC12.1.1 & R14.1F0 \\
				\rowcolor{\tablegray} UC12.1.2 & R15F0 \\
				UC13-GP & R16F0 \\
				\rowcolor{\tablegray} UC14-GP & R17F0 \\
				UC14.1.1-GP & R17.1F0 \\
				\rowcolor{\tablegray} UC14.1.2-GP & R17.2F0 \\
				UC14.1.3-GP & R17.3F0 \\

				\rowcolor{\tablegray}
				& R17.3.1F0 \\
				\rowcolor{\tablegray}
				\multirow{-2}{*}{UC14.2-GP}
				& R17.4.1F0 \\

				UC14.1.4-GP & R17.4F0 \\
				\rowcolor{\tablegray} UC15-GP & R18F0 \\
				UC15.1.1-GP & R18.1F0 \\
				\rowcolor{\tablegray} UC15.1.2-GP & R18.2F0 \\
				UC15.2-GP & R18.3F0 \\
				\rowcolor{\tablegray} UC15.3-GP & R18.4F0 \\
				UC16-GP & R19F0 \\
				\rowcolor{\tablegray} UC16.1-GP & R19.1F0 \\
				UC16.2-GP & R19.1.1F0 \\
				\rowcolor{\tablegray} UC16.1.1.1-GP & R19.2F0 \\
				UC16.1.1.2-GP & R19.3F0 \\
				\rowcolor{\tablegray} UC16.1.1.3-GP & R19.4F0 \\
				
				
				& R19.4.1F0 \\
				\multirow{-2}{*}{UC16.1.2-GP}
				& R19.5.1F0 \\

				\rowcolor{\tablegray}UC16.1.1.4-GP & R19.5F0 \\
				UC17-GP & R20F0 \\
				\rowcolor{\tablegray}UC17.1-GP & R20.1F0 \\
				UC17.2-GP & R20.2F0 \\
				\rowcolor{\tablegray}UC17.3-GP & R20.3F0 \\ 
			   \bottomrule
		   \end{oldtabularx}}
		   \caption{Elenco dei requisiti per i casi d'uso (2)}
	    \end{table}


	    \begin{table}[H]
		\centering
		%\rowcolors{2}{white}{\tablegray}
		{\def\arraystretch{1.5}
		\begin{oldtabularx}{0.7\textwidth}{YY}
			\textbf{Fonte} & \textbf{Requisito} \\
			\toprule
			\rowcolor{\tablegray} UC17.4-GP & R20.4F0 \\

			
			& R20.4.1F0 \\
			\multirow{-2}{*}{UC17.5-GP}
			& R20.4.2F0 \\

			\rowcolor{\tablegray}UC17.6-GP & R20.5F0 \\   
			UC17.7-GP & R20.5.1F0 \\
			\rowcolor{\tablegray}UC18-GP & R21F0 \\
			UC18.1-GP & R21.1F0 \\
			\rowcolor{\tablegray}UC18.2-GP & R21.2F0 \\ 	   
			UC18.3-GP & R21.3F0 \\
			\rowcolor{\tablegray}UC18.4-GP & R21.4F0 \\

			& R21.4.1F0 \\
			\multirow{-2}{*}{UC18.5-GP}
			& R21.4.2F0 \\

			\rowcolor{\tablegray}UC18.6-GP & R21.5F0 \\
			UC18.7-GP & R21.5.1F0 \\

			\bottomrule
	  	\end{oldtabularx}}
	  	\caption{Elenco dei requisiti per i casi d'uso (3)}
  		\end{table}


		\begin{table}[H]
		\centering
		\rowcolors{2}{white}{\tablegray}
		{\def\arraystretch{1.5}
		\begin{tabularx}{0.7\textwidth}{YY}
			\textbf{Fonte} & \textbf{Requisito} \\
			\toprule
			QPR001 & R1Q1 \\
			QPD001 & R2Q1 \\
			QPR004 & R3Q1 \\
			QPR006 & R4Q1 \\
			QPR007 & R5Q1 \\
			QPR008 & R6Q2 \\
			QPR009 & R7Q1 \\
			\Doc{VE\_2018-12-12} & R10Q2 \\
			\bottomrule
		\end{tabularx}}
		\caption{Elenco dei requisiti per gli obiettivi di qualità e verbali}
	\end{table}


	\setcounter{tableCounter}{1}
	\begin{table}[H]
		\centering
		{\def\arraystretch{1.6}
		\begin{oldtabularx}{0.7\textwidth}{YY}
			\textbf{Fonte} & \textbf{Requisito} \\
			\toprule
			\multirow{29}{*}{Interno}
			& R1F2 \\\cline{2-2}
			& R2F2 \\\cline{2-2}
			& R3F2 \\\cline{2-2}
			& R4F2 \\\cline{2-2}
			& R5F2 \\\cline{2-2}
			& R6F2 \\\cline{2-2}
			& R6.1F2 \\\cline{2-2}
			& R6.2F2 \\\cline{2-2}
			& R7F1 \\\cline{2-2}
			& R8.1F2 \\\cline{2-2}
			& R8.2F2 \\\cline{2-2}
			& R8.2.1F2 \\\cline{2-2}
			& R8.2.2F2 \\\cline{2-2}
			& R9F1 \\\cline{2-2}
			& R10F2 \\\cline{2-2}
			& R11F2 \\\cline{2-2}
			& R12F2 \\\cline{2-2}
			& R13F2 \\\cline{2-2}
			& R14F0 \\\cline{2-2}
			& R14.1F0 \\\cline{2-2}
			& R15F0 \\\cline{2-2}
			& R16F0 \\\cline{2-2}
			& R17F0 \\\cline{2-2}
			& R17.1F0 \\\cline{2-2}
			& R17.2F0 \\\cline{2-2}
			& R17.3F0 \\\cline{2-2}
			& R17.3.1F0 \\\cline{2-2}
			& R17.4F0 \\
			\bottomrule
		\end{oldtabularx}}
		\caption{Elenco dei requisiti da fonte interna (\thetableCounter)}
	\end{table}

	\stepcounter{tableCounter}
	\begin{table}[H]
		\centering
		{\def\arraystretch{1.6}
		\begin{oldtabularx}{0.7\textwidth}{YY}
			\textbf{Fonte} & \textbf{Requisito} \\
			\toprule
			\multirow{29}{*}{Interno}
			& R17.4.1F0 \\\cline{2-2}
			& R18F0 \\\cline{2-2}
			& R18.1F0 \\\cline{2-2}
			& R18.2F0 \\\cline{2-2}
			& R18.3F0 \\\cline{2-2}
			& R18.4F0 \\\cline{2-2}
			& R19F0 \\\cline{2-2}
			& R19.1F0 \\\cline{2-2}
			& R19.1.1F0 \\\cline{2-2}
			& R19.2F0 \\\cline{2-2}
			& R19.3F0 \\\cline{2-2}
			& R19.4F0 \\\cline{2-2}
			& R19.4.1F0 \\\cline{2-2}
			& R19.5F0 \\\cline{2-2}
			& R19.5.1F0 \\\cline{2-2}
			& R20F0 \\\cline{2-2}
			& R20.1F0 \\\cline{2-2}
			& R20.2F0 \\\cline{2-2}
			& R20.3F0 \\\cline{2-2}
			& R20.4F0 \\\cline{2-2}
			& R20.4.1F0 \\\cline{2-2}
			& R20.4.2F0 \\\cline{2-2}
			& R20.5F0 \\\cline{2-2}
			& R20.5.1F0 \\\cline{2-2}
			& R21F0 \\\cline{2-2}
			& R21.1F0 \\\cline{2-2}
			& R21.2F0 \\\cline{2-2}
			& R21.3F0 \\\cline{2-2}
			& R21.4F0 \\
			\bottomrule
		\end{oldtabularx}}
		\caption{Elenco dei requisiti da fonte interna (\thetableCounter)}
	\end{table}

	\stepcounter{tableCounter}
	\begin{table}[H]
		\centering
		{\def\arraystretch{1.6}
		\begin{oldtabularx}{0.7\textwidth}{YY}
			\textbf{Fonte} & \textbf{Requisito} \\
			\toprule
			\multirow{18}{*}{Interno}
			& R21.4.1F0 \\\cline{2-2}
			& R21.4.2F0 \\\cline{2-2}
			& R21.5F0 \\\cline{2-2}
			& R21.5.1F0 \\\cline{2-2}
			& R1Q1 \\\cline{2-2}
			& R2Q1 \\\cline{2-2}
			& R3Q1 \\\cline{2-2}
			& R4Q1 \\\cline{2-2}
			& R5Q1 \\\cline{2-2}
			& R6Q1 \\\cline{2-2}
			& R7Q1 \\\cline{2-2}
			& R7Q1 \\\cline{2-2}
			& R9Q2 \\\cline{2-2}
			& R6V0 \\\cline{2-2}
			& R7V0 \\\cline{2-2}
			& R8V0 \\\cline{2-2}
			& R9V0 \\\cline{2-2}
			& R10V0 \\				
			\bottomrule
		\end{oldtabularx}}
		\caption{Elenco dei requisiti da fonte interna (\thetableCounter)}
	\end{table}


\newcounter{V} % valore
\newcommand{\deV}{\addtocounter{V}{+1}} % Comando per l'aumento automatico del counter vaZ
\newcommand{\addC}[0]{\theV \deV} % Comando per generare
\addtocounter{V}{1}

\newcounter{Vv} % valore
\newcommand{\deVv}{\addtocounter{Vv}{+1}} % Comando per l'aumento automatico del counter vaZ
\newcommand{\addVC}[0]{\theVv \deVv} % Comando per generare
\addtocounter{Vv}{1}

\newcounter{X} % valore
\newcommand{\deX}{\addtocounter{X}{+1}} % Comando per l'aumento automatico del counter vaZ
\newcommand{\addX}[0]{\theX \deX} % Comando per generare
\addtocounter{X}{1}

		\subsubsection{Tracciamento requisito-fonte}

		% \begin{table}[H]
		% \begin{paddedtablex}[1.7]{\textwidth}{YY}
		% 	\textbf{Requisito} & \textbf{Fonte} \\\toprule
		% 	%R di F
		% 	R\addC
		% 	F2 & Interno UC1 \\
		% 	R\addC
		% 	F2 & Interno UC2 \\
		% 	R\addC
		% 	F2 & Interno UC3 \\
		% 	R\addC
		% 	F2 & Interno UC4 \\
		% 	R\addC
		% 	.1F0 & Interno UC5.1 \\
		% 	R5.1.1F0 & Interno UC5.1.1 \\
		% 	R5.2F0 & Interno UC5.2 \\
		% 	R\addC
		% 	.1.1F0 & Interno UC6.1.1 \\
		% 	R6.1.2F0 & Interno UC6.1.2 \\
		% 	R6.1.3F0 & Interno UC6.1.3 \\
		% 	R6.2.1F0 & Interno UC6.2.1 \\
		% 	R6.2.3F0 & Interno UC6.2.2 \\
		% 	R6.2.3F0 & Interno UC6.2.3 \\
		% 	R\addC
		% 	F2 & Capitolato \\
		% 	R\addC
		% 	F2 & Capitolato \\
		% 	R\addC
		% 	F2 & Capitolato \\
		% 	R\addC
		% 	F2 & Capitolato \\
		% 	R\addC
		% 	F2 & Capitolato \\
		% 	\bottomrule
		% \end{paddedtablex}
		% \caption{Elenco dei requisiti funzionali in rapporto alle fonti}
		% \end{table}

		\begin{table}[H]
			\centering
			{\def\arraystretch{1.6}
			\begin{oldtabularx}{0.7\textwidth}{YY}
				\textbf{Requisito} & \textbf{Fonte} \\
				\toprule

				\rowcolor{\tablegray}
				& Interno \\
				\rowcolor{\tablegray}
				\multirow{-2}{*}{R1F2}
				& UC1-PR \\

				& Interno \\
				\multirow{-2}{*}{R2F2}
				& UC2-PR \\

				\rowcolor{\tablegray}
				& Interno \\
				\rowcolor{\tablegray}
				\multirow{-2}{*}{R3F2}
				& UC3-PG \\

				& Interno \\
				\multirow{-2}{*}{R4F2}
				& UC4-PG \\

				\rowcolor{\tablegray}
				& Interno \\
				\rowcolor{\tablegray}
				\multirow{-2}{*}{R5F2}
				& UC5-PG \\

				& Interno \\
				\multirow{-2}{*}{R6F2}
				& UC6-GP \\

				\rowcolor{\tablegray}
				& Interno \\
				\rowcolor{\tablegray}
				\multirow{-2}{*}{R6F2}
				& UC6.1-GP \\

				& Interno \\
				\multirow{-2}{*}{R6.2F2}
				& UC6.2-GP \\

				\rowcolor{\tablegray}
				& Interno \\
				\rowcolor{\tablegray}
				\multirow{-2}{*}{R7F1}
				& UC6.3-GP \\

				& Interno \\
				\multirow{-2}{*}{R8F2}
				& UC7-GP \\

				\rowcolor{\tablegray}
				& Interno \\
				\rowcolor{\tablegray}
				\multirow{-2}{*}{R8.1F2}
				& UC7.1-GP \\

				& Interno \\
				\multirow{-2}{*}{R8.2F2}
				& UC7.2-GP \\

				\rowcolor{\tablegray}
				& Interno \\
				\rowcolor{\tablegray}
				\multirow{-2}{*}{R8.2.1F2}
				& UC7.2.1-GP \\

				\bottomrule
			\end{oldtabularx}}
			\caption{Elenco dei requisiti funzionali in rapporto alle fonti (1)}
		\end{table}



		\begin{table}[H]
			\centering
			{\def\arraystretch{1.6}
			\begin{oldtabularx}{0.7\textwidth}{YY}
				\textbf{Requisito} & \textbf{Fonte} \\
				\toprule
				
				\rowcolor{\tablegray}
				& Interno \\
				\rowcolor{\tablegray}
				\multirow{-2}{*}{R8.2.2F2}
				& UC7.2.2-GP \\
				
				& Interno \\
				\multirow{-2}{*}{R9F1}
				& UC7.2.3-GP \\
				
				\rowcolor{\tablegray}
				& Interno \\
				\rowcolor{\tablegray}
				\multirow{-2}{*}{R10F2}
				& UC8-CT \\
				
				& Interno \\
				\multirow{-2}{*}{R11F2}
				& UC9-CE \\
				
				\rowcolor{\tablegray}
				& Interno \\
				\rowcolor{\tablegray}
				\multirow{-2}{*}{R12F2}
				& UC10-BT \\
				
				& Interno \\
				\multirow{-2}{*}{R13F2}
				& UC11-SE \\
				
				\rowcolor{\tablegray}
				& Interno \\
				\rowcolor{\tablegray}
				\multirow{-2}{*}{R14F0}
				& UC12.1-GP \\
				
				& Interno \\
				\multirow{-2}{*}{R14.1F0}
				& UC12.1.1 \\
				
				\rowcolor{\tablegray}
				& Interno \\
				\rowcolor{\tablegray}
				\multirow{-2}{*}{R15F0}
				& UC12.1.2 \\
				
				& Interno \\
				\multirow{-2}{*}{R16F0}
				& UC13-GP \\
				
				\rowcolor{\tablegray}
				& Interno \\
				\rowcolor{\tablegray}
				\multirow{-2}{*}{R17F0}
				& UC14-GP \\
				
				& Interno \\
				\multirow{-2}{*}{R17.1F0}
				& UC14.1.1-GP \\
				
				\rowcolor{\tablegray}
				& Interno \\
				\rowcolor{\tablegray}
				\multirow{-2}{*}{R17.2F0}
				& UC14.1.2-GP \\

				\bottomrule
			\end{oldtabularx}}
			\caption{Elenco dei requisiti funzionali in rapporto alle fonti (2)}
		\end{table}


		\begin{table}[H]
			\centering
			{\def\arraystretch{1.6}
			\begin{oldtabularx}{0.7\textwidth}{YY}
				\textbf{Requisito} & \textbf{Fonte} \\
				\toprule
				
				\rowcolor{\tablegray}
				& Interno \\
				\rowcolor{\tablegray}
				\multirow{-2}{*}{R17.3F0}
				& UC14.1.3-GP \\
				
				& Interno \\
				\multirow{-2}{*}{R17.3.1F0}
				& UC14.2-GP \\
				
				\rowcolor{\tablegray}
				& Interno \\
				\rowcolor{\tablegray}
				\multirow{-2}{*}{R17.4F0}
				& UC14.1.4-GP \\
				
				& Interno \\
				\multirow{-2}{*}{R17.4.1F0}
				& UC14.2-GP \\
				
				\rowcolor{\tablegray}
				& Interno \\
				\rowcolor{\tablegray}
				\multirow{-2}{*}{R18F0}
				& UC15-GP \\
			
				& Interno \\
				\multirow{-2}{*}{R18.1F0}
				& UC15.1.1-GP \\
				
				\rowcolor{\tablegray}
				& Interno \\
				\rowcolor{\tablegray}
				\multirow{-2}{*}{R18.2F0}
				& UC15.1.2-GP \\
				
				& Interno \\
				\multirow{-2}{*}{R18.3F0}
				& UC15.2-GP \\
				
				\rowcolor{\tablegray}
				& Interno \\
				\rowcolor{\tablegray}
				\multirow{-2}{*}{R18.4F0}
				& UC15.3-GP \\
				
				& Interno \\
				\multirow{-2}{*}{R19F0}
				& UC16-GP \\
				
				\rowcolor{\tablegray}
				& Interno \\
				\rowcolor{\tablegray}
				\multirow{-2}{*}{R19.1F0}
				& UC16.1-GP \\
				
				& Interno \\
				\multirow{-2}{*}{R19.1.1F0}
				& UC16.2-GP \\
				
				\rowcolor{\tablegray}
				& Interno \\
				\rowcolor{\tablegray}
				\multirow{-2}{*}{R19.2F0}
				& UC16.1.1.1-GP \\
				
				\bottomrule
			\end{oldtabularx}}
			\caption{Elenco dei requisiti funzionali in rapporto alle fonti (3)}
		\end{table}


		\begin{table}[H]
			\centering
			{\def\arraystretch{1.6}
			\begin{oldtabularx}{0.7\textwidth}{YY}
				\textbf{Requisito} & \textbf{Fonte} \\
				\toprule
				
				\rowcolor{\tablegray}
				& Interno \\
				\rowcolor{\tablegray}
				\multirow{-2}{*}{R19.3F0}
				& UC16.1.1.2-GP \\
				
				& Interno \\
				\multirow{-2}{*}{R19.4F0}
				& UC16.1.1.3-GP \\
				
				\rowcolor{\tablegray}
				& Interno \\
				\rowcolor{\tablegray}
				\multirow{-2}{*}{R19.4.1F0}
				& UC16.1.2-GP \\
				
				& Interno \\
				\multirow{-2}{*}{R19.5F0}
				& UC16.1.1.4-GP \\
				
				\rowcolor{\tablegray}
				& Interno \\
				\rowcolor{\tablegray}
				\multirow{-2}{*}{R19.5.1F0}
				& UC16.1.2-GP \\
				
				& Interno \\
				\multirow{-2}{*}{R20F0}
				& UC17-GP \\
				
				\rowcolor{\tablegray}
				& Interno \\
				\rowcolor{\tablegray}
				\multirow{-2}{*}{R20.1F0}
				& UC17.1-GP \\
				
				& Interno \\
				\multirow{-2}{*}{R20.2F0}
				& UC17.2-GP \\
				
				\rowcolor{\tablegray}
				& Interno \\
				\rowcolor{\tablegray}
				\multirow{-2}{*}{R20.3F0}
				& UC17.3-GP \\
				
				& Interno \\
				\multirow{-2}{*}{R20.4F0}
				& UC17.4-GP \\
				
				\rowcolor{\tablegray}
				& Interno \\
				\rowcolor{\tablegray}
				\multirow{-2}{*}{R20.4.1F0}
				& UC17.5-GP \\
				
				& Interno \\
				\multirow{-2}{*}{R20.4.2F0}
				& UC17.5-GP \\
				
				\rowcolor{\tablegray}
				& Interno \\
				\rowcolor{\tablegray}
				\multirow{-2}{*}{R20.5F0}
				& UC17.6-GP \\
				
				\bottomrule
			\end{oldtabularx}}
			\caption{Elenco dei requisiti funzionali in rapporto alle fonti (4)}
		\end{table}


		\begin{table}[H]
			\centering
			{\def\arraystretch{1.6}
			\begin{oldtabularx}{0.7\textwidth}{YY}
				\textbf{Requisito} & \textbf{Fonte} \\
				\toprule
				
				\rowcolor{\tablegray}
				& Interno \\
				\rowcolor{\tablegray}
				\multirow{-2}{*}{R20.5.1F0}
				& UC17.7-GP \\
				
				& Interno \\
				\multirow{-2}{*}{R21F0}
				& UC18-GP \\
				
				\rowcolor{\tablegray}
				& Interno \\
				\rowcolor{\tablegray}
				\multirow{-2}{*}{R21.1F0}
				& UC18.1-GP \\
				
				& Interno \\
				\multirow{-2}{*}{R21.2F0}
				& UC18.2-GP \\
				
				\rowcolor{\tablegray}
				& Interno \\
				\rowcolor{\tablegray}
				\multirow{-2}{*}{R21.3F0}
				& UC18.3-GP \\
				
				& Interno \\
				\multirow{-2}{*}{R21.4F0}
				& UC18.4-GP \\
				
				\rowcolor{\tablegray}
				& Interno \\
				\rowcolor{\tablegray}
				\multirow{-2}{*}{R21.4.1F0}
				& UC18.5-GP \\
				
				& Interno \\
				\multirow{-2}{*}{R21.4.2F0}
				& UC18.5-GP \\
				
				\rowcolor{\tablegray}
				& Interno \\
				\rowcolor{\tablegray}
				\multirow{-2}{*}{R21.5F0}
				& UC18.6-GP \\
				
				& Interno \\
				\multirow{-2}{*}{R21.5.1F0}
				& UC18.7-GP \\
				
				\rowcolor{\tablegray} R22F2 & Capitolato \\
				R23F2 & Capitolato \\
				\rowcolor{\tablegray} R24F2 & Capitolato \\
				R25F2 & Capitolato \\
				\rowcolor{\tablegray} R26F2 & Capitolato \\
				R27F2 & Capitolato \\
				\bottomrule
			\end{oldtabularx}}
			\caption{Elenco dei requisiti funzionali in rapporto alle fonti (5)}
		\end{table}


		\begin{table}[H]
			\centering
			{\def\arraystretch{1.6}
			\begin{oldtabularx}{0.7\textwidth}{YY}
				\textbf{Requisito} & \textbf{Fonte} \\
				\toprule
				
				\rowcolor{\tablegray}
				& Interno \\
				\rowcolor{\tablegray}
				\multirow{-2}{*}{R1Q1}
				& QPR001 \\
				
				& Interno \\
				\multirow{-2}{*}{R2Q1}
				& QPD001 \\
				
				\rowcolor{\tablegray}
				& Interno \\
				\rowcolor{\tablegray}
				\multirow{-2}{*}{R3Q1}
				& QPR004 \\
				
				& Interno \\
				\multirow{-2}{*}{R4Q1}
				& QPR006 \\
				
				\rowcolor{\tablegray}
				& Interno \\
				\rowcolor{\tablegray}
				\multirow{-2}{*}{R5Q1}
				& QPR007 \\
				
				& Interno \\
				\multirow{-2}{*}{R6Q1}
				& QPR008 \\
				
				\rowcolor{\tablegray}
				& Interno \\
				\rowcolor{\tablegray}
				\multirow{-2}{*}{R7Q1}
				& QPR009 \\
				
				R8Q2 & Interno \\
				\rowcolor{\tablegray} R9Q2 & Interno \\
				
				\multirow{2}{*}{R10Q2}
				& Capitolato \\
				& \Doc{VE\_2018-12-12} \\
				
				\rowcolor{\tablegray} R11Q2 & Capitolato \\
				R12Q2 & Capitolato \\
				\rowcolor{\tablegray} R12.1Q2 & Capitolato \\
				R13Q2 & Capitolato \\
				\rowcolor{\tablegray} R13.1Q2 & Capitolato \\
				R13.2Q2 & Capitolato \\
				\rowcolor{\tablegray} R13.3Q2 & Capitolato \\
				R14Q1 & Capitolato \\
				\rowcolor{\tablegray} R14.1Q2 & Capitolato \\
				
				\bottomrule
			\end{oldtabularx}}
			\caption{Elenco dei requisiti di qualità in rapporto alle fonti (1)}
		\end{table}


		\begin{table}[H]
			\centering
			{\def\arraystretch{1.6}
			\begin{oldtabularx}{0.7\textwidth}{YY}
				\textbf{Requisito} & \textbf{Fonte} \\
				\toprule
				\rowcolor{\tablegray} R15Q2 & Capitolato \\
				R15.1Q2 & Capitolato \\
				\rowcolor{\tablegray} R15.2Q2 & Capitolato \\
				R15.3Q2 & Capitolato \\		
				\rowcolor{\tablegray} R15.3.1F2 & Capitolato \\
				R15.3.2F2 & Capitolato \\
				\bottomrule
			\end{oldtabularx}}
			\caption{Elenco dei requisiti di qualità in rapporto alle fonti (2)}
		\end{table}



		\begin{table}[H]
			\centering
			{\def\arraystretch{1.6}
			\begin{oldtabularx}{0.7\textwidth}{YY}
				\textbf{Requisito} & \textbf{Fonte} \\
				\toprule
				\rowcolor{\tablegray} R1V2 & Capitolato \\
				R1.1V0 & Capitolato \\
				\rowcolor{\tablegray} R2V2 & Capitolato \\
				R2.1V0 & Capitolato \\
				\rowcolor{\tablegray} R3V2 & Capitolato \\
				R3.1V2 & Capitolato \\
				\rowcolor{\tablegray} R4V1 & Capitolato \\
				R5V1 & Capitolato \\
				\rowcolor{\tablegray} R6V0 & Interno \\
				R7V0 & Interno \\
				\rowcolor{\tablegray} R7V0 & Interno \\
				R9V0 & Interno \\
				\rowcolor{\tablegray} R10V0 & Interno \\
				\bottomrule
			\end{oldtabularx}}
			\caption{Elenco dei requisiti di vincolo in rapporto alle fonti}
		\end{table}




% 		\begin{table}[H]
% 		\begin{paddedtablex}[1.7]{\textwidth}{YY}
% 			\textbf{Requisito} & \textbf{Fonte} \\\toprule
% 			%R di Q
% 			R\addVC
% 			Q1 & Interno QPR001 \\
% 			R\addVC
% 			Q1 & Interno QPD001 \\
% 			R\addVC
% 			Q1 & Interno QPR002 \\
% 			R\addVC
% 			Q1 & Interno QPR003 \\
% 			R\addVC
% 			Q1 & Interno QPR004 \\
% 			R\addVC
% 			Q2 & Interno QPR005 \\
% 			R\addVC
% 			Q1 & Interno QPR006 \\
% 			R\addVC
% 			Q1 & Interno QPR007 \\
% 			R\addVC
% 			Q1 & Interno QPR008 \\
% 			R\addVC
% 			Q1 & Interno QPR009 \\
% 			R\addVC
% 			Q2 & Interno QPR010 \\
% 			R\addVC
% 			Q2 & Interno \\
% 			R\addVC
% 			Q2 & Interno \\
% 			R\addVC
% 			Q2 & \Doc{VE\_2018-12-12} \\
% 			R\addVC
% 			Q2 & Capitolato \\
% 			R\addVC
% 			Q2 & Capitolato \\
% 			R\addVC
% 			Q2 & Capitolato \\
% 			R\addVC
% 			Q2 & Capitolato \\
% 			R18.1Q2 & Capitolato \\
% 			R18.2Q2 & Capitolato \\
% 			R18.3Q2 & Capitolato \\
% 			\bottomrule
% 			\end{paddedtablex}
% 		\caption{Elenco dei requisiti di qualità in rapporto alle fonti}
% 	\end{table}

% \begin{table}[H]
% 	\begin{paddedtablex}[1.7]{\textwidth}{YY}
% 		\textbf{Requisito} & \textbf{Fonte} \\\toprule
% 			%R di V
% 			R\addX
% 			V2 & Capitolato \\
% 			R1.1V0 & Capitolato \\
% 			R\addX
% 			V2 & Capitolato \\
% 			R2.1V0 & Capitolato \\
% 			R\addX
% 			V2 & Capitolato \\
% 			R3.1V2 & Capitolato \\
% 			R\addX
% 			V1 & Capitolato \\
% 			R4.1V2 & Capitolato \\
% 			R\addX
% 			V2 & Capitolato \\
% 			R5.1V2 & Capitolato \\
% 			R5.2V2 & Capitolato \\
% 			R5.3V2 & Capitolato \\
% 			R5.4V2 & Capitolato \\
% 			R5.5V2 & Capitolato \\
% 			R\addX
% 			V2 & Capitolato \\
% 			R\addX
% 			V0 & Capitolato \\
% 			R\addX
% 			V1 & Capitolato \\
% 			R\addX
% 			V1 & Capitolato \\
% 			\bottomrule \\
% 		\end{paddedtablex}
% 		\caption{Elenco dei requisiti di vincolo in rapporto alle fonti}
% 	\end{table}

	\subsection{Riepilogo}\label{Riepilogo}

		\begin{table}[H]
		\centering
		{\def\arraystretch{1.7}
		\begin{oldtabularx}{\textwidth}{YYYY}
			\textbf{Tipologia} & \textbf{Obbligatori} & \textbf{Desiderabili} & \textbf{Opzionali} \\\toprule
			\rowcolor{\tablegray} Di funzionalità & 23 & 2 & 43 \\
			Di qualità & 17 & 8 & 0 \\
			\rowcolor{\tablegray} Di vincolo & 4 & 2 & 7
			\\\bottomrule
		\end{oldtabularx}}
		\caption{Riepilogo dei requisiti}
		\end{table}

