\section{Requisiti}
Ad ogni requisito viene assegnato il codice identificativo univoco:
	\begin{center}
		\texttt{R[Numero][Tipo][Priorità]} 
	\end{center} 
	in cui ogni parte ha un significato preciso:
	\begin{itemize}
		\item \textbf{R}: requisito.
		\item \textbf{Numero}: numero progressivo.
		\item \textbf{Tipo}: la la tipologia di requisito che può essere di:
		\begin{itemize}
			\item \textbf{F}: funzionalità.
			\item \textbf{Q}: qualità.
			\item \textbf{V}: vincolo.
		\end{itemize}
		\item \textbf{Priorità}: indica il grado di urgenza di un requisito di essere soddisfatto, come:
		\begin{itemize}
			\item \textbf{0}: opzionale.
			\item \textbf{1}: desiderabile.
			\item \textbf{2}: obbligatorio.
		\end{itemize}
	\end{itemize}
	
	Esempio: \texttt{R02Q1} indica il secondo requisito di qualità ed è desiderabile.
	
%inserire il fatto che una persona può aggiungere il proprio nickname o altro da interfaccia, come requisito opzionale	
	

% Generazione automatica dei numeri 
\newcounter{vaZ} % valore
\newcommand{\decrZ}{\addtocounter{vaZ}{+1}} % Comando per l'aumento automatico del counter vaZ
\newcommand{\addNumber}[0]{\thevaZ \decrZ} % Comando per generare

\addtocounter{vaZ}{1}

	\subsection{Requisiti di funzionalità}
		\begin{paddedtablex}[1.7]{\textwidth}{cXc}%0 opz  2 obb
			\textbf{Codice} & \textbf{Requisito} & \textbf{Fonte} \\\toprule
			
			R\addNumber
			F2 & Redmine e GitLab deve essere in grado di generare delle segnalazioni da inviare ai Producer & Interno UC1.1 \\
			R\addNumber
			F0 & Se avviene un errore nella trasmissione di una segnalazione, ne vengono salvate le informazioni relative & Interno UC1.2 \\
			R\addNumber
			F2 & Il Producer deve essere in grado di inviare delle segnalazioni al Broker & Interno UC2 \\
			R\addNumber
			F2 & Il Consumer deve essere in grado di richiedere al Broker l'invio di un messaggio & Interno UC3 \\
			R\addNumber
			F2 & I Consumer devono essere in grado di inviare delle segnalazioni al server mail e Telegram & Interno UC4 \\
			R\addNumber
			F0 & L'utente può eseguire l'accesso al gestore personale & Interno UC5.1 \\
			R\addNumber
			F0 & L'utente può inserire il proprio username che lo identifica all'interno del sistema & Interno UC5.1.1 \\
			R\addNumber
			F0 & Il sistema fa apparire un messaggio di errore se il tentativo di accesso non è andato a buon fine & Interno UC5.2 \\
			R\addNumber
			F0 & L'utente può iscriversi a un topic & Interno UC6.1.1	\\
			R\addNumber
			F0 & L'utente può disiscriversi a un topic & Interno UC6.2.1 \\
			R\addNumber
			F0 & L'utente può aggiungere nel gestore personale i giorni in cui non è reperibile & Interno UC6.1.2 \\
			R\addNumber
			F0 & L'utente può rimuovere i giorni in cui ha dato la sua irreperibilità & Interno UC6.2.2 \\
			R\addNumber
			F0 & L'utente può scegliere la piattaforma di messaggistica in cui ricevere le notifiche tra Telegram e Mail 
			%e Slack
			& Interno UC6.1.3 \\
			R\addNumber
			F0 & L'utente può togliere le proprie preferenze sulle piattaforme di messaggistica & Interno UC6.2.3
			\\\bottomrule
		\end{paddedtablex}
	
	% Generazione automatica dei numeri 
	\newcounter{vaQ} % valore
	\newcommand{\decrQ}{\addtocounter{vaQ}{+1}} % Comando per l'aumento automatico del counter vaZ
	\newcommand{\addQNumber}[0]{\thevaQ \decrQ} % Comando per generare
	\addtocounter{vaQ}{0}
	
	%TODO: I test sono requisiti di qualità?
	\subsection{Requisiti di qualità}
		\begin{paddedtablex}[1.7]{\textwidth}{cXc}
			\textbf{Codice} & \textbf{Requisito} & \textbf{Fonte} \\
			\toprule
			R\addQNumber
			Q1  & L'\gloss{indice di Gulpease} di ogni documento ha indice tra 50 e 60 & Interno \\
			R\addQNumber
			Q2 & Le applicazioni sviluppate devono rispettare (quasi tutti) i fattori trattati in \gloss{The Twelve-Factor App} & Capitolato Verbale....	\\ %TODO:inserire verbale in cui ci siamo trovati con Imola
			R\addQNumber
			Q2 & È necessario fornire test unitari per ogni componente applicativa & Capitolato \\
			R\addQNumber
			Q2 & È necessario fornire test d'integrazione per ogni componente applicativa & Capitolato \\
			R\addQNumber
			Q2 & È necessario testare interamente il sistema con test di sistema & Capitolato \\
			R\addQNumber
			Q2 & È necessario presentare il bug reporting per ogni componente & Capitolato \\
			R\addQNumber
			Q2 & Deve essere redatta documentazione sulle scelte implementative effettuate & Capitolato \\
			R\addQNumber
			Q2 & Deve essere redatta documentazione sulle scelte progettuali effettuate & Capitolato \\
			R\addQNumber
			Q2 & Ogni scelta descritta nella documentazione deve essere correlata dalle proprie motivazioni & Capitolato \\
			R\addQNumber
			Q2 & Tutte le norme inserite in \NdPv\ devono essere rispettate & Interno \\
			R\addQNumber
			Q2 & Tutti i vincoli presenti nel \PdQv\ devono essere rispettare & Interno \\
			
			\\\bottomrule
		\end{paddedtablex}
	
	% Generazione automatica dei numeri 
	\newcounter{vaV} % valore
	\newcommand{\decrV}{\addtocounter{vaV}{+1}} % Comando per l'aumento automatico del counter vaZ
	\newcommand{\addVNumber}[0]{\thevaV \decrV} % Comando per generare
	\addtocounter{vaV}{0}
	
	%TODO: I requisiti di README vanno in vincolo o qualità? E quelli dei problemi?
	\subsection{Requisiti di vincolo}
		\begin{paddedtablex}[1.7]{\textwidth}{cXc}
			\textbf{Codice} & \textbf{Requisito} & \textbf{Fonte} \\
			\toprule
			R\addVNumber
			V2 & Le componenti Producer devono riuscire a recuperare le segnalazioni & Capitolato \\ % stesso di "L'applicativo Consumer/Producer deve riuscire a recuperare i messaggi da un \gloss{Topic}"
			R\addVNumber
			V2 & Le componenti Producer devono riuscire a pubblicare le segnalazioni recuperate sotto forma di messaggi nel giusto Topic & Capitolato \\
			R\addVNumber
			V2 & Le segnalazioni devono poter essere gestite in maniera automatica e personalizzabile & Capitolato \\
			R\addVNumber
			V2 & Le segnalazioni devono poter essere accentrate e standardizzate & Capitolato \\
			R\addVNumber
			V2 & Le componenti devono esporre delle \gloss{API Rest} per le interazioni con le altre componenti & Capitolato \\
			R\addVNumber
			V2 & Nel sistema deve essere presente un Broker che instanzia e gestisce i Topic & Capitolato \\
			R\addVNumber
			V2 & Le componenti Consumer devono essere in grado di abbonarsi ai Topic più adeguati & Capitolato \\
			R\addVNumber
			V2 & Devono esserci due componenti applicativi Producer tra Redmine, GitLab e Sonarqube & Capitolato \\
			R\addVNumber
			V0 & È possibile avere un terzo componente applicativo Producer oltre ai due obbligatori &  Capitolato \\
			R\addVNumber
			V2 & Devono esserci due componenti applicativi Consumer tra Telegram, Email e Slack & Capitolato \\
			R\addVNumber
			V0 & È possibile avere un terzo componente applicativo Consumer oltre ai due obbligatori & Capitolato \\
			R\addVNumber
			V0 & Come meccanismo di estensione per GitHub è possibile fare uso di \gloss{web hook} & Capitolato \\
			R\addVNumber
			V1 & Per lo sviluppo dei componenti applicativi è possibile usare come linguaggio Java(8 o più), Python o Node.js & Capitolato \\
			R\addVNumber
			V1 & È possibile utilizzare Apache Kafka come Broker & Capitolato \\
			R\addVNumber
			V2 & Docker deve essere la tecnologia di riferimento per l'istanziazione di tutte le componenti & Capitolato \\
			R\addVNumber
			V2 & Deve essere redatta documentazione su eventuali problemi riscontrati ancora aperti & Capitolato \\
			R\addVNumber
			V1 & Per ogni problema aperto documentato è possibile allegare delle soluzioni da attuare in futuro & Capitolato \\
			
			\\\bottomrule
		\end{paddedtablex}
			
			%R19V
			
		\begin{paddedtablex}[1.7]{\textwidth}{cXc}
			R\addVNumber
			V2 & Le componenti Consumer devono essere in grado di recuperare i messaggi da un Topic & Capitolato \\ 
			R\addVNumber
			V2 & Le componenti Consumer devono essere in grado di inviare i messaggi provenienti da un Topic verso il corretto destinatario & Capitolato \\ %stesso di "L'applicativo Producer/Consumer deve essere in grado di reindirizzare i messaggi verso la persona più appropriata"
			R\addVNumber
			V2 & È necessario presentare codice sorgente versionato su repository pubblico, con una delle licenze OpenSource proposte, per ogni componente & Capitolato \\
			R\addVNumber
			V2 & È necessario presentare un DockerFile per ogni componente & Capitolato \\
			R\addVNumber
			V2 & È necessario presentare un file \gloss{README} per ogni componente & Capitolato \\
			R\addVNumber
			V2 & I file README delle componenti applicative devono contenere la documentazione delle API esposte dal servizio & Capitolato \\
			R\addVNumber
			V2 & I file README delle componenti applicative devono contenere le istruzioni per il loro utilizzo & Capitolato \\
			%TODO: R.. & I DockerFile devono essere comprensivi di script e file cfg..ecc?
			R\addVNumber
			V2 & È necessario presentare un file README per il DockerFile & Capitolato \\
			R\addVNumber
			V2 & Il file README per il DockerFile deve contenere le istruzioni per l'avvio & Capitolato \\
			R\addVNumber
			V2 & Il file README per il DockerFile deve contenere la documentazione delle configurazioni custom scelte & Capitolato \\
			
			
			\\\bottomrule
		\end{paddedtablex}
	
			
	
	\subsection{Tracciamento}
	
		\subsubsection{Tracciamento fonti-requisiti}
		
		\begin{paddedtablex}[1.7]{\textwidth}{XX}
			\textbf{Fonte} & \textbf{Requisito} \\\toprule
			Interno & R1F0 \\
			
			\\\bottomrule
		\end{paddedtablex}
		
		
		\subsubsection{Tracciamento requisiti-fonte}
		
		\begin{paddedtablex}[1.7]{\textwidth}{XX}
			\textbf{Requisito} & \textbf{Fonte} \\\toprule
			R1F0 & Interno \\
			
			\\\bottomrule
		\end{paddedtablex}
		
		
		
	\subsection{Riepilogo}
	
		\begin{paddedtablex}[1.7]{\textwidth}{XXXX}
			\textbf{Tipologia} & \textbf{Obbligatori} & \textbf{Desiderabili} & \textbf{Opzionali} \\\toprule
			Di funzionalità & 0 & 0 & 0 \\
			Di qualità & 0 & 0 & 0 \\
			Di vincolo & 0 & 0 & 0 			
			\\\bottomrule
		\end{paddedtablex}
		
	