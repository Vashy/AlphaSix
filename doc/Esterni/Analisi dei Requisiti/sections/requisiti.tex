\section{Requisiti}
Ad ogni requisito viene assegnato il codice identificativo univoco:
	\begin{center}
		\texttt{R[Numero][Tipo][Priorità]} 
	\end{center} 
	in cui ogni parte ha un significato preciso:
	\begin{itemize}
		\item \textbf{R}: requisito.
		\item \textbf{Numero}: numero progressivo in 3 cifre.
		\item \textbf{Tipo}: la la tipologia di requisito che può essere di:
		\begin{itemize}
			\item \textbf{F}: funzionalità.
			\item \textbf{Q}: qualità.
			\item \textbf{V}: vincolo.
		\end{itemize}
		\item \textbf{Priorità}: indica il grado di urgenza di un requisito di essere soddisfatto, come:
		\begin{itemize}
			\item \textbf{0}: opzionale.
			\item \textbf{1}: desiderabile.
			\item \textbf{2}: obbligatorio.
		\end{itemize}
	\end{itemize}
	
	Esempio: \texttt{R002Q1} indica il secondo requisito di qualità ed è desiderabile.
	

\begin{center}
	\texttt{R[ID][Tipo][Priorità]}
\end{center}

La prima lettera indica che si tratta di un requisito, il quale possiede un numero incrementale indicato nell'ID, una tipologia e una priorità.

La tipologia è indicata con:

\begin{itemize}
	\item \textbf{F}: requisito funzionale.
	\item \textbf{Q}: requisito di qualità.
	\item \textbf{V}: requisito vincolante.
\end{itemize}

La priorità è indicata con:

\begin{itemize}
	\item \textbf{0}: opzionale.
	\item \textbf{1}: desiderabile.
	\item \textbf{2}: obbligatorio.
\end{itemize}

	%TODO: fare una roba tipo setcounter per il numero progressivo del codice dei requisiti
	\subsection{Requisiti di funzionalità}
		\begin{paddedtablex}[1.7]{\textwidth}{cXc}%0 opz  2 obb
			\textbf{Codice} & \textbf{Requisito} & \textbf{Fonte} \\\toprule
			R1F0 & L'utente può autenticarsi & Interno UC2 \\
			R2F2 & Lo stato dell'issue tracking system può cambiare & Interno UC1 \\
			R3F0 & Il sistema fa apparire un messaggio di errore se l'autenticazione non è andata a buon fine & Interno UC2 \\
			R4F0 & L'utente può iscriversi a un topic & Interno UC3.1.1	\\
			R5F0 & L'utente può selezionare i giorni in cui non è disponibile  & Interno UC3.1.2 \\
			R6F0 & L'utente può scegliere la piattaforma di messaggistica preferita tra Telegram e Mail 
			%e Slack
			& Interno UC3.1.3 \\
			R7F0 & L'utente può disiscriversi a un topic & Interno UC3.2.1 \\
			R8F0 & L'utente può modificare i giorni in cui è disponibile & Interno UC3.2.2  \\
			R9F0 & L'utente può modificare la piattaforma di messaggistica & Interno UC3.2.3
			\\\bottomrule
		\end{paddedtablex}
	
	
	%TODO: I test sono requisiti di qualità?
	\subsection{Requisiti di qualità}
		\begin{paddedtablex}[1.7]{\textwidth}{cXc}
			\textbf{Codice} & \textbf{Requisito} & \textbf{Fonte} \\
			\toprule
			R1Q1  & L'\gloss{indice di Gulpease} di ogni documento ha indice circa 55 & Interno \\
			R2Q2 & Le applicazioni sviluppate devono rispettare (quasi tutti) i fattori trattati in \gloss{The Twelve-Factor App} & Capitolato Verbale....	\\ %TODO:inserire verbale in cui ci siamo trovati con Imola
			R3Q2 & È necessario fornire test unitari per ogni componente applicativa & Capitolato \\
			R4Q2 & È necessario fornire test d'integrazione per ogni componente applicativa & Capitolato \\
			R5Q2 & È necessario testare interamente il sistema con test di sistema & Capitolato \\
			R6Q2 & È necessario presentare il bug reporting per ogni componente & Capitolato \\
			R7Q2 & Deve essere redatta documentazione sulle scelte implementative effettuate & Capitolato \\
			R8Q2 & Deve essere redatta documentazione sulle scelte progettuali effettuate & Capitolato \\
			R9Q2 & Ogni scelta descritta nella documentazione deve essere correlata dalle proprie motivazioni & Capitolato \\
			R10Q2 & Tutte le norme inserite in \NdPv\ devono essere rispettate & Interno \\
			R11Q2 & Tutti i vincoli presenti nel \PdQv\ devono essere rispettare & Interno \\
			
			\\\bottomrule
		\end{paddedtablex}
	
	
	%TODO: I requisiti di README vanno in vincolo o qualità? E quelli dei problemi?
	\subsection{Requisiti di vincolo}
		\begin{paddedtablex}[1.7]{\textwidth}{cXc}
			\textbf{Codice} & \textbf{Requisito} & \textbf{Fonte} \\
			\toprule
			R1V2 & Le componenti Producer devono riuscire a recuperare le segnalazioni & Capitolato \\ % stesso di "L'applicativo Consumer/Producer deve riuscire a recuperare i messaggi da un \gloss{Topic}"
			R2V2 & Le componenti Producer devono riuscire a pubblicare le segnalazioni recuperate sotto forma di messaggi nel giusto Topic & Capitolato \\
			R3V2 & Le segnalazioni devono poter essere gestite in maniera automatica e personalizzabile & Capitolato \\
			R4V2 & Le segnalazioni devono poter essere accentrate e standardizzate & Capitolato \\
			R5V2 & Le componenti devono esporre delle \gloss{API Rest} per le interazioni con le altre componenti & Capitolato \\
			R6V2 & Nel sistema deve essere presente un Broker che instanzia e gestisce i Topic & Capitolato \\
			R7V2 & Le componenti Consumer devono essere in grado di abbonarsi ai Topic più adeguati & Capitolato \\
			R8V2 & Devono esserci due componenti applicativi Producer tra Redmine, GitLab e Sonarqube & Capitolato \\
			R9V0 & È possibile avere un terzo componente applicativo Producer oltre ai due obbligatori &  Capitolato \\
			R10V2 & Devono esserci due componenti applicativi Consumer tra Telegram, Email e Slack & Capitolato \\
			R11V0 & È possibile avere un terzo componente applicativo Consumer oltre ai due obbligatori & Capitolato \\
			R12V0 & Come meccanismo di estensione per GitHub è possibile fare uso di \gloss{web hook} & Capitolato \\
			R13V1 & Per lo sviluppo dei componenti applicativi è possibile usare come linguaggio Java(8 o più), Python o Node.js & Capitolato \\
			
			%R14V
			
			R15V1 & È possibile utilizzare Apache Kafka come Broker & Capitolato \\
			R16V2 & Docker deve essere la tecnologia di riferimento per l'istanziazione di tutte le componenti & Capitolato \\
			R17V2 & Deve essere redatta documentazione su eventuali problemi riscontrati ancora aperti & Capitolato \\
			R18V1 & Per ogni problema aperto documentato è possibile allegare delle soluzioni da attuare in futuro & Capitolato \\
			
			\\\bottomrule
		\end{paddedtablex}
			
			%R19V
			
		\begin{paddedtablex}[1.7]{\textwidth}{cXc}
			R20V2 & Le componenti Consumer devono essere in grado di recuperare i messaggi da un Topic & Capitolato \\ 
			R21V2 & Le componenti Consumer devono essere in grado di inviare i messaggi provenienti da un Topic verso il corretto destinatario & Capitolato \\ %stesso di "L'applicativo Producer/Consumer deve essere in grado di reindirizzare i messaggi verso la persona più appropriata"
			R22V2 & È necessario presentare codice sorgente versionato su repository pubblico, con una delle licenze OpenSource proposte, per ogni componente & Capitolato \\
			
			%R23V
			
			R24V2 & È necessario presentare un DockerFile per ogni componente & Capitolato \\
			R25V2 & È necessario presentare un file \gloss{README} per ogni componente & Capitolato \\
			R26V2 & I file README delle componenti applicative devono contenere la documentazione delle API esposte dal servizio & Capitolato \\
			R27V2 & I file README delle componenti applicative devono contenere le istruzioni per il loro utilizzo & Capitolato \\
			%TODO: R.. & I DockerFile devono essere comprensivi di script e file cfg..ecc?
			R28V2 & È necessario presentare un file README per il DockerFile & Capitolato \\
			R29V2 & Il file README per il DockerFile deve contenere le istruzioni per l'avvio & Capitolato \\
			R30V2 & Il file README per il DockerFile deve contenere la documentazione delle configurazioni custom scelte & Capitolato \\
			
			
			\\\bottomrule
		\end{paddedtablex}
	
			
	
	\subsection{Tracciamento}
	
		\subsubsection{Tracciamento fonti-requisiti}
		
		\begin{paddedtablex}[1.7]{\textwidth}{XX}
			\textbf{Fonte} & \textbf{Requisito} \\\toprule
			Interno & R1F0 \\
			
			\\\bottomrule
		\end{paddedtablex}
		
		
		\subsubsection{Tracciamento requisiti-fonte}
		
		\begin{paddedtablex}[1.7]{\textwidth}{XX}
			\textbf{Requisito} & \textbf{Fonte} \\\toprule
			R1F0 & Interno \\
			
			\\\bottomrule
		\end{paddedtablex}
		
		
		
	\subsection{Riepilogo}
	
		\begin{paddedtablex}[1.7]{\textwidth}{XXXX}
			\textbf{Tipologia} & \textbf{Obbligatori} & \textbf{Desiderabili} & \textbf{Opzionali} \\\toprule
			Di funzionalità & 0 & 0 & 0 \\
			Di qualità & 0 & 0 & 0 \\
			Di vincolo & 0 & 0 & 0 			
			\\\bottomrule
		\end{paddedtablex}
		
	