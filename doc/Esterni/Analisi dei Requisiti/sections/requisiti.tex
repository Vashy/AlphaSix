\section{Requisiti}
Ad ogni requisito viene assegnato il codice identificativo univoco:
	\begin{center}
		\texttt{R[Numero][Tipo][Priorità]} 
	\end{center} 
	in cui ogni parte ha un significato preciso:
	\begin{itemize}
		\item \textbf{R}: requisito.
		\item \textbf{Numero}: numero progressivo che segue la struttura dei documenti.
		\item \textbf{Tipo}: la la tipologia di requisito che può essere di:
		\begin{itemize}
			\item \textbf{F}: funzionalità.
			\item \textbf{Q}: \gloss{qualità}.
			\item \textbf{V}: vincolo.
		\end{itemize}
		\item \textbf{Priorità}: indica il grado di urgenza di un requisito di essere soddisfatto, come:
		\begin{itemize}
			\item \textbf{0}: opzionale.
			\item \textbf{1}: desiderabile.
			\item \textbf{2}: obbligatorio.
		\end{itemize}
	\end{itemize}
	

	Esempio: \texttt{R2Q1} indica il secondo requisito di qualità ed è desiderabile.\\
	
	%I requisiti di seguito riportati sono elencati in modo tale da seguire la struttura dei documenti. Ovvero, si possono trovare raggruppati i requisiti derivanti dello stesso tipo, ad esempio solo i requisiti di funzionalità dal sesto al diciannovesimo derivano dai casi d'uso.
	%TODO: rifare id requisiti
	
%inserire il fatto che una persona può aggiungere il proprio nickname o altro da interfaccia, come requisito opzionale	...

% Generazione automatica dei numeri 
\newcounter{vaZ} % valore
\newcommand{\decrZ}{\addtocounter{vaZ}{+1}} % Comando per l'aumento automatico del counter vaZ
\newcommand{\addNumber}[0]{\thevaZ \decrZ} % Comando per generare
\addtocounter{vaZ}{1}%far partire il contatore da 1


	\subsection{Requisiti di funzionalità}
	
	\begin{table}[H]
		\begin{paddedtablex}[1.7]{\textwidth}{cXc}%0 opz  2 obb
			\textbf{Codice} & \textbf{Requisito} & \textbf{Fonte} \\\toprule
			
			%da casi d'uso
			R\addNumber.1
			F2 & Redmine e GitLab devono essere in grado di generare delle segnalazioni da inviare ai Producer & Interno UC1.1 \\
			R1.2
			F0 & Se avviene un errore nella trasmissione di una segnalazione, viene creato un log in cui vengono salvate le informazioni relative all'errore & Interno UC1.2 \\
			R\addNumber
			F2 & Il Producer deve essere in grado di inviare delle segnalazioni al Broker & Interno UC2 \\
			R\addNumber
			F2 & Il Consumer deve essere in grado di richiedere al Broker l'invio di un messaggio & Interno UC3 \\
			R\addNumber
			F2 & I Consumer devono essere in grado di inviare delle segnalazioni al server e-mail e a Telegram & Interno UC4 \\
			R\addNumber.1
			F0 & L'utente può eseguire l'accesso al gestore personale & Interno UC5.1 \\
			R5.1.1
			F0 & L'utente può inserire il proprio username che lo identifica all'interno di \progetto & Interno UC5.1.1 \\
			R5.2
			F0 & \progetto\ fa apparire un messaggio di errore se il tentativo di accesso non è andato a buon fine & Interno UC5.2 \\
			R\addNumber.1.1
			F0 & L'utente può iscriversi a un Topic & Interno UC6.1.1	\\
			R6.1.2
			F0 & L'utente può aggiungere nel gestore personale i giorni in cui non è reperibile & Interno UC6.1.2 \\
			R6.1.3
			F0 & L'utente può scegliere la piattaforma di messaggistica, tra Telegram e e-mail, in cui ricevere le notifiche 
			%e Slack
			& Interno UC6.1.3 \\
			R6.2.1
			F0 & L'utente può disiscriversi a un Topic & Interno UC6.2.1 \\
			R6.2.2
			F0 & L'utente può rimuovere i giorni in cui aveva selezionato precedentemente di non essere reperibile & Interno UC6.2.2 \\
			R6.2.3
			F0 & L'utente può togliere le proprie preferenze sulle piattaforme di messaggistica da cui ricevere notifiche inviate attraverso \progetto & Interno UC6.2.3 \\
			\bottomrule\\
		\end{paddedtablex}
		\caption{Elenco dei requisiti di funzionalità (1)}
	\end{table}
			
	\begin{table}[H]		
		\begin{paddedtablex}[1.7]{\textwidth}{cXc}
			\textbf{Codice} & \textbf{Requisito} & \textbf{Fonte} \\\toprule
			%da capitolato
			R\addNumber 
			F2 & Le componenti Consumer devono essere in grado di inviare i messaggi provenienti da un Topic verso il corretto destinatario & Capitolato \\ %stesso di "L'applicativo Producer/Consumer deve essere in grado di reindirizzare i messaggi verso la persona più appropriata"
			R\addNumber
			F2 & Le componenti Consumer devono essere in grado di abbonarsi ai Topic scelti & Capitolato \\
			R\addNumber
			F2 & Le segnalazioni devono poter essere gestite in maniera automatica e personalizzabile & Capitolato \\
			R\addNumber
			F2 & Nel sistema deve essere presente un Broker che istanzia e gestisce le segnalazioni organizzandole per Topic & Capitolato \\
			R\addNumber
			F2 & Le componenti Producer devono riuscire a pubblicare le segnalazioni recuperate sotto forma di messaggi secondo i Topic corretti & Capitolato \\
		\bottomrule
		\end{paddedtablex}
		\caption{Elenco dei requisiti di funzionalità (2)}
	\end{table}
		
	
	% Generazione automatica dei numeri 
	\newcounter{vaQ} % valore
	\newcommand{\decrQ}{\addtocounter{vaQ}{+1}} % Comando per l'aumento automatico del counter vaZ
	\newcommand{\addQNumber}[0]{\thevaQ \decrQ} % Comando per generare
	\addtocounter{vaQ}{1}
	
	%NB: molti requisiti di qualità sono stati tolti perchè non sono veri requisiti del sistema, riguardano noi e il nostro modo di fare, non il sistema
	\subsection{Requisiti di qualità}
	
	\begin{table}[H]
		\begin{paddedtablex}[1.7]{\textwidth}{cXc}
			\textbf{Codice} & \textbf{Requisito} & \textbf{Fonte} \\
			\toprule
			%presi dal PdQ processi
			R\addQNumber
			Q1 & Viene stabilito un numero massimo di giorni di ritardo per la chiusura di una issue & Interno QPR001\\
			%presi dal PdQ prodotti
			R1.1
			Q1  & L'\gloss{indice di Gulpease} di ogni documento dovrebbe rientrare all'interno di un intervallo stabilito & Interno QPD001\\  
			%			R\addQNumber
			%			Q1 & Il diario di un documento dovrebbe essere aggiornato ad ogni modifica effettuata & Interno \\
			%presi dal PdQ processi
			R\addQNumber
			Q1 & I costi previsti dalla \gloss{pianificazione} non dovrebbero variare più di \EUR{200} & Interno QPR002\\ 
			R\addQNumber 
			Q1 & Il livello di maturità dell'\gloss{ISO/IEC 15504} che \gruppo\ si prefigge di raggiungere  deve essere almeno pari al terzo & Interno QPR003\\
			R\addQNumber
			Q1 & Il team \gruppo\ si è messo d'accordo per una frequenza minima di commit effettuati in una settimana & Interno QPR004\\
			R\addQNumber
			Q2 & I requisiti obbligatori devono essere tutti completamente soddisfatti al termine del progetto & Interno QPR005\\
			R\addQNumber
			Q1 & \gruppo\ si prefigge di soddisfare un numero stabilito di requisiti desiderabili & Interno QPR006\\
			R\addQNumber
			Q1 & Nessun rischio non verificato precedentemente dovrebbe accadere nel corso del progetto & Interno QPR007\\
			R\addQNumber
			Q1 & Ogni documento dovrebbe attraversare tutte le fasi previste dal suo \gloss{ciclo di vita} & Interno QPR008\\
			R\addQNumber
			Q1 & Viene stabilito il numero massimo di modifiche che può ricevere un prodotto prima di essere verificato & Interno QPR009\\
			R\addQNumber
			Q1 & La fase di verifica di tutti i vari prodotti dovrebbe sempre essere eseguita in modo corretto & Interno QPR010\\
			R\addQNumber
			Q2 & Tutte le \gloss{norme} inserite nelle \NdP\ devono essere rispettate & Interno \\
			R\addQNumber
			Q2 & Tutti i vincoli presenti nel \PdQ\ devono essere rispettati & Interno \\
			\bottomrule
		\end{paddedtablex}
		\caption{Elenco dei requisiti di qualità (1)}
	\end{table}
		
	\begin{table}[H]
		\begin{paddedtablex}[1.7]{\textwidth}{cXc}
			\textbf{Codice} & \textbf{Requisito} & \textbf{Fonte} \\\toprule
			%presi da capitolato
			R\addQNumber
			Q2 & Le applicazioni sviluppate devono rispettare i fattori trattati in The Twelve-Factor App segnati nel \PdQd & Capitolato \Doc{VE\_12-12-2018}	\\
			R\addQNumber
			Q2 & È necessario presentare il \gloss{bug} reporting per ogni componente & Capitolato \\
			R\addQNumber
			Q2 & Deve essere redatta documentazione sulle scelte progettuali effettuate & Capitolato \\
			R\addQNumber
			Q2 & Ogni scelta descritta nella documentazione deve essere correlata dalle relative motivazioni & Capitolato \\
			R\addQNumber
			Q2 & È necessario testare ogni prodotto software considerando ogni sistema di riferimento e interazione tra le sue parti, perciò con test d'unita, d'integrazione e di sistema & Capitolato \\
			R17.1
			Q2 & È necessario fornire test unitari per ogni componente applicativo & Capitolato \\
			R17.2
			Q2 & È necessario fornire test d'integrazione per ogni componente applicativo & Capitolato \\
			R17.3
			Q2 & È necessario testare interamente il sistema con test di sistema & Capitolato \\
			\bottomrule \\
		\end{paddedtablex}
		\caption{Elenco dei requisiti di qualità (2)}
	\end{table}
	
	% Generazione automatica dei numeri 
	\newcounter{vaV} % valore
	\newcommand{\decrV}{\addtocounter{vaV}{+1}} % Comando per l'aumento automatico del counter vaZ
	\newcommand{\addVNumber}[0]{\thevaV \decrV} % Comando per generare
	\addtocounter{vaV}{1}
	
	%TODO: I requisiti di README vanno in vincolo o qualità? E quelli dei problemi?
	\subsection{Requisiti di vincolo}
			
	\begin{table}[H]
		\begin{paddedtablex}[1.7]{\textwidth}{cXc} %\rowcolors{1}{\tablegray}{\lightgray}
			\textbf{Codice} & \textbf{Requisito} & \textbf{Fonte} \\
			\toprule
			%R\addVNumber
			%V2 & Le componenti Producer devono riuscire a recuperare delle segnalazioni & Capitolato \\ % stesso di "L'applicativo Consumer/Producer deve riuscire a recuperare i messaggi da un \gloss{Topic}"
			%			R\addVNumber
			%			V2 & Le segnalazioni devono poter essere accentrate e standardizzate & Capitolato \\
			R\addVNumber
			V2 & Devono essere sviluppati due componenti applicativi Producer tra Redmine, GitLab e SonarQube & Capitolato \\
			R1.1
			V0 & È possibile avere un terzo componente applicativo Producer oltre ai due obbligatori &  Capitolato \\
			R\addVNumber
			V2 & Devono essere sviluppati due componenti applicativi Consumer tra Telegram, Email e Slack & Capitolato \\
			R2.1
			V0 & È possibile avere un terzo componente applicativo Consumer oltre ai due obbligatori & Capitolato \\
			R\addVNumber
			V2 & Docker deve essere la tecnologia di riferimento per l'istanziazione di tutte le componenti & Capitolato \\
			R3.1
			V2 & È necessario presentare un DockerFile per ogni componente & Capitolato \\
			R\addVNumber
			V1 & Per ogni problema aperto documentato è possibile allegare delle soluzioni da attuare in futuro & Capitolato\\
			R4.1
			V2 & Deve essere redatta una documentazione su eventuali problemi riscontrati rimasti ancora aperti al termine del progetto & Capitolato \\
			\bottomrule\\
		\end{paddedtablex}
		\caption{Elenco dei requisiti di vincolo (1)}
	\end{table}	

	\begin{table}[H]
		\begin{paddedtablex}[1.7]{\textwidth}{cXc}
			\textbf{Codice} & \textbf{Requisito} & \textbf{Fonte} \\
			\toprule
			R\addVNumber
			V2 & È necessario presentare un file \gloss{README} per ogni componente & Capitolato \\
			R5.1
			V2 & I file README delle componenti applicative devono contenere la documentazione delle \gloss{API} esposte dal servizio & Capitolato \\
			R5.2
			V2 & I file README delle componenti applicative devono contenere le istruzioni per il loro utilizzo & Capitolato \\
			%TODO: R.. & I DockerFile devono essere comprensivi di script e file cfg..ecc?
			R5.3
			V2 & È necessario presentare un file README per il DockerFile & Capitolato \\
			R5.4
			V2 & Il file README per il Dockerfile deve contenere le istruzioni per l'avvio & Capitolato \\
			R5.5
			V2 & Il file README per il DockerFile deve contenere la documentazione delle configurazioni custom scelte & Capitolato \\
			R\addVNumber
			V2 & Le componenti devono esporre delle \gloss{API Rest} per le interazioni con le altre componenti & Capitolato \\
			R\addVNumber
			V0 & Come meccanismo di estensione per GitLab è possibile fare uso di Webhooks & Capitolato \\
			R\addVNumber
			V1 & Per lo sviluppo dei componenti applicativi è possibile usare come linguaggio \gloss{Java} 8 o una versione più recente, \gloss{Python} o \gloss{Node.js} & Capitolato \\
			R\addVNumber
			V1 & È possibile utilizzare Apache Kafka come Broker & Capitolato \\
			\bottomrule\\
		\end{paddedtablex}
		\caption{Elenco dei requisiti di vincolo (2)}
	\end{table}	
		
					
	
	\subsection{Tracciamento}
	
		\subsubsection{Tracciamento fonti-requisiti}
			
		\begin{table}[H]
			\centering
			{\def\arraystretch{1.4}
			\begin{tabularx}{\textwidth}{YY}
				\textbf{Fonte} & \textbf{Requisito} \\
				\toprule
				\cellcolor{white} & R7F2 \\
				\cellcolor{white} & R8F2 \\
				\cellcolor{white} & R9F2 \\
				\cellcolor{white} & R10F2 \\ 
				\cellcolor{white} & R11F2 \\
				\cmidrule{2-2}
				\cellcolor{white} & R14Q2 \\
				\cellcolor{white} & R15Q2 \\
				\cellcolor{white} & R16Q2 \\
				\cellcolor{white} & R17Q2 \\
				\cellcolor{white} & R17.1Q2 \\
				\cellcolor{white} & R17.2Q2 \\
				\cellcolor{white} & R17.3Q2 \\
				\cmidrule{2-2}
				\cellcolor{white} & R1V2 \\
				\cellcolor{white} & R1.1V0 \\
				\cellcolor{white} & R2V2 \\
				\cellcolor{white} & R2.1V0 \\
				\cellcolor{white} & R3V2 \\
				\cellcolor{white} & R3.1V2 \\
				\cellcolor{white} & R4V1 \\
				\cellcolor{white} & R4.1V2 \\
				\cellcolor{white} & R5V2 \\
				\cellcolor{white} & R5.1V2 \\
				\cellcolor{white} & R5.2V2 \\
				\cellcolor{white} & R5.3V2 \\
				\cellcolor{white} & R5.4V2 \\
				\cellcolor{white} & R5.5V2 \\
				\cellcolor{white} & R6V2 \\
				\cellcolor{white} & R7V0 \\
				\cellcolor{white} & R8V1 \\
				\cellcolor{white} \multirow{-31}{*}{Capitolato} & R9V1 \\
				\bottomrule\\
			\end{tabularx}}
			\caption{Elenco dei requisiti del capitolato}
		\end{table}
	
			\begin{table}[H]
			\centering
			{\def\arraystretch{1.5}
				\begin{tabularx}{\textwidth}{YY}
					\textbf{Fonte} & \textbf{Requisito} \\
					\toprule
					\cellcolor{white} & R11Q2 \\
					\cellcolor{white} & R12Q2 \\
					\cellcolor{white} \multirow{-2}{*}{Interno} & R3Q1 \\
					\bottomrule
				\end{tabularx}}
			\caption{Elenco dei requisiti interni}
		\end{table}

		\begin{table}[H]
			\centering
			\rowcolors{2}{white}{\tablegray}
			{\def\arraystretch{1.5}
			\begin{tabularx}{\textwidth}{YY}
				\textbf{Fonte} & \textbf{Requisito} \\				
				\toprule
				UC1.1 & R1.1F2 \\
				UC1.2 & R1.2F0 \\
				UC2 & R2F2 \\
				UC3 & R3F2 \\
				UC4 & R4F0 \\
				UC5.1 & R5.1F0 \\
				UC5.1.1 & R5.1.1F0 \\
				UC5.2 & R5.2F0 \\
				UC6.1.1 & R6.1.1F0 \\
				UC6.1.2 & R6.1.2F0 \\
				UC6.1.3 & R6.1.3F0 \\
				UC6.2.1 & R6.2.1F0 \\
				UC6.2.2 & R6.2.2F0 \\
				UC6.2.3 & R6.2.3F0 \\
				\bottomrule \\
			\end{tabularx}}
			\caption{Elenco dei requisiti per i casi d'uso}
		\end{table}
	
		\begin{table}[H]
		\centering
		\rowcolors{2}{white}{\tablegray}
		{\def\arraystretch{1.5}
		\begin{tabularx}{\textwidth}{YY}
			\textbf{Fonte} & \textbf{Requisito} \\				
			\toprule
			QPR001 & R1Q1 \\
			QPD001 & R1.1Q1 \\
			QPR002 & R2Q1 \\
			QPR003 & R3Q1 \\
			QPR004 & R4Q1 \\
			QPR005 & R5Q2 \\
			QPR006 & R6Q1 \\
			QPR007 & R7Q1 \\
			QPR008 & R8Q1 \\
			QPR009 & R9Q1 \\
			QPR010 & R10Q1 \\
			\Doc{VE\_12-12-2018} & R13Q2 \\
			\bottomrule\\
		\end{tabularx}}
		\caption{Elenco dei requisiti per gli obiettivi di qualità e verbali}
	\end{table}
		
\newcounter{V} % valore
\newcommand{\deV}{\addtocounter{V}{+1}} % Comando per l'aumento automatico del counter vaZ
\newcommand{\addC}[0]{\theV \deV} % Comando per generare
\addtocounter{V}{1}		

\newcounter{Vv} % valore
\newcommand{\deVv}{\addtocounter{Vv}{+1}} % Comando per l'aumento automatico del counter vaZ
\newcommand{\addVC}[0]{\theVv \deVv} % Comando per generare
\addtocounter{Vv}{1}		

\newcounter{X} % valore
\newcommand{\deX}{\addtocounter{X}{+1}} % Comando per l'aumento automatico del counter vaZ
\newcommand{\addX}[0]{\theX \deX} % Comando per generare
\addtocounter{X}{1}	
		
		\subsubsection{Tracciamento requisiti-fonte}
		
		\begin{table}[H]
		\begin{paddedtablex}[1.7]{\textwidth}{XX}
			\textbf{Requisito} & \textbf{Fonte} \\\toprule
			%R di F
			R\addC.1
			F2 & Interno UC1.1 \\
			R1.2
			F0 & Interno UC1.2 \\
			R\addC
			F2 & Interno UC2 \\
			R\addC
			F2 & Interno UC3 \\
			R\addC
			F0 & Interno UC4 \\
			R\addC.1
			F0 & Interno UC5.1 \\
			R5.1.1
			F0 & Interno UC5.1.1 \\
			R5.2
			F0 & Interno UC5.2 \\
			R\addC.1.1
			F0 & Interno UC6.1.1 \\
			R6.1.2
			F0 & Interno UC6.1.2 \\
			R6.1.3
			F0 & Interno UC6.1.3 \\
			R6.2.1
			F0 & Interno UC6.2.1 \\
			R6.2.3
			F0 & Interno UC6.2.2 \\
			R6.2.3
			F0 & Interno UC6.2.3 \\
			R\addC
			F2 & Capitolato \\
			R\addC
			F2 & Capitolato \\
			R\addC
			F2 & Capitolato \\
			R\addC
			F2 & Capitolato \\
			R\addC
			F2 & Capitolato \\
			\bottomrule
		\end{paddedtablex}
		\caption{Elenco dei requisiti funzionali in rapporto alle fonti}
		\end{table}
			
		\begin{table}[H]	
		\begin{paddedtablex}[1.7]{\textwidth}{XX}
			\textbf{Requisito} & \textbf{Fonte} \\\toprule
			%R di Q
			R\addVC
			Q1 & Interno QPR001 \\
			R1.1
			Q1 & Interno QPD001 \\
			R\addVC
			Q1 & Interno QPR002 \\
			R\addVC
			Q1 & Interno QPR003 \\
			R\addVC
			Q1 & Interno QPR004 \\
			R\addVC
			Q2 & Interno QPR005 \\
			R\addVC
			Q1 & Interno QPR006 \\
			R\addVC
			Q1 & Interno QPR007 \\
			R\addVC
			Q1 & Interno QPR008 \\
			R\addVC
			Q1 & Interno QPR009 \\
			R\addVC
			Q2 & Interno QPR010 \\
			R\addVC
			Q2 & Interno \\
			R\addVC
			Q2 & Interno \\
			R\addVC
			Q2 & \Doc{VE\_12-12-2018} \\
			R\addVC
			Q2 & Capitolato \\
			R\addVC
			Q2 & Capitolato \\
			R\addVC
			Q2 & Capitolato \\
			R\addVC
			Q2 & Capitolato \\
			R17.1
			Q2 & Capitolato \\
			R17.2
			Q2 & Capitolato \\
			R17.3
			Q2 & Capitolato \\
			\bottomrule
			\end{paddedtablex}
		\caption{Elenco dei requisiti di qualità in rapporto alle fonti}
	\end{table}

\begin{table}[H]	
	\begin{paddedtablex}[1.7]{\textwidth}{XX}
		\textbf{Requisito} & \textbf{Fonte} \\\toprule
			%R di V
			R\addX
			V2 & Capitolato \\
			R1.1
			V0 & Capitolato \\
			R\addX
			V2 & Capitolato \\
			R2.1
			V0 & Capitolato \\
			R\addX
			V2 & Capitolato \\
			R3.1
			V2 & Capitolato \\
			R\addX
			V1 & Capitolato \\
			R4.1
			V2 & Capitolato \\
			R\addX
			V2 & Capitolato \\
			R5.1
			V2 & Capitolato \\
			R5.2
			V2 & Capitolato \\
			R5.3
			V2 & Capitolato \\
			R5.4
			V2 & Capitolato \\
			R5.5
			V2 & Capitolato \\
			R\addX
			V2 & Capitolato \\
			R\addX
			V0 & Capitolato \\
			R\addX
			V1 & Capitolato \\
			R\addX
			V1 & Capitolato \\
			\bottomrule \\
		\end{paddedtablex}
		\caption{Elenco dei requisiti di vincolo in rapporto alle fonti}
	\end{table}		
				
	\subsection{Riepilogo}
	
		\begin{table}[H]
		\begin{paddedtablex}[1.7]{\textwidth}{XXXX}
			\textbf{Tipologia} & \textbf{Obbligatori} & \textbf{Desiderabili} & \textbf{Opzionali} \\\toprule
			Di funzionalità & 9 & 0 & 10 \\
			Di qualità & 9 & 1 & 0 \\
			Di vincolo & 12 & 3 & 3 			
			\\\bottomrule
		\end{paddedtablex}
		\caption{Riepilogo dei requisiti}
		\end{table}
		
	