\section{Requisiti}
Ad ogni requisito viene assegnato il codice identificativo univoco:
	\begin{center}
		\texttt{R[Numero][Tipo][Priorità]} 
	\end{center} 
	in cui ogni parte ha un significato preciso:
	\begin{itemize}
		\item \textbf{R}: requisito.
		\item \textbf{Numero}: numero progressivo.
		\item \textbf{Tipo}: la la tipologia di requisito che può essere di:
		\begin{itemize}
			\item \textbf{F}: funzionalità.
			\item \textbf{Q}: qualità.
			\item \textbf{V}: vincolo.
		\end{itemize}
		\item \textbf{Priorità}: indica il grado di urgenza di un requisito di essere soddisfatto, come:
		\begin{itemize}
			\item \textbf{0}: opzionale.
			\item \textbf{1}: desiderabile.
			\item \textbf{2}: obbligatorio.
		\end{itemize}
	\end{itemize}
	

	Esempio: \texttt{R2Q1} indica il secondo requisito di qualità ed è desiderabile.

	
%inserire il fatto che una persona può aggiungere il proprio nickname o altro da interfaccia, come requisito opzionale	...

% Generazione automatica dei numeri 
\newcounter{vaZ} % valore
\newcommand{\decrZ}{\addtocounter{vaZ}{+1}} % Comando per l'aumento automatico del counter vaZ
\newcommand{\addNumber}[0]{\thevaZ \decrZ} % Comando per generare
\addtocounter{vaZ}{1}%far partire il contatore da 1

%TODO: mettere caption alle tabelle
	\subsection{Requisiti di funzionalità}
	
		\begin{paddedtablex}[1.7]{\textwidth}{cXc}%0 opz  2 obb
			\textbf{Codice} & \textbf{Requisito} & \textbf{Fonte} \\\toprule
			%da capitolato
			R\addNumber 
			F2 & Le componenti Consumer devono essere in grado di inviare i messaggi provenienti da un Topic verso il corretto destinatario & Capitolato \\ %stesso di "L'applicativo Producer/Consumer deve essere in grado di reindirizzare i messaggi verso la persona più appropriata"
			R\addNumber
			F2 & Le componenti Producer devono riuscire a pubblicare le segnalazioni recuperate sotto forma di messaggi secondo i Topic corretti & Capitolato \\
			R\addNumber
			F2 & Le segnalazioni devono poter essere gestite in maniera automatica e personalizzabile & Capitolato \\
			R\addNumber
			F2 & Nel sistema deve essere presente un Broker che istanzia e gestisce le segnalazioni organizzandole per Topic & Capitolato \\
			R\addNumber
			F2 & Le componenti Consumer devono essere in grado di abbonarsi ai Topic scelti & Capitolato \\
			%da casi d'uso
			R\addNumber
			F2 & Redmine e GitLab devono essere in grado di generare delle segnalazioni da inviare ai Producer & Interno UC1.1 \\
			R\addNumber
			F0 & Se avviene un errore nella trasmissione di una segnalazione, viene creato un log in cui vengono salvate le informazioni relative all'errore & Interno UC1.2 \\
			R\addNumber
			F2 & Il Producer deve essere in grado di inviare delle segnalazioni al Broker & Interno UC2 \\
			R\addNumber
			F2 & Il Consumer deve essere in grado di richiedere al Broker l'invio di un messaggio & Interno UC3 \\
			R\addNumber
			F2 & I Consumer devono essere in grado di inviare delle segnalazioni al server e-mail e a Telegram & Interno UC4 \\
			R\addNumber
			F0 & L'utente può eseguire l'accesso al gestore personale & Interno UC5.1 \\
			R\addNumber
			F0 & L'utente può inserire il proprio username che lo identifica all'interno di \progetto & Interno UC5.1.1 \\
			R\addNumber
			F0 & \progetto\ fa apparire un messaggio di errore se il tentativo di accesso non è andato a buon fine & Interno UC5.2 \\
			R\addNumber
			F0 & L'utente può iscriversi a un Topic & Interno UC6.1.1	\\
			R\addNumber
			F0 & L'utente può disiscriversi a un Topic & Interno UC6.2.1 \\
			R\addNumber
			F0 & L'utente può aggiungere nel gestore personale i giorni in cui non è reperibile & Interno UC6.1.2 \\
			R\addNumber
			F0 & L'utente può rimuovere i giorni in cui aveva selezionato precedentemente di non essere reperibile & Interno UC6.2.2 \\
			R\addNumber
			F0 & L'utente può scegliere la piattaforma di messaggistica, tra Telegram e e-mail, in cui ricevere le notifiche 
			%e Slack
			& Interno UC6.1.3 \\
		\end{paddedtablex}	
			
	\begin{table}[H]		
		\begin{paddedtablex}[1.7]{\textwidth}{cXc}	
			R\addNumber
			F0 & L'utente può togliere le proprie preferenze sulle piattaforme di messaggistica da cui ricevere notifiche inviate attraverso \progetto & Interno UC6.2.3  \\
		\end{paddedtablex}
		\caption{Elenco dei requisiti di funzionalità.}
	\end{table}
		
	
	% Generazione automatica dei numeri 
	\newcounter{vaQ} % valore
	\newcommand{\decrQ}{\addtocounter{vaQ}{+1}} % Comando per l'aumento automatico del counter vaZ
	\newcommand{\addQNumber}[0]{\thevaQ \decrQ} % Comando per generare
	\addtocounter{vaQ}{1}
	
	%NB: molti requisiti di qualità sono stati tolti perchè non sono veri requisiti del sistema, riguardano noi e il nostro modo di fare, non il sistema
	\subsection{Requisiti di qualità}
	
	\begin{table}[H]
		\begin{paddedtablex}[1.7]{\textwidth}{cXc}
			\textbf{Codice} & \textbf{Requisito} & \textbf{Fonte} \\
			\toprule
			R\addQNumber
			Q2 & Tutte le norme inserite nelle \NdP\ devono essere rispettate & Interno \\
			R\addQNumber
			Q2 & Tutti i vincoli presenti nel \PdQv\ devono essere rispettati & Interno \\
			%presi dal PdQ processi
%			R\addQNumber
%			Q1 & Non dovrebbero essere fatti più di 2 giorni di ritardo di chiusura di una issue & Interno \\
%			R\addQNumber
%			Q1 & I costi previsti dalla pianificazione non dovrebbero variare più di \EUR{200} & Interno \\ 
%			R\addQNumber 
%			Q1 & Il livello di maturità dell'ISO/IEC 15504 che \gruppo\ si prefigge di raggiungere  deve essere almeno pari al terzo & Interno \\
%			R\addQNumber
%			Q1 & La frequenza minima dei commit raggiunti in una settimana dovrebbero essere almeno pari a 25 & Interno \\
%			R\addQNumber
%			Q2 & I requisiti obbligatori devono essere tutti completamente soddisfatti alla terminazione del progetto & Interno \\
%			R\addQNumber
%			Q1 & \gruppo\ si prefigge di soddisfare almeno due requisiti desiderabili & Interno \\
%			R\addQNumber
%			Q1 & Nessun rischio non verificato precedentemente dovrebbe accadere nel corso del progetto & Interno \\
%			R\addQNumber
%			Q1 & Ogni documento dovrebbe attraversare tutte le fasi previste dal suo ciclo di vita & Interno \\
%			R\addQNumber
%			Q1 & I documenti dovrebbero essere controllati ogni 5 modifiche effettuate & Interno \\
%			R\addQNumber
%			Q1 & La fase di verifica di tutti i vari prodotti dovrebbe sempre essere eseguita in modo corretto & Interno \\
			%presi dal PdQ prodotti
			R\addQNumber
			Q1  & L'\gloss{indice di Gulpease} di ogni documento dovrebbe avere indice tra 50 e 60 & Interno \\  
%			R\addQNumber
%			Q1 & Il diario di un documento dovrebbe essere aggiornato ad ogni modifica effettuata & Interno \\
			%presi da capitolato
			R\addQNumber
			Q2 & Le applicazioni sviluppate devono rispettare i fattori trattati in \gloss{The Twelve-Factor App} segnati nel \PdQd & Capitolato Verbale....	\\ %TODO:inserire verbale in cui ci siamo trovati con Imola
			R\addQNumber
			Q2 & È necessario fornire test unitari per ogni componente applicativo & Capitolato \\
			R\addQNumber
			Q2 & È necessario fornire test d'integrazione per ogni componente applicativo & Capitolato \\
			R\addQNumber
			Q2 & È necessario testare interamente il sistema con test di sistema & Capitolato \\
			R\addQNumber
			Q2 & È necessario presentare il bug reporting per ogni componente & Capitolato \\
			R\addQNumber
			Q2 & Deve essere redatta documentazione sulle scelte progettuali effettuate & Capitolato \\
			R\addQNumber
			Q2 & Ogni scelta descritta nella documentazione deve essere correlata dalle relative motivazioni & Capitolato \\	
			\\\bottomrule

		\end{paddedtablex}
		\caption{Elenco dei requisiti di qualità.}
	\end{table}
		
	
	
	% Generazione automatica dei numeri 
	\newcounter{vaV} % valore
	\newcommand{\decrV}{\addtocounter{vaV}{+1}} % Comando per l'aumento automatico del counter vaZ
	\newcommand{\addVNumber}[0]{\thevaV \decrV} % Comando per generare
	\addtocounter{vaV}{1}
	
	%TODO: I requisiti di README vanno in vincolo o qualità? E quelli dei problemi?
	\subsection{Requisiti di vincolo}
	
   	\begin{table}[H]
		\begin{paddedtablex}[1.7]{\textwidth}{cXc} %\rowcolors{1}{\tablegray}{\lightgray}
			\textbf{Codice} & \textbf{Requisito} & \textbf{Fonte} \\
			\toprule
			%R\addVNumber
			%V2 & Le componenti Producer devono riuscire a recuperare delle segnalazioni & Capitolato \\ % stesso di "L'applicativo Consumer/Producer deve riuscire a recuperare i messaggi da un \gloss{Topic}"
%			R\addVNumber
%			V2 & Le segnalazioni devono poter essere accentrate e standardizzate & Capitolato \\
			R\addVNumber
			V2 & Le componenti devono esporre delle \gloss{API Rest} per le interazioni con le altre componenti & Capitolato \\
			R\addVNumber
			V2 & Devono essere sviluppati due componenti applicativi Producer tra Redmine, GitLab e Sonarqube & Capitolato \\
			R\addVNumber
			V0 & È possibile avere un terzo componente applicativo Producer oltre ai due obbligatori &  Capitolato \\
			R\addVNumber
			V2 & Devono essere sviluppati due componenti applicativi Consumer tra Telegram, Email e Slack & Capitolato \\
			R\addVNumber
			V0 & È possibile avere un terzo componente applicativo Consumer oltre ai due obbligatori & Capitolato \\
			R\addVNumber
			V0 & Come meccanismo di estensione per GitHub è possibile fare uso di \gloss{webhooks} & Capitolato \\
			R\addVNumber
			V1 & Per lo sviluppo dei componenti applicativi è possibile usare come linguaggio Java 8 o una versione più recente, Python o Node.js & Capitolato \\
			R\addVNumber
			V1 & È possibile utilizzare Apache Kafka come Broker & Capitolato \\
			R\addVNumber
			V2 & Docker deve essere la tecnologia di riferimento per l'istanziazione di tutte le componenti & Capitolato \\
			R\addVNumber
			V2 & Deve essere redatta una documentazione su eventuali problemi riscontrati rimasti ancora aperti al termine del progetto & Capitolato \\
			R\addVNumber
			V1 & Per ogni problema aperto documentato è possibile allegare delle soluzioni da attuare in futuro & Capitolato\\
			R\addVNumber
			V2 & È necessario presentare un DockerFile per ogni componente & Capitolato \\
			R\addVNumber
			V2 & È necessario presentare un file \gloss{README} per ogni componente & Capitolato \\
			R\addVNumber
			V2 & I file README delle componenti applicative devono contenere la documentazione delle API esposte dal servizio & Capitolato \\
			R\addVNumber
			V2 & I file README delle componenti applicative devono contenere le istruzioni per il loro utilizzo & Capitolato \\
			%TODO: R.. & I DockerFile devono essere comprensivi di script e file cfg..ecc?
			R\addVNumber
			V2 & È necessario presentare un file README per il DockerFile & Capitolato \\
			R\addVNumber
			V2 & Il file README per il DockerFile deve contenere le istruzioni per l'avvio & Capitolato \\
			R\addVNumber
			V2 & Il file README per il DockerFile deve contenere la documentazione delle configurazioni custom scelte & Capitolato \\
			\\\bottomrule

		\end{paddedtablex}
		\caption{Elenco dei requisiti di vincolo.}
	\end{table}	
		
					
	
	\subsection{Tracciamento}
	
		\subsubsection{Tracciamento fonti-requisiti}
		%\begin{table}[H]
			%\begin{tabular}%[1.7]{\textwidth}{XX}
			\begin{oldtabular}{ll}
				\textbf{Fonte} & \textbf{Requisito} \\%\toprule
				\multirow{30}{*}{Capitolato} & R1F2 \\
				 & R2F2 \\
				 & R3F2 \\
				 & R4F2 \\ 
				 & R5F2 \\
				 & R4Q2 \\
				 & R5Q2 \\
				 & R6Q2 \\
				 & R7Q2 \\
				 & R8Q2 \\
				 & R9Q2 \\
				 & R10Q2 \\
				 & R1V2 \\
				 & R2V2 \\
				 & R3V0 \\
				 & R4V2 \\
				 & R5V0 \\
				 & R6V0 \\
				 & R7V1 \\
				 & R8V1 \\
				 & R9V2 \\
				 & R10V2 \\
				 & R11V1 \\
				 & R12V2 \\
				 & R13V2 \\
				 & R14V2 \\
				 & R15V2 \\
				 & R16V2 \\
				 & R17V2 \\
				 & R18V2 \\ \hline
				\multirow{17}{*}{Interno} & R6F2 \\
				 & R7F0 \\
				 & R8F2 \\
				 & R9F2 \\
				 & R10F0 \\
				 & R11F0 \\
				 & R12F0 \\
				 & R13F0 \\
				 & R14F0 \\
				 & R15F0 \\
				 & R16F0 \\
				 & R17F0 \\
				 & R18F0 \\
				 & R19F0 \\
				 & R1Q2 \\
				 & R2Q2 \\
				 & R3Q1 \\
			  \end{oldtabular}
		  
			\begin{table}[H]	 
			\begin{oldtabular}{ll}	 
				Verbale... & R4Q2 \\
				UC1.1 & R6F2 \\
				UC1.2 & R7F0 \\
				UC2 & R8F2 \\
				UC3 & R9F2 \\
				UC4 & R10F0 \\
				UC5.1 & R11F0 \\
				UC5.1.1 & R12F0 \\
				UC5.2 & R13F0 \\
				UC6.1.1 & R14F0 \\
				UC6.2.1 & R15F0 \\
				UC6.1.2 & R16F0 \\
				UC6.2.2 & R17F0 \\
				UC6.1.3 & R18F0 \\
				UC6.2.3 & R19F0 \\
				%\\\bottomrule
			\end{oldtabular}
			\caption{Elenco di fonti-requisiti.}
		\end{table}	

		
\newcounter{V} % valore
\newcommand{\deV}{\addtocounter{V}{+1}} % Comando per l'aumento automatico del counter vaZ
\newcommand{\addC}[0]{\theV \deV} % Comando per generare
\addtocounter{V}{1}		

\newcounter{Vv} % valore
\newcommand{\deVv}{\addtocounter{Vv}{+1}} % Comando per l'aumento automatico del counter vaZ
\newcommand{\addVC}[0]{\theVv \deVv} % Comando per generare
\addtocounter{Vv}{1}		

\newcounter{X} % valore
\newcommand{\deX}{\addtocounter{X}{+1}} % Comando per l'aumento automatico del counter vaZ
\newcommand{\addX}[0]{\theX \deX} % Comando per generare
\addtocounter{X}{1}	
		
		\subsubsection{Tracciamento requisiti-fonte}
		
		\begin{paddedtablex}[1.7]{\textwidth}{XX}
			\textbf{Requisito} & \textbf{Fonte} \\\toprule
			%R di F
			R\addC
			F2 & Capitolato \\
			R\addC
			F2 & Capitolato \\
			R\addC
			F2 & Capitolato \\
			R\addC
			F2 & Capitolato \\
			R\addC
			F2 & Capitolato \\
			R\addC
			F2 & Interno UC1.1 \\
			R\addC
			F0 & Interno UC1.2 \\
			R\addC
			F2 & Interno UC2 \\
			R\addC
			F2 & Interno UC3 \\
			R\addC
			F0 & Interno UC4 \\
			R\addC
			F0 & Interno UC5.1 \\
			R\addC
			F0 & Interno UC5.1.1 \\
			R\addC
			F0 & Interno UC5.2 \\
			R\addC
			F0 & Interno UC6.1.1 \\
			R\addC
			F0 & Interno UC6.2.1 \\
			R\addC
			F0 & Interno UC6.1.2 \\
			R\addC
			F0 & Interno UC6.2.2 \\
			R\addC
			F0 & Interno UC6.1.3 \\
			R\addC
			F0 & Interno UC6.2.3 \\
			%R di Q
			R\addVC
			Q2 & Interno \\
			R\addVC
			Q2 & Interno \\
			R\addVC
			Q1 & Interno \\
			R\addVC
			Q2 & Capitolato Verbale .... \\
			R\addVC
			Q2 & Capitolato \\
			R\addVC
			Q2 & Capitolato \\
			R\addVC
			Q2 & Capitolato \\
			R\addVC
			Q2 & Capitolato \\
			R\addVC
			Q2 & Capitolato \\
			R\addVC
			Q2 & Capitolato \\
			%\\\bottomrule
		\end{paddedtablex}
			
		\begin{table}[H]	
		\begin{paddedtablex}[1.7]{\textwidth}{XX}
			%R di V
			R\addX
			V2 & Capitolato \\
			R\addX
			V2 & Capitolato \\
			R\addX
			V0 & Capitolato \\
			R\addX
			V2 & Capitolato \\
			R\addX
			V0 & Capitolato \\
			R\addX
			V0 & Capitolato \\
			R\addX
			V1 & Capitolato \\
			R\addX
			V1 & Capitolato \\
			R\addX
			V2 & Capitolato \\
			R\addX
			V2 & Capitolato \\
			R\addX
			V1 & Capitolato \\
			R\addX
			V2 & Capitolato \\
			R\addX
			V2 & Capitolato \\
			R\addX
			V2 & Capitolato \\
			R\addX
			V2 & Capitolato \\
			R\addX
			V2 & Capitolato \\
			R\addX
			V2 & Capitolato \\
			R\addX
			V2 & Capitolato \\
			
			\bottomrule
		\end{paddedtablex}
		\caption{Elenco di requisiti-fonti.}
	\end{table}		
		
		
		
	\subsection{Riepilogo}
	
		\begin{paddedtablex}[1.7]{\textwidth}{XXXX}
			\textbf{Tipologia} & \textbf{Obbligatori} & \textbf{Desiderabili} & \textbf{Opzionali} \\\toprule
			Di funzionalità & 9 & 0 & 10 \\
			Di qualità & 9 & 1 & 0 \\
			Di vincolo & 12 & 3 & 3 			
			\\\bottomrule
		\end{paddedtablex}
		
	