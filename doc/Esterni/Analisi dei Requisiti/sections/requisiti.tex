\newpage
\section{Requisiti}
Ad ogni requisito viene assegnato il codice identificativo univoco:
	\begin{center}
		\texttt{R[Numero][Tipo][Priorità]}
	\end{center}
	in cui ogni parte ha un significato preciso:
	\begin{itemize}
		\item \textbf{R}: requisito.
		\item \textbf{Numero}: numero progressivo che segue la struttura dei documenti.
		\item \textbf{Tipo}: la la tipologia di requisito che può essere di:
		\begin{itemize}
			\item \textbf{F}: funzionalità.
			\item \textbf{Q}: \gloss{qualità}.
			\item \textbf{V}: vincolo.
		\end{itemize}
		\item \textbf{Priorità}: indica il grado di urgenza di un requisito di essere soddisfatto, come:
		\begin{itemize}
			\item \textbf{0}: opzionale.
			\item \textbf{1}: desiderabile.
			\item \textbf{2}: obbligatorio.
		\end{itemize}
	\end{itemize}


	Esempio: \texttt{R2Q1} indica il secondo requisito di qualità ed è desiderabile.

	%I requisiti di seguito riportati sono elencati in modo tale da seguire la struttura dei documenti. Ovvero, si possono trovare raggruppati i requisiti derivanti dello stesso tipo, ad esempio solo i requisiti di funzionalità dal sesto al diciannovesimo derivano dai casi d'uso.
	%TODO: rifare id requisiti

%inserire il fatto che una persona può aggiungere il proprio nickname o altro da interfaccia, come requisito opzionale	...

% Generazione automatica dei numeri
%\newcounter{vaZ} % valore
%\newcommand{\incrZ}{\addtocounter{vaZ}{+1}} % Comando per l'aumento automatico del counter vaZ
%\newcommand{\addZNumber}[0]{\incrZ\thevaZ} % Comando per generare
%
%\newcommand{\Freq}[3]{R\addZNumber F#1 & #2 & #3 \\}
%\newcommand{\Fsubreq}[3]{R\thevaZ F#1 & #2 & #3 \\} % Se c'è bisogno di sottocasi

% Comando requisito generico
\newcommand{\req}[3]{%
#1 & #2 & #3 \\
}

	%COMANDI PER REQ DI FUNZIONALITÀ
	% Generazione automatica dei numeri
	\newcounter{vaF} % valore
	% \newcommand{\incrF}{\addtocounter{vaF}{+1}} % Comando per l'aumento automatico del counter vaZ
	% \newcommand{\addFNumber}{\incrF\thevaF} % Comando per generare il valore incrementato di uno rispetto a prima
	
	\newcounter{secF}[vaF] % per il secondo livello del requisito
	\newcounter{thF}[secF] % terzo livello
	\newcounter{fourthF}[thF] % quarto livello
	
	\newcommand{\ReqF}[3]{\stepcounter{vaF}R\thevaF F#1 & #2 & #3 \\} % Primo livello

	\newcommand{\subReqF}[3]{\stepcounter{secF}R\thevaF.\thesecF F#1 & #2 & #3 \\} % Secondo livello
	\newcommand{\getsubReqF}[3]{R\thevaF.\thesecF F#1 & #2 & #3 \\} % Secondo livello, get senza incremento

	\newcommand{\subsubReqF}[3]{\stepcounter{thF}R\thevaF.\thesecF.\thethF F#1 & #2 & #3 \\} % Terzo livello
	\newcommand{\getsubsubReqF}[3]{R\thevaF.\thesecF.\thethF F#1 & #2 & #3 \\} % Quarto livello, get senza incremento

	\newcommand{\subsubsubReqF}[3]{\stepcounter{fourthF}R\thevaF.\thesecF.\thethF.\thefourthF F#1 & #2 & #3 \\} % Quarto livello
	\newcommand{\getsubsubsubReqF}[3]{R\thevaF.\thesecF.\thethF.\thefourthF F#1 & #2 & #3 \\} % Quarto livello, get senza incremento

	% \newcounter{secondF} % per il secondo livello del requisito
	% \addtocounter{secondF}{1}
	% \newcommand{\secIncrF}{\addtocounter{secondF}{+1}} % Comando per l'aumento automatico del counter per il secondo livello
	% \newcommand{\addSecFNumber}{\secIncrF\thesecondF} % Comando per generare il valore incrementato di uno rispetto a prima
	% \newcommand{\resetFCounter}{\setcounter{secondF}{1}}
	% \newcommand{\decSecF}{\resetFCounter\thesecondF}
	
	% \newcounter{thirdF} % per il terzo livello del requisito
	% \addtocounter{thirdF}{1}
	% \newcommand{\thirdIncrF}{\addtocounter{thirdF}{+1}} % Comando per l'aumento automatico del counter per il secondo livello
	% \newcommand{\addThirdFNumber}{\thirdIncrF\thethirdF} % Comando per generare il valore incrementato di uno rispetto a prima
	% \newcommand{\resetthirdFCounter}{\setcounter{thirdF}{1}}
	% \newcommand{\decThirdF}{\resetthirdFCounter\thethirdF}


	% \newcommand{\Freq}[3]{R\addFNumber F#1 & #2 & #3 \\} % Nuovo requisito, maggiore del precedente
	% \newcommand{\Fsubreq}[3]{R\thevaF F#1 & #2 & #3 \\} % Requisito diverso ma con stesso numero progressivo
	% \newcommand{\Fsecondreq}[3]{R\thevaF.\addSecFNumber F#1 & #2 & #3 \\}
	% \newcommand{\Finitsecondreq}[3]{R\thevaF.\decSecF F#1 & #2 & #3 \\}
	% \newcommand{\Fthirdreq}[3]{R\thevaF.\thesecondF.\addThirdFNumber F#1 & #2 & #3 \\}
	% \newcommand{\Finitthirdreq}[3]{R\thevaF.\thesecondF.\decThirdF F#1 & #2 & #3 \\}
	

	\subsection{Requisiti di funzionalità}\label{RequisitiFunzionalità}

	\begin{table}[H]
		\begin{paddedtablex}[1.7]{\textwidth}{cXc}%0 opz  2 obb
			\textbf{Codice} & \textbf{Requisito} & \textbf{Fonte} \\\toprule

			\stepcounter{vaF} % Per allineare i requisiti ai casi d'uso

			% da casi d'uso
			\ReqF{2}{Redmine deve poter inviare la segnalazione di apertura issue al Producer Redmine}{Interno UC2-PR}
			\ReqF{2}{Redmine deve poter inviare la segnalazione di modifica issue al Producer Redmine}{Interno UC3-PR}
			\ReqF{2}{GitLab deve essere in grado di segnalare l'apertura di issue al Producer GitLab}{Interno UC4-PG}
			\ReqF{2}{GitLab deve essere in grado di segnalare la modifica issue al Producer GitLab}{Interno UC5-PG}
			\ReqF{2}{GitLab deve poter segnalare un evento di push a Producer GitLab}{Interno UC6-PG}
			\ReqF{2}{Il Producer Redmine deve essere in grado di inviare un messaggio al Gestore Personale}{Interno UC7-GP}
			\subReqF{2}{Il Producer Redmine deve essere in grado di inviare un messaggio di apertura issue al Gestore Personale}{Interno UC7.1-GP}
			\subReqF{2}{Il Producer Redmine deve essere in grado di inviare un messaggio di modifica issue al Gestore Personale}{Interno UC7.2-GP}
			\subReqF{2}{Il Producer Redmine deve essere in grado di scartare i messaggi non validi}{Interno UC7.3-GP}
			\ReqF{2}{Il Producer GitLab deve essere in grado di inviare un messaggio al Gestore Personale}{Interno UC8-GP}
			\subReqF{2}{Il Producer GitLab deve essere in grado di inviare messaggi di commit al Gestore Personale}{Interno UC8.1-GP}
			\subReqF{2}{Il Producer GitLab deve essere in grado di inviare un messaggio di issue al Gestore Personale}{Interno UC8.2-GP}
			\subsubReqF{2}{Il Producer GitLab deve essere in grado di inviare un messaggio di nuova issue al Gestore Personale}{Interno UC8.2.1-GP}
			\subsubReqF{2}{Il Producer GitLab deve essere in grado di inviare un messaggio di modifica issue al Gestore Personale}{Interno UC8.2.2-GP}
			\subsubReqF{2}{Il Producer GitLab deve essere in grado di scartare i messaggi di issue non validi}{Interno UC8.2.3-GP}
			\ReqF{2}{Il Gestore Personale deve poter inviare il messaggio finale al Consumer Telegram}{Interno UC9-CT}

			\bottomrule\\
		\end{paddedtablex}
		\caption{Elenco dei requisiti di funzionalità (1)}
	\end{table}

	\begin{table}[H]
		\begin{paddedtablex}[1.7]{\textwidth}{cXc}%0 opz  2 obb
			\textbf{Codice} & \textbf{Requisito} & \textbf{Fonte} \\\toprule

			\ReqF{2}{Il Gestore Personale deve poter inviare il messaggio finale al Consumer Email}{Interno UC10-CE}
			\ReqF{2}{Il Consumer Telegram deve poter inoltrare il messaggio finale al bot Telegram}{Interno UC11-BT}
			\ReqF{2}{Il Consumer Email deve poter inoltrare il messaggio finale al server Email}{Interno UC12-SE}
			% \stepcounter{vaF} e usare \subReqF per rientrare di un livello
			\ReqF{2}{L'utente può eseguire l'accesso al Gestore Personale}{Interno UC13.1-GP}
			\subReqF{2}{L'utente può inserire il proprio identificativo che lo fa riconoscere dal sistema}{Interno UC13.1.1}
			\subReqF{2}{\progetto\ fa apparire un messaggio di errore se il tentativo di accesso non è andato a buon fine}{Interno UC13.1.2}
			\ReqF{2}{L'utente acceduto deve poter uscire dal sistema}{Interno UC14-GP}

			\ReqF{2}{L'utente acceduto deve poter aggiungere un nuovo utente}{Interno UC15-GP}
			\stepcounter{secF}
			\subsubReqF{2}{L'utente acceduto deve poter inserire il nome dell'utente da aggiungere}{Interno UC15.1.1-GP}
			\subsubReqF{2}{L'utente acceduto deve poter inserire il cognome dell'utente da aggiungere}{Interno UC15.1.2-GP}
			\subsubReqF{2}{L'utente acceduto deve poter inserire il contatto Email dell'utente da aggiungere}{Interno UC15.1.3-GP}
			\subsubReqF{2}{L'utente acceduto deve poter inserire il contatto Telegram dell'utente da aggiungere}{Interno UC15.1.4-GP}
			\subReqF{2}{L'utente acceduto deve poter visualizzare un messaggio di errore se il contatto Telegram non è univoco}{Interno UC15.2-GP}
			\getsubReqF{2}{L'utente acceduto deve poter visualizzare un messaggio di errore se il contatto Email non è univoco}{Interno UC15.2-GP}

			\ReqF{2}{L'utente acceduto deve poter rimuovere un utente già presente}{Interno UC16-GP}
			\stepcounter{secF}
			\subsubReqF{2}{L'utente acceduto deve poter inserire il contatto Email dell'utente da rimuovere}{Interno UC16.1.1-GP}
			\subsubReqF{2}{L'utente acceduto deve poter inserire il contatto Telegram dell'utente da rimuovere}{Interno UC16.1.2-GP}
			\subReqF{2}{L'utente acceduto deve poter visualizzare un messaggio di errore se l'identificativo non è presente nel sistema}{Interno UC16.2-GP}

			\bottomrule\\
		\end{paddedtablex}
		\caption{Elenco dei requisiti di funzionalità (2)}
	\end{table}

	\begin{table}[H]
		\begin{paddedtablex}[1.7]{\textwidth}{cXc}%0 opz  2 obb
			\textbf{Codice} & \textbf{Requisito} & \textbf{Fonte} \\\toprule

			\subReqF{2}{L'utente acceduto deve potersi rimuovere dal sistema}{Interno UC16.3-GP}
			\ReqF{2}{L'utente acceduto deve poter modificare le informazioni relative a un utente}{Interno UC17-GP}
			\subReqF{2}{L'utente acceduto deve poter selezionare l'identificativo dell'utente da modificare}{Interno UC17.1-GP}

			\stepcounter{thF}
			\subsubsubReqF{2}{L'utente acceduto deve poter scegliere il nuovo nome dell'utente da modificare}{Interno UC17.1.1.1-GP}
			\subsubsubReqF{2}{L'utente acceduto deve poter scegliere il nuovo cognome dell'utente da modificare}{Interno UC17.1.1.2-GP}
			\subsubsubReqF{2}{L'utente acceduto deve poter scegliere il nuovo contatto Email dell'utente da modificare}{Interno UC17.1.1.3-GP}
			\subsubsubReqF{2}{L'utente acceduto deve poter scegliere il nuovo contatto Telegram dell'utente da modificare}{Interno UC17.1.1.4-GP}
			\subsubReqF{2}{L'utente acceduto deve poter visualizzare un messaggio di errore se il contatto Email è già presente}{Interno UC17.1.2-GP}
			\getsubsubReqF{2}{L'utente acceduto deve poter visualizzare un messaggio di errore se il contatto Telegram è già presente}{Interno UC17.1.2-GP}
			\subReqF{2}{L'utente acceduto deve poter visualizzare un messaggio di errore se l'identificativo non è riconosciuto dal sistema}{Interno UC17.2-GP}

			\stepcounter{vaF}
			\subReqF{2}{L'utente acceduto deve poter aggiungere nuovi Topic di iscrizione}{Interno UC18.1-GP}
			\subReqF{2}{L'utente acceduto deve poter aggungere nuovi giorni di indisponibilità nel calendario}{Interno UC18.2-GP}
			\subReqF{2}{L'utente acceduto deve poter aggungere una nuova piattaforma di messaggistica preferita}{Interno UC18.3-GP}
			\subReqF{2}{L'utente acceduto deve poter aggungere la propria persona di fiducia}{Interno UC18.4-GP}
			\subReqF{2}{L'utente acceduto deve poter visualizzare un messaggio di errore se il contatto Email della persona di fiducia non è presente nel sistema}{Interno UC18.5-GP}

			\bottomrule\\
		\end{paddedtablex}
		\caption{Elenco dei requisiti di funzionalità (3)}
	\end{table}

	\begin{table}[H]
		\begin{paddedtablex}[1.7]{\textwidth}{cXc}%0 opz  2 obb
			\textbf{Codice} & \textbf{Requisito} & \textbf{Fonte} \\\toprule

			\getsubReqF{2}{L'utente acceduto deve poter visualizzare un messaggio di errore se il contatto Telegram della persona di fiducia non è presente nel sistema}{Interno UC18.5-GP}
			\subReqF{2}{L'utente acceduto deve poter aggiungere nuove keyword di interesse per i messaggi di commit}{Interno UC18.6-GP}
			\subReqF{2}{L'utente acceduto deve poter visualizzare un messaggio di errore se la keyword inserita era già nella sua lista}{Interno UC18.7-GP}

			\stepcounter{vaF}
			\subReqF{2}{L'utente acceduto deve poter rimuovere i Topic a cui è iscritto}{Interno UC19.1-GP}
			\subReqF{2}{L'utente acceduto deve poter rimuovere giorni di indisponibilità nel calendario}{Interno UC19.2-GP}
			\subReqF{2}{L'utente acceduto deve poter rimuovoere una piattaforma di messaggistica preferita}{Interno UC19.3-GP}
			\subReqF{2}{L'utente acceduto deve poter rimuovere la propria persona di fiducia}{Interno UC19.4-GP}
			\subReqF{2}{L'utente acceduto deve poter visualizzare un messaggio di errore se il contatto Email della persona di fiducia non è presente nel sistema}{Interno UC19.5-GP}
			\getsubReqF{2}{L'utente acceduto deve poter visualizzare un messaggio di errore se il contatto Telegram della persona di fiducia non è presente nel sistema}{Interno UC19.5-GP}
			\subReqF{2}{L'utente acceduto deve poter rimuovere keyword di interesse per i messaggi di commit}{Interno UC19.6-GP}
			\subReqF{2}{L'utente acceduto deve poter visualizzare un messaggio di errore se la keyword da rimuovere è assente dalla sua lista}{Interno UC19.7-GP}

			\bottomrule\\
		\end{paddedtablex}
		\caption{Elenco dei requisiti di funzionalità (4)}
	\end{table}
			% \Finitsecondreq{0}{L'utente può eseguire l'accesso al gestore personale}{Interno UC5.1}
			% \Finitthirdreq{0}{L'utente può inserire il proprio username che lo identifica all'interno di \progetto}{Interno UC5.1.1}
			% \Fsecondreq{0}{\progetto\ fa apparire un messaggio di errore se il tentativo di accesso non è andato a buon fine}{Interno UC5.2}
			% \req{R\addFNumber.1.1F0}{L'utente può iscriversi a un Topic}{Interno UC6.1.1}
			% \req{R\thevaF.1.2F0}{L'utente può aggiungere nel gestore personale i giorni in cui non è reperibile}{Interno UC6.1.2}
			% \req{R\thevaF.1.3F0}{L'utente può scegliere la piattaforma di messaggistica, tra Telegram ed e-mail, in cui ricevere le notifiche}{Interno UC6.1.3}
			% \req{R\thevaF.2.1F0}{L'utente può disiscriversi da un Topic}{Interno UC6.2.1}
			% \req{R\thevaF.2.2F0}{L'utente può rimuovere i giorni in cui aveva selezionato precedentemente di non essere reperibile}{Interno UC6.2.2}
			% \req{R\thevaF.2.3F0}{L'utente può togliere le proprie preferenze sulle piattaforme di messaggistica da cui ricevere notifiche inviate attraverso \progetto}{Interno UC6.2.3}


	\begin{table}[H]
		\begin{paddedtablex}[1.7]{\textwidth}{cXc}
			\textbf{Codice} & \textbf{Requisito} & \textbf{Fonte} \\\toprule

			% da capitolato
			% stesso di "L'applicativo Producer/Consumer deve essere in grado di reindirizzare i messaggi verso la persona più appropriata"
			\ReqF{2}{Le componenti Consumer devono essere in grado di inviare i messaggi provenienti da un Topic verso il corretto destinatario}{Capitolato}
			\ReqF{2}{Le componenti Consumer devono essere in grado di abbonarsi ai Topic scelti}{Capitolato}
			\ReqF{2}{Le segnalazioni devono poter essere gestite in maniera automatica e personalizzabile}{Capitolato}
			\ReqF{2}{Nel sistema deve essere presente un Broker che istanzia e gestisce le segnalazioni organizzandole per Topic}{Capitolato}
			\ReqF{2}{Le componenti Producer devono riuscire a pubblicare le segnalazioni recuperate sotto forma di messaggi secondo i Topic corretti}{Capitolato}
			\ReqF{2}{Le componenti devono esporre delle \gloss{API Rest} per le interazioni con le altre componenti}{Capitolato}

			\bottomrule
		\end{paddedtablex}
		\caption{Elenco dei requisiti di funzionalità (5)}
	\end{table}

	%COMANDI PER REQUISITI DI FUNZIONALITÀ
	% Generazione automatica dei numeri
	\newcounter{vaQ} % valore
	%\addtocounter{vaQ}{1} % inizia da 1 il contatore
	\newcommand{\incrQ}{\addtocounter{vaQ}{+1}} % Comando per l'aumento automatico del counter vaZ
	\newcommand{\addQNumber}{\incrQ\thevaQ} % Comando per generare il valore incrementato di uno rispetto a prima

	\newcounter{secondQ} % per il secondo livello del requisito
	\addtocounter{secondQ}{1}
	\newcommand{\secIncrQ}{\addtocounter{secondQ}{+1}} % Comando per l'aumento automatico del counter per il secondo livello
	\newcommand{\addSecQNumber}{\secIncrQ\thesecondQ} % Comando per generare il valore incrementato di uno rispetto a prima
	\newcommand{\resetQCounter}{\setcounter{secondQ}{1}}
	\newcommand{\decSecQ}{\resetQCounter\thesecondQ}


	\newcommand{\Qreq}[3]{R\addQNumber Q#1 & #2 & #3 \\} % Nuovo requisito, maggiore del precedente
	\newcommand{\Qsubreq}[3]{R\thevaQ Q#1 & #2 & #3 \\} % Requisito diverso ma con stesso numero progressivo
	\newcommand{\Qsecondreq}[3]{R\thevaQ.\addSecQNumber Q#1 & #2 & #3 \\}
	\newcommand{\Qinitsecondreq}[3]{R\thevaQ.\decSecQ Q#1 & #2 & #3 \\}



	% NB: molti requisiti di qualità sono stati tolti perchè non sono veri requisiti del sistema, riguardano noi e il nostro modo di fare, non il sistema
	\subsection{Requisiti di qualità}\label{RequisitiQualità}

	\begin{table}[H]
		\begin{paddedtablex}[1.7]{\textwidth}{cXc}
			\textbf{Codice} & \textbf{Requisito} & \textbf{Fonte} \\
			\toprule
			% presi dal PdQ
			\Qreq{1}{Viene stabilito un numero massimo di giorni di ritardo per la chiusura di una issue}{Interno QPR001}
			\Qreq{1}{L'\gloss{indice di Gulpease} di ogni documento dovrebbe rientrare all'interno di un intervallo stabilito}{Interno QPD001}
			%\Qreq{1}{I costi previsti dalla \gloss{pianificazione} non dovrebbero variare più di quanto stabilito}{Interno QPR002}
			\Qreq{1}{Ci siamo accordati per una frequenza minima di commit effettuati in una settimana}{Interno QPR004}
			\Qreq{1}{Ci siamo prefissati di soddisfare un numero stabilito di requisiti desiderabili}{Interno QPR006}
			\Qreq{1}{Nessun rischio non verificato precedentemente deve accadere nel corso del progetto}{Interno QPR007}
			\Qreq{1}{Ogni documento deve attraversare tutte le fasi previste dal suo \gloss{ciclo di vita}}{Interno QPR008}
			\Qreq{1}{Viene stabilito il numero massimo di modifiche che può ricevere un prodotto prima di essere verificato}{Interno QPR009}
			\Qreq{2}{Tutte le \gloss{norme} inserite nelle \NdP\ devono essere rispettate}{Interno}
			\Qreq{2}{Tutti i vincoli presenti nel \PdQ\ devono essere rispettati}{Interno}
			\Qreq{2}{Le applicazioni sviluppate devono rispettare i fattori trattati in The Twelve-Factor App segnati nel \PdQd}{Capitolato \Doc{VE\_2018-12-12}}
			\Qreq{2}{È necessario presentare il \gloss{bug} reporting per ogni componente}{Capitolato}

			\bottomrule
		\end{paddedtablex}
		\caption{Elenco dei requisiti di qualità (1)}
	\end{table}



	\begin{table}[H]
		\begin{paddedtablex}[1.7]{\textwidth}{cXc}
			\textbf{Codice} & \textbf{Requisito} & \textbf{Fonte} \\\toprule

			\Qreq{2}{Deve essere redatta la documentazione sulle scelte progettuali effettuate}{Capitolato}
			%\req{R\thevaQ.1Q2}{Ogni scelta descritta nella documentazione deve essere correlata dalle relative motivazioni}{Capitolato}
			\Qinitsecondreq{2}{Ogni scelta descritta nella documentazione deve essere correlata dalle relative motivazioni}{Capitolato}
			\Qreq{2}{È necessario testare ogni prodotto software considerando ogni sistema di riferimento e interazione tra le sue parti, perciò con test d'unita, d'integrazione e di sistema}{Capitolato}
			%\req{R\thevaQ.1Q2}{È necessario fornire test unitari per ogni componente applicativo}{Capitolato}
			\Qinitsecondreq{2}{È necessario fornire test unitari per ogni componente applicativo}{Capitolato}
			%\req{R\thevaQ.2Q2}{È necessario fornire test d'integrazione per ogni componente applicativo}{Capitolato}
			\Qsecondreq{2}{È necessario fornire test d'integrazione per ogni componente applicativo}{Capitolato}
			\Qsecondreq{2}{È necessario testare interamente il sistema con test di sistema}{Capitolato}
			\Qreq{1}{Per ogni problema aperto documentato, si allegano delle soluzioni da attuare in futuro}{Capitolato}
			\Qinitsecondreq{2}{Deve essere redatta una documentazione su eventuali problemi riscontrati rimasti ancora aperti al termine del progetto}{Capitolato}
			\Qreq{2}{È necessario presentare un file \gloss{README} per ogni componente}{Capitolato}
			%\req{R\thevaQ.1Q2}{I file README delle componenti applicative devono contenere la documentazione delle \gloss{API} esposte dal servizio}{Capitolato}
			\Qinitsecondreq{2}{I file README delle componenti applicative devono contenere la documentazione delle \gloss{API} esposte dal servizio}{Capitolato}
			\Qsecondreq{2}{I file README delle componenti applicative devono contenere le istruzioni per il loro utilizzo}{Capitolato}
			\Qsecondreq{2}{È necessario presentare un file README per il Dockerfile}{Capitolato}
			\Qsecondreq{2}{Il file README per il Dockerfile deve contenere le istruzioni per l'avvio}{Capitolato}
			\Qsecondreq{2}{Il file README per il Dockerfile deve contenere la documentazione delle configurazioni custom scelte}{Capitolato}

			\bottomrule \\
		\end{paddedtablex}
		\caption{Elenco dei requisiti di qualità (2)}
	\end{table}

	%COMANDI PER REQ DI VINCOLO
	% Generazione automatica dei numeri
	\newcounter{vaV} % valore
	\newcommand{\incrV}{\addtocounter{vaV}{+1}} % Comando per l'aumento automatico del counter vaZ
	\newcommand{\addVNumber}{\incrV\thevaV} % Comando per generare il valore incrementato di uno rispetto a prima
	
	\newcounter{secondV} % per il secondo livello del requisito
	\addtocounter{secondV}{1}
	\newcommand{\secIncrV}{\addtocounter{secondV}{+1}} % Comando per l'aumento automatico del counter per il secondo livello
	\newcommand{\addSecVNumber}{\secIncrV\thesecondV} % Comando per generare il valore incrementato di uno rispetto a prima
	\newcommand{\resetVCounter}{\setcounter{secondV}{1}}
	\newcommand{\decSecV}{\resetVCounter\thesecondV}
	
	
	\newcommand{\Vreq}[3]{R\addVNumber V#1 & #2 & #3 \\} % Nuovo requisito, maggiore del precedente
	\newcommand{\Vsubreq}[3]{R\thevaV V#1 & #2 & #3 \\} % Requisito diverso ma con stesso numero progressivo
	\newcommand{\Vsecondreq}[3]{R\thevaV.\addSecVNumber V#1 & #2 & #3 \\}
	\newcommand{\Vinitsecondreq}[3]{R\thevaV.\decSecV V#1 & #2 & #3 \\}
	

	%TODO: aggiungere versione di Slack
	\subsection{Requisiti di vincolo}\label{RequisitiVincolo}

	\begin{table}[H]
		\begin{paddedtablex}[1.7]{\textwidth}{cXc} %\rowcolors{1}{\tablegray}{\lightgray}
			\textbf{Codice} & \textbf{Requisito} & \textbf{Fonte} \\
			\toprule
			\Vreq{2}{Devono essere sviluppati due componenti applicativi Producer tra \redmine, \gitlab\ e SonarQube 6.7}{Capitolato}
			%\req{R\thevaV.1V0}
			\Vinitsecondreq{0}{È possibile avere un terzo componente applicativo Producer oltre ai due obbligatori}{Capitolato}
			\Vreq{2}{Devono essere sviluppati due componenti applicativi Consumer tra \telegram, Email e Slack}{Capitolato}
			%\req{R\thevaV.1V0}
			\Vinitsecondreq{0}{È possibile avere un terzo componente applicativo Consumer oltre ai due obbligatori}{Capitolato}
			\Vreq{2}{\docker\ deve essere la tecnologia di riferimento per l'istanziazione di tutte le componenti}{Capitolato}
			%\req{R\thevaV.1V2}
			\Vinitsecondreq{2}{È necessario presentare un Dockerfile per ogni componente}{Capitolato}
			\Vreq{1}{Per lo sviluppo dei componenti applicativi è possibile usare come linguaggio \gloss{Java} 8 o una versione più recente, \gloss{\python} o \gloss{Node.js} 10.15.1}{Capitolato}
			\Vreq{1}{È possibile utilizzare \kafka\ come Broker}{Capitolato}

			\bottomrule\\
		\end{paddedtablex}
		\caption{Elenco dei requisiti di vincolo (1)}
	\end{table}

%
%	\begin{table}[H]
%		\begin{paddedtablex}[1.7]{\textwidth}{cXc}
%			\textbf{Codice} & \textbf{Requisito} & \textbf{Fonte} \\
%			\toprule
%
%
%			\bottomrule\\
%		\end{paddedtablex}
%		\caption{Elenco dei requisiti di vincolo (2)}
%	\end{table}



	\subsection{Tracciamento}\label{Tracciamento}

		\subsubsection{Tracciamento fonti-requisiti}

		\begin{table}[H]
			\centering
			{\def\arraystretch{1.4}
			\begin{tabularx}{\textwidth}{YY}
				\textbf{Fonte} & \textbf{Requisito} \\
				\toprule
				\cellcolor{white} & R7F2 \\
				\cellcolor{white} & R8F2 \\
				\cellcolor{white} & R9F2 \\
				\cellcolor{white} & R10F2 \\
				\cellcolor{white} & R11F2 \\
				\cmidrule{2-2}
				\cellcolor{white} & R14Q2 \\
				\cellcolor{white} & R15Q2 \\
				\cellcolor{white} & R16Q2 \\
				\cellcolor{white} & R17Q2 \\
				\cellcolor{white} & R18Q2 \\
				\cellcolor{white} & R18.1Q2 \\
				\cellcolor{white} & R18.2Q2 \\
				\cellcolor{white} \multirow{-13}{*}{Capitolato} & R18.3Q2 \\
				\bottomrule\\
			\end{tabularx}}
			\caption{Elenco dei requisiti del capitolato (1)}
		\end{table}

		\begin{table}[H]
			\centering
			{\def\arraystretch{1.4}
			\begin{tabularx}{\textwidth}{YY}
				\textbf{Fonte} & \textbf{Requisito} \\
				\toprule
		        \cellcolor{white} & R1V2 \\
				\cellcolor{white} & R1.1V0 \\
				\cellcolor{white} & R2V2 \\
				\cellcolor{white} & R2.1V0 \\
				\cellcolor{white} & R3V2 \\
				\cellcolor{white} & R3.1V2 \\
				\cellcolor{white} & R4V1 \\
				\cellcolor{white} & R4.1V2 \\
				\cellcolor{white} & R5V2 \\
				\cellcolor{white} & R5.1V2 \\
				\cellcolor{white} & R5.2V2 \\
				\cellcolor{white} & R5.3V2 \\
				\cellcolor{white} & R5.4V2 \\
				\cellcolor{white} & R5.5V2 \\
				\cellcolor{white} & R6V2 \\
				\cellcolor{white} & R7V0 \\
				\cellcolor{white} & R8V1 \\
				\cellcolor{white} \multirow{-18}{*}{Capitolato} & R9V1 \\
				\bottomrule\\
			\end{tabularx}}
			\caption{Elenco dei requisiti del capitolato (2)}
		\end{table}

		\begin{table}[H]
			\centering
			{\def\arraystretch{1.5}
				\begin{tabularx}{\textwidth}{YY}
					\textbf{Fonte} & \textbf{Requisito} \\
					\toprule
					\cellcolor{white} & R11Q1 \\
					\cellcolor{white} & R12Q2 \\
					\cellcolor{white} \multirow{-2}{*}{Interno} & R13Q2 \\
					\bottomrule
				\end{tabularx}}
			\caption{Elenco dei requisiti interni}
		\end{table}

		\begin{table}[H]
			\centering
			\rowcolors{2}{white}{\tablegray}
			{\def\arraystretch{1.5}
			\begin{tabularx}{\textwidth}{YY}
				\textbf{Fonte} & \textbf{Requisito} \\
				\toprule
				UC1 & R1F2 \\
				UC2 & R2F2 \\
				UC3 & R3F2 \\
				UC4 & R4F2 \\
				UC5.1 & R5.1F0 \\
				UC5.1.1 & R5.1.1F0 \\
				UC5.2 & R5.2F0 \\
				UC6.1.1 & R6.1.1F0 \\
				UC6.1.2 & R6.1.2F0 \\
				UC6.1.3 & R6.1.3F0 \\
				UC6.2.1 & R6.2.1F0 \\
				UC6.2.2 & R6.2.2F0 \\
				UC6.2.3 & R6.2.3F0 \\
				\bottomrule \\
			\end{tabularx}}
			\caption{Elenco dei requisiti per i casi d'uso}
		\end{table}

		\begin{table}[H]
		\centering
		\rowcolors{2}{white}{\tablegray}
		{\def\arraystretch{1.5}
		\begin{tabularx}{\textwidth}{YY}
			\textbf{Fonte} & \textbf{Requisito} \\
			\toprule
			QPR001 & R1Q1 \\
			QPD001 & R2Q1 \\
			QPR002 & R3Q1 \\
			QPR003 & R4Q1 \\
			QPR004 & R5Q1 \\
			QPR005 & R6Q2 \\
			QPR006 & R7Q1 \\
			QPR007 & R8Q1 \\
			QPR008 & R9Q1 \\
			QPR009 & R10Q1 \\
			QPR010 & R11Q1 \\
			\Doc{VE\_2018-12-12} & R14Q2 \\
			\bottomrule\\
		\end{tabularx}}
		\caption{Elenco dei requisiti per gli obiettivi di qualità e verbali}
	\end{table}

\newcounter{V} % valore
\newcommand{\deV}{\addtocounter{V}{+1}} % Comando per l'aumento automatico del counter vaZ
\newcommand{\addC}[0]{\theV \deV} % Comando per generare
\addtocounter{V}{1}

\newcounter{Vv} % valore
\newcommand{\deVv}{\addtocounter{Vv}{+1}} % Comando per l'aumento automatico del counter vaZ
\newcommand{\addVC}[0]{\theVv \deVv} % Comando per generare
\addtocounter{Vv}{1}

\newcounter{X} % valore
\newcommand{\deX}{\addtocounter{X}{+1}} % Comando per l'aumento automatico del counter vaZ
\newcommand{\addX}[0]{\theX \deX} % Comando per generare
\addtocounter{X}{1}

		\subsubsection{Tracciamento requisiti-fonte}

		\begin{table}[H]
		\begin{paddedtablex}[1.7]{\textwidth}{YY}
			\textbf{Requisito} & \textbf{Fonte} \\\toprule
			%R di F
			R\addC
			F2 & Interno UC1 \\
			R\addC
			F2 & Interno UC2 \\
			R\addC
			F2 & Interno UC3 \\
			R\addC
			F2 & Interno UC4 \\
			R\addC
			.1F0 & Interno UC5.1 \\
			R5.1.1F0 & Interno UC5.1.1 \\
			R5.2F0 & Interno UC5.2 \\
			R\addC
			.1.1F0 & Interno UC6.1.1 \\
			R6.1.2F0 & Interno UC6.1.2 \\
			R6.1.3F0 & Interno UC6.1.3 \\
			R6.2.1F0 & Interno UC6.2.1 \\
			R6.2.3F0 & Interno UC6.2.2 \\
			R6.2.3F0 & Interno UC6.2.3 \\
			R\addC
			F2 & Capitolato \\
			R\addC
			F2 & Capitolato \\
			R\addC
			F2 & Capitolato \\
			R\addC
			F2 & Capitolato \\
			R\addC
			F2 & Capitolato \\
			\bottomrule
		\end{paddedtablex}
		\caption{Elenco dei requisiti funzionali in rapporto alle fonti}
		\end{table}

		\begin{table}[H]
		\begin{paddedtablex}[1.7]{\textwidth}{YY}
			\textbf{Requisito} & \textbf{Fonte} \\\toprule
			%R di Q
			R\addVC
			Q1 & Interno QPR001 \\
			R\addVC
			Q1 & Interno QPD001 \\
			R\addVC
			Q1 & Interno QPR002 \\
			R\addVC
			Q1 & Interno QPR003 \\
			R\addVC
			Q1 & Interno QPR004 \\
			R\addVC
			Q2 & Interno QPR005 \\
			R\addVC
			Q1 & Interno QPR006 \\
			R\addVC
			Q1 & Interno QPR007 \\
			R\addVC
			Q1 & Interno QPR008 \\
			R\addVC
			Q1 & Interno QPR009 \\
			R\addVC
			Q2 & Interno QPR010 \\
			R\addVC
			Q2 & Interno \\
			R\addVC
			Q2 & Interno \\
			R\addVC
			Q2 & \Doc{VE\_2018-12-12} \\
			R\addVC
			Q2 & Capitolato \\
			R\addVC
			Q2 & Capitolato \\
			R\addVC
			Q2 & Capitolato \\
			R\addVC
			Q2 & Capitolato \\
			R18.1Q2 & Capitolato \\
			R18.2Q2 & Capitolato \\
			R18.3Q2 & Capitolato \\
			\bottomrule
			\end{paddedtablex}
		\caption{Elenco dei requisiti di qualità in rapporto alle fonti}
	\end{table}

\begin{table}[H]
	\begin{paddedtablex}[1.7]{\textwidth}{YY}
		\textbf{Requisito} & \textbf{Fonte} \\\toprule
			%R di V
			R\addX
			V2 & Capitolato \\
			R1.1V0 & Capitolato \\
			R\addX
			V2 & Capitolato \\
			R2.1V0 & Capitolato \\
			R\addX
			V2 & Capitolato \\
			R3.1V2 & Capitolato \\
			R\addX
			V1 & Capitolato \\
			R4.1V2 & Capitolato \\
			R\addX
			V2 & Capitolato \\
			R5.1V2 & Capitolato \\
			R5.2V2 & Capitolato \\
			R5.3V2 & Capitolato \\
			R5.4V2 & Capitolato \\
			R5.5V2 & Capitolato \\
			R\addX
			V2 & Capitolato \\
			R\addX
			V0 & Capitolato \\
			R\addX
			V1 & Capitolato \\
			R\addX
			V1 & Capitolato \\
			\bottomrule \\
		\end{paddedtablex}
		\caption{Elenco dei requisiti di vincolo in rapporto alle fonti}
	\end{table}

	\subsection{Riepilogo}\label{Riepilogo}

		\begin{table}[H]
		\begin{paddedtablex}[1.7]{\textwidth}{YYYY}
			\textbf{Tipologia} & \textbf{Obbligatori} & \textbf{Desiderabili} & \textbf{Opzionali} \\\toprule
			Di funzionalità & 8 & 0 & 10 \\
			Di qualità & 12 & 9 & 0 \\
			Di vincolo & 12 & 3 & 3
			\\\bottomrule
		\end{paddedtablex}
		\caption{Riepilogo dei requisiti}
		\end{table}

