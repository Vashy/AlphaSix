\newpage
\section{Requisiti}
Ad ogni requisito viene assegnato il codice identificativo univoco:
	\begin{center}
		\texttt{R[Numero][Tipo][Priorità]}
	\end{center}
	in cui ogni parte ha un significato preciso:
	\begin{itemize}
		\item \textbf{R}: requisito.
		\item \textbf{Numero}: numero progressivo che segue la struttura dei documenti.
		\item \textbf{Tipo}: la la tipologia di requisito che può essere di:
		\begin{itemize}
			\item \textbf{F}: funzionalità.
			\item \textbf{Q}: \gloss{qualità}.
			\item \textbf{V}: vincolo.
		\end{itemize}
		\item \textbf{Priorità}: indica il grado di urgenza di un requisito di essere soddisfatto, come:
		\begin{itemize}
			\item \textbf{0}: opzionale.
			\item \textbf{1}: desiderabile.
			\item \textbf{2}: obbligatorio.
		\end{itemize}
	\end{itemize}


	Esempio: \texttt{R2Q1} indica il secondo requisito di qualità ed è desiderabile.\\

	%I requisiti di seguito riportati sono elencati in modo tale da seguire la struttura dei documenti. Ovvero, si possono trovare raggruppati i requisiti derivanti dello stesso tipo, ad esempio solo i requisiti di funzionalità dal sesto al diciannovesimo derivano dai casi d'uso.
	%TODO: rifare id requisiti

%inserire il fatto che una persona può aggiungere il proprio nickname o altro da interfaccia, come requisito opzionale	...

% Generazione automatica dei numeri
\newcounter{vaZ} % valore
\newcommand{\decrZ}{\addtocounter{vaZ}{+1}} % Comando per l'aumento automatico del counter vaZ
\newcommand{\addZNumber}[0]{\thevaZ \decrZ} % Comando per generare
\addtocounter{vaZ}{1}%far partire il contatore da 1

\newcommand{\Freq}[3]{R\addZNumber F#1 & #2 & #3 \\}
\newcommand{\Fsubreq}[3]{R\thevaZ F#1 & #2 & #3 \\} % Se c'è bisogno di sottocasi

% Comando requisito generico
\newcommand{\req}[3]{%
#1 & #2 & #3 \\
}

	\subsection{Requisiti di funzionalità}

	\begin{table}[H]
		\begin{paddedtablex}[1.7]{\textwidth}{cXc}%0 opz  2 obb
			\textbf{Codice} & \textbf{Requisito} & \textbf{Fonte} \\\toprule

			% da casi d'uso
			\Freq{2}{Redmine e GitLab devono essere in grado di inviare delle segnalazioni da inviare ai Producer}{Interno UC1}
			\Freq{2}{Il Producer deve essere in grado di inviare delle segnalazioni al Broker}{Interno UC2}
			\Freq{2}{Il Consumer deve essere in grado di richiedere al Broker l'invio di un messaggio}{Interno UC3}
			\Freq{2}{I Consumer devono essere in grado di inviare delle segnalazioni al server e-mail e a Telegram}{Interno UC4}
			\req{R\thevaZ.1F0}{L'utente può eseguire l'accesso al gestore personale}{Interno UC5.1}
			\req{R\thevaZ.1.1F0}{L'utente può inserire il proprio username che lo identifica all'interno di \progetto}{Interno UC5.1.1}
			\req{R\thevaZ.2F0}{\progetto\ fa apparire un messaggio di errore se il tentativo di accesso non è andato a buon fine}{Interno UC5.2}
			\req{R\addZNumber.1.1F0}{L'utente può iscriversi a un Topic}{Interno UC6.1.1}
			\req{R\thevaZ.1.2F0}{L'utente può aggiungere nel gestore personale i giorni in cui non è reperibile}{Interno UC6.1.2}
			\req{R\thevaZ.1.3F0}{L'utente può scegliere la piattaforma di messaggistica, tra Telegram ed e-mail, in cui ricevere le notifiche}{Interno UC6.1.3}
			\req{R\thevaZ.2.1F0}{L'utente può disiscriversi da un Topic}{Interno UC6.2.1}
			\req{R\thevaZ.2.2F0}{L'utente può rimuovere i giorni in cui aveva selezionato precedentemente di non essere reperibile}{Interno UC6.2.2}
			\req{R\thevaZ.2.3F0}{L'utente può togliere le proprie preferenze sulle piattaforme di messaggistica da cui ricevere notifiche inviate attraverso \progetto}{Interno UC6.2.3}

			\bottomrule\\
		\end{paddedtablex}
		\caption{Elenco dei requisiti di funzionalità (1)}
	\end{table}

	\begin{table}[H]
		\begin{paddedtablex}[1.7]{\textwidth}{cXc}
			\textbf{Codice} & \textbf{Requisito} & \textbf{Fonte} \\\toprule

			% da capitolato
			% stesso di "L'applicativo Producer/Consumer deve essere in grado di reindirizzare i messaggi verso la persona più appropriata"
			\Freq{2}{Le componenti Consumer devono essere in grado di inviare i messaggi provenienti da un Topic verso il corretto destinatario}{Capitolato}
			\Freq{2}{Le componenti Consumer devono essere in grado di abbonarsi ai Topic scelti}{Capitolato}
			\Freq{2}{Le segnalazioni devono poter essere gestite in maniera automatica e personalizzabile}{Capitolato}
			\Freq{2}{Nel sistema deve essere presente un Broker che istanzia e gestisce le segnalazioni organizzandole per Topic}{Capitolato}
			\Freq{2}{Le componenti Producer devono riuscire a pubblicare le segnalazioni recuperate sotto forma di messaggi secondo i Topic corretti}{Capitolato}

			\bottomrule
		\end{paddedtablex}
		\caption{Elenco dei requisiti di funzionalità (2)}
	\end{table}


	% Generazione automatica dei numeri
	\newcounter{vaQ} % valore
	\newcommand{\decrQ}{\addtocounter{vaQ}{+1}} % Comando per l'aumento automatico del counter vaZ
	\newcommand{\addQNumber}[0]{\thevaQ \decrQ} % Comando per generare
	\addtocounter{vaQ}{1}


	\newcommand{\Qreq}[3]{R\addQNumber Q#1 & #2 & #3 \\}
	\newcommand{\Qsubreq}[3]{R\thevaQ Q#1 & #2 & #3 \\}

	% NB: molti requisiti di qualità sono stati tolti perchè non sono veri requisiti del sistema, riguardano noi e il nostro modo di fare, non il sistema
	\subsection{Requisiti di qualità}

	\begin{table}[H]
		\begin{paddedtablex}[1.7]{\textwidth}{cXc}
			\textbf{Codice} & \textbf{Requisito} & \textbf{Fonte} \\
			\toprule
			% presi dal PdQ processi

			\Qreq{1}{Viene stabilito un numero massimo di giorni di ritardo per la chiusura di una issue}{Interno QPR001}

			% presi dal PdQ prodotti
			\Qreq{1}{L'\gloss{indice di Gulpease} di ogni documento dovrebbe rientrare all'interno di un intervallo stabilito}{Interno QPD001}

			%			R\addQNumber
			%			Q1 & Il registro delle modifiche di un documento dovrebbe essere aggiornato ad ogni modifica effettuata & Interno \\
			% presi dal PdQ processi

			\Qreq{1}{I costi previsti dalla \gloss{pianificazione} non dovrebbero variare più di quanto stabilito}{Interno QPR002}
			\Qreq{1}{Il livello di maturità dell'\gloss{ISO/IEC 15504} che ci siamo prefissati di raggiungere  deve essere almeno pari al livello stabilito}{Interno QPR003}
			\Qreq{1}{Ci siamo accordati per una frequenza minima di commit effettuati in una settimana}{Interno QPR004}
			\Qreq{2}{I requisiti obbligatori devono essere tutti completamente soddisfatti al termine del progetto}{Interno QPR005}
			\Qreq{1}{Ci siamo prefissati di soddisfare un numero stabilito di requisiti desiderabili}{Interno QPR006}
			\Qreq{1}{Nessun rischio non verificato precedentemente dovrebbe accadere nel corso del progetto}{Interno QPR007}
			\Qreq{1}{Ogni documento dovrebbe attraversare tutte le fasi previste dal suo \gloss{ciclo di vita}}{Interno QPR008}
			\Qreq{1}{Viene stabilito il numero massimo di modifiche che può ricevere un prodotto prima di essere verificato}{Interno QPR009}
			\Qreq{1}{La fase di verifica di tutti i vari prodotti dovrebbe sempre essere eseguita in modo corretto}{Interno QPR010}
			\Qreq{2}{Tutte le \gloss{norme} inserite nelle \NdP\ devono essere rispettate}{Interno}
			\Qreq{2}{Tutti i vincoli presenti nel \PdQ\ devono essere rispettati}{Interno}

			\bottomrule
		\end{paddedtablex}
		\caption{Elenco dei requisiti di qualità (1)}
	\end{table}



	\begin{table}[H]
		\begin{paddedtablex}[1.7]{\textwidth}{cXc}
			\textbf{Codice} & \textbf{Requisito} & \textbf{Fonte} \\\toprule
			\Qreq{2}{Le applicazioni sviluppate devono rispettare i fattori trattati in The Twelve-Factor App segnati nel \PdQd}{Capitolato \Doc{VE\_2018-12-12}}
			\Qreq{2}{È necessario presentare il \gloss{bug} reporting per ogni componente}{Capitolato}
			\Qreq{2}{Deve essere redatta la documentazione sulle scelte progettuali effettuate}{Capitolato}
			\Qreq{2}{Ogni scelta descritta nella documentazione deve essere correlata dalle relative motivazioni}{Capitolato}
			\Qsubreq{2}{È necessario testare ogni prodotto software considerando ogni sistema di riferimento e interazione tra le sue parti, perciò con test d'unita, d'integrazione e di sistema}{Capitolato}
			\req{R\thevaQ.1Q2}{È necessario fornire test unitari per ogni componente applicativo}{Capitolato}
			\req{R\thevaQ.2Q2}{È necessario fornire test d'integrazione per ogni componente applicativo}{Capitolato}
			\req{R\thevaQ.3Q2}{È necessario testare interamente il sistema con test di sistema}{Capitolato}

			\bottomrule \\
		\end{paddedtablex}
		\caption{Elenco dei requisiti di qualità (2)}
	\end{table}

	% Generazione automatica dei numeri
	\newcounter{vaV} % valore
	\newcommand{\decrV}{\addtocounter{vaV}{+1}} % Comando per l'aumento automatico del counter vaZ
	\newcommand{\addVNumber}[0]{\thevaV \decrV} % Comando per generare
	\addtocounter{vaV}{1}

	\newcommand{\Vreq}[3]{R\addVNumber V#1 & #2 & #3 \\}
	\newcommand{\Vsubreq}[3]{R\thevaV V#1 & #2 & #3 \\}

	%TODO: I requisiti di README vanno in vincolo o qualità? E quelli dei problemi?
	\subsection{Requisiti di vincolo}

	\begin{table}[H]
		\begin{paddedtablex}[1.7]{\textwidth}{cXc} %\rowcolors{1}{\tablegray}{\lightgray}
			\textbf{Codice} & \textbf{Requisito} & \textbf{Fonte} \\
			\toprule
			%R\addVNumber
			%V2 & Le componenti Producer devono riuscire a recuperare delle segnalazioni & Capitolato \\ % stesso di "L'applicativo Consumer/Producer deve riuscire a recuperare i messaggi da un \gloss{Topic}"
			%			R\addVNumber
			%			V2 & Le segnalazioni devono poter essere accentrate e standardizzate & Capitolato \\

			\Vsubreq{2}{Devono essere sviluppati due componenti applicativi Producer tra Redmine, GitLab e SonarQube}{Capitolato}
			\req{R\addVNumber.1V0}{È possibile avere un terzo componente applicativo Producer oltre ai due obbligatori}{Capitolato}
			\Vsubreq{2}{Devono essere sviluppati due componenti applicativi Consumer tra Telegram, Email e Slack}{Capitolato}
			\req{R\addVNumber.1V0}{È possibile avere un terzo componente applicativo Consumer oltre ai due obbligatori}{Capitolato}
			\Vsubreq{2}{Docker deve essere la tecnologia di riferimento per l'istanziazione di tutte le componenti}{Capitolato}
			\req{R\addVNumber.1V2}{È necessario presentare un Dockerfile per ogni componente}{Capitolato}
			\Vsubreq{1}{Per ogni problema aperto documentato è possibile allegare delle soluzioni da attuare in futuro}{Capitolato}
			\req{R\addVNumber.1}{Deve essere redatta una documentazione su eventuali problemi riscontrati rimasti ancora aperti al termine del progetto}{Capitolato}

			\bottomrule\\
		\end{paddedtablex}
		\caption{Elenco dei requisiti di vincolo (1)}
	\end{table}

	\begin{table}[H]
		\begin{paddedtablex}[1.7]{\textwidth}{cXc}
			\textbf{Codice} & \textbf{Requisito} & \textbf{Fonte} \\
			\toprule

			\Vsubreq{2}{È necessario presentare un file \gloss{README} per ogni componente}{Capitolato}
			\req{R\thevaV.1V2}{I file README delle componenti applicative devono contenere la documentazione delle \gloss{API} esposte dal servizio}{Capitolato}
			\req{R\thevaV.2V2}{I file README delle componenti applicative devono contenere le istruzioni per il loro utilizzo}{Capitolato}
			%TODO: R.. & I DockerFile devono essere comprensivi di script e file cfg..ecc?
			\req{R\thevaV.3V2}{È necessario presentare un file README per il Dockerfile}{Capitolato}
			\req{R\thevaV.4V2}{Il file README per il Dockerfile deve contenere le istruzioni per l'avvio}{Capitolato}
			\req{R\addVNumber.5V2}{Il file README per il Dockerfile deve contenere la documentazione delle configurazioni custom scelte}{Capitolato}
			\Vreq{2}{Le componenti devono esporre delle \gloss{API Rest} per le interazioni con le altre componenti}{Capitolato}
			\Vreq{0}{Come meccanismo di estensione per GitLab è possibile fare uso di Webhooks}{Capitolato}
			\Vreq{1}{Per lo sviluppo dei componenti applicativi è possibile usare come linguaggio \gloss{Java} 8 o una versione più recente, \gloss{Python} o \gloss{Node.js}}{Capitolato}
			\Vreq{1}{È possibile utilizzare Apache Kafka come Broker}{Capitolato}

			\bottomrule\\
		\end{paddedtablex}
		\caption{Elenco dei requisiti di vincolo (2)}
	\end{table}



	\subsection{Tracciamento}

		\subsubsection{Tracciamento fonti-requisiti}

		\begin{table}[H]
			\centering
			{\def\arraystretch{1.4}
			\begin{tabularx}{\textwidth}{YY}
				\textbf{Fonte} & \textbf{Requisito} \\
				\toprule
				\cellcolor{white} & R7F2 \\
				\cellcolor{white} & R8F2 \\
				\cellcolor{white} & R9F2 \\
				\cellcolor{white} & R10F2 \\
				\cellcolor{white} & R11F2 \\
				\cmidrule{2-2}
				\cellcolor{white} & R14Q2 \\
				\cellcolor{white} & R15Q2 \\
				\cellcolor{white} & R16Q2 \\
				\cellcolor{white} & R17Q2 \\
				\cellcolor{white} & R18Q2 \\
				\cellcolor{white} & R18.1Q2 \\
				\cellcolor{white} & R18.2Q2 \\
				\cellcolor{white} \multirow{-13}{*}{Capitolato} & R18.3Q2 \\
				\bottomrule\\
			\end{tabularx}}
			\caption{Elenco dei requisiti del capitolato (1)}
		\end{table}

		\begin{table}[H]
			\centering
			{\def\arraystretch{1.4}
			\begin{tabularx}{\textwidth}{YY}
				\textbf{Fonte} & \textbf{Requisito} \\
				\toprule
		        \cellcolor{white} & R1V2 \\
				\cellcolor{white} & R1.1V0 \\
				\cellcolor{white} & R2V2 \\
				\cellcolor{white} & R2.1V0 \\
				\cellcolor{white} & R3V2 \\
				\cellcolor{white} & R3.1V2 \\
				\cellcolor{white} & R4V1 \\
				\cellcolor{white} & R4.1V2 \\
				\cellcolor{white} & R5V2 \\
				\cellcolor{white} & R5.1V2 \\
				\cellcolor{white} & R5.2V2 \\
				\cellcolor{white} & R5.3V2 \\
				\cellcolor{white} & R5.4V2 \\
				\cellcolor{white} & R5.5V2 \\
				\cellcolor{white} & R6V2 \\
				\cellcolor{white} & R7V0 \\
				\cellcolor{white} & R8V1 \\
				\cellcolor{white} \multirow{-18}{*}{Capitolato} & R9V1 \\
				\bottomrule\\
			\end{tabularx}}
			\caption{Elenco dei requisiti del capitolato (2)}
		\end{table}

		\begin{table}[H]
			\centering
			{\def\arraystretch{1.5}
				\begin{tabularx}{\textwidth}{YY}
					\textbf{Fonte} & \textbf{Requisito} \\
					\toprule
					\cellcolor{white} & R11Q1 \\
					\cellcolor{white} & R12Q2 \\
					\cellcolor{white} \multirow{-2}{*}{Interno} & R13Q2 \\
					\bottomrule
				\end{tabularx}}
			\caption{Elenco dei requisiti interni}
		\end{table}

		\begin{table}[H]
			\centering
			\rowcolors{2}{white}{\tablegray}
			{\def\arraystretch{1.5}
			\begin{tabularx}{\textwidth}{YY}
				\textbf{Fonte} & \textbf{Requisito} \\
				\toprule
				UC1 & R1F2 \\
				UC2 & R2F2 \\
				UC3 & R3F2 \\
				UC4 & R4F2 \\
				UC5.1 & R5.1F0 \\
				UC5.1.1 & R5.1.1F0 \\
				UC5.2 & R5.2F0 \\
				UC6.1.1 & R6.1.1F0 \\
				UC6.1.2 & R6.1.2F0 \\
				UC6.1.3 & R6.1.3F0 \\
				UC6.2.1 & R6.2.1F0 \\
				UC6.2.2 & R6.2.2F0 \\
				UC6.2.3 & R6.2.3F0 \\
				\bottomrule \\
			\end{tabularx}}
			\caption{Elenco dei requisiti per i casi d'uso}
		\end{table}

		\begin{table}[H]
		\centering
		\rowcolors{2}{white}{\tablegray}
		{\def\arraystretch{1.5}
		\begin{tabularx}{\textwidth}{YY}
			\textbf{Fonte} & \textbf{Requisito} \\
			\toprule
			QPR001 & R1Q1 \\
			QPD001 & R2Q1 \\
			QPR002 & R3Q1 \\
			QPR003 & R4Q1 \\
			QPR004 & R5Q1 \\
			QPR005 & R6Q2 \\
			QPR006 & R7Q1 \\
			QPR007 & R8Q1 \\
			QPR008 & R9Q1 \\
			QPR009 & R10Q1 \\
			QPR010 & R11Q1 \\
			\Doc{VE\_2018-12-12} & R14Q2 \\
			\bottomrule\\
		\end{tabularx}}
		\caption{Elenco dei requisiti per gli obiettivi di qualità e verbali}
	\end{table}

\newcounter{V} % valore
\newcommand{\deV}{\addtocounter{V}{+1}} % Comando per l'aumento automatico del counter vaZ
\newcommand{\addC}[0]{\theV \deV} % Comando per generare
\addtocounter{V}{1}

\newcounter{Vv} % valore
\newcommand{\deVv}{\addtocounter{Vv}{+1}} % Comando per l'aumento automatico del counter vaZ
\newcommand{\addVC}[0]{\theVv \deVv} % Comando per generare
\addtocounter{Vv}{1}

\newcounter{X} % valore
\newcommand{\deX}{\addtocounter{X}{+1}} % Comando per l'aumento automatico del counter vaZ
\newcommand{\addX}[0]{\theX \deX} % Comando per generare
\addtocounter{X}{1}

		\subsubsection{Tracciamento requisiti-fonte}

		\begin{table}[H]
		\begin{paddedtablex}[1.7]{\textwidth}{YY}
			\textbf{Requisito} & \textbf{Fonte} \\\toprule
			%R di F
			R\addC
			F2 & Interno UC1 \\
			R\addC
			F2 & Interno UC2 \\
			R\addC
			F2 & Interno UC3 \\
			R\addC
			F2 & Interno UC4 \\
			R\addC
			.1F0 & Interno UC5.1 \\
			R5.1.1F0 & Interno UC5.1.1 \\
			R5.2F0 & Interno UC5.2 \\
			R\addC
			.1.1F0 & Interno UC6.1.1 \\
			R6.1.2F0 & Interno UC6.1.2 \\
			R6.1.3F0 & Interno UC6.1.3 \\
			R6.2.1F0 & Interno UC6.2.1 \\
			R6.2.3F0 & Interno UC6.2.2 \\
			R6.2.3F0 & Interno UC6.2.3 \\
			R\addC
			F2 & Capitolato \\
			R\addC
			F2 & Capitolato \\
			R\addC
			F2 & Capitolato \\
			R\addC
			F2 & Capitolato \\
			R\addC
			F2 & Capitolato \\
			\bottomrule
		\end{paddedtablex}
		\caption{Elenco dei requisiti funzionali in rapporto alle fonti}
		\end{table}

		\begin{table}[H]
		\begin{paddedtablex}[1.7]{\textwidth}{YY}
			\textbf{Requisito} & \textbf{Fonte} \\\toprule
			%R di Q
			R\addVC
			Q1 & Interno QPR001 \\
			R\addVC
			Q1 & Interno QPD001 \\
			R\addVC
			Q1 & Interno QPR002 \\
			R\addVC
			Q1 & Interno QPR003 \\
			R\addVC
			Q1 & Interno QPR004 \\
			R\addVC
			Q2 & Interno QPR005 \\
			R\addVC
			Q1 & Interno QPR006 \\
			R\addVC
			Q1 & Interno QPR007 \\
			R\addVC
			Q1 & Interno QPR008 \\
			R\addVC
			Q1 & Interno QPR009 \\
			R\addVC
			Q2 & Interno QPR010 \\
			R\addVC
			Q2 & Interno \\
			R\addVC
			Q2 & Interno \\
			R\addVC
			Q2 & \Doc{VE\_2018-12-12} \\
			R\addVC
			Q2 & Capitolato \\
			R\addVC
			Q2 & Capitolato \\
			R\addVC
			Q2 & Capitolato \\
			R\addVC
			Q2 & Capitolato \\
			R18.1Q2 & Capitolato \\
			R18.2Q2 & Capitolato \\
			R18.3Q2 & Capitolato \\
			\bottomrule
			\end{paddedtablex}
		\caption{Elenco dei requisiti di qualità in rapporto alle fonti}
	\end{table}

\begin{table}[H]
	\begin{paddedtablex}[1.7]{\textwidth}{YY}
		\textbf{Requisito} & \textbf{Fonte} \\\toprule
			%R di V
			R\addX
			V2 & Capitolato \\
			R1.1V0 & Capitolato \\
			R\addX
			V2 & Capitolato \\
			R2.1V0 & Capitolato \\
			R\addX
			V2 & Capitolato \\
			R3.1V2 & Capitolato \\
			R\addX
			V1 & Capitolato \\
			R4.1V2 & Capitolato \\
			R\addX
			V2 & Capitolato \\
			R5.1V2 & Capitolato \\
			R5.2V2 & Capitolato \\
			R5.3V2 & Capitolato \\
			R5.4V2 & Capitolato \\
			R5.5V2 & Capitolato \\
			R\addX
			V2 & Capitolato \\
			R\addX
			V0 & Capitolato \\
			R\addX
			V1 & Capitolato \\
			R\addX
			V1 & Capitolato \\
			\bottomrule \\
		\end{paddedtablex}
		\caption{Elenco dei requisiti di vincolo in rapporto alle fonti}
	\end{table}

	\subsection{Riepilogo}

		\begin{table}[H]
		\begin{paddedtablex}[1.7]{\textwidth}{YYYY}
			\textbf{Tipologia} & \textbf{Obbligatori} & \textbf{Desiderabili} & \textbf{Opzionali} \\\toprule
			Di funzionalità & 8 & 0 & 10 \\
			Di qualità & 12 & 9 & 0 \\
			Di vincolo & 12 & 3 & 3
			\\\bottomrule
		\end{paddedtablex}
		\caption{Riepilogo dei requisiti}
		\end{table}

	\subsection{Metriche}
	Dato che non è fisso il numero dei requisiti di un progetto, abbiamo scelto una serie di metriche dove il valore ottimale da raggiungere è sempre uguale, lo zero. Abbiamo anche scelto di contare il numero di requisiti non soddisfatti invece che il contrario. Lo stesso ragionamento è valido per quanto	riguarda i rischi che possono verificarsi nel corso del progetto.

	Gli obiettivi che si vogliono raggiungere attraverso tali metriche possono essere stabiliti solo a progetto concluso.

	La denominazione delle metriche è descritta nel \NdPd.

		\subsubsection{MPR005 Requisiti obbligatori non soddisfatti}
		Per adempiere completamente alla richiesta del cliente, ci serve individuare tutti i requisiti presenti nella sua richiesta, impliciti, espliciti, diretti e derivati. Alcuni sono imprescindibili, detti obbligatori, e il loro soddisfacimento determina la buona riuscita del progetto.

		\textbf{Metrica}: numero dei requisiti obbligatori non soddisfatti.

		\subsubsection{MPR006 Requisiti desiderabili non soddisfatti}
		I requisiti desiderabili non sono necessari, ma offrono un valore aggiunto al progetto.

		\textbf{Metrica}: numero dei requisiti desiderabili non soddisfatti.

		\subsubsection{MPR007 Requisiti opzionali non soddisfatti}
		Tali requisiti dovranno essere adempiuti solo nel momento in cui tutti i requisiti obbligatori saranno soddisfatti.
		Possono essere concordati col cliente in corso d'opera.

		\textbf{Metrica}: numero dei requisiti opzionali non soddisfatti.

