\newpage
\section{Requisiti}
Ad ogni requisito viene assegnato il codice identificativo univoco:
	\begin{center}
		\texttt{R[Numero][Tipo][Priorità]}
	\end{center}
	in cui ogni parte ha un significato preciso:
	\begin{itemize}
		\item \textbf{R}: requisito.
		\item \textbf{Numero}: numero progressivo che segue la struttura dei documenti.
		\item \textbf{Tipo}: la la tipologia di requisito che può essere di:
		\begin{itemize}
			\item \textbf{F}: funzionalità.
			\item \textbf{Q}: \gloss{qualità}.
			\item \textbf{V}: vincolo.
		\end{itemize}
		\item \textbf{Priorità}: indica il grado di urgenza di un requisito di essere soddisfatto, come:
		\begin{itemize}
			\item \textbf{0}: opzionale.
			\item \textbf{1}: desiderabile.
			\item \textbf{2}: obbligatorio.
		\end{itemize}
	\end{itemize}


	Esempio: \texttt{R2Q1} indica il secondo requisito di qualità ed è desiderabile.

	%I requisiti di seguito riportati sono elencati in modo tale da seguire la struttura dei documenti. Ovvero, si possono trovare raggruppati i requisiti derivanti dello stesso tipo, ad esempio solo i requisiti di funzionalità dal sesto al diciannovesimo derivano dai casi d'uso.

%inserire il fatto che una persona può aggiungere il proprio nickname o altro da interfaccia, come requisito opzionale	...

% Generazione automatica dei numeri
%\newcounter{vaZ} % valore
%\newcommand{\incrZ}{\addtocounter{vaZ}{+1}} % Comando per l'aumento automatico del counter vaZ
%\newcommand{\addZNumber}[0]{\incrZ\thevaZ} % Comando per generare
%
%\newcommand{\Freq}[3]{R\addZNumber F#1 & #2 & #3 \\}
%\newcommand{\Fsubreq}[3]{R\thevaZ F#1 & #2 & #3 \\} % Se c'è bisogno di sottocasi

% Comando requisito generico
\newcommand{\req}[3]{%
#1 & #2 & #3 \\
}

	%COMANDI PER REQ DI FUNZIONALITÀ
	% Generazione automatica dei numeri
	\newcounter{vaF} % valore
	\newcounter{secF}[vaF] % per il secondo livello del requisito
	\newcounter{thF}[secF] % terzo livello

	\newcommand{\ReqF}[3]{\stepcounter{vaF}R\thevaF F#1 & #2 & #3 \\} % Primo livello
	\newcommand{\subReqF}[3]{\stepcounter{secF}R\thevaF.\thesecF F#1 & #2 & #3 \\} % Secondo livello
	%\newcommand{\subsubReqF}[3]{\stepcounter{thF}R\thevaF.\thesecF.\thethF F#1 & #2 & #3 \\} % Terzo livello

	\newcounter{tableCounter} % Per automatizzare conteggio tabelle

	\subsection{Requisiti di funzionalità}\label{RequisitiFunzionalità}

	\stepcounter{tableCounter}
	\begin{table}[H]
		\begin{paddedtablex}[1.7]{\textwidth}{cXc}%0 opz  2 obb
			\rowcolor{white} \textbf{Codice} & \textbf{Requisito} & \textbf{Fonte} \\\toprule

			% \stepcounter{vaF} % Per allineare i requisiti ai casi d'uso

			% da casi d'uso
			\rowcolor{\tablegray}\ReqF{2}{Redmine deve poter inviare una segnalazione al Producer Redmine}{Interno UC1-PR}
                \rowcolor{white}\subReqF{2}{Redmine deve poter inviare la segnalazione di apertura issue al Producer Redmine}{Interno UC1.1-PR}
                \rowcolor{\tablegray}\subReqF{2}{Redmine deve poter inviare la segnalazione di modifica issue al Producer Redmine}{Interno UC1.2-PR}
                \rowcolor{white}\subReqF{2}{Redmine deve poter inviare la segnalazione di commento di una issue al Producer Redmine}{Interno UC1.3-PR}
            \rowcolor{\tablegray}\ReqF{2}{GitLab deve essere in grado di inviare una segnalazione al Producer GitLab}{Interno UC2-PG}
			     \rowcolor{white}\subReqF{2}{GitLab deve essere in grado di segnalare l'apertura di issue al Producer GitLab}{Interno UC2.1-PG}
			     \rowcolor{\tablegray}\subReqF{2}{GitLab deve essere in grado di segnalare la modifica issue al Producer GitLab}{Interno UC2.2-PG}
                 \rowcolor{white}\subReqF{2}{GitLab deve essere in grado di segnalare il commento di una issue al Producer GitLab}{Interno UC2.3-PG}
			     \rowcolor{\tablegray}\subReqF{2}{GitLab deve poter segnalare un evento di push al Producer GitLab}{Interno UC2.4-PG}
                 \rowcolor{white}\subReqF{2}{GitLab deve poter segnalare un evento di commento di commit al Producer GitLab}{Interno UC2.5-PG}
			\rowcolor{\tablegray}\ReqF{2}{Il Producer Redmine deve essere in grado di inviare un messaggio al Gestore Personale}{Interno UC3-GP}
				\rowcolor{white}\subReqF{2}{Il Producer Redmine deve essere in grado di inviare un messaggio di apertura issue al Gestore Personale}{Interno UC3.1-GP}
				\rowcolor{\tablegray}\subReqF{2}{Il Producer Redmine deve essere in grado di inviare un messaggio di modifica issue al Gestore Personale}{Interno UC3.2-GP}
                \rowcolor{white}\subReqF{2}{Il Producer Redmine deve essere in grado di inviare un messaggio di commento issue al Gestore Personale}{Interno UC3.3-GP}
			\bottomrule
		\end{paddedtablex}
		\caption{Elenco dei requisiti di funzionalità (\thetableCounter)}
	\end{table}

	\stepcounter{tableCounter}
	\begin{table}[H]
		\begin{paddedtablex}[1.7]{\textwidth}{cXc}%0 opz  2 obb
			\rowcolor{white}\textbf{Codice} & \textbf{Requisito} & \textbf{Fonte} \\\toprule

            \rowcolor{\tablegray}\ReqF{2}{Il Producer GitLab deve essere in grado di inviare un messaggio al Gestore Personale}{Interno UC4-GP}
                \rowcolor{white}\subReqF{2}{Il Producer GitLab deve essere in grado di inviare un messaggio di push al Gestore Personale}{Interno UC4.1-GP}
                \rowcolor{\tablegray}\subReqF{2}{Il Producer GitLab deve essere in grado di inviare un messaggio di nuova issue al Gestore Personale}{Interno UC4.2-GP}
                \rowcolor{white}\subReqF{2}{Il Producer GitLab deve essere in grado di inviare un messaggio di modifica issue al Gestore Personale}{Interno UC4.3-GP}
                \rowcolor{\tablegray}\subReqF{2}{Il Producer GitLab deve essere in grado di inviare un messaggio di commento di issue al Gestore Personale}{Interno UC4.4-GP}
                \rowcolor{white}\subReqF{2}{Il Producer GitLab deve essere in grado di inviare messaggi di commento di commit al Gestore Personale}{Interno UC4.5-GP}
            \rowcolor{\tablegray}\ReqF{2}{Il Gestore Personale deve poter inviare il messaggio finale al Consumer Telegram}{Interno UC5-CT}
			\rowcolor{white}\ReqF{2}{Il Gestore Personale deve poter inviare il messaggio finale al Consumer Email}{Interno UC6-CE}
			\rowcolor{\tablegray}\ReqF{2}{Il Consumer Telegram deve poter inoltrare il messaggio finale al bot Telegram}{Interno UC7-BT}
			\rowcolor{white}\ReqF{2}{Il Consumer Email deve poter inoltrare il messaggio finale al server Email}{Interno UC8-SE}
			% \stepcounter{vaF} e usare \subReqF per rientrare di un livello
			\rowcolor{\tablegray}\ReqF{0}{L'utente può eseguire l'accesso al Gestore Personale}{Interno UC9.1-GP}
				\rowcolor{white}\subReqF{0}{L'utente può inserire il proprio identificativo all'interno del sistema}{Interno UC9.1.1-GP}
			\rowcolor{\tablegray}\ReqF{0}{\progetto\ deve saper notificare l'utente se il tentativo di accesso non è andato a buon fine}{Interno UC9.2-GP}
			\rowcolor{white}\ReqF{0}{L'utente acceduto deve poter uscire dal sistema}{Interno UC10-GP}
			\bottomrule\\
		\end{paddedtablex}
		\caption{Elenco dei requisiti di funzionalità (\thetableCounter)}
	\end{table}

	\stepcounter{tableCounter}
	\begin{table}[H]
		\begin{paddedtablex}[1.7]{\textwidth}{cXc}%0 opz  2 obb
			\rowcolor{white}\textbf{Codice} & \textbf{Requisito} & \textbf{Fonte} \\\toprule

            \rowcolor{\tablegray}\ReqF{0}{L'utente acceduto deve poter aggiungere un nuovo utente}{Interno UC11.1-GP}
                \rowcolor{white}\subReqF{0}{L'utente acceduto deve poter inserire il nome dell'utente da aggiungere}{Interno UC11.1.1-GP}
                \rowcolor{\tablegray}\subReqF{0}{L'utente acceduto deve poter inserire il cognome dell'utente da aggiungere}{Interno UC11.1.2-GP}
                \rowcolor{white}\subReqF{0}{L'utente acceduto deve poter inserire il contatto Email dell'utente da aggiungere}{Interno UC11.1.3-GP}
                \rowcolor{\tablegray}\subReqF{0}{L'utente acceduto deve poter inserire il contatto Telegram dell'utente da aggiungere}{Interno UC11.1.4-GP}
            \rowcolor{white}\ReqF{0}{L'utente acceduto deve poter visualizzare un messaggio di errore se il nuovo contatto Telegram inserito corrisponde a quello di un utente già registrato nel sistema}{Interno UC11.2-GP}
            \rowcolor{\tablegray}\ReqF{0}{L'utente acceduto deve poter visualizzare un messaggio di errore se il nuovo contatto Email inserito corrisponde a quello di un utente già registrato nel sistema}{Interno UC11.2-GP}
            \rowcolor{white}\ReqF{0}{L'utente acceduto deve poter rimuovere un utente presente nel sistema}{Interno UC12.1-GP}
                \rowcolor{\tablegray}\subReqF{0}{L'utente acceduto deve poter inserire il contatto Email dell'utente da rimuovere}{Interno UC12.1.1-GP}
                \rowcolor{white}\subReqF{0}{L'utente acceduto deve poter inserire il contatto Telegram dell'utente da rimuovere}{Interno UC12.1.2-GP}
            \rowcolor{\tablegray}\ReqF{0}{L'utente acceduto deve poter visualizzare un messaggio di errore se l'utente da rimuovere non è presente nel sistema}{Interno UC12.2-GP}
			\rowcolor{white}\ReqF{0}{L'utente acceduto deve poter modificare le proprie informazioni}{Interno UC13.1-GP}
				\rowcolor{\tablegray}\subReqF{0}{L'utente acceduto deve poter modificare il nome}{Interno UC13.1.1-GP}
				\rowcolor{white}\subReqF{0}{L'utente acceduto deve poter modificare il cognome}{Interno UC13.1.2-GP}
				\rowcolor{\tablegray}\subReqF{0}{L'utente acceduto deve poter modificare il contatto Email}{Interno UC13.1.3-GP}
                \rowcolor{white}\subReqF{0}{L'utente acceduto deve poter scegliere il nuovo contatto Telegram}{Interno UC13.1.4-GP}
			\rowcolor{\tablegray}\ReqF{0}{L'utente acceduto deve poter visualizzare un messaggio di errore se il contatto Email inserito corrisponde a quello di un utente presente nel sistema}{Interno UC13.2-GP}
			\bottomrule
		\end{paddedtablex}
		\caption{Elenco dei requisiti di funzionalità (\thetableCounter)}
	\end{table}

	\stepcounter{tableCounter}
	\begin{table}[H]
		\begin{paddedtablex}[1.7]{\textwidth}{cXc}%0 opz  2 obb
			\textbf{Codice} & \textbf{Requisito} & \textbf{Fonte} \\\toprule
            \ReqF{0}{L'utente acceduto deve poter visualizzare un messaggio di errore se il contatto Telegram inserito corrisponde a quello di un utente presente nel sistema}{Interno UC13.2-GP}
            \ReqF{0}{L'utente acceduto deve poter aggiungere le proprie preferenze nel sistema}{Interno UC14-GP}
                \subReqF{0}{L'utente acceduto deve poter aggiungere nuovi Topic}{Interno UC14.1-GP}
                \subReqF{0}{L'utente acceduto deve poter aggiungere nuovi giorni di indisponibilità nel calendario}{Interno UC14.2-GP}
                \subReqF{0}{L'utente acceduto deve poter aggiungere una nuova piattaforma di messaggistica in cui ricevere le notifiche}{Interno UC14.3-GP}
				\subReqF{0}{L'utente acceduto deve poter aggiungere nuove keyword di interesse per i messaggi di commit di GitLab}{Interno UC14.4-GP}
				\subReqF{0}{L'utente acceduto deve poter visualizzare un messaggio di errore se la keyword inserita era già nella sua lista}{Interno UC14.5-GP}
                \subReqF{0}{L'utente acceduto deve poter aggiungere i progetti a cui è interessato}{Interno UC14.6-GP}
                \subReqF{0}{L'utente acceduto deve poter aggiungere la priorità ai progetti a cui è interessato}{Interno UC14.7-GP}
			\ReqF{0}{L'utente acceduto deve poter rimuovere le proprie preferenze dal sistema}{Interno UC15-GP}
				\subReqF{0}{L'utente acceduto deve poter rimuovere i Topic a cui è iscritto}{Interno UC15.1-GP}
				\subReqF{0}{L'utente acceduto deve poter rimuovere giorni di indisponibilità nel calendario}{Interno UC15.2-GP}
				\subReqF{0}{L'utente acceduto deve poter rimuovere una piattaforma di messaggistica in cui ricevere le notifiche}{Interno UC15.3-GP}
				\subReqF{0}{L'utente acceduto deve poter rimuovere keyword di interesse per i messaggi di commit di GitLab}{Interno UC15.4-GP}
				\subReqF{0}{L'utente acceduto deve poter visualizzare un messaggio di errore se la keyword da rimuovere è assente dalla sua lista}{Interno UC15.5-GP}
                \subReqF{0}{L'utente acceduto vuole rimuovere un progetto a cui non è più interessato}{Interno UC15.6-GP}
                \subReqF{0}{L'utente acceduto vuole rimuovere la priorità di un progetto a cui è interessato}{Interno UC15.7-GP}
			\bottomrule\\
		\end{paddedtablex}
		\caption{Elenco dei requisiti di funzionalità (\thetableCounter)}
	\end{table}
			% \Finitsecondreq{0}{L'utente può eseguire l'accesso al gestore personale}{Interno UC5.1}
			% \Finitthirdreq{0}{L'utente può inserire il proprio username che lo identifica all'interno di \progetto}{Interno UC5.1.1}
			% \Fsecondreq{0}{\progetto\ fa apparire un messaggio di errore se il tentativo di accesso non è andato a buon fine}{Interno UC5.2}
			% \req{R\addFNumber.1.1F0}{L'utente può iscriversi a un Topic}{Interno UC6.1.1}
			% \req{R\thevaF.1.2F0}{L'utente può aggiungere nel gestore personale i giorni in cui non è reperibile}{Interno UC6.1.2}
			% \req{R\thevaF.1.3F0}{L'utente può scegliere la piattaforma di messaggistica, tra Telegram ed e-mail, in cui ricevere le notifiche}{Interno UC6.1.3}
			% \req{R\thevaF.2.1F0}{L'utente può disiscriversi da un Topic}{Interno UC6.2.1}
			% \req{R\thevaF.2.2F0}{L'utente può rimuovere i giorni in cui aveva selezionato precedentemente di non essere reperibile}{Interno UC6.2.2}
			% \req{R\thevaF.2.3F0}{L'utente può togliere le proprie preferenze sulle piattaforme di messaggistica da cui ricevere notifiche inviate attraverso \progetto}{Interno UC6.2.3}


	% \stepcounter{tableCounter}
	% \begin{table}[H]
	% 	\begin{paddedtablex}[1.7]{\textwidth}{cXc}
	% 		\textbf{Codice} & \textbf{Requisito} & \textbf{Fonte} \\\toprule

	% 		% da capitolato
	% 		% stesso di "L'applicativo Producer/Consumer deve essere in grado di reindirizzare i messaggi verso la persona più appropriata"
	% 		%\ReqF{2}{Le componenti devono esporre delle \gloss{API Rest} per le interazioni con le altre componenti}{Capitolato}

	% 		\bottomrule
	% 	\end{paddedtablex}
	% 	\caption{Elenco dei requisiti di funzionalità (\thetableCounter)}
	% \end{table}



	%COMANDI PER REQUISITI DI Qualità
	% Generazione automatica dei numeri
	\newcounter{vaQ} % valore
	\newcounter{secQ}[vaQ]
	\newcounter{thQ}[secQ] % terzo livello

	\newcommand{\ReqQ}[3]{\stepcounter{vaQ}R\thevaQ Q#1 & #2 & #3 \\}
	\newcommand{\subReqQ}[3]{\stepcounter{secQ}R\thevaQ.\thesecQ Q#1 & #2 & #3 \\}
	\newcommand{\subsubReqQ}[3]{\stepcounter{thQ}R\thevaQ.\thesecQ.\thethQ Q#1 & #2 & #3 \\} % Terzo livello




	\setcounter{tableCounter}{1}
	% NB: molti requisiti di qualità sono stati tolti perchè non sono veri requisiti del sistema, riguardano noi e il nostro modo di fare, non il sistema
	\subsection{Requisiti di qualità}\label{RequisitiQualità}

	\begin{table}[H]
		\begin{paddedtablex}[1.7]{\textwidth}{cXc}
			\rowcolor{white}\textbf{Codice} & \textbf{Requisito} & \textbf{Fonte} \\
			\toprule
			% presi dal PdQ
			\rowcolor{\tablegray}\ReqQ{1}{È stabilito un numero massimo di giorni di ritardo per la chiusura di una issue}{Interno QPR001}
			\rowcolor{white}\ReqQ{1}{L'\gloss{indice di Gulpease} di ogni documento deve rientrare all'interno di un intervallo stabilito}{Interno QPD001}
			%\Qreq{1}{I costi previsti dalla \gloss{pianificazione} non dovrebbero variare più di quanto stabilito}{Interno QPR002}
			\rowcolor{\tablegray}\ReqQ{1}{Una frequenza minima di commit devono essere effettuati in una settimana}{Interno QPR004}
            \rowcolor{white}\ReqQ{1}{Tutte le norme scelte dallo standard PEP8 devono essere rispettate}{Interno QPS005}
			\rowcolor{\tablegray}\ReqQ{1}{Un numero stabilito di requisiti desiderabili deve essere soddisfatto}{Interno QPR006}
			\rowcolor{white}\ReqQ{1}{Ogni documento deve attraversare tutte le fasi previste dal suo \gloss{ciclo di vita}}{Interno QPR008}
			\rowcolor{\tablegray}\ReqQ{1}{Viene stabilito il numero massimo di modifiche che può ricevere un prodotto prima di essere verificato}{Interno QPR009}
            \rowcolor{white}\ReqQ{1}{Ogni modulo deve essere prima progettato e solo successivamente codificato}{Interno QPR011}
            \rowcolor{\tablegray}\ReqQ{1}{L'intera progettazione del progetto non deve contenere un numero di pattern più alto di quanto segnalato nel \PdQd}{Interno QPR012}
            \rowcolor{white}\ReqQ{1}{Tutti i moduli devono possedere una suite di test allegata}{Interno QPR013}
			\rowcolor{\tablegray}\ReqQ{2}{Tutte le \gloss{norme} inserite nelle \NdPd\ devono essere rispettate}{Interno}
			\rowcolor{white}\ReqQ{2}{Tutti i vincoli presenti nel \PdQd\ devono essere rispettati}{Interno}
			\rowcolor{\tablegray}\subReqQ{2}{Le applicazioni sviluppate devono rispettare i fattori trattati in The Twelve-Factor App segnati nel \PdQd}{Capitolato \Doc{VE\_2018-12-12}}
			\rowcolor{white}\ReqQ{2}{È necessario presentare il \gloss{bug} reporting per ogni componente}{Capitolato}
			\ReqQ{2}{Deve essere redatta la documentazione sulle scelte progettuali effettuate}{Capitolato}
			%\req{R\thevaQ.1Q2}{Ogni scelta descritta nella documentazione deve essere correlata dalle relative motivazioni}{Capitolato}
			\subReqQ{2}{Ogni scelta descritta nella documentazione deve essere correlata dalle relative motivazioni}{Capitolato}

			\bottomrule
		\end{paddedtablex}
		\caption{Elenco dei requisiti di qualità (\thetableCounter)}
	\end{table}


	\stepcounter{tableCounter}
	\begin{table}[H]
		\begin{paddedtablex}[1.7]{\textwidth}{cXc}
			\textbf{Codice} & \textbf{Requisito} & \textbf{Fonte} \\\toprule


            \ReqQ{2}{È necessario testare ogni prodotto software considerando ogni sistema di riferimento e interazione tra le sue parti, perciò con test d'unità, d'integrazione e di sistema}{Capitolato}
            %\req{R\thevaQ.1Q2}{È necessario fornire test unitari per ogni componente applicativo}{Capitolato}
            \subReqQ{2}{È necessario fornire test d'unità per ogni componente applicativo}{Capitolato}
			\subReqQ{2}{È necessario fornire test d'integrazione per ogni componente applicativo}{Capitolato}
			\subReqQ{2}{È necessario testare interamente il sistema con test di sistema}{Capitolato}
			\ReqQ{1}{Per ogni problema aperto documentato, si allegano delle soluzioni da attuare in futuro}{Capitolato}
			\subReqQ{2}{Deve essere redatta una documentazione su eventuali problemi riscontrati rimasti ancora aperti al termine del progetto}{Capitolato}
			\ReqQ{2}{È necessario presentare un file \gloss{README} per ogni componente}{Capitolato}
			%\req{R\thevaQ.1Q2}{I file README delle componenti applicative devono contenere la documentazione delle \gloss{API} esposte dal servizio}{Capitolato}
			\subReqQ{2}{I file README delle componenti applicative devono contenere la documentazione delle \gloss{API} esposte dal servizio}{Capitolato}
			\subReqQ{2}{I file README delle componenti applicative devono contenere le istruzioni per il loro utilizzo}{Capitolato}
			\subReqQ{2}{È necessario presentare un file README per il Dockerfile}{Capitolato}
				\subsubReqQ{2}{Il file README per il Dockerfile deve contenere le istruzioni per l'avvio}{Capitolato}
				\subsubReqQ{2}{Il file README per il Dockerfile deve contenere la documentazione delle configurazioni custom scelte}{Capitolato}

			\bottomrule
		\end{paddedtablex}
		\caption{Elenco dei requisiti di qualità (\thetableCounter)}
	\end{table}

	% COMANDI PER REQ DI VINCOLO
	% Generazione automatica dei numeri

	\newcounter{vaV} % valore
	\newcounter{secV}[vaV]

	\newcommand{\ReqV}[3]{\stepcounter{vaV}R\thevaV V#1 & #2 & #3 \\}
	\newcommand{\subReqV}[3]{\stepcounter{secV}R\thevaV.\thesecV V#1 & #2 & #3 \\}

	% \newcommand{\incrV}{\addtocounter{vaV}{+1}} % Comando per l'aumento automatico del counter vaZ
	% \newcommand{\addVNumber}{\incrV\thevaV} % Comando per generare il valore incrementato di uno rispetto a prima

	% \newcounter{secondV} % per il secondo livello del requisito
	% \addtocounter{secondV}{1}
	% \newcommand{\secIncrV}{\addtocounter{secondV}{+1}} % Comando per l'aumento automatico del counter per il secondo livello
	% \newcommand{\addSecVNumber}{\secIncrV\thesecondV} % Comando per generare il valore incrementato di uno rispetto a prima
	% \newcommand{\resetVCounter}{\setcounter{secondV}{1}}
	% \newcommand{\decSecV}{\resetVCounter\thesecondV}


	% \newcommand{\Vreq}[3]{R\addVNumber V#1 & #2 & #3 \\} % Nuovo requisito, maggiore del precedente
	% \newcommand{\Vsubreq}[3]{R\thevaV V#1 & #2 & #3 \\} % Requisito diverso ma con stesso numero progressivo
	% \newcommand{\Vsecondreq}[3]{R\thevaV.\addSecVNumber V#1 & #2 & #3 \\}
	% \newcommand{\Vinitsecondreq}[3]{R\thevaV.\decSecV V#1 & #2 & #3 \\}

	\subsection{Requisiti di vincolo}\label{RequisitiVincolo}

	\begin{table}[H]
		\begin{paddedtablex}[1.7]{\textwidth}{cXc} %\rowcolors{1}{\tablegray}{\lightgray}
			\textbf{Codice} & \textbf{Requisito} & \textbf{Fonte} \\\toprule

			\ReqV{2}{Devono essere sviluppati due componenti applicativi Producer tra \redmine, \gitlab\ e SonarQube 6.7}{Capitolato}
			%\req{R\thevaV.1V0}
			\subReqV{0}{È possibile avere un terzo componente applicativo Producer oltre ai due obbligatori}{Capitolato}
			\ReqV{2}{Devono essere sviluppati due componenti applicativi Consumer tra \telegram, Email e Slack 3.3.8}{Capitolato}
			%\req{R\thevaV.1V0}
			\subReqV{0}{È possibile avere un terzo componente applicativo Consumer oltre ai due obbligatori}{Capitolato}
			\ReqV{2}{\docker\ deve essere la tecnologia di riferimento per l'istanziazione di tutte le componenti}{Capitolato}
			%\req{R\thevaV.1V2}
			\subReqV{2}{È necessario presentare un Dockerfile per ogni componente}{Capitolato}
			\ReqV{1}{Per lo sviluppo dei componenti applicativi è possibile usare come linguaggio \gloss{Java} 8 o una versione più recente, \gloss{\python} o \gloss{Node.js} 10.15.1}{Capitolato}
			\ReqV{1}{È possibile utilizzare \kafka\ come Broker}{Capitolato}
			\ReqV{0}{L'interfaccia è sviluppata utilizzando \html\ e \css}{Interno}
			\ReqV{0}{Il database viene sviluppato utilizzando \mongodb}{Interno}
			\ReqV{0}{Il server web viene sviluppato utilizzando \python\ con la libreria \gloss{CherryPy}}{Interno}
			\ReqV{0}{Gli URL dell'interfaccia web devono rispettare lo standard REST}{Interno}
			\ReqV{0}{Ciascun componente Docker viene istanziato tramite un file \dockercompose}{Interno}
			% I prossimi erano di funzionalità. Da ricontrollare
			\ReqV{2}{Le componenti Consumer devono essere in grado di inviare i messaggi provenienti da un Topic verso il corretto destinatario}{Capitolato}
			\ReqV{2}{Le componenti Consumer devono essere in grado di abbonarsi ai Topic scelti}{Capitolato}
			\ReqV{2}{Le segnalazioni devono poter essere gestite in maniera automatica e personalizzabile}{Capitolato}
            \ReqV{2}{Nel sistema deve essere presente un Broker che istanzia e gestisce le segnalazioni organizzandole per Topic}{Capitolato}
            \ReqV{2}{Le componenti Producer devono riuscire a pubblicare le segnalazioni recuperate sotto forma di messaggi secondo i Topic corretti}{Capitolato}
			\bottomrule
		\end{paddedtablex}
		\caption{Elenco dei requisiti di vincolo}
	\end{table}



	\subsection{Tracciamento}\label{Tracciamento}

		\subsubsection{Tracciamento fonte-requisito}

		% \begin{table}[H]
		% 	\centering
		% 	{\def\arraystretch{1.4}
		% 	\begin{oldtabularx}{\textwidth}{YY}
		% 		\textbf{Fonte} & \textbf{Requisito} \\
		% 		\toprule
		% 		\cellcolor{white} & R7F2 \\
		% 		\cellcolor{white} & R8F2 \\
		% 		\cellcolor{white} & R9F2 \\
		% 		\cellcolor{white} & R10F2 \\
		% 		\cellcolor{white} & R11F2 \\
		% 		% \cmidrule{2-2}
		% 		\cellcolor{white} & R14Q2 \\
		% 		\cellcolor{white} & R15Q2 \\
		% 		\cellcolor{white} & R16Q2 \\
		% 		\cellcolor{white} & R17Q2 \\
		% 		\cellcolor{white} & R18Q2 \\
		% 		\cellcolor{white} & R18.1Q2 \\
		% 		\cellcolor{white} & R18.2Q2 \\
		% 		\cellcolor{white} \multirow{-13}{*}{Capitolato} & R18.3Q2 \\
		% 		\bottomrule\\
		% 	\end{oldtabularx}}
		% 	\caption{Elenco dei requisiti del capitolato (1)}
		% \end{table}

		\setcounter{tableCounter}{1}
		\begin{table}[H]
			\centering
			{\def\arraystretch{1.6}
			\begin{oldtabularx}{0.7\textwidth}{YY}
				\textbf{Fonte} & \textbf{Requisito} \\
				\toprule
				& \cellcolor{\tablegray} R12.1Q2 \\
				& R13Q2 \\
				& \cellcolor{\tablegray} R14Q2 \\
				& R14.1Q2 \\
				& \cellcolor{\tablegray} R15Q2 \\
				& R15.1Q2 \\
				& \cellcolor{\tablegray} R15.2Q2 \\
				& R15.3Q2 \\
				& \cellcolor{\tablegray} R16Q1 \\
                & R16.1Q2 \\
                & \cellcolor{\tablegray} R17Q2 \\
                & R17.1Q2 \\
                & \cellcolor{\tablegray} R17.2Q2 \\
                & R17.3Q2 \\
                & \cellcolor{\tablegray} R17.3.1Q2 \\

				\multirow{-16}{*}{Capitolato} &  R17.3.2Q2 \\

				\bottomrule
			\end{oldtabularx}}
			\caption{Elenco dei requisiti del capitolato (\thetableCounter)}
		\end{table}

		\stepcounter{tableCounter}
		\begin{table}[H]
			\centering
			{\def\arraystretch{1.6}
			\begin{oldtabularx}{0.7\textwidth}{YY}
				\textbf{Fonte} & \textbf{Requisito} \\
				\toprule

                & \cellcolor{\tablegray} R1V2 \\
                & R1.1V0 \\
                & \cellcolor{\tablegray} R2V2 \\
                & R2.1V0 \\
                & \cellcolor{\tablegray} R3V2 \\
                & R3.1V2 \\
                & \cellcolor{\tablegray} R4V1 \\
                & R5V1 \\
                & \cellcolor{\tablegray} R11V2 \\
                & R12V2 \\
                & \cellcolor{\tablegray} R13V2 \\
                & R14V2 \\
				\multirow{-13}{*}{Capitolato} & \cellcolor{\tablegray} R15V2 \\

				\bottomrule
			\end{oldtabularx}}
			\caption{Elenco dei requisiti del capitolato (\thetableCounter)}
		\end{table}

		% \begin{table}[H]
		% 	\centering
		% 	{\def\arraystretch{1.5}
		% 		\begin{tabularx}{\textwidth}{YY}
		% 			\textbf{Fonte} & \textbf{Requisito} \\
		% 			\toprule
		% 			\cellcolor{white} & R11Q1 \\
		% 			\cellcolor{white} & R12Q2 \\
		% 			\cellcolor{white} \multirow{-2}{*}{Interno} & R13Q2 \\
		% 			\bottomrule
		% 		\end{tabularx}}
		% 	\caption{Elenco dei requisiti interni}
		% \end{table}

		% \begin{table}[H]
		% 	\centering
		% 	{\def\arraystretch{1.6}
		% 	\begin{oldtabularx}{0.7\textwidth}{YY}
		% 		\textbf{Fonte} & \textbf{Requisito} \\
		% 		\toprule
		% 		\multirow{3}{*}{Capitolato}
		% 		& R11Q1 \\\cline{2-2}
		% 		& R12Q2 \\\cline{2-2}
		% 		& R13Q3 \\\bottomrule
		% 	\end{oldtabularx}}
		% 	\caption{Elenco dei requisiti interni}
		% \end{table}

		\setcounter{tableCounter}{1}
		\begin{table}[H]
			\centering
			\rowcolors{2}{white}{\tablegray}
			{\def\arraystretch{1.5}
			\begin{tabularx}{0.7\textwidth}{YY}
				\textbf{Fonte} & \textbf{Requisito} \\
				\toprule
				UC1-PR & R1F2 \\
                UC1.1-PR & R1.1F2\\
                UC1.2-PR & R1.2F2\\
                UC1.3-PR & R1.3F2\\
				UC2-PR & R2F2 \\
                UC2.1-PG & R2F2.1 \\
                UC2.2-PG & R2F2.2 \\
                UC2.3-PG & R2F2.3 \\
                UC2.4-PG & R2F2.4 \\
                UC2.5-PG & R2F2.5 \\
				UC3-GP & R3F2 \\
                UC3.1-GP & R3.1F2 \\
                UC3.2-GP & R3.2F2 \\
                UC3.3-GP & R3.3F2 \\
				UC4-GP & R4F2 \\
                UC4.1-GP & R4.1F2 \\
                UC4.2-GP & R4.2F2 \\
                UC4.3-GP & R4.3F2 \\
                UC4.4-GP & R4.4F2 \\
                UC4.5-GP & R4.5F2 \\
				UC5-CT & R5F2 \\
				UC6-CE & R6F2 \\
				UC7-BT & R7F2 \\
				UC8-SE & R8F2 \\
				UC9.1-GP & R9F0 \\
                UC9.1.1-GP & R9.1F0 \\
                UC10-GP & R11F0 \\
				\bottomrule \\
			\end{tabularx}}
			\caption{Elenco dei requisiti per i casi d'uso (\thetableCounter)}
		\end{table}


		\stepcounter{tableCounter}
		\begin{table}[H]
			\centering
			%\rowcolors{2}{white}{\tablegray}
			{\def\arraystretch{1.5}
			\begin{oldtabularx}{0.7\textwidth}{YY}
				\textbf{Fonte} & \textbf{Requisito} \\
				\toprule
				\rowcolor{\tablegray}UC11.1-GP & R12F0 \\
                UC11.1.1-GP & R12.1F0 \\
                \rowcolor{\tablegray}UC11.1.2-GP & R12.2F0 \\
                UC11.1.3-GP & R12.3F0 \\
                \rowcolor{\tablegray}UC11.1.4-GP & R12.4F0 \\

                & R13F0 \\
                \multirow{-2}{*}{UC11.2-GP} & R14F0 \\

				\rowcolor{\tablegray}UC12.1-GP & R15F0 \\
				UC12.1.1-GP & R15.1F0 \\
				\rowcolor{\tablegray} UC12.1.2-GP & R15.2F0 \\
				UC12.2-GP & R16F0 \\
				\rowcolor{\tablegray} UC13.1-GP & R17F0 \\
				UC13.1.1-GP & R17.1F0 \\
				\rowcolor{\tablegray} UC13.1.2-GP & R17.2F0 \\
				UC13.1.3-GP & R17.3F0 \\
				\rowcolor{\tablegray} UC13.1.4-GP & R17.4F0 \\

                & R18F0 \\
                \multirow{-2}{*}{UC13.2-GP} & R19F0 \\

			    \rowcolor{\tablegray}UC14-GP & R20F0 \\
                UC14.1-GP & R20.1F0 \\
                \rowcolor{\tablegray}UC14.2-GP & R20.2F0 \\
                UC14.3-GP & R20.3F0 \\
                \rowcolor{\tablegray}UC14.4-GP & R20.4F0 \\
                UC14.5-GP & R20.5F0 \\
                \rowcolor{\tablegray}UC14.6-GP & R20.6F0 \\
                UC14.7-GP & R20.7F0 \\
			   \bottomrule
		   \end{oldtabularx}}
		   \caption{Elenco dei requisiti per i casi d'uso (\thetableCounter)}
	    \end{table}

        \stepcounter{tableCounter}
        \begin{table}[H]
            \centering
            %\rowcolors{2}{white}{\tablegray}
            {\def\arraystretch{1.5}
                \begin{oldtabularx}{0.7\textwidth}{YY}
                    \textbf{Fonte} & \textbf{Requisito} \\
                    \toprule
                    \rowcolor{\tablegray}UC15-GP & R21F0 \\
                    UC15.1-GP & R21.1F0 \\
                    \rowcolor{\tablegray}UC15.2-GP & R21.2F0 \\
                    UC15.3-GP & R21.3F0 \\
                    \rowcolor{\tablegray}UC15.4-GP & R21.4F0 \\
                    UC15.5-GP & R21.5F0 \\
                    \rowcolor{\tablegray}UC15.6-GP & R21.6F0 \\
                    UC15.7-GP & R21.7F0 \\
                    \bottomrule
                \end{oldtabularx}}
            \caption{Elenco dei requisiti per i casi d'uso (\thetableCounter)}
        \end{table}


		\begin{table}[H]
		\centering
		\rowcolors{2}{white}{\tablegray}
		{\def\arraystretch{1.5}
		\begin{tabularx}{0.7\textwidth}{YY}
			\textbf{Fonte} & \textbf{Requisito} \\
			\toprule
			QPR001 & R1Q1 \\
			QPD001 & R2Q1 \\
			QPR004 & R3Q1 \\
            QPS005 & R4Q1 \\
			QPR006 & R5Q1 \\
			QPR008 & R6Q1 \\
			QPR009 & R7Q2 \\
			QPR011 & R8Q1 \\
            QPR012 & R9Q1 \\
            QPR013 & R10Q1 \\
			\Doc{VE\_2018-12-12} & R12.1Q2 \\
			\bottomrule\\
		\end{tabularx}}
		\caption{Elenco dei requisiti per gli obiettivi di qualità e verbali}
	\end{table}


	\setcounter{tableCounter}{1}
	\begin{table}[H]
		\centering
		{\def\arraystretch{1.6}
		\begin{oldtabularx}{0.7\textwidth}{YY}
			\textbf{Fonte} & \textbf{Requisito} \\
			\toprule

			& \cellcolor{\tablegray} R1F2 \\
            & R1.1F2 \\
            & \cellcolor{\tablegray} R1.2F2 \\
            & R1.3F2 \\
            & \cellcolor{\tablegray} R2F2 \\
			& R2.1F2 \\
            & \cellcolor{\tablegray} R2.2F2 \\
            & R2.3F2 \\
            & \cellcolor{\tablegray} R2.4F2 \\
            & R2.5F2 \\
			& \cellcolor{\tablegray} R3F2 \\
            & R3.1F2 \\
            & \cellcolor{\tablegray} R3.2F2 \\
            & R3.3F2 \\
			& \cellcolor{\tablegray} R4F2 \\
            & R4.1F2 \\
            & \cellcolor{\tablegray} R4.2F2 \\
            & R4.3F2 \\
            & \cellcolor{\tablegray} R4.4F2 \\
            & R4.5F2 \\
			& \cellcolor{\tablegray} R5F2 \\
			& R6F2 \\
			& \cellcolor{\tablegray} R7F2 \\
			& R8F2 \\
			& \cellcolor{\tablegray} R9F0 \\
            & R9.1F0 \\
			& \cellcolor{\tablegray} R10F0 \\
			\multirow{-29}{*}{Interno} & R11F0 \\

			\bottomrule
		\end{oldtabularx}}
		\caption{Elenco dei requisiti da fonte interna (\thetableCounter)}
	\end{table}

	\stepcounter{tableCounter}
	\begin{table}[H]
		\centering
		{\def\arraystretch{1.6}
		\begin{oldtabularx}{0.7\textwidth}{YY}
			\textbf{Fonte} & \textbf{Requisito} \\
			\toprule

            & \cellcolor{\tablegray} R12F0 \\
            & R12.1F0 \\
            & \cellcolor{\tablegray} R12.2F0 \\
            & R12.3F0 \\
            & \cellcolor{\tablegray} R12.4F0 \\
            & R13F0 \\
            & \cellcolor{\tablegray} R14F0 \\
            & R15F0 \\
            & \cellcolor{\tablegray} R15.1F0 \\
            & R15.2F0 \\
            & \cellcolor{\tablegray} R16F0 \\
            & R17F0 \\
            & \cellcolor{\tablegray} R17.1F0 \\
            & R17.2F0 \\
            & \cellcolor{\tablegray} R17.3F0 \\
            & R17.4F0 \\
			& \cellcolor{\tablegray} R18F0 \\
			& R19F0 \\
			& \cellcolor{\tablegray} R20F0 \\
			& R20.1F0 \\
			& \cellcolor{\tablegray} R20.2F0 \\
			& R20.3F0 \\
			& \cellcolor{\tablegray} R20.4F0 \\
			& R20.5F0 \\
			& \cellcolor{\tablegray} R20.5F0 \\
			& R20.6F0 \\
			\multirow{-27}{*}{Interno} & \cellcolor{\tablegray} R20.7F0 \\

			\bottomrule
		\end{oldtabularx}}
		\caption{Elenco dei requisiti da fonte interna (\thetableCounter)}
	\end{table}

	\stepcounter{tableCounter}
	\begin{table}[H]
		\centering
		{\def\arraystretch{1.6}
		\begin{oldtabularx}{0.7\textwidth}{YY}
			\textbf{Fonte} & \textbf{Requisito} \\
			\toprule

            & \cellcolor{\tablegray} R21F0 \\
			& R21.1F0 \\
            & \cellcolor{\tablegray} R21.2F0 \\
            & R21.3F0 \\
            & \cellcolor{\tablegray} R21.4F0 \\
            & R21.5F0 \\
            & \cellcolor{\tablegray} R21.6F0 \\
            & R21.7F0 \\
			& \cellcolor{\tablegray} R1Q1 \\
			& R2Q1 \\
			& \cellcolor{\tablegray} R3Q1 \\
			& R4Q1 \\
			& \cellcolor{\tablegray} R5Q1 \\
			& R6Q1 \\
			& \cellcolor{\tablegray} R7Q1 \\
			& R8Q1 \\
			& \cellcolor{\tablegray} R9Q2 \\
            & R10Q1 \\
            & \cellcolor{\tablegray} R11Q1 \\
            & R12Q1 \\
			& \cellcolor{\tablegray} R6V0 \\
			& R7V0 \\
			& \cellcolor{\tablegray} R8V0 \\
			& R9V0 \\
			\multirow{-25}{*}{Interno} & \cellcolor{\tablegray} R10V0 \\
			\bottomrule
		\end{oldtabularx}}
		\caption{Elenco dei requisiti da fonte interna (\thetableCounter)}
	\end{table}


% \newcounter{V} % valore
% \newcommand{\deV}{\addtocounter{V}{+1}} % Comando per l'aumento automatico del counter vaZ
% \newcommand{\addC}[0]{\theV \deV} % Comando per generare
% \addtocounter{V}{1}

% \newcounter{Vv} % valore
% \newcommand{\deVv}{\addtocounter{Vv}{+1}} % Comando per l'aumento automatico del counter vaZ
% \newcommand{\addVC}[0]{\theVv \deVv} % Comando per generare
% \addtocounter{Vv}{1}

% \newcounter{X} % valore
% \newcommand{\deX}{\addtocounter{X}{+1}} % Comando per l'aumento automatico del counter vaZ
% \newcommand{\addX}[0]{\theX \deX} % Comando per generare
% \addtocounter{X}{1}

		\subsubsection{Tracciamento requisito-fonte}

		% \begin{table}[H]
		% \begin{paddedtablex}[1.7]{\textwidth}{YY}
		% 	\textbf{Requisito} & \textbf{Fonte} \\\toprule
		% 	%R di F
		% 	R\addC
		% 	F2 & Interno UC1 \\
		% 	R\addC
		% 	F2 & Interno UC2 \\
		% 	R\addC
		% 	F2 & Interno UC3 \\
		% 	R\addC
		% 	F2 & Interno UC4 \\
		% 	R\addC
		% 	.1F0 & Interno UC5.1 \\
		% 	R5.1.1F0 & Interno UC5.1.1 \\
		% 	R5.2F0 & Interno UC5.2 \\
		% 	R\addC
		% 	.1.1F0 & Interno UC6.1.1 \\
		% 	R6.1.2F0 & Interno UC6.1.2 \\
		% 	R6.1.3F0 & Interno UC6.1.3 \\
		% 	R6.2.1F0 & Interno UC6.2.1 \\
		% 	R6.2.3F0 & Interno UC6.2.2 \\
		% 	R6.2.3F0 & Interno UC6.2.3 \\
		% 	R\addC
		% 	F2 & Capitolato \\
		% 	R\addC
		% 	F2 & Capitolato \\
		% 	R\addC
		% 	F2 & Capitolato \\
		% 	R\addC
		% 	F2 & Capitolato \\
		% 	R\addC
		% 	F2 & Capitolato \\
		% 	\bottomrule
		% \end{paddedtablex}
		% \caption{Elenco dei requisiti funzionali in rapporto alle fonti}
		% \end{table}

		\setcounter{tableCounter}{1}
		\begin{table}[H]
			\centering
			{\def\arraystretch{1.6}
			\begin{oldtabularx}{0.7\textwidth}{YY}
				\textbf{Requisito} & \textbf{Fonte} \\
				\toprule

				\rowcolor{\tablegray}
				& Interno \\
				\rowcolor{\tablegray}
				\multirow{-2}{*}{R1F2}
				& UC1-PR \\

                & Interno \\
                \multirow{-2}{*}{R1.1F2}
                & UC1.1-PR \\

                \rowcolor{\tablegray}
                & Interno \\
                \rowcolor{\tablegray}
                \multirow{-2}{*}{R1.2F2}
                & UC1.2-PR \\

                & Interno \\
                \multirow{-2}{*}{R1.3F2}
                & UC1.3-PR \\

                \rowcolor{\tablegray}
                & Interno \\
                \rowcolor{\tablegray}
                \multirow{-2}{*}{R1.4F2}
                & UC1.4-PR \\

            	& Interno \\
                \multirow{-2}{*}{R2F2}
				& UC2-PG \\

                \rowcolor{\tablegray}
                & Interno \\
                \rowcolor{\tablegray}
                \multirow{-2}{*}{R2.1F2}
                & UC2.1-PG \\

                & Interno \\
                \multirow{-2}{*}{R2.2F2}
                & UC2.2-PG \\

                \rowcolor{\tablegray}
                & Interno \\
                \rowcolor{\tablegray}
                \multirow{-2}{*}{R2.3F2}
                & UC2.3-PG \\

                & Interno \\
                \multirow{-2}{*}{R2.4F2}
                & UC2.4-PG \\

                \rowcolor{\tablegray}
                & Interno \\
                \rowcolor{\tablegray}
                \multirow{-2}{*}{R2.5F2}
                & UC2.5-PG \\

				\bottomrule
			\end{oldtabularx}}
			\caption{Elenco dei requisiti funzionali in rapporto alle fonti (\thetableCounter)}
		\end{table}


		\stepcounter{tableCounter}
		\begin{table}[H]
			\centering
			{\def\arraystretch{1.6}
			\begin{oldtabularx}{0.7\textwidth}{YY}
				\textbf{Requisito} & \textbf{Fonte} \\
				\toprule

                \rowcolor{\tablegray}
                & Interno \\
                \rowcolor{\tablegray}
                \multirow{-2}{*}{R3F2}
                & UC3-GP \\

                & Interno \\
                \multirow{-2}{*}{R3.1F2}
                & UC3.1-GP \\

                \rowcolor{\tablegray}
                & Interno \\
                \rowcolor{\tablegray}
                \multirow{-2}{*}{R3.2F2}
                & UC3.2-GP \\

                & Interno \\
                \multirow{-2}{*}{R3.3F2}
                & UC3.3-GP \\

                \rowcolor{\tablegray}
                & Interno \\
                \rowcolor{\tablegray}
                \multirow{-2}{*}{R4F2}
                & UC4-GP \\

                & Interno \\
                \multirow{-2}{*}{R4.1F2}
                & UC4.1-GP \\

                \rowcolor{\tablegray}
                & Interno \\
                \rowcolor{\tablegray}
                \multirow{-2}{*}{R4.2F2}
                & UC4.2-GP \\

                & Interno \\
                \multirow{-2}{*}{R4.3F2}
                & UC4.3-GP \\

                \rowcolor{\tablegray}
                & Interno \\
                \rowcolor{\tablegray}
                \multirow{-2}{*}{R4.4F2}
                & UC4.4-GP \\

                & Interno \\
                \multirow{-2}{*}{R4.5F2}
                & UC4.5-GP \\

                \rowcolor{\tablegray}
                & Interno \\
                \rowcolor{\tablegray}
                \multirow{-2}{*}{R5F2}
                & UC5-CT \\

                & Interno \\
                \multirow{-2}{*}{R6F2}
                & UC6-CE \\

                \rowcolor{\tablegray}
                & Interno \\
                \rowcolor{\tablegray}
                \multirow{-2}{*}{R7F2}
                & UC7-BT \\

				& Interno \\
				\multirow{-2}{*}{R8F2}
				& UC8-SE \\

				\bottomrule
			\end{oldtabularx}}
			\caption{Elenco dei requisiti funzionali in rapporto alle fonti (\thetableCounter)}
		\end{table}


		\stepcounter{tableCounter}
		\begin{table}[H]
			\centering
			{\def\arraystretch{1.6}
			\begin{oldtabularx}{0.7\textwidth}{YY}
				\textbf{Requisito} & \textbf{Fonte} \\
				\toprule

                \rowcolor{\tablegray}
                & Interno \\
                \rowcolor{\tablegray}
                \multirow{-2}{*}{R9F2}
                & UC9.1-GP \\

                & Interno \\
                \multirow{-2}{*}{R9.1F0}
                & UC9.1.1-GP \\

                \rowcolor{\tablegray}
                & Interno \\
                \rowcolor{\tablegray}
                \multirow{-2}{*}{R10F0}
                & UC9.2-GP \\

                & Interno \\
                \multirow{-2}{*}{R11F0}
                & UC10-GP \\

                \rowcolor{\tablegray}
                & Interno \\
                \rowcolor{\tablegray}
                \multirow{-2}{*}{R12F0}
                & UC11.1-GP \\

                & Interno \\
                \multirow{-2}{*}{R12.1F0}
                & UC11.1.1-GP \\

                \rowcolor{\tablegray}
                & Interno \\
                \rowcolor{\tablegray}
                \multirow{-2}{*}{R12.2F0}
                & UC11.1.2-GP \\

                & Interno \\
                \multirow{-2}{*}{R12.3F0}
                & UC11.1.3-GP \\

                \rowcolor{\tablegray}
                & Interno \\
                \rowcolor{\tablegray}
                \multirow{-2}{*}{R12.4F0}
                & UC11.1.4-GP \\

                & Interno \\
                \multirow{-2}{*}{R13F0}
                & UC11.2-GP \\

                \rowcolor{\tablegray}
                & Interno \\
                \rowcolor{\tablegray}
                \multirow{-2}{*}{R14F0}
                & UC11.1-GP \\

                & Interno \\
                \multirow{-2}{*}{R15F0}
                & UC12.1-GP \\

                \rowcolor{\tablegray}
                & Interno \\
                \rowcolor{\tablegray}
                \multirow{-2}{*}{R15.1F0}
                & UC12.1.1-GP \\

                & Interno \\
                \multirow{-2}{*}{R15.2F0}
                & UC12.1.2-GP \\

                \rowcolor{\tablegray}
                & Interno \\
                \rowcolor{\tablegray}
                \multirow{-2}{*}{R16F0}
                & UC12.2-GP \\

				\bottomrule
			\end{oldtabularx}}
			\caption{Elenco dei requisiti funzionali in rapporto alle fonti (\thetableCounter)}
		\end{table}


		\stepcounter{tableCounter}
		\begin{table}[H]
			\centering
			{\def\arraystretch{1.6}
			\begin{oldtabularx}{0.7\textwidth}{YY}
				\textbf{Requisito} & \textbf{Fonte} \\
				\toprule

                \rowcolor{\tablegray}
                & Interno \\
                \rowcolor{\tablegray}
                \multirow{-2}{*}{R17F0}
                & UC13.1-GP \\

                & Interno \\
                \multirow{-2}{*}{R17.1F0}
                & UC13.1.1-GP \\

                \rowcolor{\tablegray}
                & Interno \\
                \rowcolor{\tablegray}
                \multirow{-2}{*}{R17.2F0}
                & UC13.1.2-GP \\

                & Interno \\
                \multirow{-2}{*}{R17.3F0}
                & UC13.1.3-GP \\

                \rowcolor{\tablegray}
                & Interno \\
                \rowcolor{\tablegray}
                \multirow{-2}{*}{R17.4F0}
                & UC13.1.4-GP \\

                & Interno \\
                \multirow{-2}{*}{R18F0}
                & UC13.2-GP \\

                \rowcolor{\tablegray}
                & Interno \\
                \rowcolor{\tablegray}
                \multirow{-2}{*}{R19F0}
                & UC13.2-GP \\

				& Interno \\
				\multirow{-2}{*}{R20F0}
				& UC14-GP \\

				\rowcolor{\tablegray}
				& Interno \\
				\rowcolor{\tablegray}
				\multirow{-2}{*}{R20.1F0}
				& UC14.1-GP \\

				& Interno \\
				\multirow{-2}{*}{R20.2F0}
				& UC14.2-GP \\

				\rowcolor{\tablegray}
				& Interno \\
				\rowcolor{\tablegray}
				\multirow{-2}{*}{R20.3F0}
				& UC14.3-GP \\

				& Interno \\
				\multirow{-2}{*}{R20.4F0}
				& UC14.4-GP \\

				\rowcolor{\tablegray}
				& Interno \\
				\rowcolor{\tablegray}
				\multirow{-2}{*}{R20.5F0}
				& UC14.5-GP \\

				& Interno \\
				\multirow{-2}{*}{R20.6F0}
				& UC14.6-GP \\

				\rowcolor{\tablegray}
				& Interno \\
				\rowcolor{\tablegray}
				\multirow{-2}{*}{R20.7F0}
				& UC14.7-GP \\

				\bottomrule
			\end{oldtabularx}}
			\caption{Elenco dei requisiti funzionali in rapporto alle fonti (\thetableCounter)}
		\end{table}


		\stepcounter{tableCounter}
		\begin{table}[H]
			\centering
			{\def\arraystretch{1.6}
			\begin{oldtabularx}{0.7\textwidth}{YY}
				\textbf{Requisito} & \textbf{Fonte} \\
				\toprule

                \rowcolor{\tablegray}
				& Interno \\
                \rowcolor{\tablegray}
				\multirow{-2}{*}{R21F0}
				& UC15-GP \\

				& Interno \\
				\multirow{-2}{*}{R21.1F0}
				& UC15.1-GP \\

                \rowcolor{\tablegray}
				& Interno \\
                \rowcolor{\tablegray}
				\multirow{-2}{*}{R21.2F0}
				& UC15.2-GP \\

				& Interno \\
				\multirow{-2}{*}{R21.3F0}
				& UC15.3-GP \\

                \rowcolor{\tablegray}
				& Interno \\
                \rowcolor{\tablegray}
				\multirow{-2}{*}{R21.4F0}
				& UC15.4-GP \\

				& Interno \\
				\multirow{-2}{*}{R21.5F0}
				& UC15.5-GP \\

                \rowcolor{\tablegray}
				& Interno \\
                \rowcolor{\tablegray}
				\multirow{-2}{*}{R21.6F0}
				& UC15.6-GP \\

				& Interno \\
				\multirow{-2}{*}{R21.7F0}
				& UC15.7-GP \\
				\bottomrule
			\end{oldtabularx}}
			\caption{Elenco dei requisiti funzionali in rapporto alle fonti (\thetableCounter)}
		\end{table}


		\setcounter{tableCounter}{1}
		\begin{table}[H]
			\centering
			{\def\arraystretch{1.6}
			\begin{oldtabularx}{0.7\textwidth}{YY}
				\textbf{Requisito} & \textbf{Fonte} \\
				\toprule

				\rowcolor{\tablegray}
				& Interno \\
				\rowcolor{\tablegray}
				\multirow{-2}{*}{R1Q1}
				& QPR001 \\

				& Interno \\
				\multirow{-2}{*}{R2Q1}
				& QPD001 \\

				\rowcolor{\tablegray}
				& Interno \\
				\rowcolor{\tablegray}
				\multirow{-2}{*}{R3Q1}
				& QPR004 \\

                & Interno \\
                \multirow{-2}{*}{R4Q1}
                & QPS005 \\

                \rowcolor{\tablegray}
				& Interno \\
                \rowcolor{\tablegray}
				\multirow{-2}{*}{R5Q1}
				& QPR006 \\

				& Interno \\
				\multirow{-2}{*}{R6Q1}
				& QPR008 \\

				\rowcolor{\tablegray}
				& Interno \\
				\rowcolor{\tablegray}
				\multirow{-2}{*}{R7Q1}
				& QPR009 \\

                & Interno \\
                \multirow{-2}{*}{R8Q1}
                & QPR011 \\

                \rowcolor{\tablegray}
                & Interno \\
                \rowcolor{\tablegray}
                \multirow{-2}{*}{R9Q1}
                & QPR012 \\

                & Interno \\
                \multirow{-2}{*}{R10Q1}
                & QPR013 \\

                \rowcolor{\tablegray}R11Q2 & Interno \\
				R12Q2 & Interno \\

                \rowcolor{\tablegray}
				& Capitolato \\
                \rowcolor{\tablegray}
				\multirow{-2}{*}{R12.1Q2} & \Doc{VE\_2018-12-12} \\

                R13Q2 & Capitolato \\
                \rowcolor{\tablegray}R14Q2 & Capitolato \\
				R14.1Q2 & Capitolato \\
				\bottomrule
			\end{oldtabularx}}
			\caption{Elenco dei requisiti di qualità in rapporto alle fonti (\thetableCounter)}
		\end{table}


		\stepcounter{tableCounter}
		\begin{table}[H]
			\centering
			{\def\arraystretch{1.6}
			\begin{oldtabularx}{0.7\textwidth}{YY}
				\textbf{Requisito} & \textbf{Fonte} \\
				\toprule

                \rowcolor{\tablegray} R15Q2 & Capitolato \\
                R15.1Q2 & Capitolato \\
                \rowcolor{\tablegray} R15.2Q2 & Capitolato \\
                R15.3Q2 & Capitolato \\
                \rowcolor{\tablegray} R16Q1 & Capitolato \\
                R16.1Q2 & Capitolato \\
				\rowcolor{\tablegray} R17Q2 & Capitolato \\
				R17.1Q2 & Capitolato \\
				\rowcolor{\tablegray} R17.3Q2 & Capitolato \\
				R17.3.1Q2 & Capitolato \\
                \rowcolor{\tablegray} R17.3.2Q2 & Capitolato \\
				\bottomrule
			\end{oldtabularx}}
			\caption{Elenco dei requisiti di qualità in rapporto alle fonti (\thetableCounter)}
		\end{table}



		\begin{table}[H]
			\centering
			{\def\arraystretch{1.6}
			\begin{oldtabularx}{0.7\textwidth}{YY}
				\textbf{Requisito} & \textbf{Fonte} \\
				\toprule
				\rowcolor{\tablegray} R1V2 & Capitolato \\
				R1.1V0 & Capitolato \\
				\rowcolor{\tablegray} R2V2 & Capitolato \\
				R2.1V0 & Capitolato \\
				\rowcolor{\tablegray} R3V2 & Capitolato \\
				R3.1V2 & Capitolato \\
				\rowcolor{\tablegray} R4V1 & Capitolato \\
				R5V1 & Capitolato \\
				\rowcolor{\tablegray} R6V0 & Interno \\
				R7V0 & Interno \\
				\rowcolor{\tablegray} R8V0 & Interno \\
				R9V0 & Interno \\
				\rowcolor{\tablegray} R10V0 & Interno \\
                R11V2 & Capitolato \\
                \rowcolor{\tablegray} R12V2 & Capitolato \\
                R13V2 & Capitolato \\
                \rowcolor{\tablegray} R14V2 & Capitolato \\
                R15V2 & Capitolato \\
				\bottomrule
			\end{oldtabularx}}
			\caption{Elenco dei requisiti di vincolo in rapporto alle fonti}
		\end{table}




% 		\begin{table}[H]
% 		\begin{paddedtablex}[1.7]{\textwidth}{YY}
% 			\textbf{Requisito} & \textbf{Fonte} \\\toprule
% 			%R di Q
% 			R\addVC
% 			Q1 & Interno QPR001 \\
% 			R\addVC
% 			Q1 & Interno QPD001 \\
% 			R\addVC
% 			Q1 & Interno QPR002 \\
% 			R\addVC
% 			Q1 & Interno QPR003 \\
% 			R\addVC
% 			Q1 & Interno QPR004 \\
% 			R\addVC
% 			Q2 & Interno QPR005 \\
% 			R\addVC
% 			Q1 & Interno QPR006 \\
% 			R\addVC
% 			Q1 & Interno QPR007 \\
% 			R\addVC
% 			Q1 & Interno QPR008 \\
% 			R\addVC
% 			Q1 & Interno QPR009 \\
% 			R\addVC
% 			Q2 & Interno QPR010 \\
% 			R\addVC
% 			Q2 & Interno \\
% 			R\addVC
% 			Q2 & Interno \\
% 			R\addVC
% 			Q2 & \Doc{VE\_2018-12-12} \\
% 			R\addVC
% 			Q2 & Capitolato \\
% 			R\addVC
% 			Q2 & Capitolato \\
% 			R\addVC
% 			Q2 & Capitolato \\
% 			R\addVC
% 			Q2 & Capitolato \\
% 			R18.1Q2 & Capitolato \\
% 			R18.2Q2 & Capitolato \\
% 			R18.3Q2 & Capitolato \\
% 			\bottomrule
% 			\end{paddedtablex}
% 		\caption{Elenco dei requisiti di qualità in rapporto alle fonti}
% 	\end{table}

% \begin{table}[H]
% 	\begin{paddedtablex}[1.7]{\textwidth}{YY}
% 		\textbf{Requisito} & \textbf{Fonte} \\\toprule
% 			%R di V
% 			R\addX
% 			V2 & Capitolato \\
% 			R1.1V0 & Capitolato \\
% 			R\addX
% 			V2 & Capitolato \\
% 			R2.1V0 & Capitolato \\
% 			R\addX
% 			V2 & Capitolato \\
% 			R3.1V2 & Capitolato \\
% 			R\addX
% 			V1 & Capitolato \\
% 			R4.1V2 & Capitolato \\
% 			R\addX
% 			V2 & Capitolato \\
% 			R5.1V2 & Capitolato \\
% 			R5.2V2 & Capitolato \\
% 			R5.3V2 & Capitolato \\
% 			R5.4V2 & Capitolato \\
% 			R5.5V2 & Capitolato \\
% 			R\addX
% 			V2 & Capitolato \\
% 			R\addX
% 			V0 & Capitolato \\
% 			R\addX
% 			V1 & Capitolato \\
% 			R\addX
% 			V1 & Capitolato \\
% 			\bottomrule \\
% 		\end{paddedtablex}
% 		\caption{Elenco dei requisiti di vincolo in rapporto alle fonti}
% 	\end{table}

	\subsection{Riepilogo}\label{Riepilogo}

		\begin{table}[H]
		\centering
		{\def\arraystretch{1.7}
		\begin{oldtabularx}{\textwidth}{YYYY}
			\textbf{Tipologia} & \textbf{Obbligatori} & \textbf{Desiderabili} & \textbf{Opzionali} \\\toprule
			\rowcolor{\tablegray} Di funzionalità & 24 & 0 & 38 \\
			Di qualità & 17 & 11 & 0 \\
			\rowcolor{\tablegray} Di vincolo & 9 & 2 & 7
			\\\bottomrule
		\end{oldtabularx}}
		\caption{Riepilogo dei requisiti}
		\end{table}
