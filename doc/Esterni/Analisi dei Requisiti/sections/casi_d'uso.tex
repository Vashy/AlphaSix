\newpage
\section{Casi d'uso}
testo

	\subsection{Introduzione}
	testo
	
	\subsection{Attori}
	\begin{itemize}
		\item Producer
		\item Utente che interagisce con il gestore personale
		\item Consumer (secondario)
	\end{itemize}
	
	\subsection{Elenco casi d'uso}
	Messaggio - tecnologia che invia il messaggio - tipo di messaggio di quella tecnologia.
	Classificazioni in base al topic.




\subsubsection{UC1 - Redmine/GitLab genera una segnalazione}
	\begin{itemize}
		\item \textbf{Codice}: UC1.
		\item \textbf{Titolo}: Redmine/GitLab genera una segnalazione.
		\item \textbf{Attori primari}: Redmine/Gitlab.
		\item \textbf{Descrizione}: se il Producer è offline, c'è la possibilità che il messaggio venga perso... il sistema qui è il Producer ed è interno al sistema Butterfly. Cambio stato repository tramite commit per GitLab. Cambio di stato issue tracking system per GitLab e Redmine.
		\item \textbf{Precondizione}:
		\item \textbf{Postcondizione}:
		\item \textbf{Scenario principale}: 
		\begin{enumerate}
			\item Redmine/GitLab prepara una segnalazione da mandare
			\item Redmine/GitLab invia la segnalazione al Producer.
		\end{enumerate}
		\item \textbf{Estensioni}: messaggio creato con parametri errati.
	\end{itemize}

\subsubsection{UC2 - Redmine/GitLab genera una segnalazione errata}
	\begin{itemize}
		\item \textbf{Codice}: UC2.
		\item \textbf{Titolo}: Redmine/GitLab genera una segnalazione errata.
		\item \textbf{Attori primari}: Redmine/Gitlab.
		\item \textbf{Descrizione}:
		\item \textbf{Precondizione}:
		\item \textbf{Postcondizione}:
		\item \textbf{Scenario principale}: 
	\end{itemize}

\subsubsection{UC3 - Il Producer invia la segnalazione al Broker}
	\begin{itemize}
		\item \textbf{Codice}: UC3.
		\item \textbf{Titolo}: il Producer invia la segnalazione al Broker.
		\item \textbf{Attori primari}: Producer.
		\item \textbf{Descrizione}: ..il sistema qui è Broker ed è interno al sistema Butterfly.
		\item \textbf{Precondizione}:
		\item \textbf{Postcondizione}:
		\item \textbf{Scenario principale}: 
	\end{itemize}

%Opzionale Sonarqube

\subsubsection{UC4 - Il Consumer interroga il Broker}
	\begin{itemize}
		\item \textbf{Codice}: UC4.
		\item \textbf{Titolo}: il Consumer interroga il Broker.
		\item \textbf{Attori primari}: Consumer.
		\item \textbf{Descrizione}: il Consumer chiede al Broker per acquisire il messaggio... il sistema di riferimento qui è Broker ed è interno al sistema Butterfly.
		\item \textbf{Precondizione}:
		\item \textbf{Postcondizione}:
		\item \textbf{Scenario principale}: 
	\end{itemize}

%TODO: rivedere UC5
\subsubsection{UC5 - Telegram/e-mail riceve un messaggio dal Consumer} 
	\begin{itemize}
		\item \textbf{Codice}: UC5.\footnote{Sebbene questo caso d'uso non sia del tutto corretto perchè l'attore Telegram/e-mail è un attore passivo, che riceve e non che agisce, è stato scelto di creare questo caso d'uso al fine di inserire la funzionalità "Il Consumer invia un messaggio a Telegram/e-mail". Ma scrivere un caso d'uso così scritto sarebbe stato del tutto scorretto in quanto il sistema di riferimento in questo caso sarebbe stato Telegram/e-mail. Cosa non possibile dato che Telegram/e-mail è esterno a Butterfly e quindi non può essere considerato un suo sottosistema.}
		\item \textbf{Titolo}: Telegram/e-mail riceve un messaggio dal Consumer.
		\item \textbf{Attori}: Telegram/server mail.
		\item \textbf{Descrizione}: ..il sistema di riferimento qui è Consumer ed è interno al sistema Butterfly.
		\item \textbf{Precondizione}:
		\item \textbf{Postcondizione}:
		\item \textbf{Scenario principale}: 
	\end{itemize}

%Opzionale Slack



\subsubsection{UC6 - Autenticazione dell'utente nel sistema}

	\begin{itemize}
		\item \textbf{Codice}: UC6.
		\item \textbf{Titolo}: autenticazione dell'utente nel sistema.
		\item \textbf{Attori primari}: utente non autenticato.
		\item \textbf{Descrizione}: l'utente richiede di autenticarsi al sistema attraverso un form php dove inserisce username e password.
		\item \textbf{Precondizione}: il sistema considera l’utilizzatore come un utente non autenticato.
		\item \textbf{Postcondizione}: il sistema riconosce l'utilizzatore come utente autenticato.
		\item \textbf{Scenario Principale}: l'utente non ancora riconosciuto dal sistema effettua l'autenticazione.
		\item \textbf{Estensioni}: 
	\end{itemize}
	
	\paragraph{UC6.1}%TODO: rivedere
		\begin{itemize}
			\item \textbf{Codice}: UC6.1.
			\item \textbf{Titolo}: autenticazione
			\item \textbf{Attori}: utente non autenticato
			\item \textbf{Descrizione}: l'utente attende l'autenticazione da parte del sistema
			\item \textbf{Precondizione}: il sistema riconosce l'utilizzatore come un utente non autenticato
			\item \textbf{Postcondizione}: il sistema riconosce l'utente come autenticato
			\item \textbf{Scenario Principale}: l’utente non ancora riconosciuto dal sistema richiede l'autenticazione
	\end{itemize}
		\subparagraph{UC6.1.1}
			\begin{itemize}
			\item \textbf{Codice}: UC6.1.1.	
			\item \textbf{Titolo}: inserimento username
			\item \textbf{Attori}: utente non autenticato
			\item \textbf{Descrizione}: l'utente inserisce l'username
			\item \textbf{Precondizione}: il sistema offre l'interfaccia grafica adatta all'inserimento dell'username
			\item \textbf{Postcondizione}: l'utente ha inserito l'username desiderato
			\item \textbf{Scenario Principale}: l'utente inserisce l'username per autenticarsi
		\end{itemize}
		\subparagraph{UC6.1.2}
			\begin{itemize}
			\item \textbf{Titolo}: inserimento password
			\item \textbf{Attori}: utente non autenticato
			\item \textbf{Descrizione}: l'utente inserisce la password
			\item \textbf{Precondizione}: il sistema offre l'interfaccia grafica adatta all'inserimento della password
			\item \textbf{Postcondizione}: l'utente ha inserito la password desiderata
			\item \textbf{Scenario Principale}: l'utente inserisce la password per autenticarsi
		\end{itemize}
	
	%TODO: da mettere come un altro caso d'uso distinto
	\paragraph{UC6.2}
		\begin{itemize}
			\item \textbf{Titolo}: visualizzazione errore autenticazione fallita
			\item \textbf{Attori}: utente non autenticato
			\item \textbf{Descrizione}: l'utente viene avvisato che ha inserito username o password sbagliate
			\item \textbf{Precondizione}: il sistema riceve una richiesta di autenticazione da parte di un utente che
			fornisce username o password sbagliate
			\item \textbf{Postcondizione}: il sistema comunica all'utilizzatore l'errore
			\item \textbf{Scenario Principale}: l'utente visualizza il messaggio d'errore
		\end{itemize}

\subsubsection{UC7}
configurazione delle preferenze dell'utente nel sistema

	\paragraph{UC7.1}
	aggiunta preferenza
		\subparagraph{UC7.1.1}:
		iscrizione topic
			data la lista di topic presenti l'utente seleziona quelli a cui è interessato ricevere notifica (suddiviso per tecnologia e per tag [BUG, FIX, ISSUE, ecc.] )
		\subparagraph{UC7.1.2}:
		prima selezione dei giorni in calendario:
			dato il calendario lavorativo l'utente selezionerà i giorni in cui sarà assente
		\subparagraph{UC7.1.3}
		piattaforma di messaggistica:
			telegram: aggiunta nickname
			slack: aggiunta nickname
			mail: aggiunta email

	\paragraph{UC7.2}
	rimozione preferenza
		\subparagraph{UC7.2.1}
		disiscrizione topic %si può non essere iscritti a nessun topic?
			data la lista di topic a cui è iscritto l'utente elimina quelli a cui non è più interessato ricevere notifica
		\subparagraph{UC7.2.2}
		rimozione giorno di calendario
			data una lista dei giorni in cui non si è reperibili, vengono rimossi uno o più di questi giorni
		\subparagraph{UC7.2.3}
		rimozione piattaforma di messaggistica %dire che bisogna lasciarne almeno una?
			telegram: modifica nickname
			slack: modifica nickname
			mail: modifica email
			
	\paragraph{UC7.3}
	applico le modifiche fatte
	
	\paragraph{UC7.4}
	Annullo le modifiche fatte
	\iffalse %inizio paragrafo commentato
	\paragraph{UC7.2}
	aggiornamento delle preferenze
	\subparagraph{UC7.2.1}
	modifica calendario:
	dato il calendario lavorativo l'utente modifica i giorni disponibili
	\subparagraph{UC7.2.2}
	cambio piattaforma di messaggistica
	telegram: modifica nickname
	slack: modifica nickname
	mail: modifica email
	\fi
			
		%Possibilità di aggiungere funzionalità specifiche di Gitlab, di Sonar, di Redmine ecc
		
		
		
