\newpage
\section{Casi d'uso}
Questa sezione elenca le funzionalità offerte da \progetto\ descritte attraverso il linguaggio \gloss{UML}.
\progetto\ può essere visto come l'insieme di più sottosistemi che verranno di seguito elencati e che sono stati descritti in modo molto generale anche attraverso la Figura \ref{fig:butterfly}.\\
Questa mostra la suddivisione di \progetto\ nei sottosistemi in cui è composto e ne facilita l'analisi per la stesura dei casi d'uso.
Abbiamo quindi come attori non soltanto le applicazioni che mandano messaggi al sistema ma anche componente inoltre quali Producer e Consumer che hanno interazioni con il Broker.
	
	\subsection{Attori}
	\begin{itemize}
		\item Redmine/GitLab
		\item Producer
		\item Utente non acceduto
		\item Utente (acceduto), che interagisce col Gestore Personale
		\item Consumer
		\item Telegram/e-mail
	\end{itemize}
	
	\subsection{Elenco casi d'uso}

%TODO: da ricordarsi: se qualcuno è offline, c'è la possibilità che il messaggio venga perso.

\newcounter{uccount}

\stepcounter{uccount}

\subsubsection{UC\theuccount-R - Redmine segnala apertura issue a Producer Redmine}
    \begin{figure}[H]
		\centering
		\includegraphics[width=0.7\textwidth]{img/UC1.png}\\
		\caption{UCR\theuccount-R - Redmine segnala apertura issue a Producer
			Redmine}
	\end{figure}
	\begin{itemize}
		\item \textbf{Codice}: UCR\theuccount-R.
		\item \textbf{Titolo}: Redmine segnala apertura issue a Producer Redmine.
		\item \textbf{Attori primari}: Redmine.
		\item \textbf{Descrizione}: il sistema qui è il Producer Redmine ed è interno al sistema \progetto.
		 L'apertura di una issue in un particolare progetto su Redmine
		 contiene i seguenti campi di interesse:
		 \begin{itemize}
		 	\item Tracker
		 	\item Subject
		 	\item Status
		 	\item Priority e opzionalmente:
		 	\begin{itemize}
		 		\item Description
		 		\item Assignee
		 	\end{itemize}
		 \end{itemize}
		\item \textbf{Precondizione}: Viene aperta una issue su Redmine e
		segnalata a \progetto.
		\item \textbf{Postcondizione}: il Producer Redmine riceve la segnalazione da Redmine.
		\item \textbf{Scenario principale}: 
		\begin{enumerate}
			\item Redmine procede all'invio della segnalazione di issue al Producer Redmine.
		\end{enumerate}
		
	\end{itemize}


\stepcounter{uccount}

\include{casi_d'uso/UC2-PR}

\stepcounter{uccount}

	\subsubsection{UC\theuccount-G - Gitlab segnala la modifica di una issue al Producer Gitlab}
	\begin{figure}[H]
		\centering
		\includegraphics[width=0.7\textwidth]{img/UC1.png}\\
		\caption{UC\theuccount-G - Gitlab segnala la modifica di una issue al Producer Gitlab}
	\end{figure}
	\begin{itemize}
		\item \textbf{Codice}: UC\theuccount-G.
		\item \textbf{Titolo}: Gitlab segnala la modifica di una issue al Producer Gitlab.
		\item \textbf{Attori primari}: GitLab.
		\item \textbf{Descrizione}: l'invio di una segnalazione avviene
		da parte di GitLab tramite webhook, quando una issue viene modificata.
		\item \textbf{Precondizione}: Viene modificata una issue già aperta su un
		progetto di GitLab e segnalata a \progetto.
		\item \textbf{Postcondizione}: il Producer GitLab riceve la segnalazione da GitLab.
		\item \textbf{Scenario principale}: 
		\begin{enumerate}
			\item GitLab procede all'invio della segnalazione di modifica issue al Producer GitLab
		\end{enumerate}
		
	\end{itemize}

\stepcounter{uccount}

\include{casi_d'uso/UC4-PG}

\stepcounter{uccount}

\subsubsection{UC\theuccount-PG - GitLab segnala evento di push a Producer GitLab}
	\begin{figure}[H]
		\centering
		\includegraphics[width=0.7\textwidth]{img/UC1.png}\\
		\caption{UC\theuccount-PG - GitLab segnala evento di push a Producer GitLab}
	\end{figure}
	\begin{itemize}
		\item \textbf{Codice}: UC\theuccount-PG.
		\item \textbf{Titolo}: GitLab segnala evento di push a Producer GitLab.
		\item \textbf{Attori primari}: GitLab.
		\item \textbf{Descrizione}: l'invio di una segnalazione avviene da parte di GitLab tramite webhook. L'evento di
		push può essere composto da uno o più commit.
		\item \textbf{Precondizione}: Viene effettuato un push su GitLab e segnalato a \progetto.
		\item \textbf{Postcondizione}: il Producer GitLab riceve la segnalazione da GitLab.
		\item \textbf{Scenario principale}: 
		\begin{enumerate}
			\item GitLab procede all'invio della segnalazione di push al Producer GitLab.
		\end{enumerate}
		
	\end{itemize}

\stepcounter{uccount}

\subsubsection{UC\theuccount-GP - GPoducer Redmine invia messaggio al Gestore Personale}
    \begin{figure}[H]
		\centering
		\includegraphics[width=0.7\textwidth]{img/UC1.png}\\
		\caption{UC\theuccount-GP - GPoducer Redmine invia messaggio al Gestore Personale}
	\end{figure}
	\begin{itemize}
		\item \textbf{Codice}: UC\theuccount-GP.
		\item \textbf{Titolo}: GPoducer Redmine invia messaggio al Gestore Personale.
		\item \textbf{Attori GPimari}: GPoducer Redmine.
		\item \textbf{Descrizione}: il GPoducer Redmine, dopo aver ricevuto una
		 segnalazione da Redmine, elabora il messaggio e lo invia al Gestore Personale.
		 Il messaggio finale, una volta terminata l'elaborazione, conterrà i campi:
		 \begin{itemize}
		 	\item GPoject
		 	\item Topic
		 	\item Subject e opzionalmente:
		 	\begin{itemize}
		 		\item Description
		 	\end{itemize}
		 \end{itemize}
		\item \textbf{GPecondizione}: il GPoducer Redmine ha ricevuto una segnalazione da Redmine.
		\item \textbf{Postcondizione}: il GPoducer Redmine ha elaborato e inviato al Gestore Personale il messaggio.
		\item \textbf{Scenario GPincipale}: 
		\begin{enumerate}
			\item GPoducer Redmine GPocede all'invio del messaggio al gestore personale.
		\end{enumerate}
		
	\end{itemize}
	
		\paragraph{UC\theuccount.1-GP - GPoducer Redmine invia messaggio di apertura issue al Gestore Personale}
		\begin{figure}[H]
			\centering
			\includegraphics[width=0.7\textwidth]{img/UC1.png}\\
			\caption{UC\theuccount.1-GP - GPoducer Redmine invia messaggio di apertura issue al Gestore Personale}
		\end{figure}
		\begin{itemize}
			\item \textbf{Codice}: UC\theuccount.1-GP.
			\item \textbf{Titolo}: GPoducer Redmine invia messaggio di apertura issue al Gestore Personale.
			\item \textbf{Attori GPimari}: GPoducer Redmine.
			\item \textbf{Descrizione}: il GPoducer Redmine, dopo aver
			ricevuto una segnalazione di apertura issue da Redmine, elabora
			il messaggio e lo invia al Gestore Personale.
			Il messaggio finale, una volta terminata l'elaborazione, conterrà i campi:
			\begin{itemize}
				\item GPoject
				\item Topic
				\item Subject e opzionalmente:
				\begin{itemize}
					\item Description
				\end{itemize}
			\end{itemize}
			\item \textbf{GPecondizione}: il GPoducer Redmine ha ricevuto una segnalazione da Redmine.
			\item \textbf{Postcondizione}: il GPoducer Redmine ha elaborato e inviato al Gestore Personale il messaggio di apertura issue.
			\item \textbf{Scenario GPincipale}: 
			\begin{enumerate}
				\item GPoducer Redmine GPocede all'invio del messaggio di
				apertura issue al gestore personale.
			\end{enumerate}
			
		\end{itemize}
		
		\paragraph{UC\theuccount.2-GP - GPoducer Redmine invia messaggio di modifica issue al Gestore Personale}
			\begin{figure}[H]
				\centering
				\includegraphics[width=0.7\textwidth]{img/UC1.png}\\
				\caption{UC\theuccount.2-GP - GPoducer Redmine invia messaggio di modifica issue al Gestore Personale}
			\end{figure}
			\begin{itemize}
				\item \textbf{Codice}: UC\theuccount.2-GP.
				\item \textbf{Titolo}: GPoducer Redmine invia messaggio di modifica issue al Gestore Personale
				\item \textbf{Attori GPimari}: GPoducer Redmine.
				\item \textbf{Descrizione}: il GPoducer Redmine, dopo aver
				ricevuto una segnalazione di modifica issue da Redmine, elabora
				il messaggio e lo invia al Gestore Personale.
				Il messaggio finale, una volta terminata l'elaborazione, conterrà i campi:
				\begin{itemize}
					\item GPoject
					\item Topic
					\item Subject e opzionalmente:
					\begin{itemize}
						\item Description
					\end{itemize}
				\end{itemize}
				\item \textbf{GPecondizione}: il GPoducer Redmine ha ricevuto una segnalazione da Redmine.
				\item \textbf{Postcondizione}: il GPoducer Redmine ha elaborato e inviato al Gestore Personale il messaggio di modifica issue.
				\item \textbf{Scenario GPincipale}: 
				\begin{enumerate}
					\item GPoducer Redmine GPocede all'invio del messaggio di
					modifica issue al gestore personale.
				\end{enumerate}
				
			\end{itemize}

\stepcounter{uccount}

\subsubsection{UC\theuccount-GP - Producer GitLab invia messaggio al Gestore Personale}
	\begin{figure}[H]
		\centering
		\includegraphics[width=0.7\textwidth]{img/UC1.png}\\
		\caption{UC\theuccount-GP - Producer GitLab invia messaggio al Gestore Personale}
	\end{figure}
	\begin{itemize}
		\item \textbf{Codice}: UC\theuccount-GP.
		\item \textbf{Titolo}: Producer GitLab invia messaggio al Gestore Personale.
		\item \textbf{Attori primari}: Producer GitLab.
		\item \textbf{Descrizione}: il Producer GitLab, dopo aver ricevuto una segnalazione da GitLab, elabora un messaggio da inviare al Gestore Personale.
		\item \textbf{Precondizione}: il Producer GitLab ha ricevuto una segnalazione da GitLab.
		\item \textbf{Postcondizione}: l Producer GitLab ha inviato al Gestore Personale il messaggio elaborato.
		\item \textbf{Scenario principale}: 
		\begin{enumerate}
			\item Producer GitLab procede all'invio del messaggio al Gestore Personale.
		\end{enumerate}
		
	\end{itemize}
	
	\paragraph{UC\theuccount.1-GP - Producer GitLab invia uno o più messaggi di commit al Gestore Personale}
		\begin{figure}[H]
			\centering
			\includegraphics[width=0.7\textwidth]{img/UC1.png}\\
			\caption{UC\theuccount.1-GP - Producer GitLab invia uno o più messaggi di commit al Gestore Personale}
		\end{figure}
		\begin{itemize}
			\item \textbf{Codice}: UC\theuccount.1-GP.
			\item \textbf{Titolo}: Producer GitLab invia uno o più messaggi di commit al Gestore Personale.
			\item \textbf{Attori primari}: Producer GitLab.
			\item \textbf{Descrizione}: il Producer GitLab, dopo aver ricevuto una segnalazione di push da GitLab,
			elabora un messaggio per commit che verrà catalogato sotto il Topic "commits".
			Il messaggio elaborato conterrà i campi:
			\begin{itemize}
				\item Project
				\item Topic
				\item Message
			\end{itemize}
			\item \textbf{Precondizione}: il Producer GitLab ha ricevuto una segnalazione da GitLab.
			\item \textbf{Postcondizione}: il Producer GitLab ha inviato uno o più messaggi elaborati di commit.
			\item \textbf{Scenario principale}: 
			\begin{enumerate}
				\item Producer GitLab procede all'invio di uno o più messaggi
			 di commit al Gestore Personale.
			\end{enumerate}
			
		\end{itemize}
	
		\paragraph{UC\theuccount.2-GP -  Producer GitLab invia messaggio di issue al Gestore Personale}
			\begin{figure}[H]
				\centering
				\includegraphics[width=0.7\textwidth]{img/UC1.png}\\
				\caption{UC\theuccount.2-GP -  Producer GitLab invia messaggio di issue al Gestore Personale}
			\end{figure}
			\begin{itemize}
				\item \textbf{Codice}: UC\theuccount.2-GP.
				\item \textbf{Titolo}:  Producer GitLab invia messaggio di issue al Gestore Personale.
				\item \textbf{Attori primari}: Producer GitLab.
				\item \textbf{Descrizione}: il Producer GitLab, dopo aver ricevuto una segnalazione di issue da GitLab,
				controlla se la issue è appena stata creata o si tratta di una modifica di
				una issue preesistente. Il messaggio elaborato, una volta elaborato, conterrà i campi:
				\begin{itemize}
					\item Project
					\item Topic
					\item Subject e opzionalmente:
					\begin{itemize}
						\item Description
						\item Due Date
						\item Milestone
						\item Assignee
					\end{itemize}
				\end{itemize}
				\item \textbf{Precondizione}: il Producer GitLab ha ricevuto una segnalazione da GitLab.
				\item \textbf{Postcondizione}: il Producer GitLab ha inviato al Gestore Personale il messaggio \newline elaborato.
				\item \textbf{Scenario principale}: 
				\begin{enumerate}
					\item Producer GitLab procede all'invio di un messaggio di
					issue al Gestore Personale.
				\end{enumerate}
				
			\end{itemize}
		
			\subparagraph{UC\theuccount.2.1-GP - Producer GitLab invia messaggio di una nuova issue al Gestore Personale}
				\begin{figure}[H]
					\centering
					\includegraphics[width=0.7\textwidth]{img/UC1.png}\\
					\caption{UC\theuccount.2.1-GP - Producer GitLab invia messaggio di una nuova issue al Gestore Personale}
				\end{figure}
				\begin{itemize}
					\item \textbf{Codice}: UC\theuccount.2.1-GP.
					\item \textbf{Titolo}: Producer GitLab invia messaggio di una nuova issue al Gestore Personale.
					\item \textbf{Attori primari}: Producer GitLab.
					\item \textbf{Descrizione}: il Producer GitLab, dopo aver ricevuto una segnalazione di una nuova issue
					da GitLab, elabora il messaggio che conterrà i campi:
					\begin{itemize}
						\item Project
						\item Topic
						\item Subject e opzionalmente:
						\begin{itemize}
							\item Description
							\item Due Date
							\item Milestone
							\item Assignee
						\end{itemize}
					\end{itemize}
					\item \textbf{Precondizione}: il Producer GitLab ha ricevuto una segnalazione da GitLab.
					\item \textbf{Postcondizione}: il Producer GitLab ha inviato al Gestore Personale il messaggio elaborato di nuova issue.
					\item \textbf{Scenario principale}: 
					\begin{enumerate}
						\item Producer GitLab procede all'invio di un messaggio di
						nuova issue al Gestore Personale.
					\end{enumerate}
					
				\end{itemize}
		
			\subparagraph{UC\theuccount.2.2-GP - Producer GitLab invia messaggio di modifica di una issue al Gestore Personale}
				\begin{figure}[H]
					\centering
					\includegraphics[width=0.7\textwidth]{img/UC1.png}\\
					\caption{UC\theuccount.2.2-GP - Producer GitLab invia messaggio di modifica di una issue al Gestore Personale}
				\end{figure}
				\begin{itemize}
					\item \textbf{Codice}: UC\theuccount.2.2-GP.
					\item \textbf{Titolo}: Producer GitLab invia messaggio di modifica di una issue al Gestore Personale.
					\item \textbf{Attori primari}: Producer GitLab.
					\item \textbf{Descrizione}: il Producer GitLab, dopo aver ricevuto una segnalazione di modifica di una issue da
					GitLab, controlla se sono stati modificati i campi Label o Title.
					In caso positivo, viene inviato un messaggio elaborato al Gestore Personale, il quale conterrà:
					\begin{itemize}
						\item Project
						\item Topic
						\item Subject e opzionalmente:
						\begin{itemize}
							\item Description
							\item Due Date
							\item Milestone
							\item Assignee
						\end{itemize}
					\end{itemize}
					\item \textbf{Precondizione}: il Producer GitLab ha ricevuto una segnalazione da GitLab.
					\item \textbf{Postcondizione}: il Producer GitLab ha inviato al Gestore Personale il messaggio elaborato di modifica issue.
					\item \textbf{Scenario principale}: 
					\begin{enumerate}
						\item Producer GitLab procede all'invio di un messaggio di modifica issue al Gestore Personale.
					\end{enumerate}
					\item \textbf{Estensioni}: 
					\begin{enumerate}
						\item Ci sono dei messaggi non validi e vengono scartati [UCGP\theuccount.2.3].
					\end{enumerate}
				\end{itemize}
		
			\subparagraph{UC\theuccount.2.3-GP - Producer GitLab scarta i messaggi non validi}
				\begin{figure}[H]
					\centering
					\includegraphics[width=0.7\textwidth]{img/UC1.png}\\
					\caption{UC\theuccount.2.3-GP - Producer GitLab scarta i messaggi non validi}
				\end{figure}
				\begin{itemize}
					\item \textbf{Codice}: UC\theuccount.2.3-GP.
					\item \textbf{Titolo}: Producer GitLab scarta i messaggi non validi.
					\item \textbf{Attori primari}: Producer GitLab.
					\item \textbf{Descrizione}: il Producer GitLab, dopo aver ricevuto una segnalazione di una modifica issue da GitLab, controlla
					se sono state modificati i campi Label o Title. In caso negativo, il messaggio viene scartato.
					\item \textbf{Precondizione}: il Producer GitLab ha ricevuto una segnalazione da GitLab.
					\item \textbf{Postcondizione}: il Producer GitLab ha scartato il messaggio.
					\item \textbf{Scenario principale}: 
					\begin{enumerate}
						\item Producer GitLab scarta i messaggi non validi.
					\end{enumerate}
				\end{itemize}

\stepcounter{uccount}

\include{casi_d'uso/UC8-CT}

\stepcounter{uccount}

\subsubsection{UC\theuccount-CE - Gestore Personale invia il messaggio finale al Consumer Email}
	\begin{figure}[H]
		\centering
		\includegraphics[width=0.7\textwidth]{img/UC1.png}\\
		\caption{UC\theuccount-CE - Gestore Personale invia il messaggio finale al Consumer Email}
	\end{figure}
	\begin{itemize}
		\item \textbf{Codice}: UC\theuccount-CE.
		\item \textbf{Titolo}: Gestore Personale invia il messaggio finale al Consumer Email.
		\item \textbf{Attori primari}: Gestore Personale.
		\item \textbf{Descrizione}: il Gestore Personale, dopo aver ricevuto il messaggio elaborato dai Producer Redmine o GitLab,
		valuta il campo Topic del messaggio, controlla chi è iscritto a quel Topic, se la persona è disponibile, e se vuole ricevere
		il messaggio tramite email. Se tutte queste condizioni sono verificate, viene preparato il messaggio finale da inviare al
		Consumer Email. Il messaggio finale, una volta elaborato, conterrà i campi:
		\begin{itemize}
			\item Email del destinatario
			\item Applicazione di provenienza
			\item Ora di invio
			\item Tipo di segnalazione(commit, issue)
			\item Project
			\item Topic
			\item Subject e opzionalmente
		 	\begin{itemize}
				\item Description
				\item Due date
				\item Milestone
				\item Assignee
			\end{itemize}
		\end{itemize}
		\item \textbf{Precondizione}: il Gestore Personale ha ricevuto il messaggio elaborato dai Producer Redmine o GitLab.
		\item \textbf{Postcondizione}: Il Gestore Personale ha inviato il messaggio finale al Consumer Email.
		\item \textbf{Scenario principale}: 
		\begin{enumerate}
			\item Gestore Personale procede all'invio del messaggio finale al Consumer Email.
		\end{enumerate}
		
	\end{itemize}

\stepcounter{uccount}

\include{casi_d'uso/UC10-BT}

\stepcounter{uccount}

\include{casi_d'uso/UC11-SE}

\stepcounter{uccount}

\subsubsection{UC\theuccount-GP - Accesso}
		\begin{figure}[H]
			\centering
				\includegraphics[width=\columnwidth]{img/UC5.png}\\
			\caption{UC\theuccount-GP - Accesso}
		\end{figure}
	\begin{itemize}
		\item \textbf{Codice}: UC\theuccount-GP.
		\item \textbf{Titolo}: accesso.
		\item \textbf{Attori primari}: utente non acceduto.
		\item \textbf{Descrizione}: l'utente richiede di accedere al sistema attraverso un form dove inserisce l'username.
		\item \textbf{Precondizione}: il sistema considera l’utilizzatore di esso come un utente non acceduto.
		\item \textbf{Postcondizione}: il sistema riconosce l'utilizzatore di esso come utente acceduto.
		\item \textbf{Scenario Principale}:
		\begin{enumerate}
			\item L'utente non ancora riconosciuto dal sistema effettua l'accesso inserendo il proprio username.
		\end{enumerate}
	\end{itemize}
	
	\paragraph{UC\theuccount.1-GP - Accesso dell'utente nel sistema}
		\begin{figure}[H]
			\centering
				\includegraphics[width=\columnwidth]{img/UC5_1.png}\\
			\caption{UC\theuccount.1-GP - Accesso dell'utente nel sistema}
		\end{figure}
		\begin{itemize}
			\item \textbf{Codice}: UC\theuccount.1-GP.
			\item \textbf{Titolo}: accesso dell'utente nel sistema.
			\item \textbf{Attori primari}: utente non acceduto.
			\item \textbf{Descrizione}: l'utente attende l'accesso al sistema.
			\item \textbf{Precondizione}: il sistema riconosce l'utilizzatore come un utente non acceduto.
			\item \textbf{Postcondizione}: il sistema riconosce l'utente con successo.
			\item \textbf{Scenario Principale}:
			\begin{enumerate}
				\item L’utente non ancora riconosciuto dal sistema richiede l'accesso attraverso l'inserimento dell'username.
			\end{enumerate}
			\item \textbf{Estensioni}:
			\begin{enumerate}
				\item L'accesso non va a buon fine e viene visualizzato un errore avvisando l'utente [UC5.2].
			\end{enumerate}
		\end{itemize}

		\subparagraph{UC\theuccount.1.1-GP - Inserimento username}
			\begin{itemize}
				\item \textbf{Codice}: UC\theuccount.1.1-GP.
				\item \textbf{Titolo}: inserimento username.
				\item \textbf{Attori primari}: utente non acceduto.
				\item \textbf{Descrizione}: l'utente inserisce l'username.
				\item \textbf{Precondizione}: il sistema offre l'interfaccia grafica adatta all'inserimento dell'username.
				\item \textbf{Postcondizione}: l'utente ha inserito l'username desiderato.
				\item \textbf{Scenario Principale}:
				\begin{enumerate}
					\item L'utente inserisce l'username per autenticarsi.
				\end{enumerate}
			\end{itemize}

	\paragraph{UC\theuccount.2-GP - Errore username inesistente}
		\begin{itemize}
			\item \textbf{Codice}: UC\theuccount.1.2-GP.
			\item \textbf{Titolo}: errore username inesistente.
			\item \textbf{Attori primari}: utente non acceduto.
			\item \textbf{Descrizione}: l'utente viene avvisato che ha inserito un username errato.
			\item \textbf{Precondizione}: il sistema riceve una richiesta di accesso da parte di un utente che
			fornisce uno username errato. 
			\item \textbf{Postcondizione}: il sistema comunica all'utilizzatore l'errore.
			\item \textbf{Scenario Principale}:
			\begin{enumerate}
				\item L'utente visualizza il messaggio d'errore.
			\end{enumerate}
		\end{itemize}


\stepcounter{uccount}

\include{casi_d'uso/UC13-GP}

\stepcounter{uccount}

\subsubsection{UC\theuccount-GP - Aggiunta nuovo utente}
		\begin{figure}[H]
			\centering
				\includegraphics[width=\columnwidth]{img/UC5.png}\\
			\caption{UC\theuccount-GP - Aggiunta nuovo utente}
		\end{figure}
	\begin{itemize}
		\item \textbf{Codice}: UC\theuccount-GP.
		\item \textbf{Titolo}: aggiunta nuovo utente.
		\item \textbf{Attori primari}: utente acceduto.
		\item \textbf{Descrizione}: l'utente aggiunge un nuovo utente nel sistema.
		\item \textbf{Precondizione}: un nuovo utente deve essere aggiunto nel sistema.
		\item \textbf{Postcondizione}: un utente viene aggiunto al sistema.
		\item \textbf{Scenario Principale}:
		\begin{enumerate}
			\item Utente acceduto procede all'aggiunta di un nuovo utente.
		\end{enumerate}
	\end{itemize}

	\paragraph{UC\theuccount.1-GP - Utente aggiunto con successo}
		\begin{figure}[H]
			\centering
			\includegraphics[width=\columnwidth]{img/UC5.png}\\
			\caption{UC\theuccount.1-GP - Utente aggiunto con successo}
		\end{figure}
		\begin{itemize}
			\item \textbf{Codice}: UC\theuccount.1-GP.
			\item \textbf{Titolo}: utente aggiunto con successo.
			\item \textbf{Attori primari}: utente acceduto.
			\item \textbf{Descrizione}: un nuovo utente viene inserito con successo nel sistema.
			\item \textbf{Precondizione}: un nuovo utente deve essere aggiunto nel sistema.
			\item \textbf{Postcondizione}: un utente viene aggiunto al sistema.
			\item \textbf{Scenario Principale}:
			\begin{enumerate}
				\item Utente acceduto procede all'aggiunta di un nuovo utente.
			\end{enumerate}
			\item \textbf{Estensioni}:
			\begin{itemize}
				\item Errore utente già presente nel sistema[UC13-GP].
			\end{itemize}
		\end{itemize}

		\subparagraph{UC\theuccount.1.1-GP - Inserimento nome utente}
			\begin{figure}[H]
				\centering
				\includegraphics[width=\columnwidth]{img/UC5.png}\\
				\caption{UC\theuccount.1.1-GP - Inserimento nome utente}
			\end{figure}
			\begin{itemize}
				\item \textbf{Codice}: UC\theuccount.1.1-GP.
				\item \textbf{Titolo}: inserimento nome utente.
				\item \textbf{Attori primari}: utente acceduto.
				\item \textbf{Descrizione}: l'utente inserisce il nominativo dell'utente appena aggiunto.
				\item \textbf{Precondizione}: un nuovo utente deve essere aggiunto nel sistema.
				\item \textbf{Postcondizione}: il nome è stato aggiunto.
				\item \textbf{Scenario Principale}:
				\begin{enumerate}
					\item Utente acceduto procede all'aggiunta del nominativo del nuovo utente.
				\end{enumerate}
			\end{itemize}

			\subparagraph{UC\theuccount.1.2-GP - Inserimento cognome utente}
				\begin{figure}[H]
					\centering
					\includegraphics[width=\columnwidth]{img/UC5.png}\\
					\caption{UC\theuccount.1.2-GP - Inserimento cognome utente}
				\end{figure}
				\begin{itemize}
					\item \textbf{Codice}: UC\theuccount.1.2-GP.
					\item \textbf{Titolo}: inserimento cognome utente.
					\item \textbf{Attori primari}: utente acceduto.
					\item \textbf{Descrizione}: l'utente inserisce il cognome dell'utente appena aggiunto.
					\item \textbf{Precondizione}: un nuovo utente deve essere aggiunto nel sistema.
					\item \textbf{Postcondizione}: il cognome è stato aggiunto.
					\item \textbf{Scenario Principale}:
					\begin{enumerate}
						\item Utente acceduto procede all'aggiunta del cognome del nuovo utente.
					\end{enumerate}
				\end{itemize}

				\subparagraph{UC\theuccount.1.3-GP - Inserimento contatto Email}
					\begin{figure}[H]
						\centering
						\includegraphics[width=\columnwidth]{img/UC5.png}\\
						\caption{UC\theuccount.1.3-GP - Inserimento contatto Email}
					\end{figure}
					\begin{itemize}
						\item \textbf{Codice}: UC\theuccount.1.3-GP.
						\item \textbf{Titolo}: inserimento contatto Email.
						\item \textbf{Attori primari}: utente acceduto.
						\item \textbf{Descrizione}: l'utente inserisce il contatto Email dell'utente appena aggiunto.
						\item \textbf{Precondizione}: un nuovo utente deve essere aggiunto nel sistema.
						\item \textbf{Postcondizione}: il contatto Email è stato aggiunto.
						\item \textbf{Scenario Principale}:
						\begin{enumerate}
							\item Utente acceduto procede all'aggiunta del contatto Email del nuovo utente.
						\end{enumerate}
				\end{itemize}

				\subparagraph{UC\theuccount.1.4-GP - Inserimento contatto Telegram}
					\begin{figure}[H]
						\centering
						\includegraphics[width=\columnwidth]{img/UC5.png}\\
						\caption{UC\theuccount.1.4-GP - Inserimento contatto Telegram}
					\end{figure}
					\begin{itemize}
						\item \textbf{Codice}: UC\theuccount.1.4-GP.
						\item \textbf{Titolo}: inserimento contatto Telegram.
						\item \textbf{Attori primari}: utente acceduto.
						\item \textbf{Descrizione}: l'utente inserisce il contatto Telegram dell'utente appena aggiunto.
						\item \textbf{Precondizione}: un nuovo utente deve essere aggiunto nel sistema.
						\item \textbf{Postcondizione}: il contatto Telegram è stato aggiunto.
						\item \textbf{Scenario Principale}:
						\begin{enumerate}
							\item Utente acceduto procede all'aggiunta del contatto Telegram del nuovo utente.
						\end{enumerate}
					\end{itemize}

			\paragraph{UC\theuccount.2-GP - Errore utente già presente nel sistema}
				\begin{figure}[H]
					\centering
					\includegraphics[width=\columnwidth]{img/UC5.png}\\
					\caption{UC\theuccount.2-GP - Errore utente già presente nel sistema}
				\end{figure}
				\begin{itemize}
					\item \textbf{Codice}: UC\theuccount.2-GP.
					\item \textbf{Titolo}: errore utente già presente nel sistema.
					\item \textbf{Attori primari}: utente acceduto.
					\item \textbf{Descrizione}: l’utente viene avvisato che i contatti Telegram o email immessi non sono univoci.
					\item \textbf{Precondizione}: un nuovo utente deve essere aggiunto nel sistema.
					\item \textbf{Postcondizione}: il sistema comunica all’utilizzatore l’errore e l'utente non viene inserito.
					\item \textbf{Scenario Principale}:
					\begin{enumerate}
						\item Utente acceduto visualizza il messaggio d'errore.
					\end{enumerate}
				\end{itemize}

\stepcounter{uccount}

\subsubsection{UC\theuccount-GP - Rimozione utente dal sistema}
		\begin{figure}[H]
			\centering
				\includegraphics[width=\columnwidth]{img/UC5.png}\\
			\caption{UC\theuccount-GP - Rimozione utente dal sistema}
		\end{figure}
	\begin{itemize}
		\item \textbf{Codice}: UC\theuccount-GP.
		\item \textbf{Titolo}: Rimozione utente dal sistema.
		\item \textbf{Attori primari}: utente acceduto.
		\item \textbf{Descrizione}: l'utente rimuove l'utente desiderato dal sistema.
		\item \textbf{Precondizione}: un utente già presente deve essere rimosso dal sistema.
		\item \textbf{Postcondizione}: un utente viene rimosso dal sistema.
		\item \textbf{Scenario Principale}:
		\begin{enumerate}
			\item Utente acceduto procede alla rimozione di un utente.
		\end{enumerate}
\end{itemize}

	\paragraph{UC\theuccount.1-GP - Rimozione avvenuta con successo}
		\begin{figure}[H]
			\centering
			\includegraphics[width=\columnwidth]{img/UC5.png}\\
			\caption{UC\theuccount.1-GP - Rimozione avvenuta con successo}
		\end{figure}
		\begin{itemize}
			\item \textbf{Codice}: UC\theuccount.1-GP.
			\item \textbf{Titolo}: rimozione avvenuta con successo.
			\item \textbf{Attori primari}: utente acceduto.
			\item \textbf{Descrizione}: il contatto Email o Telegram desiderato è presente nel sistema,
			per cui la rimozione avviene con successo.
			\item \textbf{Precondizione}: un utente già presente deve essere rimosso dal sistema.
			\item \textbf{Postcondizione}: un utente con il contatto Email o Telegram inserito viene rimosso dal sistema.
			\item \textbf{Scenario Principale}:
			\begin{enumerate}
				\item Un utente viene rimosso.
			\end{enumerate}
			\item \textbf{Estensioni}:
			\begin{itemize}
				\item Errore contatto non presente nel sistema[UC14.2-GP].
			\end{itemize}
		\end{itemize}

			\subparagraph{UC\theuccount.1.1-GP - Inserimento contatto Email}
				\begin{figure}[H]
					\centering
					\includegraphics[width=\columnwidth]{img/UC5.png}\\
					\caption{UC\theuccount.1.1-GP - Inserimento contatto Email}
				\end{figure}
				\begin{itemize}
					\item \textbf{Codice}: UC\theuccount.1.1-GP.
					\item \textbf{Titolo}: inserimento contatto Email.
					\item \textbf{Attori primari}: utente acceduto.
					\item \textbf{Descrizione}: l'utente ha aggiunto il contatto Email relativo all'utente che vuole rimuovere.
					\item \textbf{Precondizione}: un utente già presente deve essere rimosso dal sistema.
					\item \textbf{Postcondizione}: il contatto Email è stato inserito.
					\item \textbf{Scenario Principale}:
					\begin{enumerate}
						\item Utente acceduto procede all'inserimento del contatto Email del utente da rimuovere.
					\end{enumerate}
				\end{itemize}

				\subparagraph{UC\theuccount.1.2-GP - Inserimento contatto Telegram}
					\begin{figure}[H]
						\centering
						\includegraphics[width=\columnwidth]{img/UC5.png}\\
						\caption{UC\theuccount.1.2-GP - Inserimento contatto Telegram}
					\end{figure}
					\begin{itemize}
						\item \textbf{Codice}: UC\theuccount.1.2-GP.
						\item \textbf{Titolo}: inserimento contatto Telegram.
						\item \textbf{Attori primari}: utente acceduto.
						\item \textbf{Descrizione}: l'utente ha aggiunto il contatto Telegram relativo all'utente che vuole rimuovere.
						\item \textbf{Precondizione}: un utente già presente deve essere rimosso dal sistema.
						\item \textbf{Postcondizione}: il contatto Telegram è stato inserito.
						\item \textbf{Scenario Principale}:
						\begin{enumerate}
							\item Utente acceduto procede all'inserimento del contatto Telegram del utente da rimuovere.
						\end{enumerate}
					\end{itemize}

				\paragraph{UC\theuccount.2-GP - Errore contatto non presente nel sistema}
						\begin{figure}[H]
							\centering
							\includegraphics[width=\columnwidth]{img/UC5.png}\\
							\caption{UC\theuccount.2-GP - Errore contatto non presente nel sistema}
						\end{figure}
						\begin{itemize}
							\item \textbf{Codice}: UC\theuccount.2-GP.
							\item \textbf{Titolo}: errore contatto non presente nel sistema.
							\item \textbf{Attori primari}: utente acceduto.
							\item \textbf{Descrizione}: l’utente viene avvisato che il contatto inserito non è presente nel sistema.
							\item \textbf{Precondizione}: un utente già presente deve essere rimosso dal sistema.
							\item \textbf{Postcondizione}: il sistema comunica all’utilizzatore l’errore e nessuno user viene rimosso.
							\item \textbf{Scenario Principale}:
							\begin{enumerate}
								\item Un utente non viene rimosso perchè non presente nel sistema.
							\end{enumerate}
						\end{itemize}

\stepcounter{uccount}

\subsubsection{UC\theuccount-GP - Modifica utente}
		\begin{figure}[H]
			\centering
				\includegraphics[width=\columnwidth]{img/UC5.png}\\
			\caption{UC\theuccount-GP - Modifica utente}
		\end{figure}
	\begin{itemize}
		\item \textbf{Codice}: UC\theuccount-GP.
		\item \textbf{Titolo}: modifica utente.
		\item \textbf{Attori primari}: utente acceduto.
		\item \textbf{Descrizione}: l’utente vuole modificare le informazioni relative a un utente.
		\item \textbf{Precondizione}: l'utente acceduto vuole modificare un utente già presente.
		\item \textbf{Postcondizione}: i campi dell'utente sono stati modificati correttamente.
		\item \textbf{Scenario Principale}:
		\begin{enumerate}
			\item Utente acceduto procede alla modifica di un utente.
		\end{enumerate}
	\end{itemize}

	\paragraph{UC\theuccount.1-GP - Selezione ID utente}
		\begin{figure}[H]
			\centering
			\includegraphics[width=\columnwidth]{img/UC5.png}\\
			\caption{UC\theuccount.1-GP - Selezione ID utente}
		\end{figure}
		\begin{itemize}
			\item \textbf{Codice}: UC\theuccount.1-GP.
			\item \textbf{Titolo}: selezione ID utente.
			\item \textbf{Attori primari}: utente acceduto.
			\item \textbf{Descrizione}: l'utente ha aggiunto il nuovo nome dell'utente che vuole modificare.
			\item \textbf{Precondizione}: l'utente acceduto vuole modificare un utente già presente.
			\item \textbf{Postcondizione}: l'ID utente è stato inserito.
			\item \textbf{Scenario Principale}:
			\begin{enumerate}
				\item Utente acceduto procede all'inserimento dell'ID utente da modificare.
			\end{enumerate}
			\item \textbf{Estensioni}:
			\begin{itemize}
				\item Errore user ID inesistente[UC15.2-GP]
			\end{itemize}
		\end{itemize}
	
		\subparagraph{UC\theuccount.1.1-GP - Modifica utente avvenuta con successo}
			\begin{figure}[H]
				\centering
				\includegraphics[width=\columnwidth]{img/UC5.png}\\
				\caption{UC\theuccount.1.1-GP - Modifica utente avvenuta con successo}
			\end{figure}
			\begin{itemize}
				\item \textbf{Codice}: UC\theuccount.1.1-GP.
				\item \textbf{Titolo}: modifica utente avvenuta con successo.
				\item \textbf{Attori primari}: utente acceduto.
				\item \textbf{Descrizione}: l'ID utente è presente nel sistema e ne vengono modificati i relativi campi con successo.
				\item \textbf{Precondizione}: l'utente acceduto vuole modificare un utente già presente.
				\item \textbf{Postcondizione}: l'utente è stato modificato con successo.
				\item \textbf{Scenario Principale}:
				\begin{enumerate}
					\item Utente viene modificato con successo.
				\end{enumerate}
			\end{itemize}
		
			\subsubparagraph{UC\theuccount.1.1.1-GP - Inserimento del nuovo nome}
				\begin{figure}[H]
					\centering
					\includegraphics[width=\columnwidth]{img/UC5.png}\\
					\caption{UC\theuccount.1.1.1-GP - Inserimento del nuovo nome}
				\end{figure}
				\begin{itemize}
					\item \textbf{Codice}: UC\theuccount.1.1.1-GP.
					\item \textbf{Titolo}: inserimento del nuovo nome.
					\item \textbf{Attori primari}: utente acceduto.
					\item \textbf{Descrizione}: l'utente aggiunge il nuovo nome relativo all'ID utente inserito che vuole modificare.
					\item \textbf{Precondizione}: l'utente acceduto vuole modificare un utente già presente.
					\item \textbf{Postcondizione}: il nome è stato inserito.
					\item \textbf{Scenario Principale}:
					\begin{enumerate}
						\item Utente acceduto inserisce il nuovo nome dell'utente che vuole modificare.
					\end{enumerate}
				\end{itemize}
			
			\subsubparagraph{UC\theuccount.1.1.2-GP - Inserimento del nuovo cognome}
				\begin{figure}[H]
					\centering
					\includegraphics[width=\columnwidth]{img/UC5.png}\\
					\caption{UC\theuccount.1.1.2-GP - Inserimento del nuovo cognome}
				\end{figure}
				\begin{itemize}
					\item \textbf{Codice}: UC\theuccount.1.1.2-GP.
					\item \textbf{Titolo}: inserimento del nuovo cognome.
					\item \textbf{Attori primari}: utente acceduto.
					\item \textbf{Descrizione}: l'utente aggiunge il nuovo cognome relativo all'ID utente inserito che vuole modificare.
					\item \textbf{Precondizione}: l'utente acceduto vuole modificare un utente già presente.
					\item \textbf{Postcondizione}: il cognome è stato inserito.
					\item \textbf{Scenario Principale}:
					\begin{enumerate}
						\item Utente acceduto inserisce il nuovo cognome dell'utente che vuole modificare.
					\end{enumerate}
				\end{itemize}
			
			\subsubparagraph{UC\theuccount.1.1.3-GP - Inserimento del nuovo contatto Email}
				\begin{figure}[H]
					\centering
					\includegraphics[width=\columnwidth]{img/UC5.png}\\
					\caption{UC\theuccount.1.1.3-GP - Inserimento del nuovo contatto Email}
				\end{figure}
				\begin{itemize}
					\item \textbf{Codice}: UC\theuccount.1.1.3-GP.
					\item \textbf{Titolo}: inserimento del nuovo contatto Email.
					\item \textbf{Attori primari}: utente acceduto.
					\item \textbf{Descrizione}: l'utente aggiunge il nuovo contatto Email relativo all'ID utente inserito che vuole modificare.
					\item \textbf{Precondizione}: l'utente acceduto vuole modificare un utente già presente.
					\item \textbf{Postcondizione}: il contatto Email è stato inserito.
					\item \textbf{Scenario Principale}:
					\begin{enumerate}
						\item Utente acceduto inserisce il nuovo contatto Email dell'utente che vuole modificare.
					\end{enumerate}
				\end{itemize}
			
			\subsubparagraph{UC\theuccount.1.1.4-GP - Inserimento del nuovo contatto Telegram}
				\begin{figure}[H]
					\centering
					\includegraphics[width=\columnwidth]{img/UC5.png}\\
					\caption{UC\theuccount.1.1.4-GP - Inserimento del nuovo contatto Telegram}
				\end{figure}
				\begin{itemize}
					\item \textbf{Codice}: UC\theuccount.1.1.4-GP.
					\item \textbf{Titolo}: inserimento del nuovo contatto Telegram.
					\item \textbf{Attori primari}: utente acceduto.
					\item \textbf{Descrizione}: l'utente aggiunge il nuovo contatto Telegram relativo all'ID utente inserito che vuole modificare.
					\item \textbf{Precondizione}: l'utente acceduto vuole modificare un utente già presente.
					\item \textbf{Postcondizione}: il contatto Telegram è stato inserito.
					\item \textbf{Scenario Principale}:
					\begin{enumerate}
						\item Utente acceduto inserisce il nuovo contatto Telegram dell'utente che vuole modificare.
					\end{enumerate}
				\end{itemize}
		
	\paragraph{UC\theuccount.2-GP - Errore ID utente inesistente}
		\begin{figure}[H]
			\centering
			\includegraphics[width=\columnwidth]{img/UC5.png}\\
			\caption{UC\theuccount.2-GP - Errore ID utente inesistente}
		\end{figure}
		\begin{itemize}
			\item \textbf{Codice}: UC\theuccount.2-GP.
			\item \textbf{Titolo}: errore ID utente inesistente.
			\item \textbf{Attori primari}: utente acceduto.
			\item \textbf{Descrizione}:  l’utente viene avvisato che ha inserito un'ID utente errato.
			\item \textbf{Precondizione}: l'utente acceduto vuole modificare un utente già presente.
			\item \textbf{Postcondizione}: il sistema comunica all’utilizzatore l’errore.
			\item \textbf{Scenario Principale}:
			\begin{enumerate}
				\item L'utente ha inserito un ID utente errato e il sistema comunica all’utilizzatore l’errore.
			\end{enumerate}
		\end{itemize}

\stepcounter{uccount}

\subsubsection{UC\theuccount-GP - Aggiunta preferenze}
		\begin{figure}[H]
			\centering
				\includegraphics[width=\columnwidth]{img/UC5.png}\\
			\caption{UC\theuccount-GP - Aggiunta preferenze}
		\end{figure}
	\begin{itemize}
		\item \textbf{Codice}: UC\theuccount-GP.
		\item \textbf{Titolo}: aggiunta preferenze.
		\item \textbf{Attori primari}: utente acceduto.
		\item \textbf{Descrizione}: l’utente, date le varie opzioni per configurare Butterfly, aggiunge una
		preferenza tra Topic, giorni di calendario, piattaforma di messaggistica (Telegram o e-mail)
		preferita e la persona di fiducia che lo può sostituire.
		\item \textbf{Precondizione}: l’utente ha acceduto con le sue credenziali corrette nel sistema e non
		ha già selezionato tutte le preferenze possibili proposte da Butterfly.
		\item \textbf{Postcondizione}: la nuova configurazione contiene una o più preferenze in aggiunta rispetto a quella precedente.
		\item \textbf{Scenario Principale}:
		\begin{enumerate}
			\item Utente acceduto procede all'aggiunta di una o più preferenze.
		\end{enumerate}
	\end{itemize}


	\paragraph{UC\theuccount.1-GP - Iscrizione Topic}
		\begin{figure}[H]
			\centering
			\includegraphics[width=\columnwidth]{img/UC5.png}\\
			\caption{UC\theuccount.1-GP - Iscrizione Topic}
		\end{figure}
		\begin{itemize}
			\item \textbf{Codice}: UC\theuccount.1-GP.
			\item \textbf{Titolo}: iscrizione Topic.
			\item \textbf{Attori primari}: utente acceduto.
			\item \textbf{Descrizione}: data la lista di Topic presenti, l’utente ne seleziona uno o
			più a cui è interessato, ricevendone una notifica. I Topic sono divisi per categoria e
			comprendono etichette, progetto a cui sono legate e l'applicazione di provenienza: Redmine o GitLab.
			\item \textbf{Precondizione}: l’utente ha acceduto correttamente nel sistema e non ha già
			selezionato tutti i Topic possibili proposti da Butterfly.
			\item \textbf{Postcondizione}: il numero di Topic a cui è interessato l’utente è aumentato.
			\item \textbf{Scenario Principale}:
			\begin{enumerate}
				\item Utente acceduto procede all'iscrizione di uno o più Topic.
			\end{enumerate}
		\end{itemize}
	
		\paragraph{UC\theuccount.2-GP - Aggiunta dei giorni di indisponibilità nel calendario}
			\begin{figure}[H]
				\centering
				\includegraphics[width=\columnwidth]{img/UC5.png}\\
				\caption{UC\theuccount.2-GP - Aggiunta dei giorni di indisponibilità nel calendario}
			\end{figure}
			\begin{itemize}
				\item \textbf{Codice}: UC\theuccount.2-GP.
				\item \textbf{Titolo}: aggiunta dei giorni di indisponibilità nel calendario.
				\item \textbf{Attori primari}: utente acceduto.
				\item \textbf{Descrizione}: dato il calendario lavorativo, l’utente aggiunge uno o più
				giorni in cui non è reperibile e non vuole ricevere notifiche.
				\item \textbf{Precondizione}: l’utente ha acceduto correttamente nel sistema e non
				ha già selezionato tutti i giorni di indisponibilità.
				\item \textbf{Postcondizione}: il numero di giorni in cui l’utente non si rende disponibile è aumentato.
				\item \textbf{Scenario Principale}:
				\begin{enumerate}
					\item Utente acceduto procede all'inserimento di uno o più giorni di indisponibilità.
				\end{enumerate}
			\end{itemize}
		
			\paragraph{UC\theuccount.3-GP - Aggiunta della piattaforma di messaggistica preferita}
				\begin{figure}[H]
					\centering
					\includegraphics[width=\columnwidth]{img/UC5.png}\\
					\caption{UC\theuccount.3-GP - Aggiunta della piattaforma di messaggistica preferita}
				\end{figure}
				\begin{itemize}
					\item \textbf{Codice}: UC\theuccount.3-GP.
					\item \textbf{Titolo}: aggiunta della piattaforma di messaggistica preferita.
					\item \textbf{Attori primari}: utente acceduto.
					\item \textbf{Descrizione}: l’utente aggiunge la sua preferenza tra Telegram e email dove
					vuole ricevere le notifiche.
					\item \textbf{Precondizione}: l’utente ha acceduto correttamente nel sistema e non ha
					già selezionato tutte le piattaforme di messaggistica possibili proposte da Butterfly.
					\item \textbf{Postcondizione}: il numero di piattaforme di messaggistica selezionate dall’utente è aumentato.
					\item \textbf{Scenario Principale}:
					\begin{enumerate}
						\item Utente acceduto procede all'aggiunta di una o più piattaforme di messaggistica.
					\end{enumerate}
				\end{itemize}
			
			\paragraph{UC\theuccount.4-GP - Aggiunta persona di fiducia}
				\begin{figure}[H]
					\centering
					\includegraphics[width=\columnwidth]{img/UC5.png}\\
					\caption{UC\theuccount.4-GP - Aggiunta persona di fiducia}
				\end{figure}
				\begin{itemize}
					\item \textbf{Codice}: UC\theuccount.4-GP.
					\item \textbf{Titolo}: aggiunta persona di fiducia.
					\item \textbf{Attori primari}: utente acceduto.
					\item \textbf{Descrizione}: l’utente acceduto aggiunge l'utente legato a un ID di sua preferenza a
					cui inoltrare i messaggi in caso di indisponibilità.
					\item \textbf{Precondizione}: l’utente ha acceduto con le sue credenziali corrette nel
					sistema e non ha già selezionato la persona a cui inoltrare le notifiche.
					\item \textbf{Postcondizione}: la preferenza viene aggiunta correttamente.
					\item \textbf{Scenario Principale}:
					\begin{enumerate}
						\item Utente acceduto procede all'aggiunta della sua persona di fiducia.
					\end{enumerate}
					\item \textbf{Estensioni}:
					\begin{enumerate}
						\item Errore ID persona di fiducia inesistente[UC16.5-GP].
					\end{enumerate}
				\end{itemize}
			
			\paragraph{UC\theuccount.5-GP - Errore ID persona di fiducia inesistente}
				\begin{figure}[H]
					\centering
					\includegraphics[width=\columnwidth]{img/UC5.png}\\
					\caption{UC\theuccount.5-GP - Errore ID persona di fiducia inesistente}
				\end{figure}
				\begin{itemize}
					\item \textbf{Codice}: UC\theuccount.5-GP.
					\item \textbf{Titolo}: errore ID persona di fiducia inesistente.
					\item \textbf{Attori primari}: utente acceduto.
					\item \textbf{Descrizione}: l’utente viene avvisato che ha inserito un ID utente errato.
					\item \textbf{Precondizione}: l’utente ha acceduto con le sue credenziali corrette nel
					sistema e non ha già selezionato la persona a cui inoltrare le notifiche.
					\item \textbf{Postcondizione}: il sistema comunica all’utilizzatore l’errore di preferenza.
					\item \textbf{Scenario Principale}:
					\begin{enumerate}
						\item Utente acceduto procede all'aggiunta della sua persona di fiducia ma questa non esiste e
						visualizza l'errore.
					\end{enumerate}
				\end{itemize}
			
			\paragraph{UC\theuccount.6-GP - Aggiunta keyword per i push di GitLab}
				\begin{figure}[H]
					\centering
					\includegraphics[width=\columnwidth]{img/UC5.png}\\
					\caption{UC\theuccount.6-GP - Aggiunta keyword per i push di GitLab}
				\end{figure}
				\begin{itemize}
					\item \textbf{Codice}: UC\theuccount.6-GP.
					\item \textbf{Titolo}: aggiunta keyword per i push di GitLab.
					\item \textbf{Attori primari}: utente acceduto.
					\item \textbf{Descrizione}: l’utente aggiunge le keyword che vuole che siano contenute
					nei messaggi di commit dei push di cui vuole ricevere la notifica.
					\item \textbf{Precondizione}: l’utente ha acceduto con le sue credenziali corrette nel sistema.
					\item \textbf{Postcondizione}: nelle nuove configurazioni dell'utente selezionato sono
					presenti una o più nuove keyword per ricevere notifiche da push di GitLab.
					\item \textbf{Scenario Principale}:
					\begin{enumerate}
						\item Utente acceduto procede all'aggiunta di una o più nuove keyword.
					\end{enumerate}
				\end{itemize}

\stepcounter{uccount}

\subsubsection{UC\theuccount-GP - Rimozione preferenze}
		\begin{figure}[H]
			\centering
				\includegraphics[width=\columnwidth]{img/UC5.png}\\
			\caption{UC\theuccount-GP - Rimozione preferenze}
		\end{figure}
	\begin{itemize}
		\item \textbf{Codice}: UC\theuccount-GP.
		\item \textbf{Titolo}: rimozione preferenze.
		\item \textbf{Attori primari}: utente acceduto.
		\item \textbf{Descrizione}: l’utente, dopo aver selezionato delle preferenze dalle opzioni
		di configurazione, ne rimuove una o più. Le preferenze consistono in Topic, date di calendario,
		piattaforma di messaggistica (Telegram e email) e persona di fiducia che lo può sostituire.
		\item \textbf{Precondizione}: l’utente ha eseguito l'accesso nel sistema ed è presente almeno
		una preferenza selezionata tra quelle proposte da Butterfly.
		\item \textbf{Postcondizione}: la nuova configurazione contiene una o più preferenze in meno rispetto
		a quella precedente.
		\item \textbf{Scenario Principale}:
		\begin{enumerate}
			\item Utente acceduto procede alla rimozione di una o più preferenze.
		\end{enumerate}
	\end{itemize}


	\paragraph{UC\theuccount.1-GP - Disiscrizione Topic}
		\begin{figure}[H]
			\centering
			\includegraphics[width=\columnwidth]{img/UC5.png}\\
			\caption{UC\theuccount.1-GP - Disiscrizione Topic}
		\end{figure}
		\begin{itemize}
			\item \textbf{Codice}: UC\theuccount.1-GP.
			\item \textbf{Titolo}: disiscrizione Topic.
			\item \textbf{Attori primari}: utente acceduto.
			\item \textbf{Descrizione}: l’utente si disiscrive da uno o più Topic dai quali prima riceveva
			delle notifiche.
			\item \textbf{Precondizione}: l’utente ha acceduto correttamente nel sistema e non ha già
			selezionato tutti i Topic possibili proposti da Butterfly.
			\item \textbf{Postcondizione}: il numero di Topic a cui è iscritto l’utente è diminuito.
			\item \textbf{Scenario Principale}:
			\begin{enumerate}
				\item Utente acceduto procede alla disiscrizione di uno o più Topic.
			\end{enumerate}
		\end{itemize}
		
		\paragraph{UC\theuccount.2-GP - Rimozione di uno o più giorni di irreperibilità nel calendario}
		\begin{figure}[H]
			\centering
			\includegraphics[width=\columnwidth]{img/UC5.png}\\
			\caption{UC\theuccount.2-GP - Rimozione di uno o più giorni di irreperibilità nel calendario}
		\end{figure}
		\begin{itemize}
			\item \textbf{Codice}: UC\theuccount.2-GP.
			\item \textbf{Titolo}: rimozione di uno o più giorni di irreperibilità nel calendario.
			\item \textbf{Attori primari}: utente acceduto.
			\item \textbf{Descrizione}: l’utente rimuove i giorni di calendario in cui precedentemente
			non era reperibile, tornando disponibile.
			\item \textbf{Precondizione}: l’utente ha acceduto correttamente nel sistema ed è presente
			almeno un giorno di calendario selezionato tra quelli proposti da Butterfly.
			\item \textbf{Postcondizione}: il numero di giorni di calendario in cui l’utente non è reperibile è diminuito.
			\item \textbf{Scenario Principale}:
			\begin{enumerate}
				\item Utente acceduto procede alla rimozione di uno o più giorni di irreperibilità.
			\end{enumerate}
		\end{itemize}
		
		\paragraph{UC\theuccount.3-GP - Rimozione preferenza piattaforma di messaggistica}
		\begin{figure}[H]
			\centering
			\includegraphics[width=\columnwidth]{img/UC5.png}\\
			\caption{UC\theuccount.3-GP - Rimozione preferenza piattaforma di messaggistica}
		\end{figure}
		\begin{itemize}
			\item \textbf{Codice}: UC\theuccount.3-GP.
			\item \textbf{Titolo}: rimozione preferenza piattaforma di messaggistica.
			\item \textbf{Attori primari}: utente acceduto.
			\item \textbf{Descrizione}: l’utente rimuove una o più preferenze tra Telegram e email dalle
			quali non vuole più ricevere notifiche tramite Butterfly.
			\item \textbf{Precondizione}: l’utente ha acceduto correttamente nel sistema ed è presente
			almeno una piattaforma di messaggistica selezionata tra quelle proposte da Butterfly.
			\item \textbf{Postcondizione}: il numero di piattaforme di messaggistica da cui l’utente vuole ricevere notifiche è diminuito.
			\item \textbf{Scenario Principale}:
			\begin{enumerate}
				\item Utente acceduto procede alla rimozione di una o più piattaforme di messaggistica.
			\end{enumerate}
		\end{itemize}
		
		\paragraph{UC\theuccount.4-GP - Rimozione persona di fiducia}
		\begin{figure}[H]
			\centering
			\includegraphics[width=\columnwidth]{img/UC5.png}\\
			\caption{UC\theuccount.4-GP - Aggiunta persona di fiducia}
		\end{figure}
		\begin{itemize}
			\item \textbf{Codice}: UC\theuccount.4-GP.
			\item \textbf{Titolo}: aggiunta persona di fiducia.
			\item \textbf{Attori primari}: utente acceduto.
			\item \textbf{Descrizione}:  l’utente rimuove lo user legato a un ID di sua preferenza a cui inoltrare
			i messaggi in caso di indisponibilità.
			\item \textbf{Precondizione}: l’utente ha eseguito l'accesso nel sistema ed è presente almeno
			uno user con l'ID selezionato tra quelle proposte da Butterfly.
			\item \textbf{Postcondizione}: la preferenza viene rimossa correttamente.
			\item \textbf{Scenario Principale}:
			\begin{enumerate}
				\item Utente acceduto procede alla rimozione della sua persona di fiducia.
			\end{enumerate}
			\item \textbf{Estensioni}:
			\begin{enumerate}
				\item Errore ID persona di fiducia inesistente[UC17.5-GP].
			\end{enumerate}
		\end{itemize}
		
		\paragraph{UC\theuccount.5-GP - Errore ID persona di fiducia inesistente}
		\begin{figure}[H]
			\centering
			\includegraphics[width=\columnwidth]{img/UC5.png}\\
			\caption{UC\theuccount.5-GP - Errore ID persona di fiducia inesistente}
		\end{figure}
		\begin{itemize}
			\item \textbf{Codice}: UC\theuccount.5-GP.
			\item \textbf{Titolo}: errore ID persona di fiducia inesistente.
			\item \textbf{Attori primari}: utente acceduto.
			\item \textbf{Descrizione}: l’utente viene avvisato che ha inserito un ID utente errato.
			\item \textbf{Precondizione}: l’utente ha acceduto con le sue credenziali corrette nel
			sistema e non ha già selezionato la persona a cui inoltrare le notifiche.
			\item \textbf{Postcondizione}: il sistema comunica all’utilizzatore l’errore di preferenza.
			\item \textbf{Scenario Principale}:
			\begin{enumerate}
				\item Utente acceduto procede alla rimozione della sua persona di fiducia ma questa non esiste e
				visualizza l'errore.
			\end{enumerate}
		\end{itemize}
		
		\paragraph{UC\theuccount.6-GP - Errore keyword da rimuovere non presente}
		\begin{figure}[H]
			\centering
			\includegraphics[width=\columnwidth]{img/UC5.png}\\
			\caption{UC\theuccount.6-GP - Errore keyword da rimuovere non presente}
		\end{figure}
		\begin{itemize}
			\item \textbf{Codice}: UC\theuccount.6-GP.
			\item \textbf{Titolo}: errore keyword da rimuovere non presente.
			\item \textbf{Attori primari}: utente acceduto.
			\item \textbf{Descrizione}: la keyword che l'utente intende rimuovere non è registrata nel sistema.
			\item \textbf{Precondizione}: l’utente ha acceduto con le sue credenziali corrette nel sistema.
			\item \textbf{Postcondizione}: viene visualizzato un messaggio d'errore con indicato che la keyword
			selezionata non è presente nel sistema.
			\item \textbf{Scenario Principale}:
			\begin{enumerate}
				\item Utente acceduto procede alla rimozione di una keyword che non è presente nel sistema e
				visualizza l'errore.
			\end{enumerate}
		\end{itemize}
		