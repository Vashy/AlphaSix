\subsubsection{UC\theuccount-GP - Producer GitLab invia messaggio al Gestore Personale}
	\begin{figure}[H]
		\centering
		\includegraphics[width=0.7\textwidth]{img/UC1.png}\\
		\caption{UC\theuccount-GP - Producer GitLab invia messaggio al Gestore Personale}
	\end{figure}
	\begin{itemize}
		\item \textbf{Codice}: UC\theuccount-GP.
		\item \textbf{Titolo}: Producer GitLab invia messaggio al Gestore Personale.
		\item \textbf{Attori primari}: Producer GitLab.
		\item \textbf{Descrizione}: il Producer GitLab, dopo aver ricevuto una segnalazione da GitLab, elabora un messaggio da inviare al Gestore Personale.
		\item \textbf{Precondizione}: il Producer GitLab ha ricevuto una segnalazione da GitLab.
		\item \textbf{Postcondizione}: l Producer GitLab ha inviato al Gestore Personale il messaggio elaborato.
		\item \textbf{Scenario principale}: 
		\begin{enumerate}
			\item Producer GitLab procede all'invio del messaggio al Gestore Personale.
		\end{enumerate}
		
	\end{itemize}
	
	\paragraph{UC\theuccount.1-GP - Producer GitLab invia uno o più messaggi di commit al Gestore Personale}
		\begin{figure}[H]
			\centering
			\includegraphics[width=0.7\textwidth]{img/UC1.png}\\
			\caption{UC\theuccount.1-GP - Producer GitLab invia uno o più messaggi di commit al Gestore Personale}
		\end{figure}
		\begin{itemize}
			\item \textbf{Codice}: UC\theuccount.1-GP.
			\item \textbf{Titolo}: Producer GitLab invia uno o più messaggi di commit al Gestore Personale.
			\item \textbf{Attori primari}: Producer GitLab.
			\item \textbf{Descrizione}: il Producer GitLab, dopo aver ricevuto una segnalazione di push da GitLab,
			elabora un messaggio per commit che verrà catalogato sotto il Topic "commits".
			Il messaggio elaborato conterrà i campi:
			\begin{itemize}
				\item Project
				\item Topic
				\item Message
			\end{itemize}
			\item \textbf{Precondizione}: il Producer GitLab ha ricevuto una segnalazione da GitLab.
			\item \textbf{Postcondizione}: il Producer GitLab ha inviato uno o più messaggi elaborati di commit.
			\item \textbf{Scenario principale}: 
			\begin{enumerate}
				\item Producer GitLab procede all'invio di uno o più messaggi
			 di commit al Gestore Personale.
			\end{enumerate}
			
		\end{itemize}
	
		\paragraph{UC\theuccount.2-GP -  Producer GitLab invia messaggio di issue al Gestore Personale}
			\begin{figure}[H]
				\centering
				\includegraphics[width=0.7\textwidth]{img/UC1.png}\\
				\caption{UC\theuccount.2-GP -  Producer GitLab invia messaggio di issue al Gestore Personale}
			\end{figure}
			\begin{itemize}
				\item \textbf{Codice}: UC\theuccount.2-GP.
				\item \textbf{Titolo}:  Producer GitLab invia messaggio di issue al Gestore Personale.
				\item \textbf{Attori primari}: Producer GitLab.
				\item \textbf{Descrizione}: il Producer GitLab, dopo aver ricevuto una segnalazione di issue da GitLab,
				controlla se la issue è appena stata creata o si tratta di una modifica di
				una issue preesistente. Il messaggio elaborato, una volta elaborato, conterrà i campi:
				\begin{itemize}
					\item Project
					\item Topic
					\item Subject e opzionalmente:
					\begin{itemize}
						\item Description
						\item Due Date
						\item Milestone
						\item Assignee
					\end{itemize}
				\end{itemize}
				\item \textbf{Precondizione}: il Producer GitLab ha ricevuto una segnalazione da GitLab.
				\item \textbf{Postcondizione}: il Producer GitLab ha inviato al Gestore Personale il messaggio \newline elaborato.
				\item \textbf{Scenario principale}: 
				\begin{enumerate}
					\item Producer GitLab procede all'invio di un messaggio di
					issue al Gestore Personale.
				\end{enumerate}
				
			\end{itemize}
		
			\subparagraph{UC\theuccount.2.1-GP - Producer GitLab invia messaggio di una nuova issue al Gestore Personale}
				\begin{figure}[H]
					\centering
					\includegraphics[width=0.7\textwidth]{img/UC1.png}\\
					\caption{UC\theuccount.2.1-GP - Producer GitLab invia messaggio di una nuova issue al Gestore Personale}
				\end{figure}
				\begin{itemize}
					\item \textbf{Codice}: UC\theuccount.2.1-GP.
					\item \textbf{Titolo}: Producer GitLab invia messaggio di una nuova issue al Gestore Personale.
					\item \textbf{Attori primari}: Producer GitLab.
					\item \textbf{Descrizione}: il Producer GitLab, dopo aver ricevuto una segnalazione di una nuova issue
					da GitLab, elabora il messaggio che conterrà i campi:
					\begin{itemize}
						\item Project
						\item Topic
						\item Subject e opzionalmente:
						\begin{itemize}
							\item Description
							\item Due Date
							\item Milestone
							\item Assignee
						\end{itemize}
					\end{itemize}
					\item \textbf{Precondizione}: il Producer GitLab ha ricevuto una segnalazione da GitLab.
					\item \textbf{Postcondizione}: il Producer GitLab ha inviato al Gestore Personale il messaggio elaborato di nuova issue.
					\item \textbf{Scenario principale}: 
					\begin{enumerate}
						\item Producer GitLab procede all'invio di un messaggio di
						nuova issue al Gestore Personale.
					\end{enumerate}
					
				\end{itemize}
		
			\subparagraph{UC\theuccount.2.2-GP - Producer GitLab invia messaggio di modifica di una issue al Gestore Personale}
				\begin{figure}[H]
					\centering
					\includegraphics[width=0.7\textwidth]{img/UC1.png}\\
					\caption{UC\theuccount.2.2-GP - Producer GitLab invia messaggio di modifica di una issue al Gestore Personale}
				\end{figure}
				\begin{itemize}
					\item \textbf{Codice}: UC\theuccount.2.2-GP.
					\item \textbf{Titolo}: Producer GitLab invia messaggio di modifica di una issue al Gestore Personale.
					\item \textbf{Attori primari}: Producer GitLab.
					\item \textbf{Descrizione}: il Producer GitLab, dopo aver ricevuto una segnalazione di modifica di una issue da
					GitLab, controlla se sono stati modificati i campi Label o Title.
					In caso positivo, viene inviato un messaggio elaborato al Gestore Personale, il quale conterrà:
					\begin{itemize}
						\item Project
						\item Topic
						\item Subject e opzionalmente:
						\begin{itemize}
							\item Description
							\item Due Date
							\item Milestone
							\item Assignee
						\end{itemize}
					\end{itemize}
					\item \textbf{Precondizione}: il Producer GitLab ha ricevuto una segnalazione da GitLab.
					\item \textbf{Postcondizione}: il Producer GitLab ha inviato al Gestore Personale il messaggio elaborato di modifica issue.
					\item \textbf{Scenario principale}: 
					\begin{enumerate}
						\item Producer GitLab procede all'invio di un messaggio di modifica issue al Gestore Personale.
					\end{enumerate}
					\item \textbf{Estensioni}: 
					\begin{enumerate}
						\item Ci sono dei messaggi non validi e vengono scartati [UCGP\theuccount.2.3].
					\end{enumerate}
				\end{itemize}
		
			\subparagraph{UC\theuccount.2.3-GP - Producer GitLab scarta i messaggi non validi}
				\begin{figure}[H]
					\centering
					\includegraphics[width=0.7\textwidth]{img/UC1.png}\\
					\caption{UC\theuccount.2.3-GP - Producer GitLab scarta i messaggi non validi}
				\end{figure}
				\begin{itemize}
					\item \textbf{Codice}: UC\theuccount.2.3-GP.
					\item \textbf{Titolo}: Producer GitLab scarta i messaggi non validi.
					\item \textbf{Attori primari}: Producer GitLab.
					\item \textbf{Descrizione}: il Producer GitLab, dopo aver ricevuto una segnalazione di una modifica issue da GitLab, controlla
					se sono state modificati i campi Label o Title. In caso negativo, il messaggio viene scartato.
					\item \textbf{Precondizione}: il Producer GitLab ha ricevuto una segnalazione da GitLab.
					\item \textbf{Postcondizione}: il Producer GitLab ha scartato il messaggio.
					\item \textbf{Scenario principale}: 
					\begin{enumerate}
						\item Producer GitLab scarta i messaggi non validi.
					\end{enumerate}
				\end{itemize}