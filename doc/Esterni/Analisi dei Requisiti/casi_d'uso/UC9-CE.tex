\subsubsection{UC\theuccount-CE - Gestore Personale invia il messaggio finale al Consumer Email}
	\begin{figure}[H]
		\centering
		\includegraphics[width=0.7\textwidth]{img/UC1.png}\\
		\caption{UC\theuccount-CE - Gestore Personale invia il messaggio finale al Consumer Email}
	\end{figure}
	\begin{itemize}
		\item \textbf{Codice}: UC\theuccount-CE.
		\item \textbf{Titolo}: Gestore Personale invia il messaggio finale al Consumer Email.
		\item \textbf{Attori primari}: Gestore Personale.
		\item \textbf{Descrizione}: il Gestore Personale, dopo aver ricevuto il messaggio elaborato dai Producer Redmine o GitLab,
		valuta il campo Topic del messaggio, controlla chi è iscritto a quel Topic, se la persona è disponibile, e se vuole ricevere
		il messaggio tramite email. Se tutte queste condizioni sono verificate, viene preparato il messaggio finale da inviare al
		Consumer Email. Il messaggio finale, una volta elaborato, conterrà i campi:
		\begin{itemize}
			\item Email del destinatario
			\item Applicazione di provenienza
			\item Ora di invio
			\item Tipo di segnalazione(commit, issue)
			\item Project
			\item Topic
			\item Subject e opzionalmente
		 	\begin{itemize}
				\item Description
				\item Due date
				\item Milestone
				\item Assignee
			\end{itemize}
		\end{itemize}
		\item \textbf{Precondizione}: il Gestore Personale ha ricevuto il messaggio elaborato dai Producer Redmine o GitLab.
		\item \textbf{Postcondizione}: Il Gestore Personale ha inviato il messaggio finale al Consumer Email.
		\item \textbf{Scenario principale}: 
		\begin{enumerate}
			\item Gestore Personale procede all'invio del messaggio finale al Consumer Email.
		\end{enumerate}
		
	\end{itemize}