\subsubsection{UCGL\theuccount\ - GitLab segnala apertura issue al Producer GitLab}
	\begin{figure}[H]
		\centering
		\includegraphics[width=0.7\textwidth]{img/UC1.png}\\
		\caption{UCGL\theuccount\ - GitLab segnala apertura issue al Producer GitLab}
	\end{figure}
	\begin{itemize}
		\item \textbf{Codice}: UCGL\theuccount.
		\item \textbf{Titolo}: GitLab segnala apertura issue al Producer GitLab.
		\item \textbf{Attori primari}: GitLab.
		\item \textbf{Descrizione}:
		il sistema Producer GitLab ed è interno al sistema Butterfly. L'invio di
		una segnalazione avviene da parte di GitLab tramite webhook. L'apertura di
		una issue su GitLab contiene:
		\begin{itemize}
			\item Title e opzionalmente
			\begin{itemize}
				\item Label
				\item Milestone
				\item Assignees
				\item Due Date
			\end{itemize}
		\end{itemize}
		\item \textbf{Precondizione}: Viene aperta una issue su GitLab e 
		segnalata a \progetto.
		\item \textbf{Postcondizione}: il Producer GitLab riceve la segnalazione da GitLab.
		\item \textbf{Scenario principale}: 
		\begin{enumerate}
			\item GitLab invia la segnalazione di issue al Producer GitLab
		\end{enumerate}
		
	\end{itemize}
	
	\stepcounter{uccount}
	
	\subsubsection{UCGL\theuccount\ - Gitlab segnala la modifica di una issue al Producer Gitlab}
	\begin{figure}[H]
		\centering
		\includegraphics[width=0.7\textwidth]{img/UC1.png}\\
		\caption{UCGL\theuccount\ - Gitlab segnala la modifica di una issue al Producer Gitlab}
	\end{figure}
	\begin{itemize}
		\item \textbf{Codice}: UCGL\theuccount.
		\item \textbf{Titolo}: Gitlab segnala la modifica di una issue al Producer Gitlab.
		\item \textbf{Attori primari}: GitLab.
		\item \textbf{Descrizione}:
		il sistema Producer GitLab ed è interno al sistema Butterfly. L'invio di una segnalazione avviene da parte di GitLab tramite webhook, quando una issue viene modificata.
		\item \textbf{Precondizione}: Viene modificata una issue già aperta su un
		progetto di GitLab e segnalata a \progetto.
		\item \textbf{Postcondizione}: il Producer GitLab riceve la segnalazione da GitLab.
		\item \textbf{Scenario principale}: 
		\begin{enumerate}
			\item GitLab invia la segnalazione di modifica issue al Producer GitLab
		\end{enumerate}
		
	\end{itemize}
	
	\stepcounter{uccount}
	
	\subsubsection{UCGL\theuccount\ - GitLab segnala evento di push a Producer GitLab}
	\begin{figure}[H]
		\centering
		\includegraphics[width=0.7\textwidth]{img/UC1.png}\\
		\caption{UCGL\theuccount\ - GitLab segnala evento di push a Producer GitLab}
	\end{figure}
	\begin{itemize}
		\item \textbf{Codice}: UCGL\theuccount.
		\item \textbf{Titolo}: GitLab segnala evento di push a Producer GitLab.
		\item \textbf{Attori primari}: GitLab.
		\item \textbf{Descrizione}:
		il sistema Producer GitLab ed è interno al sistema Butterfly. L'invio di
		una segnalazione avviene da parte di GitLab tramite webhook. L'evento di
		push può essere composto da uno o più commit.
		\item \textbf{Precondizione}: Viene effettuato un push su GitLab e segnalato a \progetto.
		\item \textbf{Postcondizione}: il Producer GitLab riceve la segnalazione da GitLab.
		\item \textbf{Scenario principale}: 
		\begin{enumerate}
			\item GitLab invia la segnalazione di push al Producer GitLab
		\end{enumerate}
		
	\end{itemize}