\subsubsection{UC\theuccount-GP - Modifica utente}
		\begin{figure}[H]
			\centering
				\includegraphics[width=0.9\textwidth]{img/casi_d'uso/UC\theuccount.png}\\
			\caption{UC\theuccount-GP - Modifica utente}
		\end{figure}
	\begin{itemize}
		\item \textbf{Codice}: UC\theuccount-GP.
		\item \textbf{Titolo}: modifica utente.
		\item \textbf{Attori primari}: utente.
		\item \textbf{Descrizione}: l’utente vuole modificare le informazioni a esso relative.
		\item \textbf{Precondizione}: l'utente vuole modificare alcuni dei dati presenti nel sistema.
		\item \textbf{Postcondizione}: i campi dell'utente sono stati modificati correttamente.
		\item \textbf{Scenario principale}:
		\begin{enumerate}
			\item L'utente modifica i dati
		\end{enumerate}
	\end{itemize}

	\stepcounter{subuccount}

		\subsubsection{UC\theuccount.\thesubuccount-GP - Modifica di un utente del sistema}
			\begin{figure}[H]
				\centering
				\includegraphics[width=0.6\columnwidth]{img/casi_d'uso/UC\theuccount_\thesubuccount.png}\\
				\caption{UC\theuccount.\thesubuccount-GP - Modifica di un utente del sistema}
			\end{figure}
			\begin{itemize}
				\item \textbf{Codice}: UC\theuccount.\thesubuccount-GP.
				\item \textbf{Titolo}: modifica di un utente del sistema.
				\item \textbf{Attori primari}: utente.
				\item \textbf{Descrizione}: l'utente vuole modificare i suoi dati compilando i relativi campi.
				\item \textbf{Precondizione}: l'utente vuole modificare alcuni dei dati presenti nel sistema.
				\item \textbf{Postcondizione}: l'utente è stato modificato.
				\item \textbf{Scenario principale}:
				\begin{enumerate}
					\item L'utente inserisce i nuovi dati che vuole rimpiazzare ai precedenti
					\item L'utente viene modificato
				\end{enumerate}
				\item \textbf{Estensioni}:
				\begin{itemize}
					\item Errore, nuovi dati dell'utente già esistenti [UC\theuccount.2-GP]
				\end{itemize}
			\end{itemize}

			\stepcounter{subsubuccount}
			\subsubsection{UC\theuccount.\thesubuccount.\thesubsubuccount-GP - Inserimento del nuovo nome}
				\begin{itemize}
					\item \textbf{Codice}: UC\theuccount.\thesubuccount.\thesubsubuccount-GP.
					\item \textbf{Titolo}: inserimento del nuovo nome.
					\item \textbf{Attori primari}: utente.
					\item \textbf{Descrizione}: l'utente aggiunge il nuovo nome che vuole modificare.
					\item \textbf{Precondizione}: l'utente vuole modificare alcuni dei dati presenti nel sistema.
					\item \textbf{Postcondizione}: il nome è stato inserito.
					\item \textbf{Scenario principale}:
					\begin{enumerate}
						\item L'utente inserisce il suo nuovo nome
					\end{enumerate}
				\end{itemize}

			\stepcounter{subsubuccount}

			\subsubsection{UC\theuccount.\thesubuccount.\thesubsubuccount-GP - Inserimento del nuovo cognome}
				\begin{itemize}
					\item \textbf{Codice}: UC\theuccount.\thesubuccount.\thesubsubuccount-GP.
					\item \textbf{Titolo}: inserimento del nuovo cognome.
					\item \textbf{Attori primari}: utente.
					\item \textbf{Descrizione}: l'utente aggiunge il nuovo cognome che vuole modificare.
					\item \textbf{Precondizione}: l'utente vuole modificare alcuni dei dati presenti nel sistema.
					\item \textbf{Postcondizione}: il cognome è stato inserito.
					\item \textbf{Scenario principale}:
					\begin{enumerate}
						\item L'utente inserisce il suo nuovo cognome
					\end{enumerate}
				\end{itemize}

			\stepcounter{subsubuccount}
			\subsubsection{UC\theuccount.\thesubuccount.\thesubsubuccount-GP - Inserimento del nuovo contatto Email}
				\begin{itemize}
					\item \textbf{Codice}: UC\theuccount.\thesubuccount.\thesubsubuccount-GP.
					\item \textbf{Titolo}: inserimento del nuovo contatto Email.
					\item \textbf{Attori primari}: utente.
					\item \textbf{Descrizione}: l'utente aggiunge il nuovo contatto Email rimpiazzando il suo contatto precedente o aggiungendone uno per la prima volta.
					\item \textbf{Precondizione}: l'utente vuole modificare alcuni dei dati presenti nel sistema.
					\item \textbf{Postcondizione}: il contatto Email è stato inserito.
					\item \textbf{Scenario principale}:
					\begin{enumerate}
						\item L'utente inserisce il nuovo contatto Email
					\end{enumerate}
				\end{itemize}

			\stepcounter{subsubuccount}
			\subsubsection{UC\theuccount.\thesubuccount.\thesubsubuccount-GP - Inserimento del nuovo contatto Telegram}

				\begin{itemize}
					\item \textbf{Codice}: UC\theuccount.\thesubuccount.\thesubsubuccount-GP.
					\item \textbf{Titolo}: inserimento del nuovo contatto Telegram.
					\item \textbf{Attori primari}: utente.
					\item \textbf{Descrizione}: l'utente aggiunge il nuovo contatto Telegram rimpiazzando il suo contatto precedente o aggiungendone uno per la prima volta.
					\item \textbf{Precondizione}: l'utente vuole modificare alcuni dei dati presenti nel sistema.
					\item \textbf{Postcondizione}: il contatto Telegram è stato inserito.
					\item \textbf{Scenario principale}:
					\begin{enumerate}
						\item L'utente inserisce il nuovo contatto Telegram
					\end{enumerate}
				\end{itemize}

		\stepcounter{subuccount}
		\subsubsection{UC\theuccount.\thesubuccount-GP - Errore, nuovi dati dell'utente già esistenti}

		\begin{itemize}
			\item \textbf{Codice}: UC\theuccount.\thesubuccount-GP.
			\item \textbf{Titolo}: errore, nuovi dati dell'utente già esistenti.
			\item \textbf{Attori primari}: utente.
			\item \textbf{Descrizione}: i nuovi dati dell'utente da modificare, che sono stati inseriti, sono già presenti nel sistema. Questo vuol dire che i nuovi campi corrispondono a quelli di un utente già esistente. In particolare il contatto Telegram o Email, perchè una persona può avere lo stesso nome e cognome di un altro, ma non la stessa Email e nemmeno lo stesso identificativo Telegram.
			\item \textbf{Precondizione}: l'utente vuole modificare alcuni dei dati presenti nel sistema.
			\item \textbf{Postcondizione}: l'utente non è stato modificato e viene visualizzato un messaggio di errore.
			\item \textbf{Scenario principale}:
			\begin{enumerate}
				\item L'utente inserisce i campi richiesti dal sistema per la modifica
				\item L'utente visualizza un messaggio di errore
			\end{enumerate}
		\end{itemize}

%	\stepcounter{subuccount}
%	\subsubsection{UC\theuccount.\thesubuccount-GP - Errore identificativo inesistente}
%
%		\begin{itemize}
%			\item \textbf{Codice}: UC\theuccount.\thesubuccount-GP.
%			\item \textbf{Titolo}: errore identificativo inesistente.
%			\item \textbf{Attori primari}: utente.
%			\item \textbf{Descrizione}:  l’utente inserisce l'identificativo dell'utente di cui vuole modificare le informazioni, ma viene avvisato che ha inserito un'identificativo errato perchè esso non è presente all'interno del sistema.
%			\item \textbf{Precondizione}: l'utente vuole modificare un utente già presente.
%			\item \textbf{Postcondizione}: il sistema comunica all’utilizzatore l’errore.
%			\item \textbf{Scenario principale}:
%			\begin{enumerate}
%				\item L'utente inserisce un identificativo errato
%				\item Il sistema comunica all’utilizzatore l’errore
%			\end{enumerate}
%		\end{itemize}
