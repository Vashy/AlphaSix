\lettera{M}\label{M}

    \parola{Machine Learning}{%
        Termine anglofono di apprendimento automatico, è un insieme di metodi volti a far apprendere alle macchine (intesi come software)
        senza essere state esplicitamente o preventivamente programmate.
    }\label{Machine Learning}

	\parola{Macro-processo}{
		Indica un processo "macro" (esteso), ovvero un grande processo composto da altri processi minori.
	}\label{Macro-processo}

    \parola{Major Release}{%
        Specifica versione di un prodotto che ne determina l'incremento di uno dei numeri di versione più significativi (dipendente dalle specifiche
        e dal contesto ove è definito), a seguito di modifiche importanti.
        Ne determina la distribuzione.
    }\label{Major Release}

    \parola{Macro}{%
        Indica una procedura o "blocco" di comandi tipicamente ricorrente durante l'esecuzione di un programma,
        volta a favorire il riutilizzo del codice.
    }\label{Macro}

    \parola{Markdown}{%
        Linguaggio di markup con una sintassi molto semplice, convertibile in altri formati quali HTML con un \gloss{tool} omonimo.
    }\label{Markdown}

	\parola{Metadato}{%
		Particolare dato che descrive insiemi di altri dati.
	}\label{Metadato}

	\parola{Metrica}{
		Una metrica software è uno standard per la misura di alcune proprietà del software o delle sue specifiche.
	}\label{Metrica}

	\parola{Microservizio}{%
	 Un microservizio è un servizio "piccolo" e autonomo che interagisce con altri microservizi e che ha come finalità quella di fare una cosa e di farla nel modo corretto.
	}\label{Microservizio}

    \parola{Milestone}{%
        In italiano pietra miliare, indica importanti traguardi intermedi nello svolgimento di un progetto. Più nello specifico è un momento nel tempo in cui vengono a concludersi n \gloss{baseline}. Quindi una milestone è associata a una o più baseline.
    }\label{Milestone}

    \parola{Modello di sviluppo}{%
        Principio teorico indicante il metodo da seguire in fase di progettazione e codifica.
    }\label{Modello di sviluppo}