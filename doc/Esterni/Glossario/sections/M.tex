\lettera{M}\label{M}

    \parola{Macchina virtuale}{%
		Una macchina virtuale (VM) indica un software che, attraverso un processo di virtualizzazione, crea un ambiente virtuale che emula il comportamento di una macchina fisica (PC client o server) grazie all'assegnazione di risorse hardware ed in cui alcune applicazioni possono essere eseguite come se interagissero con tale macchina.
	}\label{Macchina virtuale}

    \parola{Machine Learning}{%
        Termine anglofono di apprendimento automatico, è un insieme di metodi volti a far apprendere alle macchine (intesi come software)
        senza essere state esplicitamente o preventivamente programmate.
    }\label{Machine Learning}
    
    \parola{Macro}{%
        Indica una procedura o ``blocco'' di comandi tipicamente ricorrente durante l'esecuzione di un programma, volta a favorire il riutilizzo del codice.
    }\label{Macro}

	\parola{Macro-processo}{
		Indica un processo ``macro'' (esteso), ovvero un grande processo composto da altri processi minori.
	}\label{Macro-processo}

    \parola{Major Release}{%
        Specifica versione di un prodotto che ne determina l'incremento di uno dei numeri di versione più significativi (dipendente dalle specifiche
        e dal contesto ove è definito), a seguito di modifiche importanti.
        Ne determina la distribuzione.
    }\label{Major Release}

    \parola{Markdown}{%
        Linguaggio di markup con una sintassi molto semplice, convertibile in altri formati quali HTML con un \gloss{tool} omonimo.
    }\label{Markdown}

	\parola{Markup}{%
		Un linguaggio di markup è un insieme di regole che descrivono i meccanismi di rappresentazione (strutturali, semantici, presentazionali) di un testo; facendo uso di convenzioni rese standard, tali regole sono utilizzabili su più supporti. La tecnica di formattazione con marcatori (detti espressioni codificate) richiederà una serie di convenzioni, proprie appunto di un 'linguaggio a marcatori di documenti'. 
	}\label{Markup}

	\parola{Metadato}{%
		Particolare dato che descrive insiemi di altri dati.
	}\label{Metadato}

	\parola{Metrica}{
		Una metrica software è uno standard per la misura di alcune proprietà del software o delle sue specifiche.
	}\label{Metrica}

	\parola{Microservizio}{%
		 Un microservizio è un servizio ``piccolo'' e autonomo che interagisce con altri microservizi e che ha come finalità quella di fare una cosa e di farla nel modo corretto.
	}\label{Microservizio}

    \parola{Milestone}{%
        In italiano pietra miliare, indica importanti traguardi intermedi nello svolgimento di un progetto. Più nello specifico è un momento nel tempo in cui vengono a concludersi n \gloss{baseline}. Quindi una milestone è associata a una o più baseline.
    }\label{Milestone}

	\parola{Modello a componenti}{%
		Il modello a componenti è un modello di sviluppo basato sul riuso di unità software che possono avere diverse dimensioni:
		\begin{itemize}
			\item System reuse			
			\item Application reuse		
			\item Component reuse
			\item Object and function reuse
		\end{itemize}
	}\label{Modello a componenti}

    \parola{Modello di sviluppo}{%
        Principio teorico indicante il metodo da seguire in fase di progettazione e codifica, serie di passi o percorso da svolgere per ottenere risultati di alta qualità e in tempi prefissati nello sviluppo di un prodotto o sistema software.
    }\label{Modello di sviluppo}

	\parola{MongoDB}{%
		È un \gloss{DBMS} non relazionale, orientato ai documenti. MongoDB si allontana dalla struttura tradizionale basata su tabelle dei database relazionali in favore di documenti in stile \gloss{JSON} con schema dinamico.
	}\label{MongoDB}