\lettera{S}\label{S}

	\parola{Scala}{
		Scala è un linguaggio di programmazione di tipo general-purpose multi-paradigma creato per integrare le caratteristiche dei linguaggi orientati agli oggetti come \gloss{Java} e dei linguaggi funzionali come Python. La compilazione di codice sorgente Scala produce Java bytecode per l'esecuzione su una JVM.
	}\label{Scala}
	
	\parola{Scripting}{%
		\`E un linguaggio di programmazione interpretato destinato in genere a compiti di automazione del sistema operativo (batch) o delle applicazioni (\gloss{macro}), o a essere usato nella programmazione Web all'interno delle pagine Web.
	}\label{Scripting}
	
    \parola{SCSS}{%
        Linguaggio compilato per sviluppo di CSS.
    }\label{SCSS}

	\parola{Sistema}{%
		Insieme di elementi tra loro interdipendenti per ottenere un determinato scopo funzionale.
	}\label{Sistema}

    \parola{Skill}{%
        Capacità in italiano, sono il sistema fornito da Amazon per permettere
        al cliente di \gloss{Alexa} di creare funzioni ad hoc per ottenere un'esperienza
        personalizzabile con l'assistente virtuale.
    }\label{Skill}

	\parola{Slack}{%
		Strumento di messaggistica istantanea utilizzato per la collaborazione aziendale tra singoli o tramite canali accessibili a tutto il team o solo ad alcuni membri.
		Permette l'integrazione con strumenti di terze parti come \hyperref[Google Drive]{\gloss{Google Drive}}, Trello, \hyperref[GitHub]{\gloss{GitHub}}, Google Calendar e altre applicazioni popolari. 
	}\label{Slack}

    \parola{Slack Time}{%
		Letteralmente "tempo allentato", è il tempo di possibile ritardo di un'attività che non comporta un ritardo per il progetto di cui fa parte. 
	}\label{slack-time}
    
    \parola{Slide}{
    	Slide è il termine inglese equivalente a "diapositiva".
	}\label{Slide}
    
    \parola{Smart Contract}{%
		Contratto implementato attraverso codice eseguito dalla \gloss{\textit{EVM}}. 
	}\label{Smart Contract}

	\parola{Software-as-a-Service}{%
		Tipo di software, spesso fruibile attraverso un'applicazione Web, che cerca di:
		
		\begin{itemize}
			\item Automatizzare la configurazione dell'applicazione per ridurre il costo d'ingresso dello sviluppatore al progetto.
			\item Possedere un'alta portabilità.
			\item Adattarsi ai recenti cloud platform.
			\item Effettuare \gloss{continuous deployment}.
			\item Possono scalare significativamente senza troppi cambiamenti ai tool, all'architettura e al processo di sviluppo.
		\end{itemize}
	}\label{Software-as-a-Service}

	\parola{SonarQube}{%
		Applicazione di Continuous Inspection che permette di ispezionare il codice e analizzare il codice in maniera statica e dinamica da remoto.
	}\label{SonarQube}

	\parola{Sorgente}{%
		Termine informale con cui ci si riferisce ai file contenenti codice sorgente.
	}\label{Sorgente}

	\parola{Sotto-processo}{
		Un sotto-processo è una parte del processo che comprende più attività e ha propri attributi in termini di obiettivo, contribuendo però nel contempo al raggiungimento dell'obiettivo più generale del processo.
	}\label{Sotto-processo}

	\parola{SPY}{%
		Software Process Assessment \& Improvement è uno standard di qualità per la valutazione dei processi. Cerca di dare un giudizio oggettivo a questi in termini di maturità e ne individua azioni migliorative.
	}\label{SPY}

	\parola{Stakeholder}{%
		Individui o gruppi che hanno un interesse legittimo nei confronti dell’impresa e delle sue attività, passate, presenti e future, e il cui contributo (volontario o involontario) è essenziale al suo successo.
	}\label{Stakeholder}

    \parola{Swift}{%
        Linguaggio di programmazione orientato agli oggetti utilizzato per lo sviluppo di
        sistemi e applicativi Apple (iOS, MacOS, etc..).
    }\label{Swift}

	\parola{System testing}{%
		Rappresenta l'ultima frontiera di controllo prima del cliente finale. Esso controlla che il funzionamento globale del sistema sia assicurato entro standard ben definiti.
	}\label{System testing}
