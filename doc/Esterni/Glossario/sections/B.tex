\lettera{B}\label{B}
	
	\parola{Backend}{%
		Parte del programma con il quale l’utente interagisce indirettamente, di solito attraverso l’utilizzo di
		un’applicazione \gloss{frontend}.
	}\label{Backend}
	
	\parola{Baseline}{%
		Strumento utilizzato per registrare lo stato di avanzamento del software. Essa raggruppa più modifiche sostanziali del software rilasciandone una versione non completa ma stabile.
	}\label{Baseline}
    
    \parola{Blockchain}{%
        Database distribuito che consente la tracciabilità di transazioni riguardanti \gloss{criptovaluta}.
    }\label{Blockchain}
    
    \parola{Bootstrap}{%
        Framework \gloss{open source} sviluppato da Twitter, atto a offrire un libero strumento
        che semplifichi la creazione di siti web grazie a modelli di progettazione responsivi
        e riutilizzabili.
    }\label{Bootstrap}

	\parola{Bot}{%
		 Un bot è un programma che accede alla rete attraverso lo stesso tipo di canali utilizzati dagli utenti che utilizzano il programma. Il loro scopo in genere è legato all'automazione di compiti che sarebbero troppo gravosi o complessi per gli utenti.
	}\label{Bot}
    
    \parola{Broker}{%
        Componente che gestisce i messaggi inviati da Publisher a Subscriber nel relativo modello architetturale.
        Mette a disposizione i Topic nei quali i Publisher inviano i messaggi, mentre i Subscriber possono
        iscriversi a essi per ricevere i messaggi.
    }\label{Broker}

	\parola{Bug}{%
		Errore che non permette il corretto funzionamento o la corretta compilazione del prodotto software o della documentazione creata con \LaTeX.
	}\label{Bug}    