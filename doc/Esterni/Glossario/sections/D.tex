\lettera{D} \label{D}

	\parola{Dashboard}{%
		Schermata che permette di monitorare in tempo reale l’andamento dei report e delle metriche aziendali più importanti.
	}\label{Dashboard}

	\parola{Design Pattern}{%
		Soluzione progettuale generale a un problema ricorrente. \`E la descrizione di un modello da applicare per risolvere un problema, il quale può presentarsi in diverse situazioni durante la progettazione e lo sviluppo del software. Ogni design pattern ha il suo campo applicativo in base alle precondizioni del problema, coi suoi pregi e difetti.
	}\label{Design Pattern}
	
	\parola{Deployment}{%
		Si riferisce alla consegna o rilascio al cliente, con relativa installazione e messa in funzione di
		un'applicazione o di un sistema software.
	}\label{Deployment}
	
    \parola{Development}{%
    	Fase del \gloss{ciclo di vita} che prevede parte di sviluppo del software, intesa come la vera e propria scrittura del codice.
    }\label{Development}

	\parola{DevOps}{%
		\`E l'unione delle parole \gloss{Development\G} e \gloss{Operation}. Vuole intendere la pratica di veder collaborare in modo molto stretto queste due particolari fasi dello sviluppo software. \`E una strategia che prevede un rilascio del prodotto molto frequente aiutando le aziende nello standardizzare gli ambienti di sviluppo.
	}\label{DevOps}
    
    \parola{Docker}{
		Piattaforma che consente di automatizzare il \gloss{deployment} di applicazioni all'interno di contenitori software.
	}\label{Docker}
	
	\parola{Dockerfile}{
		Contiene le configurazioni necessarie per il container sul quale si andrà a eseguire l'\gloss{applicativo}.
	}\label{Dockerfile}

	\parola{Documento}{
		Scritto tecnico che convalida o certifica la realtà di un'attività specifica del progetto in atto. 
	}\label{Documento}

	\parola{Draw.io}{
		È un software online gratis per diagrammi per creare diagrammi di flusso, \gloss{UML} e altri tipi di diagrammi vari.
	}\label{Draw.io}