\lettera{E} \label{E}

    \parola{E-commerce}{%
		Piattaforma attraverso la quale è possibile acquistare e vendere prodotti attraverso Internet.
	}\label{E-commerce}

	\parola{Efficacia}{%
		Misura della capacità di riuscire a raggiungere i compiti fissati. Essa si calcola in base al grado di raggiungimento degli obiettivi.
	}\label{Efficace}

	\parola{Efficienza}{%
		Misura della capacità di poter raggiungere i propri obiettivi cercando di usare meno risorse possibili.
	}\label{Efficienza}

    \parola{EIP712}{%
        Standard per la firma dei contratti su Ethereum.
    }\label{EIP712}

    \parola{Enhancement}{%
    	Attività con lo scopo di accrescere la qualità del prodotto o migliorarne (e in caso aumentarne) le funzionalità.
    }\label{Enhancement}

    \parola{ERC20}{%
        Standard per i gettoni di Ethereum (beni digitali).
    }\label{ERC20}

    \parola{Ethereum}{%
        Piattaforma che consente la scrittura facilitata di applicazioni
        distribuite che usano Blockchain.
    }\label{Ethereum}

    \parola{Event Driven}{%
        È un paradigma di programmazione dell'informatica. Mentre in un programma tradizionale l'esecuzione delle istruzioni segue percorsi fissi, che si ramificano soltanto in punti ben determinati predefiniti dal programmatore, nei programmi scritti utilizzando la tecnica a eventi il flusso del programma è largamente determinato dal verificarsi di eventi esterni.
    }\label{EVM}

    \parola{EVM}{%
        Ethereum Virtual Machine, computer distribuito contenente milioni di oggetti che hanno le capacità di mantenere uno stato interno,
        eseguire codice e comunicare tra loro.
    }\label{EVM}
