\lettera{I}\label{I}

    \parola{IDE}{%
        Acronimo di Integrated Development Environment, è un software che aiuta i programmatori nella scrittura di codice sorgente offrendo svariati benefici.
    }\label{IDE}

	\parola{Incapsulamento}{
	Un meccanismo del linguaggio di programmazione atto a limitare l'accesso diretto agli elementi dell'oggetto.
	}\label{Incapsulamento}

	\parola{Indice di Gulpease}{
		Indice di leggibilità di un testo per la lingua italiana. Utilizza la lunghezza delle parole in lettere anziché in sillabe, semplificandone il calcolo automatico. I risultati sono compresi tra 0 e 100, dove il valore ``100'' indica la leggibilità più alta e ``0'' la leggibilità più bassa. 
	}\label{Indice di Gulpease}

    \parola{Information hiding}{%
        Buona pratica secondo la quale le varie informazioni e dati n un software devo poter essere nascoste a più livelli, per impedire a chi non ne possiede il diritto di accedervi.
    }\label{Information hiding}
	
    \parola{Integration Hell}{%
        Punto che impedisce ai membri di un team di integrare senza conflitti il loro lavoro, portando a ore o addirittura giorni di correzioni
        prima di riuscire a integrare correttamente il codice.
    }\label{Integration Hell}

    \parola{Intelligenza Artificiale}{%
        Abilità di un sistema di avere prestazioni che, per un osservatore, sono simili a quelle dell'intelligenza umana.
	}\label{Intelligenza Artificiale}

	\parola{Interfaccia}{% Intesa come le interfacce di Java
		In Java ha una struttura simile a una classe, ma può contenere solo costanti e metodi d'istanza
		astratti (quindi non può contenere né costruttori, né variabili	statiche, né variabili di istanza,
		né metodi statici).
	}\label{Interfaccia}

	\parola{iOS}{%
		\`E un sistema operativo sviluppato da Apple per iPhone, iPod touch e iPad.
	}\label{iOS}
	
	\parola{ISO/IEC 15504}{
		ISO/IEC 15504, anche conosciuta come SPICE (``Software Process Improvement and Capability Determination''), è un insieme di nove documenti di standard tecnici relativi ai processi di sviluppo del software e relative funzioni di business e, in particolare, alla loro valutazione.
	}\label{ISO/IEC 15504}
	
    \parola{ISO/IEC 9000}{%
    	\`E lo standard riguardante la \gloss{qualità} applicato a ogni ambito lavorativo, analizzando la gestione dei processi e il loro sviluppo.
    }\label{ISO/IEC 9000}

	\parola{Issue}{%
		Attività da svolgere.
	}\label{Issue}

	\parola{Issue Tracking System}{
		Strumento che facilita la gestione del processo di sviluppo attraverso la gestione di attività diverse.
		Ogni singola attività del progetto è gestita mediante un \gloss{workflow} e mantenuta all'interno di un’unica piattaforma e di un'unica \gloss{repository}.
	}\label{Issue Tracking System}
