\lettera{T}\label{T}

	\parola{Task}{%
		Termine inglese utilizzato per definire compito.
	}\label{Task}
	
	\parola{Team di sviluppo}{%
		Il team di sviluppo (in lingua inglese ``Development Team'') è una squadra di persone che sviluppano un progetto.
	}\label{Team di sviluppo}
	
	\parola{Telegram}{%
    	Servizio di messaggistica istantanea tramite Internet che offre la possibilità di scrivere messaggi tra due o più persone (gruppi). 
    	Si possono creare \gloss{bot} con varie funzioni in grado di ricevere, elaborare e mandare messaggi. 
	}\label{Telegram}
	
	\parola{Template}{%
		Documento o programma nel quale è definito un ``modello'' riutilizzabile in più ambiti.
	}\label{Template}

    \parola{Test Driven Development (TDD)}{%
        Si tratta di una metodologia di sviluppo software che prevede la creazione dei test di un'unità prima della sua codifica. Perciò l'unità viene creta esclusivamente in relazione al superamento dei test.
    }\label{Test Driven Development}

	\parola{The Twelve-Factor App}{%
		È una metodologia di sviluppo che può essere applicata a qualunque software, scritto in qualsiasi linguaggio di programmazione. L'obiettivo che si pone è orientare l'applicazione a seguire un formato dichiarativo, interfacciarsi in modo pulito, adattarsi alle più recenti piattaforme cloud e minimizzare la divergenza tra sviluppo e produzione.
	}\label{12}

	\parola{Ticket}{%
		Richiesta di assistenza, tracciata da un sistema informatico di gestione di tali richieste.
	}\label{Ticket}

	\parola{Tool}{%
		È il termine inglese equivalente all'italiano ``strumento''. Nel contesto informatico si intende un'applicazione che svolge un determinato compito.
	}\label{Tool}

	\parola{Topic}{%
		Equivalente di ``argomento'' in italiano. Nel contesto di un \gloss{Broker} si intende un canale di messaggi associato a uno specifico argomento.
	}\label{Topic}

	\parola{Tox}{%
		È uno strumento che mira ad automatizzare e standardizzare i test in Python. Serve a
		facilitare il processo di packaging, testing e release del software Python.
	}\label{Tox}
	