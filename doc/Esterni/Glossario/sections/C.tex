\lettera{C}\label{C}

	\parola{Callback}{
		Riferimento a una funzione oppure a un ``blocco di codice''.
	}\label{Callback}

	\parola{Capability}{%
		Capacità di un processo di essere cognitivamente capace di raggiungere il suo scopo.
	}\label{Capability}
	
    \parola{Capitolato}{%
        Documento tecnico, possibilmente allegato a un contratto d'appalto che intercorre tra il cliente ed un committente, in
        quest'ultimo indica costi, modalit\`a e tempi di realizzazione dell'opera oggetto del contratto.
    }\label{Capitolato}

	\parola{Caption}{
		Termine generico col quale viene definito un titolo o una didascalia per qualsiasi documento o immagine. Dall'inglese: titolo.
	}\label{Caption}
    
    \parola{Car Sharing}{%
        È un servizio di mobilità urbana che permette agli utenti di utilizzare un veicolo su prenotazione noleggiandolo per un
        periodo di tempo breve, nell'ordine di minuti o ore, e pagando in ragione dell'utilizzo effettuato.
    }\label{Car Sharing}

	\parola{Caso d'uso}{
		Il caso d'uso è una serie di azioni in sequenza, che determinano uno scenario, miranti ad un obiettivo che un utente, chiamato \gloss{attore}, vuole raggiungere.
	}\label{Caso d'uso}

	\parola{Change Significance}{%
		Schema in cui i cambiamenti sono definiti da livelli di importanza differenti. Ogni livello è indicato da un numero in una posizione diversa: generalmente
		il numero più a sinistra è quello con un importanza maggiore e che determina un significativo numero di cambiamenti, e man mano che il numero si trova più a
		destra, il numero di cambiamenti necessario a incrementarlo è minore (e.g. nella forma X.Y.Z, l'incremento della X determina un grosso cambiamento,
		mentre la Z potrebbe cambiare a ogni minima modifica).
	}\label{Change Significance}

	\parola{CharryPy}{
		È un framework per lo sviluppo web object-oriented in linguaggio \gloss{Python}. Fornisce le fondamenta sopra le quali
		applicazioni web complesse possono essere scritte.
	}\label{CharryPy}

	\parola{CI/CD}{%
		Sigla con cui ci si riferisce a \gloss{Continuous Integration} e \gloss{Continuous Delivery} contemporaneamente.
    }\label{CI/CD}

	\parola{Ciclo di Deming}{
		Il ciclo di Deming (o ciclo di PDCA, acronimo dall'inglese ``Plan–Do–Check–Act'') è un metodo di gestione iterativo in quattro fasi utilizzato per il controllo e il miglioramento continuo dei processi e dei prodotti.	
	}\label{Ciclo di Deming}

	\parola{Ciclo di vita}{%
		Insieme delle attività per la realizzazione di un software. Questo parte dal momento in cui una persona lo pensa (conception) fino al suo ritiro. La sua struttura deve essere conosciuta preventivamente per valutare i costi e le risorse per lo sviluppo del software. Il ciclo di vita può perciò essere visto come un automa a stati finiti.
		
		Esistono diversi cicli di vita che vengono scelti in base al tipo di software da creare:
		
		\begin{itemize}
			\item Cascata/sequenziale
			\item Iterativo
			\item Incrementale
			\item Evolutivo
			\item Componenti
			\item Agile
			\item Scrum
			\item Spirale
			\item SEMAT
		\end{itemize}
	}\label{Ciclo di vita}

	\parola{Client}{%
		Un client è un qualunque componente che accede ai servizi o alle risorse di un altro componente detto server. In questo contesto si può quindi parlare di client riferendosi all'hardware oppure al software.
	}\label{Client}
	
	\parola{Cloud}{%
		Indica un paradigma di erogazione di servizi offerti da un fornitore a un cliente finale attraverso la rete
		Internet (come l'archiviazione, l'elaborazione o la trasmissione dati), a partire da un insieme di risorse
		preesistenti, configurabili e disponibili in remoto.
	}\label{Cloud}
	
	\parola{CMMI}{%
		Acronimo di \gloss{Capability Maturity Model Integration}, si tratta di uno standard per ottenere una valutazione sulla \gloss{qualità} dai fornitori in modo uniforme.

		Questo standard fa leva sul determinare la capability di un processo, intesa come la misura della capacità e dell'efficienza
		per un \gloss{processo} di ottenere risultati, e maturity, intesa quanto l'azienda sia capace di organizzare e governare i suoi processi.

		Il CMMI prevede cinque livelli di maturità a cui si associa ogni processo, ed essi sono:

		\begin{enumerate}
			\item Initial
			\item Managed
			\item Defined
			\item Quantitatively managed
			\item Optimizing
		\end{enumerate}
	}\label{CMMI}

	\parola{Codebase}{
		Il termine codebase è usato nello sviluppo del software per indicare l'intera collezione di codice sorgente usata per costruire una particolare applicazione o un particolare componente.
	}\label{Codebase}

	\parola{Code smell}{
		L'espressione code smell (in inglese, letteralmente:"puzza del codice") viene usata per indicare una serie di caratteristiche che il codice sorgente può avere e che sono generalmente riconosciute come probabili indicazioni di un difetto di programmazione. I code smell non rilevano \gloss{bug} o errori, bensì debolezze di progettazione che riducono la qualità del software.
	}\label{Code smell}

	\parola{Commit}{
		Il termine inglese ``commit'' in questo contesto si riferisce ad una o più modifiche effettuate in una \gloss{repository}, a cui viene associato un nome e un codice identificativo per il \gloss{tracciamento}.
	}\label{Commit}

	\parola{Committente}{%
		Il committente è colui che richiede del software, però solo da
		usare e non mantenere o implementare, in cambio di denaro.
	}\label{Committente}

	\parola{Componente}{%
		È una parte identificabile di un programma più ampio. Fornisce funzioni, in genere correlate tra loro.
	}\label{Componente}

    \parola{Consumer}{%
    	Componenti che avranno il compito di abbonarsi ai Topic adeguati, recuperandone i messaggi.
	}\label{Consumer}

	\parola{Container}{% di Docker
	TODO
	}\label{Container}

	\parola{Continuous Delivery}{%
		Passo successivo alla \gloss{CI} (Continuous Integration) in cui ogni cambiamento può potenzialmente essere rilasciato in produzione,
		in modo semplice e immediato.
		Serve per migliorare la cooperazione e la comunicazione tra developer, operational e tester.
	}\label{Continuous Delivery}

	\parola{Continuous Deployment}{%
		Metodologia di sviluppo del software che prevede il frequente rilascio del prodotto funzionante preferibilmente in una repository specifica. 
	}\label{Continuous Deployment}

	\parola{Continuous Integration}{%
		Pratica dell'Ingegeria del software in cui l'allineamento dell'ambiente di lavoro degli sviluppatori verso l'ambiente condiviso
		è molto frequente.
		Il codice di un progetto che adotta la \gloss{CI} prevede che vi sia un processo di build e verifica automatico.
	}\label{Continuous Integration}
	
	\parola{Criptovaluta}{%
		\`E una valuta paritaria, decentralizzata e digitale la cui implementazione si basa sui principi della crittografia per convalidare le transazioni e la generazione di moneta in sé.
	}\label{Criptovaluta}
	
	\parola{CSS3}{%
        Versione pi\`u moderna del Cascading Style Sheets, linguaggio usato per definire la formattazione dello stile dei documenti 
        HTML/XML.
    }\label{CSS3}
