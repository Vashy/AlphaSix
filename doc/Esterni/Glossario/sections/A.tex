\lettera{A} \label{A}

    \parola{Alexa}{%
        Assistente vocale di Amazon basato sul cloud.
        \`E l’intelligenza artificiale utilizzata da alcuni dispositivi
        creati appositamente dall'azienda,
        come ad esempio Amazon Echo.
    }\label{Alexa}

    \parola{Amazon API Gateway}{%
        Servizio completamente gestito che semplifica agli sviluppatori la creazione,
        la pubblicazione, la manutenzione, il monitoraggio e la protezione delle API su qualsiasi scala.
    }\label{Gateway}

    \parola{Amazon Aurora Serverless}{%
        Configurazione on demand per Aurora (versione compatibile con MySQL) con scalabilità
        automatica in cui avvio, arresto e ricalibrazione della capacità del database avvengono
        in base alle necessità dell'applicazione senza interventi manuali.
    }\label{Amazon Aurora Serverless}

    \parola{Amazon Web Services (AWS)}{%
        Piattaforma di servizi cloud di Amazon, utilizzabile per ospitare
        in un server sicuro contenuti quali ad esempio storage di database
        e siti web. 
    }\label{AWS}
    
    \parola{Android}{%
        È un sistema operativo per dispositivi mobili sviluppato da Google Inc. e basato sul kernel Linux. Progettato
        principalmente per smartphone e tablet, con interfacce utente specializzate per televisori, automobili, orologi 
        da polso, occhiali, ecc.
    }\label{Android}

	\parola{Apache Kafka}{%
		\`E un sistema \gloss{open source} di messaggistica istantanea, che consente la gestione di un elevato numero di operazioni in tempo reale da migliaia di client, sia in lettura che in scrittura.
	}\label{Apache Kafka}
    
    \parola{API Rest}{%
		Metodi con cui è possibile fornire l'interazione con le componenti del sistema basate su representational state transfer,
        dove le risorse sono uniche e indirizzabili mediante URI.
	}\label{API Rest}
    
    \parola{Applicativo}{%
    	Programma software con lo scopo di rendere possibile una o più funzionalità, servizi o strumenti utili
    	tramite una interfaccia utente.
    }\label{Applicativo}
    
    \parola{AWS Lambda}{%
        API che consente di eseguire codice senza la necessità di effettuare provisioning o gestire un server.
    }\label{AWS Lambda}
