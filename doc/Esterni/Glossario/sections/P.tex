\lettera{P}\label{P}

	\parola{Pattern Publisher / Subscriber}{%
		\gloss{Design pattern} utilizzato per la comunicazione asincrona fra diversi processi, oggetti o altri agenti.
		In questo schema, mittenti e destinatari di messaggi dialogano attraverso un tramite, detto dispatcher o broker. Il mittente di un messaggio (detto publisher) non deve essere consapevole dell'identità dei destinatari (detti subscriber); esso si limita a "pubblicare" (in inglese to publish) il proprio messaggio al dispatcher. I destinatari si rivolgono a loro volta al dispatcher "abbonandosi" (in inglese to subscribe) alla ricezione di messaggi. Il dispatcher quindi inoltra ogni messaggio inviato da un publisher a tutti i subscriber interessati a quel messaggio.
	}\label{Pattern Publisher / Subscriber}
	
    \parola{Peer to Peer}{%
        \`E un'espressione indicante un modello di architettura logica di rete informatica in cui i nodi non sono gerarchizzati
        unicamente sotto forma di clienti o serventi fissi, ma anche sotto forma di nodi equivalenti, potendo fungere al
        contempo da cliente e servente verso gli altri nodi terminali della rete.
    }\label{Peer to Peer}

	\parola{Pianificazione}{%
		Fase che precede lo svolgimento delle attività dove si decide la durata delle stesse e le risorse da esse impiegate.
	}\label{Pianificazione}
	
	\parola{Pipeline}{
		Il termine pipeline viene utilizzato per indicare un insieme di componenti software collegati tra loro in cascata, in modo che il risultato prodotto da uno degli elementi (output) sia l'ingresso di quello immediatamente successivo (input).
	}\label{Pipeline}
	
	\parola{Plug-in}{%
		\`E un programma non autonomo che interagisce con un altro programma per ampliarne o estenderne le funzionalità originarie.
	}\label{Plug-in}

	\parola{Preventivo}{
		Calcolo di previsione del costo di un determinato lavoro.
	}\label{Preventivo}
	
	\parola{Processo}{%
		Indica le attività da svolgere per permettere il corretto passaggio da uno
		stato del \gloss{ciclo di vita} del software a un altro, esso cioè trasforma dei bisogni (input) in prodotti (output), attraverso regole e consumo di risorse.
	}\label{Processo}

	\parola{Processo stateless}{%
		Tipo di processo che non necessita, nel corso della sua esecuzione, di avere uno stato.
	}\label{Processo stateless}

	\parola{Prodotto}{%
		Risultato finito e funzionante del lavoro fatto durante tutto lo svolgimento di un progetto. Il prodotto è ciò che viene consegnato al cliente alla fine. 
	}\label{Prodotto}

	\parola{Producer}{%
		Componente di Butterfly con lo scopo di raccogliere i messaggi e pubblicarli sotto forma di messaggi all'interno dei Topic adeguati.
	}\label{Producer}

	\parola{Progetto}{%
		Insieme di attività e compiti che prevedono il conseguimento di obiettivi e specifiche prefissati in una fascia di tempo definita. Sono definite date
		di inizio e date di fine in cui si potrebbero avere limitate risorse a disposizione.
	}\label{Progetto}

	\parola{Proof of Concept (PoC)}{%
		Realizzazione abbozzata di un progetto o metodo, provandone la fattibilità.
	}\label{PoC}

	\parola{Push}{%
		Comando che aggiorna i riferimenti remoti di una o più repository con i riferimenti locali, mandando gli oggetti necessari per allineare il branch remoto
		a quello locale.
	}\label{Push}
	
	\parola{Python}{%
		\`E un linguaggio di programmazione ad alto livello, orientato agli oggetti, adatto, tra gli altri usi, a sviluppare applicazioni distribuite, scripting, computazione numerica e \gloss{system testing}.
	}\label{Python}