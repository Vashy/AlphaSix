\lettera{Z}\label{Z}

    \parola{Zero laxity}{%
		%Laxity tradotto significa ``dilazionabilità'' ovvero la differenza tra scadenza e tempo di esecuzione netto ancora mancante.
		%Tale tempo che intercorre tra un'attività ed un'altra non è pianificato, questo per poter essere impiegato nel caso in cui si verificassero imprevisti che farebbero altrimenti slittare tutte le attività successive.\par
		%Per zero laxity si intende quindi non pianificare un breve lasso di tempo tra un'attività e l'altra.\par
		%Ci si ritroverebbe così a lavorare senza margine di tempo a disposizione.
		Per zero laxity s'intende terminare un'attività all'ultimo momento poichè c'è ``zero margine'', ovvero nessun margine di tempo a disposizione.
    }\label{Zero laxity}
