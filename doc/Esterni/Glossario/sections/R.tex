\lettera{R}\label{R}
    
    \parola{React/Redux}{%
	    Framework per lo sviluppo frontend di applicazioni basato su Javascript.
	}\label{ReactRedux}

	\parola{README}{
		File che contiene informazioni riguardo ai file contenuti in un archivio o in una directory.
	}\label{README}

	\parola{Redattore}{%
		Termine con cui ci si riferisce genericamente a chi redige un documento.
	}\label{Redattore}

	\parola{Redmine}{%
		\`E una piattaforma web open-source per la gestione di progetti attraverso diversi tool inerenti all'\gloss{Issue Tracking System}. Permette di creare le wiki dei progetti, diagrammi di \gloss{Gantt}, fare time-tracking ed interfacciarsi con diversi Version Control System come \gloss{GitLab}.
	}\label{Redmine}
	
	\parola{Regexp}{%
		Abbreviazione della traduzione inglese di espressione regolare, sono insiemi di simboli che identificano insiemi di stringhe. La notazione dipende dal programma
		utilizzato, per questo non esiste uno standard che le definisce.
	}\label{Regexp}

	\parola{Repository}{%
		Ambiente di un sistema informativo in cui vengono gestiti metadati attraverso tabelle relazionali, regole e motori di calcolo.
	}\label{Repository}

	\parola{Requisito}{%
		Caratteristica richiesta dal committente che deve possedere il prodotto finale. I requisiti, rispetto al capitolato, presentato possono essere:
		
		\begin{itemize}
			\item \textbf{Obbligatori}: da soddisfare obbligatoriamente;
			\item \textbf{Opzionali}: non obbligatori, ma che attribuiscono un valore aggiunto;
			\item \textbf{Desiderabili}: da soddisfare solo se avanzano risorse una volta completati i requisiti obbligatori;
		\end{itemize}
	}\label{Requisito}

	\parola{Rete Bayesiana}{%
	    Particolare tipo di rete finalizzato a calcolare la probabilità che un evento avvenga. In nodi della rete rappresentano tutte le parti da cui dipende la probabilità finale da calcolare, ognuno con la sua percentuale di successo. Gli archi invece rappresentano le dipendenze tra le parti.
	}\label{ReteBayesiana}
    
    \parola{Reti Ethereum}{%
        Reti per lo scambio di criptovaluta, ad esempio MainNet. Richiedono tempi lunghi per le transazioni.
	}\label{Reti Ethereum}
    
    \parola{Reti Raiden}{%
        Reti alternative per lo scambio di criptovaluta, consentono tempi istantanei per il completamento delle transazioni.
	}\label{Reti Raiden}

	\parola{Risorsa}{%
		
	}\label{Risorsa}

    \parola{Routine}{%
        Insieme di attivit\`a, consuetudini, che vengono ripetute costantemente a intervalli
        possibilmente riconoscibili (ad esempio, giorno per giorno).
    }\label{Routine}
