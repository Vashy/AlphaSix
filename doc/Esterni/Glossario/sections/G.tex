\lettera{G} \label{G}

    \parola{Gamification}{%
        Utilizzo di meccanismi tipici del gioco e in particolare dei videogiochi (punti, livelli, premi, beni virtuali,
        classifiche), per rendere gli utenti interessati ai servizi offerti.
    }\label{Gamification}

	\parola{Gantt}{%
		I diagrammi di Gantt servono per organizzare le attività e i ruoli all'interno di un progetto.
		
		L'asse orizzontale contiene una linea temporale, mentre quella verticale i componenti del team di sviluppo o le attività del progetto. All'interno del diagramma viene indicato in che finestra temporale ogni attività o ruolo sarà attivo.  
	}\label{Gantt}
    
    \parola{Gas}{%
        Costo dovuto alla computazione perché una transazione sia effettiva.
    }\label{Gas}
    
    \parola{Git}{%
		\`E un software di controllo versione distribuito utilizzabile da interfaccia a riga di comando,
		creato da Linus Torvalds nel 2005.
    }\label{Git}

	\parola{GitHub}{%
		\`E un servizio di hosting per l'implementazione di software che implementa lo strumento di controllo di sviluppo Git.
	}\label{GitHub}

	\parola{GitLab}{%
		Come GitHub, servizio di hosting per progetti software che utilizza lo strumento di controllo di versione Git.
	}\label{GitLab}
    
    \parola{Google Cloud Platform}{%
        Piattaforma che permette agli sviluppatori di costruire, testare e distribuire applicazioni.
    }\label{Google Cloud Platform}

    \parola{Google Drive}{%
        Servizio di \gloss{cloud} offerto da Google, con un piano gratuito di 15GB, e possibilità di espandere lo spazio di cloud con i piani a pagamento.
    }\label{Google Drive}

    \parola{Google Maps}{%
        Servizio di mappe e navigazione satellitare offerto da Google.
    }\label{Google Maps}
    
    \parola{Grafana}{%
		Software scritto in \gloss{JavaScript} che analizza i dati raccolti da vari tipi di database. Si occupa di mostrare i dati attraverso molti tipi di metriche, segnalare quando i valori raggiungono una soglia critica o li visualizza attraverso una vasta gamma di \gloss{dashboard}. Grafana offre la possibilità di creare \gloss{plug-in} ad esso collegati per personalizzare l'analisi dei dati. 
	}\label{Grafana}
