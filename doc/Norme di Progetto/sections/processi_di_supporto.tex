
\section{Processi di supporto}\label{PS}

	\subsection{Documentazione}\label{PS:Documentazione}


		\subsubsection{Implementazione}\label{PS:Documentazione:Implementazione}

			\paragraph{Template}\label{PS:Documentazione:Implementazione:Template}
			Prima di iniziare a redarre i documenti, è stato creato un template per \LaTeX \ (\S\ref{LaTeX})
			contenente tutte le impostazioni grafiche condivise tra i vari documenti, per sfruttare il riutilizzo
			del codice e semplificare enormemente la manutenzione dei \gloss{sorgenti}.\par
			Nello specifico, è presente un file per ognuna delle seguenti utilità:
			\begin{itemize}
				\item Layout delle pagine
				\item \gloss{Macro} personalizzate volte a semplificare l'utilizzo di strutture o comandi ricorrenti
				\item Codice per la generazione della prima pagina
					(struttura definita in \S\ref{PS:Documentazione:Struttura:Frontespizio})
				\item Diario delle modifiche
			\end{itemize}

			\paragraph{Ciclo di vita dei documenti}\label{PS:Documentazione:Implementazione:CicloVita}
			Durante il suo ciclo di vita, ogni documento potrà trovarsi in una delle seguenti fasi:
			\begin{itemize}
				\item \textbf{Redazione}: fase che inizia con la creazione del documento e dura fino alla sua ultima approvazione.
					Il \Res\ assegna ai \gloss{redattori} le varie sezioni di ogni documento da redarre, i quali aggiorneranno la versione nel diario delle modifiche
					come normato in \S\ref{Versionamento}.
				\item \textbf{Verifica}: il documento entra in questa fase nel momento in cui i redattori hanno terminato la stesura del lavoro loro assegnato
					segnalandolo al \Res, il quale assegnerà ai Verificatori la verifica della qualità del prodotto, secondo quanto riportato nelle norme di verifica.
					Essi potranno approvare il documento oppure notificare il \Res\ su eventuali errori o incongruenze emerse durante la fase di verifica, che provvederà
					a riassegnare il lavoro.
				\item \textbf{Approvazione}: fase che inizia dall'accettazione del documento da parte dei Verificatori nella fase di verifica. Spetta al \Res\
					l'approvazione ufficiale del documento, seguita dal rilascio di una \gloss{major release}.
			\end{itemize}

		\subsubsection{Struttura}\label{PS:Documentazione:Struttura}

			\paragraph{Frontespizio}\label{PS:Documentazione:Struttura:Frontespizio}
			Descrivere la prima pagina..

			\paragraph{Storico delle versioni}\label{PS:Documentazione:Struttura:StoricoVersioni}
			Contenuto

			\paragraph{Indice}\label{PS:Documentazione:Struttura:Indice}

			\paragraph{Contenuto}\label{PS:Documentazione:Struttura:Contenuto}
			La struttura di ogni pagina presenta:
			\begin{itemize}
				\item Intestazione con:
				\begin{itemize}
					\item a sinistra: logo di \emph{AlphaSix}
					\item a destra: nome del capitolato e documento corrente
				\end{itemize}
				\item Piè pagina con:
				\begin{itemize}
					\item a sinistra: nome del gruppo e mail di riferimento del gruppo
					\item a destra: numero della pagina corrente
				\end{itemize}
			\end{itemize}


		\subsubsection{Design}\label{PS:Documentazione:Design}

			\paragraph{Norme tipografiche}\label{PS:Documentazione:Design:NormeT}
			Le norme tipografiche qui di seguito elencate sono state decise in modo che ogni membro del gruppo concorra a mantenere una forma coerente e univoca
			per tutti i documenti redatti.

			\subparagraph{Stile del testo}\label{PS:Documentazione:Design:NormeT:StileTesto}
			\begin{itemize}
				\item \textbf{Corsivo}: solo per i nomi dei documenti citati.
				\item \textbf{Maiuscolo}: la prima lettera per
				\begin{itemize}
					\item tutte le parole appartenenti ai nomi dei documenti tranne gli articoli
					\item i nomi dei ruoli
				\end{itemize}
			\end{itemize}

			\subparagraph{Elenchi puntati}\label{PS:Documentazione:Design:NormeT:ElenchiPuntati}
			\begin{itemize}
				\item \textbf{Simboli di livello}: un pallino nero per il primo livello, un trattino per il secondo livello.
				\item \textbf{Punteggiatura}: nessuna punteggiatura alla fine di una frase, tranne nel caso in cui sia presente una descrizione.
					In quel caso la descrizione è preceduta dai due punti ``:'' e termina con un punto ``.''.
				\item \textbf{Grassetto}: solo se è presente una descrizione, allora sono in grassetto tutte le parole prima dei due punti ``:''.
			\end{itemize}

			\subparagraph{Altri formati testuali comuni} \label{PS:Documentazione:Design:NormeT:AltriFormati}
			\begin{itemize}
				\item \textbf{Orari}: [HH]:[MM] secondo la norma ISO 8601 nel formato 24 ore dove:
				\begin{itemize}
					\item HH indica le ore, da 00 a 23
					\item MM i minuti, da 00 a 59
				\end{itemize}
				\item \textbf{Date}: [DD]-[MM]-[YYYY] formato adottato in Europa dove:
				\item \textbf{Date}: \texttt{[DD]-[MM]-[YYYY]} formato adottato in Europa dove:
				\item \textbf{Date}: \texttt{DD-MM-YYYY} formato adottato in Europa dove:
				\begin{itemize}
					\item DD indica il numero del giorno, da 01 a 31
					\item MM del mese, da 01 a 12
					\item YYYY dell'anno
				\end{itemize}
				\item \textbf{Nota a piè pagina}: serve ad inserire elementi aggiuntivi, come osservazioni o riferimenti a parti interne al documento,
				che sono utili alla comprensione del testo, ma se inseriti all'interno del discorso ne interromperebbero la lettura rendendola meno scorrevole.
			\end{itemize}


			\paragraph{Elementi grafici}

			\subparagraph{Figure}
			Ogni immagine inserita nei documenti deve sempre essere centrata rispetto al foglio e adeguatamente separata dal testo. Deve inoltre essere
			accompagnata da una breve \gloss{caption} che permetta al lettore di capire esattamente che cosa sta guardando. È presente nell'indice l'Elenco
			delle Figure che raccoglie la lista di tutte le immagini presenti.

			\subparagraph{Tabelle}
			Come per le figure, ogni tabella sarà accompagnata da una caption e sarà della dimensione del testo, o se più piccola, centrata.
			Tutte le tabelle saranno raccolte nell'Elenco delle tabelle.\par
			Saranno presenti due tipologie di tabelle:
			\begin{itemize}
				\item \textbf{Semplici}: tabelle standard senza uno stile particolare, in cui le celle sono separate da bordi neri.
				\item \textbf{Complesse}: tabelle con un'alternanza di colori tra le righe delle celle (grigio e bianco) e senza bordi verticali.
					Le celle sono separate orizzontalmente da una corretta spaziatura e allineamento e verticalmente dall'alternanza dei due colori.
					La riga dell'header può essere bianca o di un grigio più scuro in base al contesto, con il testo che può essere in grassetto.
			\end{itemize}


		\subsubsection{Produzione}

			\paragraph{Suddivisione dei documenti}

			\subparagraph{Documenti interni}
			Sono considerati interni documenti quali \Doc{Studio di Fattibilità v1.0.0} e \Doc{Norme di Progetto v1.0.0} che non sono visibili ad entità esterne,
			ma solo ad AlphaSix.

			\subparagraph{Documenti esterni}
			Questi documenti sono ufficiali e approvati direttamente dal \Res\ e comprendono, per esempio: \Doc{Piano di Progetto v1.0.0},
			\Doc{Piano di Qualifica v1.0.0} e \Doc{Analisi dei Requisiti v1.0.0}. Si chiamano esterni perché accessibili al committente.

			\subparagraph{Glossario}
			Il documento denominato \Doc{Glossario v1.0.0} raccoglie in ordine alfabetico tutti i termini utilizzati che necessitano di una spiegazione più approfondita.
			Sono identificabili all'interno degli altri documenti da un font differente e una G a pedice la prima volta che appaiono.

			\subparagraph{Verbali}
			Questi documenti vengono redatti quando AlphaSix tiene delle riunioni o ci sono incontri esterni, per esempio con \II. Vengono redatti da una singola persona
			e presentano tutti le stesse sezioni:
			\begin{itemize}
				\item \textbf{Informazioni incontro}: lista delle informazioni principali riguardanti la riunione quali luogo, data, orario, ordine del giorno, ecc\dots
				\item \textbf{Argomenti}: lista dei principali argomenti trattati con descrizione di cosa si è discusso nel dettaglio.
				%TODO da riempire se viene aggiunto altro
			\end{itemize}


			\paragraph{Strumenti di supporto}

			\subparagraph{\LaTeX} \label{LaTeX}
			È stato scelto \LaTeX \ perché presenta molti vantaggi quali:

			\subparagraph{TexStudio}
			Fornisce scorciatoie...

			\subparagraph{Visual Studio Code}


			\subparagraph{GanttProject}


			\subparagraph{Gulpease}



		\subsubsection{Mantenimento}

			\paragraph{Versionamento} \label{Versionamento}
			Tutti i documenti redatti supporteranno il versionamento, in modo da essere univoci e rendere disponibile la possibilità di consultare versioni precedenti
			in qualsiasi fase del loro ciclo di vita.
			Il modello di versionamento adottato segue lo schema \gloss{change significance}. La versione di un file è espressa secondo la notazione
			\begin{center}
				\texttt{vX.Y.Z}
			\end{center}
			\indent dove:
			\begin{itemize}
				\item \texttt{X} indica il numero di versione principale. Inizia da 0 e viene incrementato ogni volta che il \Res\ approva il documento, determinando
					una \gloss{major release}.
				\item \texttt{Y} indica il numero di versione secondario, contatore delle fasi di verifica effettuate dal \Ver\ superate positivamente. Inizia da 0. Viene riportato a zero ad
					ogni incremento della \texttt{X}.
				\item \texttt{Z} è l'indice di modifica minore, incrementato ogni volta che viene effettuato un aggiornamento inferiore, quale l'aggiunta di una sezione
					o correzioni grammaticali/sintattiche di un certo peso. Viene azzerato ad ogni incremento della \texttt{Y} o della \texttt{X}.
			\end{itemize}

			\texttt{X}, \texttt{Y} e \texttt{Z} hanno dominio $[0,+\infty)$, possono assumere pertanto un valore $> 9$.

			\paragraph{Continuous Integration}
			Per quanto riguarda la stesura dei documenti, verrà adottato il principio di \gloss{continuous integration}, che sarà tuttavia limitato in questo periodo
			a sincronizzarsi il prima possibile con il repository remoto (non essendoci
			una vera e propria build o dei test da effettuare), sia per quanto riguarda il \gloss{fetch} che per quanto riguarda il \gloss{push}.

			Questo serve a rendere più remota possibile la probabilità di incappare nell'\gloss{integration hell}.


			\paragraph{Nomenclatura}

			\subparagraph{Verbali}	\label{NomenclaturaVerbali}
			I Verbali  possono essere interni oppure esterni, nel caso in cui il team incontri gli esponenti di \II.
			Il nominativo del file in cui sono formalizzati è il seguente:
			\begin{itemize}
				\item \texttt{VI\_dd-mm-yyyy.pdf} per i verbali interni
				\item \texttt{VE\_dd-mm-yyyy.pdf} per i verbali esterni
			\end{itemize}
			dove dd-mm-yyyy è la data in cui sono stati tenuti, nel formato descritto nel paragrafo \S\ref{PS:Documentazione:Design:NormeT:AltriFormati}.

			\subparagraph{Documenti vari}
			Saranno presenti due tipologie di file: file interni al gruppo e file esterni.
			\begin{itemize}
				\item La prima categoria include moduli di \LaTeX\ contenenti le varie sezioni, che non verranno mai esposti esternamente. Questi file verranno
					denominati usando la convenzione \texttt{snake\_case.tex}, dove snake\_case è il nome della sezione o modulo.
				\item La seconda categoria include i file \texttt{.tex} principali che produrranno i PDF da consegnare al committente. Essi verranno denominati
					con la convenzione \texttt{CamelCase\_vX.Y.X.tex} / \texttt{CamelCase\_vX.Y.Z.pdf}, dove CamelCase sarà il nome del documento generico mentre
					\texttt{vX.Y.Z} sarà la versione che identifica univocamente il documento come descritto in \S\ref{Versionamento}.
			\end{itemize}

	\subsection{Verifica}

		\subsubsection{Scopo}
		Verificare che le metriche vengano rispettate..

		%\subsubsection{Aspettative} %da mettere nel PdQ

		\subsubsection{Descrizione}

		\subsubsection{Metriche}

			\paragraph{Metriche per i documenti}

			\paragraph{Metriche per i processi}

			%\paragraph{Analisi statica e dinamica?}

			%\paragraph{...Tutti i test}	%UTILE??

			\paragraph{Strumenti}

			%\subparagraph{Verifica di documentazione}
			%Google Docs?

			\subparagraph{Verifica ortografica}
			Integrazione di TexStudio

			%\subparagraph{Integrazione continua} % Jenkins? Più avanti?
