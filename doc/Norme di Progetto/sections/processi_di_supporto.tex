
\section{Processi di supporto}\label{PS}

	\subsection{Documentazione}\label{PS:Documentazione}


		\subsubsection{Implementazione}\label{PS:Documentazione:Implementazione}
		Contenuto

			\paragraph{Template}\label{PS:Documentazione:Implementazione:Template}
			Prima di iniziare a redarre i documenti, è stato creato un template per \LaTeX \ (\S\ref{LaTeX})
			contenente tutte le impostazioni grafiche condivise tra i vari documenti, per sfruttare il riutilizzo
			del codice e semplificare enormemente la manutenzione dei \gloss{sorgenti}.\par
			Nello specifico, è presente un file per ognuna delle seguenti utilità:
			\begin{itemize}
				\item Layout delle pagine
				\item \gloss{Macro} personalizzate volte a semplificare l'utilizzo di strutture o comandi ricorrenti
				\item Codice per la generazione della prima pagina
					(struttura definita in \S\ref{PS:Documentazione:Struttura:Frontespizio})
				\item Diario delle modifiche
			\end{itemize}

			\paragraph{Ciclo di vita dei documenti}\label{PS:Documentazione:Implementazione:CicloVita}
			Durante il suo ciclo di vita, ogni documento potrà trovarsi in una delle seguenti fasi:
			\begin{itemize}
				\item \textbf{Redazione}: fase che inizia con la creazione del documento e dura fino alla sua ultima approvazione.
					Il \Res\ assegna ai \gloss{redattori} le varie sezioni di ogni documento da redarre, i quali aggiorneranno la versione nel diario delle modifiche
					come normato in \S\ref{Versionamento}.
				\item \textbf{Verifica}: il documento entra in questa fase nel momento in cui i redattori hanno terminato la stesura del lavoro loro assegnato
					segnalandolo al \Res, il quale assegnerà ai Verificatori la verifica della qualità del prodotto, secondo quanto riportato nelle norme di verifica.
					Essi potranno approvare il documento oppure notificare il \Res\ su eventuali errori o incongruenze emerse durante la fase di verifica, che provvederà
					a riassegnare il lavoro.
				\item \textbf{Approvazione}: fase che inizia dall'accettazione del documento da parte dei Verificatori nella fase di verifica. Spetta al \Res\
					l'approvazione ufficiale del documento, seguita dal rilascio di una \gloss{major release}.
			\end{itemize}

		\subsubsection{Struttura}\label{PS:Documentazione:Struttura}

			\paragraph{Frontespizio}\label{PS:Documentazione:Struttura:Frontespizio}


			\paragraph{Storico delle versioni}\label{PS:Documentazione:Struttura:StoricoVersioni}
			Contenuto

			\paragraph{Indice}\label{PS:Documentazione:Struttura:Indice}

			\paragraph{Contenuto}\label{PS:Documentazione:Struttura:Contenuto}
			La struttura di ogni pagina presenta:
			\begin{itemize}
				\item Intestazione con:
				\begin{itemize}
					\item a sinistra: logo di \textit{AlphaSix};
					\item a destra: nome del capitolato e documento corrente;
				\end{itemize}
				\item Piè pagina con:
				\begin{itemize}
					\item a sinistra: nome del gruppo e mail di riferimento del gruppo;
					\item a destra: numero della pagina corrente;
				\end{itemize}
			\end{itemize}


		\subsubsection{Design}\label{PS:Documentazione:Design}

			\paragraph{Norme tipografiche}\label{PS:Documentazione:Design:NormeT}
			Le norme tipografiche qui di seguito elencate sono state decise in modo che ogni membro del gruppo concorra a mantenere una forma coerente e univoca per tutti i documenti redatti.

			\subparagraph{Stile del testo}\label{PS:Documentazione:Design:NormeT:StileTesto}
			\begin{itemize}
				\item Corsivo: nomi documenti.
				\item Maiuscolo: nomi documenti tranne articoli, nomi di ruoli
			\end{itemize}

			\subparagraph{Elenchi puntati}\label{PS:Documentazione:Design:NormeT:ElenchiPuntati}
			\begin{itemize}
				\item Simboli di livello:
				\item Punteggiatura: nessuna punteggiatura alla fine di una frase, tranne nel caso in cui sia presente una descrizione. In quel caso si finisce con un punto ".".
				\item Grassetto: solo se è presente una descrizione, allora è in grassetto tutta la parte prima dei due punti ":".
			\end{itemize}

			\subparagraph{Altri formati testuali comuni} %"testuali" o "di scrittura"?
			\begin{itemize}
				\item Orari:  :
				\item Date:		-
				\item Nomi propri:
				\item Nomi documenti: tutto in maiuscolo tranne articoli
				\item Nomi ruoli: con maiuscola
			\end{itemize}



			\paragraph{Elementi grafici}

			\subparagraph{Figure}
			 Sempre caption che descrive

			\subparagraph{Tabelle}
			Contenuto



		\subsubsection{Produzione}

			\paragraph{Suddivisione dei documenti}

			\subparagraph{Documenti interni}
			Quelli interni..

			\subparagraph{Documenti esterni}
			Quelli esterni..

			\subparagraph{Glossario}

			\subparagraph{Verbali}



			\paragraph{Strumenti di supporto}

			\subparagraph{\LaTeX} \label{LaTeX}
			È stato scelto \LaTeX \ perché presenta molti vantaggi quali:

			\subparagraph{TexStudio}
			Fornisce scorciatoie...

			\subparagraph{Visual Studio Code}


			\subparagraph{GanttProject}


			\subparagraph{Gulpease}



		\subsubsection{Mantenimento}

			\paragraph{Versionamento} \label{Versionamento}
			Tutti i documenti redatti supporteranno il versionamento, in modo da essere univoci e rendere disponibile la possibilità di consultare versioni precedenti
			in qualsiasi fase del loro ciclo di vita.
			Il modello di versionamento adottato segue lo schema \gloss{change significance}. La versione di un file è espressa secondo la notazione
			\begin{center}
				\texttt{X.Y.Z}
			\end{center}
			\indent dove:
			\begin{itemize}
				\item \texttt{X} indica il numero di versione principale. Inizia da 0 e viene incrementato ogni volta che il \Res\ approva il documento, determinando
					una \emph{major release}.
				\item \texttt{Y} indica il numero di versione che indica una versione che ha superato la fase di verifica, effettuata dal \Ver. Inizia da 0. Viene riportato a zero ad
					ogni incremento della \texttt{X}.
				\item \texttt{Z} è l'indice di modifica minore, incrementato ogni volta che viene effettuato un aggiornamento inferiore, quale l'aggiunta di una sezione
					o correzioni grammaticali/sintattiche di un certo peso. Viene azzerato ad ogni incremento della \texttt{Y} o della \texttt{X}.
			\end{itemize}

			\texttt{X}, \texttt{Y} e \texttt{Z} hanno dominio $[0,+\infty)$, possono assumere pertanto un valore $> 9$.

			\paragraph{Continuous Integration}
			Per quanto riguarda la stesura dei documenti, verrà adottato il principio di \gloss{continuous integration}, che sarà tuttavia limitato in questo periodo
			a sincronizzarsi il prima possibile con il repository remoto (non essendoci
			una vera e propria \emph{build} o dei test da effettuare), sia per quanto riguarda il \gloss{fetch} che per quanto riguarda il \gloss{push}.

			Questo serve a rendere più remota possibile la probabilità di incappare nell'\gloss{integration hell}.


			\paragraph{Nomenclatura}

			\subparagraph{Verbali}
			I Verbali  possono essere interni oppure esterni, nel caso in cui il team incontri gli esponenti di \II.
			Il nominativo del file in cui sono formalizzati sarà univoco:
			\begin{itemize}
				\item \texttt{VI\_dd-mm-yyyy.pdf} per i verbali interni
				\item \texttt{VE\_dd-mm-yyyy.pdf} per i verbali esterni
			\end{itemize}
			dove dd-mm-yyyy è la data in cui sono stati tenuti, nel formato giorno-mese-anno.

			\subparagraph{Documenti vari}
			Saranno presenti due tipologie di file: file interni al gruppo e file esterni.
			\begin{itemize}
				\item La prima categoria include moduli di \LaTeX\ contenenti le varie sezioni, che non verranno mai esposti esternamente. Questi file verranno
					denominati usando la convenzione \texttt{snake\_case.tex}, dove snake\_case è il nome della sezione o modulo.
				\item La seconda categoria include i file \texttt{.tex} principali che produrranno i PDF da consegnare al committente. Essi verranno denominati
					con la convenzione \texttt{CamelCase\_vX.Y.X.tex} / \texttt{CamelCase\_vX.Y.Z.pdf}, dove CamelCase sarà il nome del documento generico mentre
					\texttt{X.Y.Z} sarà la versione che identifica univocamente il documento.
			\end{itemize}

	\subsection{Verifica}

		\subsubsection{Scopo}
		Verificare che le metriche vengano rispettate..

		%\subsubsection{Aspettative} %da mettere nel PdQ

		\subsubsection{Descrizione}

		\subsubsection{Metriche}

			\paragraph{Metriche per i documenti}

			\paragraph{Metriche per i processi}

			%\paragraph{Analisi statica e dinamica?}

			%\paragraph{...Tutti i test}	%UTILE??

			\paragraph{Strumenti}

			%\subparagraph{Verifica di documentazione}
			%Google Docs?

			\subparagraph{Verifica ortografica}
			Integrazione di TexStudio

			%\subparagraph{Integrazione continua} % Jenkins? Più avanti?
