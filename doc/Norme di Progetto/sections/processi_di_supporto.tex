
\section{Processi di supporto}\label{PS}	

	\subsection{Documentazione}\label{PS:Documentazione}
	
		
		\subsubsection{Implementazione}\label{PS:Documentazione:Implementazione}
		Contenuto
			
			\paragraph{Template}\label{PS:Documentazione:Implementazione:Template}
			Contenuto
				
			\paragraph{Ciclo di vita}\label{PS:Documentazione:Implementazione:CicloVita}
			Contenuto
	
		\subsubsection{Struttura}\label{PS:Documentazione:Struttura}
		
			\paragraph{Frontespizio}\label{PS:Documentazione:Struttura:Frontespizio}
			Contenuto
			
			\paragraph{Storico delle versioni}\label{PS:Documentazione:Struttura:StoricoVersioni}	
			Contenuto
			
			\paragraph{Indice}\label{PS:Documentazione:Struttura:Indice}
			
			\paragraph{Contenuto}\label{PS:Documentazione:Struttura:Contenuto}
			La struttura di ogni pagina presenta:
			\begin{itemize}
				\item Intestazione con:
				\begin{itemize}
					\item a sinistra: logo di \textit{AlphaSix};
					\item a destra: nome del capitolato e documento corrente;		
				\end{itemize}
				 \item Piè pagina con:
				 \begin{itemize}
				 	\item a sinistra: nome del gruppo e mail di riferimento del gruppo;
				 	\item a destra: numero della pagina corrente;
				\end{itemize}	
			\end{itemize}
	
	
		\subsubsection{Design}\label{PS:Documentazione:Design}
		
			\paragraph{Norme tipografiche}\label{PS:Documentazione:Design:NormeT}
			Le norme tipografiche qui di seguito elencate sono state decise in modo che ogni membro del gruppo concorra a mantenere una forma coerente e univoca per tutti i documenti redatti.
			
			\subparagraph{Stile del testo}\label{PS:Documentazione:Design:NormeT:StileTesto}
			\begin{itemize}
				\item Grassetto..
				\item Corsivo..
			\end{itemize}
		
			\subparagraph{Elenchi puntati}\label{PS:Documentazione:Design:NormeT:ElenchiPuntati}
			\begin{itemize}
				\item Siboli di livello:
				\item Punteggiatura:
				\item Grassetto:
			\end{itemize}
		
			\subparagraph{Altri formati testuali comuni} %"testuali" o "di scrittura"?
			\begin{itemize}
				\item Orari:
				\item Date:
				\item Nomi propri:
			\end{itemize}
		
		
		
			\paragraph{Elementi grafici}
			
			\subparagraph{Figure}
			Qunto spazio? Sempre caption?
			
			\subparagraph{Tabelle}
			Contenuto
			
			
			
		\subsubsection{Produzione}

			\paragraph{Suddivisione dei documenti}
			
			\subparagraph{Documenti informali}
			Quelli interni..
			
			\subparagraph{Documenti formali}
			Quelli esterni..
			
			\subparagraph{Glossario}
			
			\subparagraph{Verbali}
			
			
			
			\paragraph{Strumenti di supporto}
			
			\subparagraph{LATEX}
			È stato scelto Latex perché presenta molti vantaggi quali:
			
			\subparagraph{TexStudio} 
			Fornisce scorciatoie...
			
			%\subparagraph{Redmine} ?
			
		\subsubsection{Mantenimento}
		
			\paragraph{Versionamento}
			v[..]\_[..]\_[..]	%!! ?
			
			\paragraph{Nomenclatura}
			
			\subparagraph{Verbali}
			snake\_case?
				
			\subparagraph{Documenti vari}	
			snake\_case?

	\subsection{Verifica}
		
		\subsubsection{Scopo}
		Verificare che le metriche vengano rispettate..
		
		\subsubsection{Aspettative}
		
		\subsubsection{Descrizione}
		
		\subsubsection{Metriche}
		
			%\paragraph{Metriche p}
			
			%\paragraph{Analisi statica e dinamica?}
			
			%\paragraph{...Tutti i test}	%UTILE?? 
			
			\paragraph{Strumenti}
			
			\subparagraph{Verifica di documentazione}
			Google Docs?
			
			\subparagraph{Verifica ortografica}
			Integrazione di TexStudio
			
			%\subparagraph{Integrazione continua} %Jenkins? Più avanti?

