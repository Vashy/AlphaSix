\section{Processi Primari}\label{PP}

    \subsection{Processo di fornitura}\label{PP:Fornitura}	%istanziare Management Process, Infrastructure Process e Improvement Process

        \subsubsection{Scopo}\label{PP:Fornitura:Scopo}
		La sezione corrente ha lo scopo di riportare le  attività principali del fornitore in rapporto con il cliente. 
		
		\subsubsection{Ricerca delle tecnologie}
		Il gruppo approfondisce la propria conoscenza su tecnologie e framework vari, in modo da discutere insieme quali sono quelle di interesse comune e di utilità per il progetto. Questo prevede poi l'auto-formazione di quelle scelte in comune accordo per averne una buona padronanza.

        \subsubsection{Studio di Fattibilità}\label{PP:Fornitura:SdF} 
        In quest'attività viene prodotto il documento \DAlt{\textit{Studio di Fattibilità v1.0}} al fine di analizzare ogni capitolato e scegliere quale contratto accettare. 
        Nello specifico, il documento in ogni sezione contiene:
        	\begin{itemize}
        		\item \textbf{Descrizione generale}: breve descrizione del capitolato.
        		\item \textbf{Obiettivo finale}: punto focale su cui si concentra.
        		\item \textbf{Tecnologie coinvolte}: elenco delle tecnologie direttamente coinvolte esplicitate nel capitolato.
        		\item \textbf{Valutazione conclusiva}: giudizio finale del gruppo.
        	\end{itemize}
        
        \subsubsection{Preparazione in vista della revisione}
        In questa parte il gruppo prepara tutto il materiale necessario al buon superamento della revisione, come per esempio lettera di presentazione, documentazione e slide per l'esposizione.


    \subsection{Processo di sviluppo}\label{PP:Sviluppo}

        \subsubsection{Scopo}\label{PP:Sviluppo:Scopo}
        Cont


        \subsubsection{Analisi dei Requisiti}\label{PP:Sviluppo:AdR}
        Contenuto generico
        
        

		 \paragraph{Denominazione dei requisiti}\label{PP:Sviluppo:AdR:DenominazioneRequisiti}
		 Scrivere codice univoco per requisiti
		 Priorità - Tipo (se funzionale ecc) - Codice (?)
	
		 \paragraph{Casi d'uso}\label{PP:Sviluppo:AdR:CasiUso}
		 %descrivere cos'è?
		 La denominazione scelta per i casi d'uso è la seguente
		 \begin{itemize}
		 	\item UC[NUMERO] per uno principale (ad esempio, per il primo caso d'uso UC1)
		 	\item UC[NUMERO].[NUMERO] per un sotto caso (ad esempio, per il primo sotto caso del primo caso d'uso UC1.1)	
		 \end{itemize}
	 	 %ecc..
		        
	        %\subparagraph{Gerarchie di casi d'uso}\label{PP:Sviluppo:AdR:CasiDUso:GerarchieCasiDUso}	% SOLO SE UTILE
	        %Contenuto


        %\subsubsection{Progettazione}\label{PP:Sviluppo:Progettazione}	%PIÙ AVANTI
        %Contenuto generico
        

		 %\paragraph{Obiettivi}\label{PP:Sviluppo:Progettazione:Obiettivi}
		 %Contenuto



		\subsubsection{Diagrammi UML}\label{PP:Sviluppo:UML}	
		
		
        %\paragraph{Codifica}\label{PP:Sviluppo:Codifica} %PIÙ AVANTI

        
        \subsubsection{Strumenti}\label{PP:Sviluppo:Strumenti}
        
	    \paragraph{Ambiente di sviluppo}\label{PP:Sviluppo:Strumenti:AmbienteSviluppo}
	    Vengono qui riportate le componenti software utilizzate da ogni membro del gruppo per lo sviluppo del progetto.
	         	
	        	
	    \subparagraph{Sistema operativo}\label{PP:Sviluppo:Strumenti:AmbienteSviluppo:SistemaOperativo}
	    
	    Ubuntu 18.
	        		
	    %\subparagraph{IDE}\label{PP:Sviluppo:Strumenti:AmbienteSviluppo:IDE}
	    %Atom - Intellij?
        
		
		
	