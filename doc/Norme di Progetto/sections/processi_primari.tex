\section{Processi Primari}\label{PP}

    \subsection{Processo di fornitura}\label{PP:Fornitura}	%istanziare Management Process, Infrastructure Process e Improvement Process

        \subsubsection{Scopo}\label{PP:Fornitura:Scopo}
		La sezione corrente ha lo scopo di riportare le  attività principali del fornitore in rapporto con il cliente.

		\subsubsection{Ricerca delle tecnologie}
		Il gruppo approfondisce la propria conoscenza su tecnologie e \gloss{framework} vari, in modo da discutere insieme quali sono
		quelle di interesse comune e di utilità per il progetto. Questo prevede poi l'auto-formazione di quelle scelte in comune accordo per averne una buona padronanza.

        \subsubsection{Studio di Fattibilità}\label{PP:Fornitura:SdF}
        In quest'attività viene prodotto il documento \Doc{Studio di Fattibilità v1.0.0} al fine di analizzare ogni capitolato e scegliere quale contratto accettare.
        Nello specifico, il documento in ogni sezione contiene:
        	\begin{itemize}
        		\item \textbf{Descrizione generale}: breve descrizione del capitolato.
        		\item \textbf{Obiettivo finale}: punto focale su cui si concentra.
        		\item \textbf{Tecnologie coinvolte}: elenco delle tecnologie direttamente coinvolte esplicitate nel capitolato.
        		\item \textbf{Valutazione conclusiva}: giudizio finale del gruppo.
        	\end{itemize}

        \subsubsection{Preparazione in vista della revisione}
		In questa parte il gruppo prepara tutto il materiale necessario al buon superamento della revisione, come per esempio lettera di presentazione,
		documentazione e slide per l'esposizione.


    \subsection{Processo di sviluppo}\label{PP:Sviluppo}

		\subsubsection{Scopo}\label{PP:Sviluppo:Scopo}
		Lo sviluppo consiste nell'affrontare le attività volte a sviluppare il software richiesto dal proponente. Per una corretta implementazione di tale processo è
		necessario:
		\begin{itemize}
			\item Fissare degli obiettivi di sviluppo
			\item Realizzare un prodotto che sia conforme:
			\begin{itemize}
				\item ai requisiti definiti dal proponente
				\item ai test di definiti nelle norme di qualità
			\end{itemize}
		\end{itemize}
		Lo standard ISO/IEC 12207:1995\footnote{Fonte in \S\ref{rifinfo}} definisce il processo di sviluppo quel processo
		contenente tutte le attività relative al prodotto finale, quali:
		\begin{itemize} % [noitemsep]
			\item Analisi dei requisiti
			\item Progettazione
			\item Codifica
			\item Integrazione ed installazione
		\end{itemize}


        \subsubsection{Analisi dei Requisiti}\label{PP:Sviluppo:AdR}
		Gli Analisti si occupano di redigere l'\Doc{Analisi dei Requisiti v1.0.0}, composta dal seguente contenuto:
		\begin{itemize}
			\item Descrizione generale del prodotto
			\item Modellazione concettuale del sistema tramite la definizione dei vari casi d'uso
			\item Classificazione e tracciamento dei requisiti
		\end{itemize}

		\paragraph{Denominazione dei requisiti}\label{PP:Sviluppo:AdR:DenominazioneRequisiti}
		Ogni requisito che è stato individuato duranti l'analisi, presenta il seguente identificativo univoco \texttt{R[Numero][Tipo][Priorità]}, in cui:
		\begin{itemize}
		 	\item \textbf{Numero}: corrisponde ad un numero progressivo.
		 	\item \textbf{Tipo}: segnala la tipologia di requisito che può essere:
		 	\begin{itemize}
		 		\item \textbf{F}: requisito funzionale.
		 		\item \textbf{Q}: requisito di qualità.
		 		\item \textbf{V}: requisito di vincolo.
		 	\end{itemize}
	 		\item \textbf{Priorità}: indica il grado di urgenza di un requisito di essere soddisfatto, come:
	 		\begin{itemize}
	 			\item \textbf{0}: opzionale, di grado basso e solo marginalmente utile.
	 			\item \textbf{1}: desiderabile, di medio livello quindi non necessario.
	 			\item \textbf{2}: obbligatorio, di grado alto quindi importante per il committente e impossibile da tralasciare.
	 		\end{itemize}
		\end{itemize}

		\paragraph{Casi d'uso}\label{PP:Sviluppo:AdR:CasiUso}
		% descrivere cos'è?
		Un \gloss{caso d'uso} è una tecnica per identificare i requisiti funzionali che descrive le interazioni tra il sistema di riferimento e un utente ad esso esterno. \\
		Ogni caso d'uso che si vuol descrivere presenta:
		\begin{itemize}
		 	\item \textbf{Codice}: d'identificazione.
		 	\item \textbf{Titolo}: breve per dare una breve denominazione.
		 	\item \textbf{Attori}: tutti quelli coinvolti, sia primari che secondari.
		 	\item \textbf{Descrizione}: per spiegare più nel dettaglio le azioni.
		 	\item \textbf{Precondizione}: per rappresentare l'istante prima.
		 	\item \textbf{Postcondizione}: per rappresentare l'istante successivo.
		 	\item \textbf{Scenario principale}: contenente la serie di azioni da compiere numerate nell'ordine in cui vengono compiute.
		 	\item \textbf{Estensioni}: per azioni inerenti a scenari alternativi (d'eccezione o errore).
		\end{itemize}
		La rappresentazione scelta per il codice dei casi d'uso è:
		\begin{itemize}
		 	\item \textbf{UC[Numero]}: per un caso principale (ad esempio, per il primo caso d'uso avremo UC1)
			\item \textbf{UC[Numero].[Numero]}: per un sotto caso (ad esempio, per il primo figlio del primo caso d'uso avremo UC1.1)
		\end{itemize}
	 	 %ecc..


        %\subsubsection{Progettazione}\label{PP:Sviluppo:Progettazione}	%PIÙ AVANTI
        %Contenuto generico

		 %\paragraph{Obiettivi}\label{PP:Sviluppo:Progettazione:Obiettivi}
		 %Contenuto



		\subsubsection{Diagrammi UML}\label{PP:Sviluppo:UML}	%SOLO di attività e sequenza PIÙ AVANTI


		\paragraph{Diagrammi dei casi d'uso}	\label{DiagrammiCasiUso}
		Per i diagrammi dei casi d'uso viene utilizzato il linguaggio UML 2.0. Essi descrivono la visione di un utente esterno al sistema e non danno nessun dettaglio implementativo. I componenti di questo tipo di diagrammi sono:
		\begin{itemize}
			\item Attore
			\item Use case
			\item Relazioni, che possono essere di:
			\begin{itemize}
				\item Associazione
				\item Inclusione
				\item Estensione
				\item Generalizzazione
			\end{itemize}
		\end{itemize}
		Per la specifica di un caso d'uso si fa riferimento al paragrafo \S\ref{PP:Sviluppo:AdR:CasiUso}.


        %\paragraph{Codifica}\label{PP:Sviluppo:Codifica} %PIÙ AVANTI


        \subsubsection{Strumenti di base}\label{PP:Sviluppo:Strumenti}

	    \paragraph{Ambiente di sviluppo}\label{PP:Sviluppo:Strumenti:AmbienteSviluppo}
	    Vengono qui riportate le componenti software utilizzate da ogni membro del gruppo per lo sviluppo del progetto.


	    \subparagraph{Sistema operativo}\label{PP:Sviluppo:Strumenti:AmbienteSviluppo:SistemaOperativo}
	    Il sistema operativo scelto da tutti è Ubuntu, in particolare ognuno ha una versione di Ubuntu 18.

	    %\subparagraph{IDE}\label{PP:Sviluppo:Strumenti:AmbienteSviluppo:IDE}
	    %Atom - Intellij?



