\section{Introduzione}

\subsection{Premessa}
    Il documento che segue verrà prodotto incrementalmente al presentarsi della necessità di redigere nuove norme.
    Per questo motivo, non è da considerare al pari di un documento completo (e.g. la parte relativa alla codifica non ci sarà fino
    al presentarsi di quella necessità).

\subsection{Scopo del documento}
    Il presente \gloss{documento} ha l’obiettivo di mettere in chiaro le norme, le convenzioni e le tecnologie
    che verranno adottate dal gruppo AlphaSix durante lo svolgimento del \gloss{progetto}. Ogni membro del team
    \`e tenuto ad osservarlo rigorosamente, per mantenere consistenza ed omogeneit\`a in ogni aspetto, durante lo sviluppo
    del software che sarà prodotto durante il progetto.\par
    Questo documento verr\`a redatto incrementalmente, in base alle esigenze che verranno incontrate durante lo sviluppo del
    progetto stesso.

% DA FARE?
% \subsection{Scopo del prodotto}
    %D ire cosa fa Butterfly

\subsection{Glossario}
Tutti i termini qui presenti che richiedono una spiegazione più dettagliata, per evitare ambiguit\`a,
sono riconoscibili dal pedice G (e.g. \gloss{glossario})
e possono essere trovati, insieme alla loro definizione, nel documento allegato denominato \Doc{Glossario v1.0.0}.

\subsection{Riferimenti}

    \subsubsection{Normativi}	\label{rifnorma}
    \begin{itemize}
    	\item ISO 8601: \url{https://it.wikipedia.org/wiki/ISO\_8601#Orari}
    	\item ISO/IEC 12207: \url{https://en.wikipedia.org/wiki/ISO/IEC_12207}
    	\item Formato della data: \url{https://it.wikipedia.org/wiki/Formato\_della\_data}
    \end{itemize}

    \subsubsection{Informativi}	\label{rifinfo}
    \begin{itemize}
        \item Descrizione dei ruoli di progetto: \\\url{https://www.math.unipd.it/~tullio/IS-1/2018/Progetto/RO.html}
        \item Visual Studio Code: \url{https://code.visualstudio.com/docs}
        \item TexStudio: \url{https://www.texstudio.org/}
        \item GanttProject: \url{https://www.ganttproject.biz/}
	\end{itemize}
