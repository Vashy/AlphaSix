\section{Processi Organizzativi}


    \subsection{Gestione del progetto}

	    \subsubsection{Scopo}
	    Questa sezione ha lo scopo di delineare in cosa consiste l'attività di gestione del progetto.

	    % \subsubsection{Istanziazione dei processi}	%da rivedere
		% Derivano dagli standard...

		\subsubsection{Pianificazione delle attività}

			\paragraph{Scopo delle pianificazione}
			Si occupa di generare il PdP e comprende:
			\begin{itemize}
				\item Risorse
				\item Assegnazione delle risorse alle attività
				\item ...
				..
			\end{itemize}

			\paragraph{Obiettivi}
			L'attività è importante per...
			Struttura generale del PdP:
			\begin{itemize}
				\item Introduzione e scopo;
				\item Analisi;
				\item Modello di sviluppo adottato..
				..
			\end{itemize}

			\paragraph{Procedura}
			Procedura generale per l’organizzazione della pianificazione.
			con politiche di gestione dei rischi..

			\paragraph{Ruoli di progetto}
			I ruoli\footnote{Per la descrizione completa, vedere ``Descrizione dei ruoli di progetto'' in \S\ref{rifinfo}.} scelti per lo sviluppo del progetto sono:
			\begin{itemize}[noitemsep]
				\item Analista
				\item Progettista
				\item Responsabile
				\item Amministratore
				\item Programmatore
				\item Verificatore
			\end{itemize}


		\subsubsection{Pianificazione della qualità}
		Per cercare la qualità nei prodotti è necessario che vengano misurate diverse caratteristiche di questi ultimi.
		Per ogni caratteristica deve essere fissato un livello minimo è un livello ottimale che questa deve avere.
		
		In tale sezione viene descritto il modo di utilizzo di metriche e strumenti utili ad ottenere qualità nei processi, nei documenti e nei prodotti.
		Gli strumenti utilizzati possono essere software o standard di qualità descritti nel \Doc{\PdQ}.
		Gli obiettivi che ci si pone di raggiungere con questi strumenti sono descritti invece nel \Doc{\PdQ}.
			
			\paragraph{Classificazione degli obiettivi}
			Gli obiettivi per la qualità (Q) concordati dall'\Amm~e dal \Ver~ descritti nel \PdQ~avranno il seguente codice identificativo:
			
			\begin{center}
				Q[Tipo][ID]: Nome
			\end{center}
		
			\begin{itemize}
				\item \textbf{Tipo}: indica il la tipologia dell'oggetto di cui viene valutata la qualità, e questa può essere:
				\begin{itemize}
					\item \textbf{PR}: processi;
					\item \textbf{PD}: prodotti che risultano essere documenti;
					\item \textbf{PS}: prodotti software;
				\end{itemize}
			
				\item \textbf{ID}: ogni tipo di obiettivo possiede una lista ordinata attraverso un numero incrementale di tre cifre;
				\item \textbf{Nome}: offre un'informazione più chiara dell'obiettivo attraverso una breve frase;
			\end{itemize}
		
			Ad esempio:
			
			\begin{itemize}
				\item \textbf{QPD001: Leggibilità del testo}
			\end{itemize}
			
			
			\paragraph{Classificazione delle metriche}
			Ogni obiettivo della qualità deve per quanto possibile essere collegato ad una metrica, anch'essa scelta dall'\Amm~e dal \Ver. Questo per valutare quantitativamente il raggiungimento o meno degli obiettivi stabiliti.
			
			Le metriche (M) verranno classificate nel seguente modo:
			
			\begin{center}
				M[Tipo][ID]: Nome
			\end{center}
			
			\begin{itemize}
				\item \textbf{Tipo}: indica il la tipologia dell'oggetto di cui viene applicata la metrica, e questa può essere:
				\begin{itemize}
					\item \textbf{PR}: processi;
					\item \textbf{PD}: prodotti che risultano essere documenti;
					\item \textbf{PS}: prodotti software;
				\end{itemize}
				
				\item \textbf{ID}: ogni tipo di obiettivo possiede una lista ordinata attraverso un numero incrementale di tre cifre;
				\item \textbf{Nome}: offre un'informazione più chiara dell'obiettivo attraverso una breve frase;
			\end{itemize}
		
			Ad esempio:
			
			\begin{itemize}
				\item \textbf{MPD001: Indice Gulpease}
			\end{itemize}
		
			Quando sarà possibile, la metrica e l'obiettivo di qualità collegati fra loro avranno lo stesso Tipo e ID.
			
			Ad esempio:
		
			\begin{table}[H]
				\begin{detailtable}{\columnwidth}{YYYY}
					\thead{ID} & 
					\thead{Nome} &
					\thead{Descrizione} &
					\thead{Metrica}\\\hline\rowcolor{gray!15}
					QPD001 &Leggibilità del documento &La lettura del documento deve essere comprensibile &MPD001: Gulpease				\end{detailtable}
			\end{table}
			
		\subsubsection{Controllo del progetto}

			\paragraph{Monitoraggio del progetto}

			\subparagraph{Monitoraggio dell'esecuzione dei processi}
			Ogni processo verrà controllato periodicamente durante tutta la sua esecuzione in modo da non intralciare il way of working,
			mediante misurazioni associate a metriche descritte dai processi di verifica, controllate tramite l'utilizzo di strumenti automatizzati.
			Le metriche adottate sono, per il processo considerato, indicatori di efficacia dei prodotti rispetto ai requisiti di funzionalità e qualità,
			stabiliti nel \Doc{\PdQ\ v1.0.0} e nell'\Doc{\AdR v1.0.0}. Risultano validi indicatori per la valutazione di:
			\begin{itemize}
				\item Aderenza al way of working
				\item Stato di avanzamento del processo rispetto alla pianificazione
				\item Identificazione dei problemi
				\item Eventuali ripianificazioni
			\end{itemize}

			\subparagraph{Procedure di comunicazione}
			Per coordinare il team e il committente sullo stato del progetto sono state stabilite le seguenti norme:
			\begin{itemize}
				\item Per le comunicazioni interne verranno utilizzati gli strumenti segnalati in \S\ref{pianificazione e coordinamento}, in particolare
					con l'utilizzo di \gloss{Slack} e, all'occorrenza, verranno fissate riunioni per le questioni più importanti:
					\begin{itemize}
						\item concordare data, ora e luogo;
						\item stabilire un ordine del giorno da discutere;
						\item appuntare il contenuto della riunione per poter poi redigere formalmente il verbale della riunione
					\end{itemize}
				\item Per le comuncazioni esterne verrà utilizzata prevalentemente la comunicazione via email. % HELP WANTED fissare riunioni con l'azienda?
			\end{itemize}


    		\paragraph{Gestione dei problemi emersi}
			Problemi emersi durante l'esecuzione dei processi verranno segnalati tramite ticket, utilizzando il sistema integrato di Github come descritto
			in dettaglio in \S\ref{Github}. Questo permette di tracciare i problemi insorti durante le attività e di pianificare la risoluzione degli stessi.

    		% \paragraph{Riportare lo stato di avanzamento del progetto}
			% vedere Breaking Bug, io non riesco a capire il senso di questo paragrafo e lo toglierei ~ Tim

		\subsubsection{Chiusura dei processi}

    		\paragraph{Archiviazione dei prodotti} % TODO Rivedere
			I prodotti che, una volta terminate le attività e processi per completarli, soddisfano le aspettative in termini di qualità,
			verranno archiviati in apposite \gloss{repository}. Alla versione attuale di questo documento, RIVEDERE

    		\paragraph{Archiviazione delle misurazioni} % TODO
    		necessario archiviare le misurazioni effettuate attraverso le metriche definite nel presente documento..

    	\subsubsection{Strumenti di pianificazione e coordinamento}\label{pianificazione e coordinamento}

    		\paragraph{Slack}
			Slack è uno strumento di collaborazione sviluppato appositamente per coordinare il lavoro tra i team, permettendo la comunicazione in tempo
			reale, e mettendo a disposizione molte altre utilità indispensabili. Tra queste, c'è la possibilità di dividere il \gloss{workspace} in vari canali
			specifici, marcare parole o frasi importanti con diversi stili (grassetto, corsivo, codice, barrato), fissare i messaggi importanti, aprire una discussione
			sotto ogni messaggio per non creare ingorghi sul canale, ecc \dots

			Il workspace usato dal team è stato suddiviso in diversi canali, in base alle esigenze del periodo di lavoro. Nel periodo attuale, i vari canali sono:
			\begin{itemize}
				\item \textbf{\# general}: canale in cui verranno discusse tematiche generali riguardanti il progetto e le sue attività.
				\item \textbf{\# Analisi dei Requisiti}: discussioni inerenti alla stesura del documento \Doc{\AdR}. Nello specifico verranno discussi
					 i requisiti e i casi d'uso.
				\item \textbf{\# Piano di Progetto}: discussioni inerenti la pianificazione, la suddivisione del lavoro, la valutazione dei rischi, la consuntivazione e
					altre attività volte alla redazione del documento \Doc{\PdP}.
				\item \textbf{\# Piano di Qualifica}: discussioni inerenti principalmente la qualità di processo e di prodotto, riportate nel documento \Doc{\PdQ}.
				\item \textbf{\# Norme di Progetto}: discussioni riguardanti le norme che il gruppo adotterà e che ogni membro sarà tenuto a rispettare. Esse
					saranno riportate nel documento \Doc{\NdP}.
				\item \textbf{\# Studio di Fattibilità}: canale in cui verrà discussa la fattibilità di ogni capitolato proposto, convergenti nel documento
					\Doc{\SdF}.
				\item \textbf{\# git}: canale in cui un bot correttamente configurato manderà una notifica ogni volta che verrà effettuato un commit, oppure alla chiusura o apertura
					di una issue nel repository principale.
				\item \textbf{\# latex}: discussioni riguardanti gli aspetti tecnici di \LaTeX.
				\item \textbf{\# random}: discussioni off-topic del team.
			\end{itemize}

    		\paragraph{GitHub}\label{Github}
			Github fornisce un servizio di tracking delle issue per ogni repository che ospita. Questo strumento offre più funzionalità di quanto potrebbe
			sembrare, poichè permette anche di trattare le issue come delle \gloss{task}, marcandole con una opportuna label. Infatti, l'issue tracker offre le seguenti
			funzionalità:
			\begin{itemize}
				\item Aprire una issue, attribuendone un titolo e una descrizione più dettagliata. Ogni issue è legata a un ID che verrà automaticamente generato (e.g. \texttt{\#32}),
					al quale sarà possibile fare riferimento nei messaggi di commit.
				\item Creare delle \gloss{milestone}, e aggregare ad esse le issue.
				\item Assegnare delle label alle issue, per marcarne la natura. Ad esempio, una issue marcata come bug sarà un problema da risolvere,
					una issue marcata come enhancement sarà un miglioramento o una task. È possibile inoltre creare delle label personalizzate.
				\item Assegnare un issue a un collaboratore, che sarà tenuto a correggerla.
				\item Supporto di \gloss{markdown} esteso, che permette di formare delle pagine per le issue di aspetto delizioso e allo stesso tempo estremamente funzionali.
			\end{itemize}

			%\paragraph{Redmine} % ? da vedere piu avanti.
			% Per ora Github per issue-task tracker ~Tim



	\subsection{Formazione}

		La formazione di ogni membro del gruppo avviene tramite studio autonomo dei \gloss{framework} menzionati da \II\ durante la presentazione del progetto e incontri con AlphaSix.

		\subsubsection{Piano di formazione}
		Il piano di formazione prevede:
		\begin{itemize}
			\item Lo studio della documentazione di tecnologie e metodologie quali:
			\begin{itemize}
				\item Redmine
				\item GitLab
				\item SonarQube
				\item Apache Kafka
				\item Telegram
				\item Slack
				\item Docker
				\item API Rest
				\item The Twelve-Factor App %!! TODO aggiungere in Studio di fat
			\end{itemize}
			\item Lo studio delle possibilità d'integrazione tra di esse
		\end{itemize}

