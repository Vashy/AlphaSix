\section{Processi Organizzativi}

    \subsection{Gestione del progetto}

	    \subsubsection{Scopo}
	    Lo scopo è..
	    
	    %\subsubsection{Istanziazione dei processi}	%da rivedere
		%Derivano dagli standard...
		
		\subsubsection{Pianificazione delle attività}
		
			\paragraph{Scopo delle pianificazione}
			Si occupa di generare il PdP e comprende:
			\begin{itemize}
				\item Risorse
				\item Assegnazione delle risorse alle attività
				\item ...
				..
			\end{itemize}
		
			\paragraph{Obiettivi}
			L'attività è importante per...
			Struttura generale del PdP:
			\begin{itemize}
				\item Introduzione e scopo;
				\item Analisi;
				\item Modello di sviluppo adottato..
				..
			\end{itemize}
		
			\paragraph{Procedura}	
			Procedura generale per l’organizzazione della pianificazione.
			con politiche di gestione dei rischi..
			
			\paragraph{Ruoli di Progetto}
			I ruoli\footnote{La definizione di ogni ruolo: \url{https://www.math.unipd.it/~tullio/IS-1/2018/Progetto/RO.html}} presenti in questo progetto sono:
			\begin{itemize}
				\item Analista
				\item Progettista
				\item Responsabile
				\item 
			\end{itemize}
			 
		
		\subsubsection{Pianificazione della qualità}
		Citare Norme di prog e Piano di qualifica
		Ricordarsi di adattare il paragrafo alla figura del verificatore.
		
		\subsubsection{Controllo del progetto}
		
			\paragraph{Monitoraggio del progetto}
			
			\subparagraph{Monitoraggio dell'esecuzione dei processi}
			Citare PdQ ecc.
			
			\subparagraph{Procedure di comunicazione}
    		Prassi per le riunioni interne ed esterne..
    		
    		
    		\paragraph{Gestione dei problemi emersi}
    		Che trattiamo issue da risolvere..
    		
    		\paragraph{Riportare lo stato di avanzamento del progetto}
    		
    		
		\subsubsection{Chiusura dei processi}
		
    		\paragraph{Archiviazione dei prodotti}
			I prodotti che, una volta terminate le attività e processi per completarli, soddisfano le aspettative in termini di qualità,
			verranno archiviati in apposite \gloss{repository}. Alla versione attuale di questo documento,
    		
    		\paragraph{Archiviazione delle misurazioni} % TODO
    		necessario archiviare le misurazioni effettuate attraverso le metriche definite nel presente documento..
    		
    	\subsubsection{Strumenti di pianificazione e coordinamento}
    	
    		\paragraph{Slack} %da Telegram, trasferirsi
			\emph{Slack} è uno strumento di collaborazione sviluppato appositamente per coordinare il lavoro tra i team, permettendo la comunicazione in tempo
			reale, e mettendo a disposizione molte altre utilità indispensabili. Tra queste, c'è la possibilità di dividere il \gloss{workspace} in vari canali
			specifici, marcare parole o frasi importanti con diversi stili (grassetto, corsivo, codice, barrato), fissare i messaggi importanti, aprire un \gloss{thread}
			sotto ogni messaggio per non creare \gloss{spam} sul canale, ecc \dots

			Il workspace usato dal team è stato suddiviso in diversi canali, in base alle esigenze del periodo di lavoro. Nel periodo attuale, i vari canali sono:
			\begin{itemize}
				\item \textbf{\# general}: canale in cui verranno discusse tematiche generali riguardanti il progetto e le sue attività.
				\item \textbf{\# Analisi dei Requisiti}: discussioni inerenti alla stesura del documento \Doc{Analisi dei Requisiti}. Nello specifico verranno discussi
					 i requisiti e i casi d'uso.
				\item \textbf{\# Piano di Progetto}: discussioni inerenti la pianificazione, la suddivisione del lavoro, la valutazione dei rischi, la consuntivazione e
					altre attività volte alla redazione del documento \Doc{Piano di Progetto}.
				\item \textbf{\# Piano di Qualifica}: discussioni inerenti principalmente la qualità di processo e di prodotto, riportate nel documento \Doc{Piano di Qualifica}.
				\item \textbf{\# Norme di Progetto}: discussioni riguardanti le norme che il gruppo adotterà e che ogni membro sarà tenuto a rispettare. Esse
					saranno riportate nel documento \Doc{Norme di Progetto}.
				\item \textbf{\# Studio di Fattibilità}: canale in cui verrà discussa la fattibilità di ogni capitolato proposto, convergenti nel documento
					\Doc{Studio di Fattiblità}.
				\item \textbf{\# git}: canale in cui un bot correttamente configurato manderà una notifica ogni volta che verrà effettuato un commit o aperta una issue
					nella repository principale.
				\item \textbf{\# latex}: discussioni riguardanti gli aspetti tecnici di \LaTeX.
				\item \textbf{\# random}: discussioni \gloss{off-topic} del team.
			\end{itemize}

    		\paragraph{GitHub}
			\emph{Github} fornisce un servizio di tracking delle \emph{issue} per ogni repository che ospita. Questo strumento offre più funzionalità di quanto potrebbe
			sembrare, poichè permette anche di trattare le issue come delle \gloss{task}, marcandole con una opportuna label. Infatti, l'issue tracker offre le seguenti 
			funzionalità:
			\begin{itemize}
				\item Aprire una issue, attribuendone un titolo e una descrizione più dettagliata. Ogni issue è legata a un ID che verrà automaticamente generato (e.g. \texttt{\#32}),
					al quale sarà possibile fare riferimento nei messaggi di commit.
				\item Creare delle \gloss{milestone}, e aggregare ad esse le issue.
				\item Assegnare delle \emph{label} alle issue, per marcarne la natura. Ad esempio, una issue marcata come \emph{bug} sarà un problema da risolvere,
					una issue marcata come \emph{enhancement} sarà un miglioramento o una task. È possibile inoltre creare delle label personalizzate.
				\item Assegnare le issue a un collaboratore, che sarà tenuto a correggerla.
				\item Supporto di \gloss{markdown} esteso, che permette di formare delle pagine per le issue di aspetto delizioso e allo stesso tempo estremamente funzionali.
			\end{itemize}
    		
			%\paragraph{Redmine} % ? da vedere piu avanti. 
			% Per ora Github per issue-task tracker ~Tim

			
	
	\subsection{Formazione}		
	
		\subsubsection{Reperimento del materiale necessario al training}
		La formazione del personale avviene tramite studio autonomo dei framework proposti dall'azienda..
		
		\subsubsection{Piano di training}
		Fasi..
		scrivere delle varie tecnologie
