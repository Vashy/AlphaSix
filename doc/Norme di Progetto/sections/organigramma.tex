\newpage
\section{Organigramma}

	\subsection{Redazione}		
		\begin{table}[H]
			\centering
			\begin{orgtable}{\columnwidth}{YYm{7cm}}
				\rowcolor{gray!40}
				\thead{Nome} & \thead{Data} & \thead{Firma} \\\hline
				\CV & 16-12-2018 & \includegraphics[width=7cm]{img/firma_cv.png}  \\
			\end{orgtable}
			\caption{Redazione}
		\end{table}
	
	\subsection{Approvazione}
		\begin{table}[H]
			\centering
			\begin{orgtable}{\columnwidth}{YYm{7cm}}
				\rowcolor{gray!40}
				\thead{Nome} &\thead{Data} &\thead{Firma} \\\hline			
				\CV & 16-12-2018 & \includegraphics[width=7cm]{img/firma_cv.png}\\\hline\rowcolor{gray!15}
				Tullio Vardanega &  &  \includegraphics[width=7cm]{img/firma_cv.png}\\\hline
				Riccardo Cardin &  &  \includegraphics[width=7cm]{img/firma_cv.png}\\
			\end{orgtable}
			\caption{Approvazione}
		\end{table}
	
	\subsection{Accettazione componenti}
		\begin{table}[H]
			\centering
				\begin{orgtable}{\columnwidth}{YYm{7cm}}
				\rowcolor{gray!40}					
				\thead{Nome} &\thead{Data} &\thead{Firma} \\\hline
				\CV & 16-12-2018 & \includegraphics[width=7cm]{img/firma_cv.png}\\\hline\rowcolor{gray!15}
				\LC & 16-12-2018 & \includegraphics[width=7cm]{img/firma_cv.png}\\\hline
				\SG & 16-12-2018 & \includegraphics[width=7cm]{img/firma_cv.png}\\\hline\rowcolor{gray!15}
				\MM & 16-12-2018 & \includegraphics[width=7cm]{img/firma_cv.png}\\\hline
				\NC & 16-12-2018 & \includegraphics[width=7cm]{img/firma_cv.png}\\\hline\rowcolor{gray!15}
				\TG & 16-12-2018 & \includegraphics[width=7cm]{img/firma_cv.png}\\
			\end{orgtable}
			\caption{Accettazione componenti}
		\end{table}

	\subsection{Componenti}
		\begin{table}[H]
			\centering
			\begin{orgtable}{\columnwidth}{YcY}
				\rowcolor{gray!40}						
				\thead{Nome} & \thead{Matricola} & \thead{Indirizzo} \\
				\hline
				\CV & 1143057 & \href{mailto:stefanciprian.voinea@studenti.unipd.it}{stefanciprian.voinea@studenti.unipd.it} \\\hline\rowcolor{gray!15}						
				\LC & 1143488 & \href{mailto:laura.cameran@studenti.unipd.it}{laura.cameran@studenti.unipd.it} \\\hline
				\SG & 1111111 & \href{mailto:@studenti.unipd.it}{@studenti.unipd.it} \\\hline\rowcolor{gray!15}						
				\MM & 1143333 & \href{mailto:matteo.marchiori.4@studenti.unipd.it}{matteo.marchiori.4@studenti.unipd.it} \\\hline
				\NC & 1123257 & \href{mailto:nicola.carlesso.2@studenti.unipd.it}{nicola.carlesso.2@studenti.unipd.it} \\\hline\rowcolor{gray!15}		
				\TG & 1123442 & \href{mailto:timoty.granziero@studenti.unipd.it}{timoty.granziero@studenti.unipd.it} \\
			\end{orgtable}
			\caption{Componenti}
		\end{table}
	
	%DA RIMUOVERE?
	\subsection{Definizione dei ruoli}

		% RIVEDERE
		Nel corso del progetto i membri del gruppo si impegneranno a ricorpire diversi ruoli che rappresentano figure aziendali specializzate indispensabili per lo sviluppo ottimale e per la qualità del prodotto finale.\\
		Ciascun componente dovrà ricoprire almeno una volta ogni ruolo (con la possibilità che più persone ricoprano lo stesso ruolo nello stesso momento) facendo in modo che non ci siano sovrapposizioni di compiti e che chi fa da verificatore non vada a controllare il proprio lavoro ma sempre quello altrui, garantendo quindi assenza di conflitto di interessi tra i ruoli assunti.\\
		Questi ruoli sono:
	
		\begin{itemize}
			
			\item \textbf{Responsabile}: è garante del progetto, dei suoi risultati e della sua riuscita; lo rappresenta, per conto del suo gruppo, al committente.\\
			Possiede il potere decisionale, le sue responsabilità principali sono:
			%RIFORMULARE DA QUA
			\begin{itemize}
				\item Pianificazione, coordinamento e controllo delle attività;
				\item Gestione e controllo delle risorse;
				\item Analisi e gestione dei rischi;
				\item Approvazione dei documenti;
				\item Approvazione dell’offerta economica.
			\end{itemize}
			Redige il Piano di Progetto e collabora alla stesura del Piano di Qualifica, in particolare nella sezione relativa alla pianificazione;
			
			\item \textbf{Amministratore}: responsabile del controllo, dell’efficienza e dell’operatività dell’ambiente di lavoro.
			Le sue responsabilità principali sono:
			\begin{itemize}
				\item Ricerca di strumenti che possano automatizzare qualsiasi compito che possa essere tolto all’umano;
				\item Risoluzione dei problemi legati alle difficoltà di gestione e controllo dei processi e delle risorse. La risoluzione di tali problemi richiede l’adozione di strumenti adatti;
				\item Controllo delle versioni e delle configurazioni del prodotto;
				\item Gestione dell’archiviazione e del versionamento della documentazione di progetto;
				\item Fornire procedure e strumenti per il monitoraggio e segnalazione per il controllo qualità.
			\end{itemize}
			Redige le Norme di Progetto, dove spiega e norma l’utilizzo degli strumenti, redige la sezione del Piano di Qualifica dove vengono descritti strumenti e metodi di verifica;
		
			\item \textbf{Analista}: responsabile delle attività di analisi.
			Le sue responsabilità principali sono:			
			\begin{itemize}
				\item Produrre una specifica di progetto comprensibile, sia per il proponente, sia per il committente che per i progettisti, e motivata in ogni suo punto;
				\item Comprendere appieno la natura e la complessità del problema.
			\end{itemize}
			Redige lo Studio di Fattibilità, l’Analisi dei Requisiti e parte del Piano di Qualifica. 
			Partecipa alla redazione del Piano di Qualifica in quanto conosce l’ambito del progetto ed ha chiari i livelli di qualità richiesta e le procedure da applicare per ottenerla;
		
			\item \textbf{Progettista}: responsabile delle attività di progettazione.
			Le sue responsabilità principali sono:			
			\begin{itemize}
				\item Produrre una soluzione attuabile, comprensibile e motivata;
				\item Effettuare scelte su aspetti progettuali che applichino al prodotto soluzioni note ed ottimizzate;
				\item Effettuare scelte su aspetti progettuali e tecnologici che rendano il prodotto facilmente manutenibile.
			\end{itemize}
			Redige la Specifica Tecnica, la Definizione di Prodotto e le sezioni inerenti le metriche di verifica della programmazione del Piano di Qualifica; 
		
			\item \textbf{Programmatore}:responsabile delle attività di codifica e delle componenti di ausilio necessarie per l’esecuzione delle prove di verifica e validazione.
			Le sue responsabilità principali sono:			
			\begin{itemize}
				\item Implementare rigorosamente le soluzioni descritte dal progettista, da cui seguirà quindi la realizzazione del prodotto;
				\item Scrivere codice: documentato, versionato, manutenibile e che rispetti gli standard stabiliti per la scrittura del codice;
				\item Implementare i test sul codice scritto, necessari per prove di verifica e validazione.\item 
			\end{itemize}
			Redige il Manuale Utente e produce una abbondante documentazione del codice.
		
			\item \textbf{Verificatore}: responsabile delle attività di verifica.
			Le sue responsabilità principali sono:			
			\begin{itemize}
				\item Assicurare che l’attuazione delle attività sia conforme alle norme stabilite;
				\item Controllare la conformità di ogni stadio del ciclo di vita del prodotto.
			\end{itemize}
			Redige la sezione del Piano di Qualifica che illustra l’esito e la completezza delle verifiche e delle prove effettuate;

		\end{itemize}
	
		\noindent
		A ciascun ruolo viene assegnato un costo euro/ora in differente, in base ai compiti da svolgere e alle responsabilità, come segue:
	
		\begin{table}[H]
			\centering
			\begin{orgtable}{7cm}{YY}
				\rowcolor{gray!40}					
				\thead{Ruolo} & \thead{Costo \euro/h}\\\hline
				Responsabile& \EUR{30} \\\hline\rowcolor{gray!15}
				Amministratore& \EUR{20} \\\hline
				Analista& \EUR{25} \\\hline\rowcolor{gray!15}
				Progettista& \EUR{22} \\\hline
				Programmatore& \EUR{15} \\\hline\rowcolor{gray!15}
				Verificatore& \EUR{15} \\
			\end{orgtable}
			\caption{Costo \euro/h per ruolo}
		\end{table}