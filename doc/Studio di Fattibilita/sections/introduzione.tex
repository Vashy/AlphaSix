\section{Introduzione} \label{introduzione}
    \subsection{Scopo del documento}
    Il documento corrente ha lo scopo di descrivere brevemente ma con adeguata precisione ognuno dei sei capitolati proposti,
    in modo da evidenziare le ragioni per cui è stato scelto il \gloss{capitolato} C1 \textit{Butterfly}.

    \subsection{Glossario}
    Tutti i termini qui presenti che richiedono una spiegazione più dettagliata, per evitare ambiguit\`a,
sono riconoscibili dal pedice G (e.g. \gloss{glossario})
e possono essere trovati, insieme alla loro definizione, nel documento allegato denominato \Doc{Glossario v1.0.0}.

    \subsection{Riferimenti}

        \subsubsection{Riferimenti normativi}
            \begin{itemize}
                \item \textbf{Norme di progetto}: \Doc{Norme di Progetto v1.0.0}.
            \end{itemize}

        \subsubsection{Riferimenti informativi}
            \begin{itemize}
                \item Capitolato d'appalto C1, \textbf{Butterfly}\footnote{\url{http://www.math.unipd.it/~tullio/IS-1/2018/Progetto/C1.pdf}}:
                    monitor per processi CI/CD.
                \item Capitolato d'appalto C2, \textbf{Colletta}\footnote{\url{http://www.math.unipd.it/~tullio/IS-1/2018/Progetto/C2.pdf}}:
                    piattaforma raccolta dati di analisi di testo.
                \item Capitolato d'appalto C3, \textbf{G\&B}\footnote{\url{http://www.math.unipd.it/~tullio/IS-1/2018/Progetto/C3.pdf}}:
                    monitoraggio intelligente di processi DevOps.
                \item Capitolato d'appalto C4, \textbf{MegAlexa}\footnote{\url{http://www.math.unipd.it/~tullio/IS-1/2018/Progetto/C4.pdf}}:
                    arricchitore di skill di Amazon Alexa.
                \item Capitolato d'appalto C5, \textbf{P2PCS}\footnote{\url{http://www.math.unipd.it/~tullio/IS-1/2018/Progetto/C5.pdf}}:
                    piattaforma di peer-to-peer car sharing.
                \item Capitolato d'appalto C6, \textbf{Soldino}\footnote{\url{http://www.math.unipd.it/~tullio/IS-1/2018/Progetto/C6.pdf}}:
                    piattaforma Ethereum per pagamenti IVA.
            \end{itemize}
