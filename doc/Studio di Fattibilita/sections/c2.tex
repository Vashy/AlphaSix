\section{Colletta - C2} \label{c2}
    \subsection{Descrizione generale}
    Il progetto Colletta proposto dall'azienda Mivoq prevede la realizzazione di una piattaforma di collezione dati inerenti piccoli esercizi di grammatica. In essa si identificano tre attori principali: insegnanti, allievi e sviluppatori. L'insegnante fornisce gli esercizi di grammatica che vengono svolti dagli allievi. I dati derivanti da queste interazioni tra insegnati e allievi possono in seguito essere utilizzate dagli sviluppatori per migliorare il sistema di riconoscimento delle frasi.

    \subsection{Obiettivo finale}
    L'obiettivo del progetto è creare un'applicazione (web o mobile) con una struttura come quella descritta precedentemente, in cui la raccolta di dati avviene in modo implicito tramite il solo utilizzo da parte degli utenti. Questi dati possono essere impiegati successivamente per la produzione di servizi utili basati sull'apprendimento automatico.

    \subsection{Tecnologie coinvolte}
    La scelta delle tecnologie viene lasciata molto libera ma vengono consigliate:
    	\begin{itemize}
    		%TODO: da inserire nel glossario
    		\item \textbf{Firebase} (piattaforma mobile di Google di aiuto per lo sviluppo veloce di app di alta qualità) o altri servizi esistenti per l'immagazzinamento dei dati;
    		\item \textbf{Hunpos} (software open-source della proponente scritto in OCaml per l'analisi di parti del discorso), \textbf{Freeling} (libreria C++ che fornisce delle funzionalità per l'analisi del linguaggio) o altri software open-source per lo svolgimento degli esercizi;
    	\end{itemize}

    \subsection{Valutazione conclusiva}
    Il capitolato appena descritto non è stato scelto dal gruppo perchè tocca solo marginalmente l'ambito del machine learning. Tratta solamente di una piattaforma per la raccolta dati, motivo per cui non risulta particolarmente accattivante.
