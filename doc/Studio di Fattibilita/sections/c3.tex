\section{G\&B - C3} \label{c3}
    \subsection{Descrizione generale}
    Il capitolato C3 parte dall'esigenza di dover monitorare i sistemi che nel prossimo futuro di occuperanno della fatture emesse in formato digitale. Data la grande mole di dati emessi, tali sistemi devono essere soggetti ad un continuo controllo. Per fare ciò si ricerca una stretta collaborazione tra \textit{Development}\G e \textit{Operation}\GAlt, detta anche \textit{DevOps}\GAlt, grazie ad un efficiente sistema di notifica per eventuali problemi nei sistemi utilizzati.

    \subsection{Obiettivo finale}
    L'azienda proponente, come strumento di monitoraggio per questo tipo di sistemi, si affida a \textit{Grafana}\GAlt. Purtroppo questo prodotto si limita a segnalare eventuali problematiche principalmente nel momento in cui i parametri osservati superano un valore soglia. L'azienda vorrebbe dunque creare un plug-in per Grafana che sfrutti l'intelligenza artificiale attraverso una \textit{rete Bayesiana}\G affinché il monitoraggio sia più efficace.

    \subsection{Tecnologie coinvolte}
    L'azienda consiglia / richiede di utilizzare:
    \begin{itemize}
    	\item \textit{\textbf{AWS}}\GAlt, \textit{\textbf{Azure}}\G
    	\item \textit{\textbf{JavaScript}}\G
    	\item \textit{\textbf{Grafana}}\G
    	\item \textit{\textbf{Rete Bayesiana}}\GAlt: attraverso la libreria \textit{jsbayes}.
    	\item \textit{\textbf{GitHub}}\G
    \end{itemize}

    \subsection{Valutazione conclusiva}
    Il capitolato richiede lo studio di strumenti inerenti l'intelligenza artificiale, un argomento ritenuto molto interessante dal gruppo. Purtroppo il dover creare un plug-in per un software già ben strutturato, quale Grafana, ha spinto il team a pensare che le tecnologie utilizzate fossero toccate in modo marginale.
