\section{G\&B - C3} \label{c3}
    \subsection{Descrizione generale}
    Il capitolato C3 parte dall'esigenza di dover monitorare i sistemi che nel prossimo futuro di occuperanno della fatture emesse in formato digitale. Data la grande mole di dati emessi, tali sistemi devono essere soggetti ad un continuo controllo. Per fare ciò si ricerca una stretta collaborazione tra "Development" e "Operation", detta anche "DevOps", e un'efficiente sistema di notifica di eventuali problemi nei sistemi utilizzati.

    \subsection{Obiettivo finale}
    L'azienda proponente, come strumento di monitoraggio per questo tipo di sistemi, si affida a Grafana. Purtroppo Grafana si limita a segnalare eventuali problematiche principalmente nel momento in cui i parametri osservati superano un valore soglia. L'azienda vorrebbe dunque creare un plug-in per Grafana che sfrutti l'intelligenza artificiale attraverso una rete Bayesiana affinché il monitoraggio sia più efficacie.

    \subsection{Tecnologie coinvolte}
    \begin{itemize}
    	\item \textbf{JavaScript}
    	\item \textbf{Grafana}
    	\item \textbf{Rete Bayesiana} (attraverso la libreria "jsbayes")
    	\item \textbf{GitHub}
    \end{itemize}

    \subsection{Valutazione conclusiva}
    Il capitolato richiede lo studio di tecnologie inierenti all'intelligenza artificiale, un argomento ritenuto molto interessante. Purtroppo, per quanto concerne agli interessi del gruppo, il dover creare un plug-in per una tecnologia già ben strutturata, quale è Grafana, ha fatto ritenere al team che le tecnologie affrontate fossero poche e toccate in modo marginale.
