\section{P2PCS - C5} \label{c5}
    \subsection{Descrizione generale}
    Il quinto capitolato propone la creazione di un'applicazione Android in grado di gestire una piattaforma di car sharing peer to peer.

    \subsection{Obiettivo finale}
    È richiesto lo sviluppo di un sistema software che consente ad un utente la condivisione della propria auto. Tale condivisione si potrà avere una volta che l'operatore avrà inserito nel calendario i giorni in cui non utilizzerà il mezzo, lasciando così la possibilità di poterlo usufruire ad un'altra figura.

    \subsection{Tecnologie coinvolte}
    	\begin{itemize}
    		\item Google Maps: servizio di mappe e navigazione satellitare.
		\item Google Cloud Platform: Piattaforma che permette agli sviluppatori di costruire, testare e distribuire applicazioni.
		\item Henshin - movens platform: piattaforma Open Source per la mobilità.
		\item Octalysis: È un progetto che cerca di ottimizzare le scelte umane servendosi del coinvolgimento, degli interessi e della motivazione.
		\item Node.js: piattaforma per il motore Javascript V8.
		\item ?Java EE: insieme di specifiche indicate per la programmazione in Java?
	\end{itemize}
	
    \subsection{Valutazione conclusiva}
    Fin da subito, l'idea di "condividere la propria macchina con altre persone, amici o meno" non ha suscitato interesse e motivazione al gruppo. Inoltre la tecnologia Octalysis vincolava troppo l'andamento del progetto, richiedendone un uso stringente. Dopo queste considerazioni il team ha deciso di spostare i propri interessi verso altri capitolati.
