\section{P2PCS - C5} \label{c5}
    \subsection{Descrizione generale}
    Il quinto capitolato propone la creazione di un'applicazione \textit{Android}\G in grado di gestire una piattaforma di \textit{car sharing}\G \textit{peer to peer}\G.

    \subsection{Obiettivo finale}
    È richiesto lo sviluppo di un sistema software che consenta ad un utente la condivisione della propria auto. Tale condivisione si potrà avere solo una volta che l'operatore avrà inserito nel calendario i giorni in cui non utilizzerà il mezzo.
    In questo modo si lascia così la possibilità di poter dare le chiavi della propria autovettura in mano ad un'altra figura.

    \subsection{Tecnologie coinvolte}
	L'azienda consiglia di utilizzare:
    	\begin{itemize}
    		\item \textit{\textbf{Google Maps}}\GAlt
		\item \textit{\textbf{Google Cloud Platform}}\GAlt
		\item \textit{\textbf{Henshin - movens platform}}\GAlt
		\item \textit{\textbf{Octalysis}}\GAlt
		\item \textit{\textbf{Node.js}}\GAlt
	\end{itemize}
	
    \subsection{Valutazione conclusiva}
    Fin da subito, l'idea di ``condividere la propria macchina con altre persone amiche o meno'' (cit da capitolato) non ha suscitato interesse e motivazione al gruppo. Inoltre la tecnologia \textit{Octalysis}\G sembra vincolare troppo l'andamento del progetto, richiedendone un uso stringente e diventa solamente un'applicazione che, secondo il parere del gruppo, sfrutta il \textit{gamification}\G per convincere gli utenti. Dopo queste considerazioni il team ha deciso di spostare i propri interessi verso altri capitolati.
