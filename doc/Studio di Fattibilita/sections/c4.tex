\section{MegaAlexa - C4} \label{c4}
    \subsection{Descrizione generale}
    Il \textit{capitolato}\G propone lo sviluppo di una piattaforma
    dedita alla creazione di una routine, in grado di eseguire una sequenza
    di skill per \textit{Alexa}\GAlt, l'assistente vocale di Amazon.

    \subsection{Obiettivo finale}
    Nello specifico, \`e richiesto lo sviluppo di una piattaforma multilingue web e
    mobile (iOS o Android) che sia in grado di 
    creare workflow personalizzati per Alexa creati dagli utenti.

    \subsection{Tecnologie coinvolte}
    L'azienda consiglia di utilizzare:
    \begin{itemize}
    	\item Linguaggi per la piattaforma web:
    	\begin{itemize}
    		\item Bootstrap
    	\end{itemize}
        \item \textit{\textbf{HTML5}}\GAlt, \textit{\textbf{CSS3}}\GAlt, \textit{\textbf{Javscript}}\GAlt:
        \item \textit{\textbf{Node.js}}\GAlt:
        \item \textit{\textbf{Kotlin}}\GAlt \textit{\textbf{Swift}}\GAlt:
        \item \textit{\textbf{Amazon Web Services}}\GAlt
            \begin{itemize}
                \item \textit{\textbf{API Gateway}}\GAlt;
                \item \textit{\textbf{Lambda}}\GAlt;
                \item \textit{\textbf{Aurora Serverless}}\GAlt.
            \end{itemize}
    \end{itemize}

    \subsection{Valutazione conclusiva}
    Dall'analisi fatta dal team è stato deciso di non sviluppare questo capitolato per la mole di lavoro elevata.
    Non è semplice inoltre gestire le varie routine per gli utenti oltre che sviluppare un'applicazione web e mobile nativa multilingua.
    Nonostante questo le tecnologie di AWS e di workflow di assistenti vocali risultano interessanti.

