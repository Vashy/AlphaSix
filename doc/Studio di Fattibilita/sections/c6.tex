\section{Soldino - C6} \label{c6}
    \subsection{Descrizione generale}
    Redbabel Studio propone lo sviluppo di un sistema che consenta l'amministrazione delle \gloss{V.A.T.} imposte su beni venduti tramite \gloss{e-commerce}.
    \subsection{Obiettivo finale}

    L'obiettivo è quello di ottenere un prodotto che coinvolga l'intera gestione delle tasse, dal pagamento effettuato dal cliente per il singolo bene
    fino alla ricezione e gestione da parte del governo, passando per la ricezione della tassa da parte del fornitore.
    L'applicativo è composto da una parte che si occupa di gestire gli \gloss{smart contracts} e da una parte Web che permetta l'accesso alla \gloss{EVM}.

    \subsection{Tecnologie coinvolte}
	L'azienda consiglia:
    	\begin{itemize}
        \item \gloss{Ethereum}
		\item \gloss{Blockchain}
        \item \gloss{Smart Contracts}
		\item \gloss{EIP712}
		\item \gloss{Gas}
		\item \gloss{ERC20}
		\item \gloss{Reti Ethereum}
        \item \gloss{Reti Raiden}
        \item \gloss{Javascript ES8}
        \item \gloss{React/Redux}
        \item \gloss{SCSS}
	\end{itemize}

    \subsection{Valutazione conclusiva}
    Nonostante l'uso di tecnologie innovative come Blockchain, smart contracts e gestione delle tasse in modo automatico, \gruppo non ha scelto
    il progetto perché richiede molto tempo da dedicare allo studio approfondito di tutti gli argomenti coinvolti.
    Essendo queste tecnologie nuove il team era indeciso se scegliere questo capitolato o optare su strumenti più consolidati.