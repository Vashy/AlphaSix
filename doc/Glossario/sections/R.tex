\lettera{R}\label{R}
    
    \parola{React/Redux}{%
	    Framework per lo sviluppo frontend di applicazioni basato su Javascript.
	}\label{ReactRedux}

	\parola{Redattore}{%
		Termine con cui ci si riferisce genericamente a chi redige un documento.
	}\label{Redattore}

	\parola{Redmine}{%
		\'E una piattaforma web open-source per la gestione di progetti attraverso diversi tool inerenti all'Issue Tracking System. Permette di creare le wiki dei progetti, diagrammi di \gloss{Gantt}, fare time-tracking ed interfacciarsi con diversi Version Control System come \gloss{GitLab}.
	}\label{Redmine}
	
	\parola{Regexp}{%
		Abbreviazione della traduzione inglese di espressione regolare, sono insiemi di simboli che identificano insiemi di stringhe. La notazione dipende dal programma
		utilizzato, per questo non esiste uno standard che le definisce.
	}\label{Regexp}

	\parola{Repository}{%
		Ambiente di un sistema informativo in cui vengono gestiti metadati attraverso tabelle relazionali, regole e motori di calcolo.
	}\label{Repository}

	\parola{Rete Bayesiana}{%
	    Particolare tipo di rete finalizzato a calcolare la probabilità che un evento avvenga. In nodi della rete rappresentano tutte le parti da cui dipende la probabilità finale da calcolare, ognuno con la sua percentuale di successo. Gli archi invece rappresentano le dipendenza tra le parti.
	}\label{ReteBayesiana}
    
    \parola{Reti Ethereum}{%
        Reti per lo scambio di cryptocurrency, ad esempio MainNet. Richiedono tempi lunghi per le transazioni.
	}\label{Reti Ethereum}
    
    \parola{Reti Raiden}{%
        Reti alternative per lo scambio di cryptocurrency, consentono tempi istantanei per il completamento delle transazioni.
	}\label{Reti Raiden}

    \parola{Routine}{%
        Insieme di attivit\`a, consuetudini, che vengono ripetute costantemente a intervalli
        possibilmente riconoscibili (ad esempio, giorno per giorno).
    }\label{Routine}
