\lettera{P}\label{P}

	%Copiato da wikipedia, va rivisto (cv)
	\parola{Pattern Publisher / Subscriber}{
		Design pattern utilizzato per la comunicazione asincrona fra diversi processi, oggetti o altri agenti.
		Per la sua natura di integrazione tra diverse sorgenti software, il pattern publish/subscribe può essere considerato un middleware.
		In questo schema, mittenti e destinatari di messaggi dialogano attraverso un tramite, che può essere detto dispatcher o broker. Il mittente di un messaggio (detto publisher) non deve essere consapevole dell'identità dei destinatari (detti subscriber); esso si limita a "pubblicare" (in inglese to publish) il proprio messaggio al dispatcher. I destinatari si rivolgono a loro volta al dispatcher "abbonandosi" (in inglese to subscribe) alla ricezione di messaggi. Il dispatcher quindi inoltra ogni messaggio inviato da un publisher a tutti i subscriber interessati a quel messaggio.
		In genere, il meccanismo di sottoscrizione consente ai subscriber di precisare nel modo più specifico possibile a quali messaggi sono interessati. Per esempio, un subscriber potrebbe "abbonarsi" solo alla ricezione di messaggi da determinati publisher, oppure aventi certe caratteristiche.
		Questo schema implica che ai publisher non sia noto quanti e quali siano i subscriber e viceversa. Questo può contribuire alla scalabilità del sistema
	}\label{Pattern Publisher / Subscriber}
	
    \parola{Peer to Peer}{%
        È un'espressione indicante un modello di architettura logica di rete informatica in cui i nodi non sono gerarchizzati
        unicamente sotto forma di clienti o serventi fissi, ma pure sotto forma di nodi equivalenti, potendo fungere al
        contempo da cliente e servente verso gli altri nodi terminali della rete.
    }\label{Peer to Peer}

	\parola{Producer}{
		Componente di Butterfly con lo scopo di raccogliere i messaggi e pubblicarli sotto forma di messaggi all'interno dei topic adeguati
	}\label{Producer}