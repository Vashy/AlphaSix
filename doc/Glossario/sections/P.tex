\lettera{P}\label{P}

	\parola{Pattern Publisher / Subscriber}{%
		Design pattern utilizzato per la comunicazione asincrona fra diversi processi, oggetti o altri agenti.
		In questo schema, mittenti e destinatari di messaggi dialogano attraverso un tramite, che può essere detto dispatcher o broker. Il mittente di un messaggio (detto publisher) non deve essere consapevole dell'identità dei destinatari (detti subscriber); esso si limita a "pubblicare" (in inglese to publish) il proprio messaggio al dispatcher. I destinatari si rivolgono a loro volta al dispatcher "abbonandosi" (in inglese to subscribe) alla ricezione di messaggi. Il dispatcher quindi inoltra ogni messaggio inviato da un publisher a tutti i subscriber interessati a quel messaggio.
	}\label{Pattern Publisher / Subscriber}
	
    \parola{Peer to Peer}{%
        \`E un'espressione indicante un modello di architettura logica di rete informatica in cui i nodi non sono gerarchizzati
        unicamente sotto forma di clienti o serventi fissi, ma pure sotto forma di nodi equivalenti, potendo fungere al
        contempo da cliente e servente verso gli altri nodi terminali della rete.
    }\label{Peer to Peer}

	\parola{Producer}{%
		Componente di Butterfly con lo scopo di raccogliere i messaggi e pubblicarli sotto forma di messaggi all'interno dei topic adeguati
	}\label{Producer}