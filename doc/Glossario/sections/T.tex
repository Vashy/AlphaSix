\lettera{T}\label{T}

	\parola{Task}{%
		Termine inglese utilizzato per definire compito.
	}\label{Task}

	\parola{Telegram}{%
    	Servizio di messaggistica istantanea tramite Internet che offre la possibilità di chattare tra due o più persone (gruppi). 
    	Si possono creare bot con varie funzioni in grado di ricevere, elaborarare e mandare messaggi. 
	}\label{Telegram}
	
	\parola{Template}{%
		Documento o programma nel quale è definito un ``modello'' o un'``ossatura generale'' riutilizzabile in più ambiti.
	}\label{Template}

	\parola{The Twelve-Factor App}{
		È una metodologia di sviluppo che può essere applicata a qualunque software, scritto in qualsiasi linguaggio di programmazione. L'obiettivo che si pone è orientare l'applicazione a seguire un formato dichiarativo, interfacciarsi in modo pulito, adattarsi alle più recenti piattaforme cloud e minimizzare la divergenza tra sviluppo e produzione.
	}\label{12}

	\parola{Tool}{
		Tool è il termine inglese equivalente all'italiano "strumento". Nel contesto informatico si intende un'applicazione che svolge un determinato compito.
	}\label{Tool}

	\parola{Topic}{
		Equivalente di "argomento" in italiano. Nel contesto del capitolato C1 si intende un canale di messaggi associato ad uno specifico argomento.
	}\label{Topic}
	