\lettera{M}\label{M}

    \parola{Machine Learning}{%
        Termine anglofono di apprendimento automatico, è un insieme di metodi volti a far apprendere alle macchine (intesi come software)
        senza essere state esplicitamente o preventivamente programmate.
    }\label{Machine Learning}

    \parola{Major Release}{%
        Specifica versione di un prodotto che ne determina l'incremento di uno dei numeri di versione più significativi (dipendente dalle specifiche
        e dal contesto ove è definito), a seguito di modifiche importanti, e ne determina la distribuzione.
    }\label{Major Release}

    \parola{Marco}{%
        Indica una procedura o ``blocco'' di comandi tipicamente ricorrente durante l'esecuzione di un programma,
        volta a favorire il riutilizzo del codice.
    }\label{Macro}

    \parola{Markdown}{%
        Linguaggio di markup con una sintassi molto semplice, convertibile in altri formati quali HTML con un tool omonimo.
    }\label{Markdown}

	\parola{Metadato}{%
		particolare dato che descrive insiemi di altri dati.
	}\label{Metadato}

    \parola{Milestone}{%
        In italiano pietra miliare, indica importanti traguardi intermedi nello svoglimento di un progetto. Più nello specifico è un momento nel tempo in cui vengono a concludersi n \gloss{baseline}, perciò una milestone è associata ad una o più baseline.
    }\label{Milestone}
