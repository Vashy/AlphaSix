\lettera{C}\label{C}

	\parola{Callback}{
		Riferimento a una funzione oppure a un "blocco di codice".
	}\label{Callback}

    \parola{Capitolato}{%
        Documento tecnico, possibilmente allegato a un contratto d'appalto che intercorre tra il cliente ed un committente, in
        quest'ultimo indica costi, modalit\`a e tempi di realizzazione dell'opera oggetto del contratto.
    }\label{Capitolato}

	\parola{Caption}{
		Termine generico col quale viene definito un titolo o una didascalia per qualsiasi documento o immagine. Dall'inglese: titolo.
	}\label{Caption}
    
    \parola{Car Sharing}{%
        È un servizio di mobilità urbana che permette agli utenti di utilizzare un veicolo su prenotazione noleggiandolo per un
        periodo di tempo breve, nell'ordine di minuti o ore, e pagando in ragione dell'utilizzo effettuato.
    }\label{Car Sharing}

	\parola{Caso d'uso}{
		Il caso d'uso è una serie di azioni in sequenza, che determinano uno scenario, miranti ad un obiettivo che un utente, chiamato attore, vuole raggiungere.
	}\label{Caso d'uso}

	\parola{Change Significance}{%
		Schema in cui i cambiamenti sono definiti da livelli di importanza differenti. Ogni livello è indicato da un numero in una posizione diversa: generalmente
		il numero più a sinistra è quello con un importanza maggiore e che determina un significativo numero di cambiamenti, e man mano che il numero si trova più a
		destra, il numero di cambiamenti necessario a incrementarlo è minore (e.g. nella forma X.Y.Z, l'incremento della X determina un grosso cambiamento,
		mentre la Z potrebbe cambiare ad ogni minima modifica).
	}\label{Change Significance}

	\parola{CI/CD}{%
		Sigla con cui ci si riferisce a \gloss{Continuous Integration} e \gloss{Continuous Delivery} contemporaneamente.
    }\label{CI/CD}

	\parola{Ciclo di vita}{%
		Insieme delle attività per la realizzazione di un software. Questo parte dal momento in cui una persona lo pensa (conception) fino al suo ritiro, che	avviene dopo anche la fase di manutenzione. La sua struttura deve essere conosciuta preventivamente per valutare i costi e le risorse per lo sviluppo del software. Il ciclo di vita può perciò essere visto come un automa a stati finiti.
		
		Esistono diversi cicli di vita che vengono scelti in base al tipo di software da creare:
		
		\begin{itemize}
			\item \textbf{Cascata/sequenziale}
			\item \textbf{Iterativo}
			\item \textbf{Incrementale}
			\item \textbf{Evolutivo}
			\item \textbf{Componenti}
			\item \textbf{Agile}
			\item \textbf{Scrum}
			\item \textbf{Spirale}
			\item \textbf{SEMAT}
		\end{itemize}
	}\label{Ciclo di vita}

	\parola{CMMI}{%
		Acronimo di \textit{Capability Maturity Model Integration}, si tratta di uno standard per ottenere una valutazione sulla qualità dai fornitori in modo uniforme.

		Questo standard fa leva sul determinare la capability di un processo, intesa come la misura della capacità e dell'efficienza
		per un processo di ottenere risultati, e maturity, intesa quanto l'azienda sia capace di organizzare e governare i suoi processi.

		Il CMMI prevede cinque livelli di maturità a cui si associa ogni processo, ed essi sono:

		\begin{enumerate}
			\item \textbf{Initial}
			\item \textbf{Managed}
			\item \textbf{Defined}
			\item\textbf{Quantitatively managed}
			\item \textbf{Optimizing}
		\end{enumerate}
	}\label{CMMI}

    \parola{Consumer}{%
    	Componenti che avranno il compito di abbonarsi ai Topic adeguati, recuperandone i messaggi.
	}\label{Consumer}

	\parola{Continuous Delivery}{%
		Passo successivo alla CI (Continuous Integration) in cui ogni cambiamento può potenzialmente essere rilasciato in produzione,
		in modo semplice e immediato.
		Serve per migliorare la cooperazione e la comunicazione tra developer, operational e tester.
	}\label{Continuous Delivery}

	\parola{Continuous Deployment}{%
		Metodologia di sviluppo del software che prevede il frequente rilascio del prodotto funzionante preferibilmente in una repository specifica. 
	}\label{Continuous Deployment}

	\parola{Continuous Integration}{%
		Pratica dell'Ingegeria del software in cui l'allineamento dell'ambiente di lavoro degli sviluppatori verso l'ambiente condiviso
		è molto frequente.
		Il codice di un progetto che adotta la CI prevede che vi sia un processo di build e verifica automatico.
	}\label{Continuous Integration}

	\parola{CSS3}{%
        Versione pi\`u moderna del Cascading Style Sheets, linguaggio usato per definire la formattazione dei documenti 
        HTML/XML.
    }\label{CSS3}
