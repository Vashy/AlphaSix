\section{Casi d'uso}
testo

	\subsection{Introduzione}
	testo
	
	\subsection{Attori}
	\begin{itemize}
		\item Producer
		\item Utente che interagisce con il gestore personale
		\item Consumer (secondario)
	\end{itemize}
	
	\subsection{Elenco casi d'uso}
	Messaggio - tecnologia che invia il messaggio - tipo di messaggio di quella tecnologia.
	Classificazioni in base al topic.

\subsubsection{UC 1}
producer genera un messaggio
	\paragraph{UC1.1}
	api gitlab
		\subparagraph{UC1.1.1}
		aggiornamento issue tracking system (apertura issue / chiusura)
		\subparagraph{UC1.1.2}
		analisi tag:
			se non presente {si aggiunge alla lista dei tag} --> estensione / scenario alternativo
		\subparagraph{UC1.1.3}
		cambio stato repository (commit)

	\paragraph{UC1.2}
	api redmine
		\subparagraph{UC1.2.1}
		aggiornamento issue tracking system (apertura issue / chiusura / avviso ritardo)

	\paragraph{UC1.3}
	api sonarqube
		\subparagraph{UC1.3.1}
		esecuzione build

\subsubsection{UC 2}
	Autenticazione dell'utente nel sistema
	\paragraph{UC2.1}
		Autenticazione
			\subparagraph{UC2.1.1}
				Inserimento username
			\subparagraph{UC2.1.2}
				Inserimento password
	\paragraph{UC2.2}
		Visualizzazione errore autenticazione fallita


\subsubsection{UC 3}
configurazione delle preferenze dell'utente nel sistema

	\paragraph{UC3.1}
	aggiunta preferenza
		\subparagraph{UC3.1.1}:
		iscrizione topic
			data la lista di topic presenti l'utente seleziona quelli a cui è interessato ricevere notifica (suddiviso per tecnologia e per tag [BUG, FIX, ISSUE, ecc.] )
		\subparagraph{UC3.1.2}:
		prima selezione dei giorni in calendario:
			dato il calendario lavorativo l'utente selezionerà i giorni in cui sarà assente
		\subparagraph{UC3.1.3}
		piattaforma di messaggistica:
			telegram: aggiunta nickname
			slack: aggiunta nickname
			mail: aggiunta email

	\paragraph{UC3.2}
	rimozione preferenza
		\subparagraph{UC3.2.1}
		disiscrizione topic %si può non essere iscritti a nessun topic?
			data la lista di topic a cui è iscritto l'utente elimina quelli a cui non è più interessato ricevere notifica
		\subparagraph{UC3.2.2}
		rimozione giorno di calendario
			data una lista dei giorni in cui non si è reperibili, vengono rimossi uno o più di questi giorni
		\subparagraph{UC3.2.3}
		rimozione piattaforma di messaggistica %dire che bisogna lasciarne almeno una?
			telegram: modifica nickname
			slack: modifica nickname
			mail: modifica email
			
	\paragraph{UC3.3}
	applico le modifiche fatte
	
	\paragraph{UC3.4}
	Annullo le modifiche fatte
\iffalse %inizio paragrafo commentato
\paragraph{UC3.2}
aggiornamento delle preferenze
\subparagraph{UC1.2.1}
modifica calendario:
dato il calendario lavorativo l'utente modifica i giorni disponibili
\subparagraph{UC1.2.2}
cambio piattaforma di messaggistica
telegram: modifica nickname
slack: modifica nickname
mail: modifica email
\fi
		
		%Possibilità di aggiungere funzionalità specifiche di Gitlab, di Sonar, di Redmine ecc
		
		
		
