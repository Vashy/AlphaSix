\section{Casi d'uso}
testo

	\subsection{Introduzione}
	testo
	
	\subsection{Attori}
	\begin{itemize}
		\item Producer
		\item Broker
		\item Consumer
		\item Utente finale che vede sul gestore personale i report.
		\item Utente specializzato a configurare tutto il nostro SW (sistemista).
	\end{itemize}
	
	\subsection{Elenco casi d'uso}
	Messaggio - tecnologia che invia il messaggio - tipo di messaggio di quella tecnologia.
	Classificazioni in base al topic.
	
		\subsubsection{UC 1}
		 Producer genera un messaggio 
		
			\paragraph{UC 1.1}
			Analisi di chi è il Producer che ha eseguito l'azione.
			
			\subparagraph{UC 1.1.1}
			Se il Producer..
		
		\subsubsection{UC 2}
		Broker che riceve messaggio.
		
		\subsubsection{UC 3}
		Broker invia un messaggio al Consumer.
			
			\paragraph{UC 3.1}
			Reindirizzamento alla persona sostitutiva più adatta (vedere di dare una priorità).
		
		\subsubsection{UC 4}
		Consumer riceve messaggio dal Broker.
		
		\subsubsection{UC 5}
		Utente che va sul gestore personale e vede i report creati.
		
		\subsubsection{UC 6}
		Configurazione del container platform.
		
		%Possibilità di aggiungere funzionalità specifiche di Gitlab, di Sonar, di Redmine ecc
		
		
		