\section{Casi d'uso}
testo

	\subsection{Introduzione}
	testo
	
	\subsection{Attori}
	\begin{itemize}
		\item Producer
		\item Utente che interagisce con il gestore personale
		\item Consumer (secondario)
	\end{itemize}
	
	\subsection{Elenco casi d'uso}
	Messaggio - tecnologia che invia il messaggio - tipo di messaggio di quella tecnologia.
	Classificazioni in base al topic.

\subsubsection{UC 1}
producer genera un messaggio
	\paragraph{UC1.1}
	api gitlab
		\subparagraph{UC1.1.1}
		aggiornamento issue tracking system (apertura issue / chiusura)
		\subparagraph{UC1.1.1}
		analisi tag:
			se non presente {si aggiunge alla lista dei tag} --> estensione / scenario alternativo
		\subparagraph{UC1.1.1}
		cambio stato repository (commit)

	\paragraph{UC1.1}
	api redmine
		\subparagraph{UC1.1.1}
		aggiornamento issue tracking system (apertura issue / chiusura / avviso ritardo)

	\paragraph{UC1.1}
	api sonarqube
		\subparagraph{UC1.1.1}
		esecuzione build


\subsubsection{UC 2}
autenticazione dell'utente nel sistema
	\paragraph{UC1.1}
	inserimento username
	\paragraph{UC1.1}
	inserimento password


\subsubsection{UC 3}
configurazione delle preferenze dell'utente nel sistema

	\paragraph{UC1.1}
	aggiunta preferenza
		\subparagraph{UC1.1}:
		iscrizione topic
			data la lista di topic presenti l'utente seleziona quelli a cui è interessato ricevere notifica (suddiviso per tecnologia e per tag [BUG, FIX, ISSUE, ecc.] )
		\subparagraph{UC1.1}:
		prima selezione dei giorni in calendario:
			dato il calendario lavorativo l'utente selezionerà i giorni in cui sarà assente
		\subparagraph{UC1.1}
		piattaforma di messaggistica:
			telegram: aggiunta nickname
			slack: aggiunta nickname
			mail: aggiunta email
	\paragraph{UC1.1}
	aggiornamento delle preferenze
		\subparagraph{UC1.1}
		modifica calendario:
			dato il calendario lavorativo l'utente modifica i giorni disponibili
		\subparagraph{UC1.1}
		cambio piattaforma di messaggistica
			telegram: modifica nickname
			slack: modifica nickname
			mail: modifica email
	\paragraph{UC1.1}
	rimozione preferenza
		\subparagraph{UC1.1}
		disiscrizione topic
			data la lista di topic a cui è iscritto l'utente elimina quelli a cui non è più interessato ricevere notifica
		\subparagraph{UC1.1}
		rimozione piattaforma di messaggistica
			telegram: modifica nickname
			slack: modifica nickname
			mail: modifica email		
		
		%Possibilità di aggiungere funzionalità specifiche di Gitlab, di Sonar, di Redmine ecc
		
		
		