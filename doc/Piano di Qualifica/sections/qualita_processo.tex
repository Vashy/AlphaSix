\section{Qualità di processo}\label{QualitaProcesso}

\subsection{Scopo}
La qualità di un prodotto è fortemente influenzata dal processo utilizzato nell'arco di creazione del prodotto stesso: da un tubo sporco non può uscire acqua pulita.

Per questo è necessario operare con un buon ciclo di vita per i processi, che devono essere verificati e valutati. Per questo viene seguito lo schema del ciclo di Deming e dell'ISO 15504 descritti all'Appendice §A.

\subsection{Processi}
I processi saranno elencati nel seguente modo:

\begin{center}
	PROC[ID] Nome
\end{center}

L'ID sarà un numero incrementale per indicare in modo univoco il processo e il nome sarà una breve frase per indicare la funzione del processo.

Per ogni processo sono elencate le sue funzioni principali, gli aspetti da misurare per ottenere la qualità desiderata e le metriche utilizzate. Gli obiettivi di qualità elencati, quando è possibile, sono affiancati da una particolare metrica. 

La nomenclatura per le metriche e gli obiettivi è descritta nelle \Doc{\NdP} nella sezione §3.1.3.

	\subsubsection{PROC001 Pianificazione del progetto, organizzazione e struttura}
	Tale macro-processo ha lo scopo di pianificare il lavoro da svolgere per soddisfare i requisiti richiesti.
	
		\paragraph*{Funzioni}
	
		\begin{itemize}
			\item \textbf{Sviluppare sotto-processi}: i vari obiettivi devono poter associati ad azioni ben precise, ognuna delle quali appartiene ad un sotto-processo;
			\item \textbf{Suddividere i compiti}: è compito dell'\Amm assegnare i vari compiti ai vari ruoli del team;
			\item \textbf{Candelarizzare i documenti}: altro compito dell'\Amm e quello di stabilire delle \gloss{baseline} durante il progetto, sarà poi compito dei vari membri del team organizzare i loro impegni per rispettare tali scadenze;
			\item \textbf{Formazione personale}: l'inesperienza del team richiede di un periodo di formazione personale più lungo del normale. Questo periodo deve essere contato dall'\Amm all'interno della calendarizzazione;
			\item \textbf{Standard}: vengono scelti gli standard da seguire;
			\item \textbf{Budget}: è necessario conoscere le proprie risorse in termini di tempo $\frac{Costo}{Persona}$ in modo tale da restare il più fedeli possibile al preventivo stilato;
		\end{itemize}
	
		\paragraph*{Metriche}
		
		\begin{itemize}
			\item \textbf{MPR001 Varianza della programmazione}
			\item \textbf{MPR002 Varianza dei costi}
			\item \textbf{MPR003 Aderenza agli standard}
		\end{itemize}
	
		\paragraph{Obiettivi}
		
		\begin{itemize}
			\item \textbf{QPR001 Rispetto delle fasi dell'organigramma}: all'interno dell'organigramma presente alla sezione (inserire sezione organigramma) del \Doc{\PdP} sono presenti le date di scadenza delle varie attività, l'obiettivo è quello di rispettarle il più possibile;
			\item \textbf{QPR001 Variazione del budget}: le risorse messe a disposizione all'inizio del progetto devono potersi mantenere tali in tutta la sua durata;
			\item \textbf{QPR002 Rispetto delle fasi del ciclo di vita}: ogni processo deve rispettare le fasi del ciclo di Deming;
		\end{itemize}
	
	\subsubsection{PROC002 Analisi}
	
	\subsubsection{PROC003 Produzione documenti}
	
	\subsubsection{PROC003 Verifica}

\subsection{Eventuali tabelle}
Testo