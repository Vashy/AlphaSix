\section{Qualità di prodotto}

\subsection{Scopo}
testo

\subsection{Prodotti}
testo

	\subsubsection{Documenti}
	testo
	
\subsection{Eventuali tabelle}
testo

\subsection{The Twelve-Factor App}
The Twelve-Factor App è presente tra i requisiti dell'\Doc{Analisi dei Requisiti} ed è una serie di dodici regole destinate ad chi vuole sviluppare \gloss{software-as-a-service} (SaaS).

I suoi principi sono:

\begin{enumerate}
	\item \textbf{Codebase}: deve essere presente una sola code base versionata da un \textit{Version Control System} come \gloss{GitLAb} da cui possono derivare diversi deploy;
	\item \textbf{Dipendenze}: le librerie usate dal codice devono essere presenti nella directory della singola applicazione e non attive a livello di sistema. In questo modo l'applicazione è il meno dipendente possibile dal sistema di esecuzione;
	\item \textbf{Configurazione}: i parametri di configurazione dell'applicazione devono essere completamente separati dalla sua implementazione;
	\item \textbf{Backing Service}: l'applicazione non deve far distinzione tra funzionalità uguali usate in locale o remoto;
	\item \textbf{Build, release, esecuzione}: bisogna separare in modo netto la fase di build, quella di deploy e quella esecuzione, usando tool differenti e diverse repository per salvare i risultati delle varie fasi;
	\item \textbf{Processi}: l'esecuzione dell'applicazione deve essere vista come l'insieme di uno o più processi che restituiscono un risultato. Questi sono di tipo \gloss{stateless};
	\item \textbf{Binding delle Porte}: l'applicazione è completamente contenuta in se stessa e non fa affidamento ad un altro servizio nell'ambiente di esecuzione. effettua invece il binding delle porte diventando un servizio per le richieste esterne;
	\item \textbf{Concorrenza}: sviluppare i processi in modo tale che possano lavorare su un sistema decentralizzato;
	\item \textbf{Rilasciabilità}: i processi dell'applicazione devono poter essere avviati e fermati quando se ne ha bisogno senza passaggi bruschi;
	\item \textbf{Parità tra Sviluppo e Produzione}: deve esserci meno differenza possibile tra lo stato di sviluppo e quello di produzione. Questo si ottiene facendo un rilascio contuno del prodotto;
	\item \textbf{Log}: l'applicazione dovrebbe poter offrire un sistema di login;
	\item \textbf{Processi di Amministrazione}: porre attenzione a quei processi che devono essere eseguiti una tantum dagli sviluppatori ad esempio. Questi processi devono poter essere accessibili solo ad alcuni e indicati in una specifica release;
\end{enumerate}

In accordo col cliente \textit{Imola Informatica} e da quanto concerne il capitolato non tutti i punti della Twelve-Factor App possono essere rispettati. Questi sono:

\begin{itemize}
	\item \textbf{11}: dato che l'applicazione verrà eseguita nella rete interna dell'azienda, non sarà necessaria una fase di autenticazione dell'utente;
\end{itemize}