%%%%%%%%%%%%%%%%%%%%%%%%%%%%%%%%%%%%%%%%%
% Thin Formal Letter
% LaTeX Template
% Version 2.0 (7/2/17)
%
% This template has been downloaded from:
% http://www.LaTeXTemplates.com
%
% Original author:
% WikiBooks (http://en.wikibooks.org/wiki/LaTeX/Letters) with modifications by 
% Vel (vel@LaTeXTemplates.com)
%
% License:
% CC BY-NC-SA 3.0 (http://creativecommons.org/licenses/by-nc-sa/3.0/)
%
%%%%%%%%%%%%%%%%%%%%%%%%%%%%%%%%%%%%%%%%%

%----------------------------------------------------------------------------------------
%	DOCUMENT CONFIGURATIONS
%----------------------------------------------------------------------------------------

\documentclass[12pt]{letter} % Copiata dal nostro stile

\usepackage{geometry} % Per modificare margini e dimensioni varie
\usepackage{graphicx} % Per inserire immagini

\geometry{
	paper=a4paper, % Change to letterpaper for US letter
	top=3cm, % Top margin
	bottom=1.5cm, % Bottom margin
	left=2.2cm, % Left margin
	right=2.2cm, % Right margin
	% showframe, % Uncomment to show how the type block is set on the page
}

\usepackage[T1]{fontenc} % Output font encoding for international characters
\usepackage[utf8]{inputenc} % Required for inputting international characters
\usepackage[italian]{babel}
\usepackage{charter} % stix di default, copiato dagli answer credo
\usepackage{scrextend} % indentazione per il corpo della lettera
\usepackage{microtype} % Improve justification
\usepackage{eurosym} % wanna see the muney


% TABELLE
\usepackage{tabularx,booktabs}
\usepackage[table,usenames,dvipsnames]{xcolor}
% Definizione di nuovi colori da poter usare per le tabelle
\definecolor{lightgray}{gray}{0.9}
\definecolor{lightblue}{rgb}{0.93,0.95,1.0}
\definecolor{lightorange}{rgb}{1,0.75,0.35} % orange

% Ridefinizione dell'env tabular. Il vecchio è utilizzabile con l'env oldtabular
\let\oldtabular\tabular
\let\endoldtabular\endtabular
\renewenvironment{tabular}{\rowcolors{2}{white}{lightgray}\oldtabular}{\endoldtabular}

% Ridefinizione dell'env tabular. Il vecchio è utilizzabile con l'env oldtabular
\let\oldtabularx\tabularx
\let\endoldtabularx\endtabularx
\renewenvironment{tabularx}{\rowcolors{2}{white}{lightgray}\oldtabularx}{\endoldtabularx}
\renewcommand{\arraystretch}{1.4}

% D di documento a pedice, con e senza spazio
\newcommand{\DAlt}{\ped{\tiny{D}}}
\newcommand{\D}{\ped{\tiny{D }}}

% e.g. \Doc{Norme di Progetto}
\newcommand{\Doc}[1]{\textit{#1}\DAlt}

% \date{inseirre-la-data} % comment to show today date

\signature{
	Responsabile AlphaSix
	\\\includegraphics[width=5cm]{img/firma_sg.png}
	\\Samuele Gardin
} % firma a piè di pagina

%\address{123 Broadway \\ City, State 12345 \\ (000) 111-1111} % Your address and phone number


\begin{document}

	%----------------------------------------------------------------------------------------
	%	INDIRIZZO
	%----------------------------------------------------------------------------------------

		
	\begin{letter}{%
		Alla cortese attenzione di
		\\Professor Tullio Vardanega
		\\Professor Riccardo Cardin
		\vspace{0.7em}
		\\Università degli Studi di Padova
		\\Dipartimento di Matematica "Tullio Levi-Civita"
		\\Via Trieste, 63
		\\35121 Padova (PD)
	} % Name/title of the addressee		

	\begin{center}
		\includegraphics[width=5cm]{../template/icons/a6.png}
		\\ alpha.six.unipd@gmail.com
	\end{center}

	%----------------------------------------------------------------------------------------
	%	CONTENUTO
	%----------------------------------------------------------------------------------------
	\opening{%
		Egregi professori,
	}

	\begin{addmargin}[2em]{2em}
		\hspace{1cm} % add indetation	
		Con la presente lettera, AlphaSix intende comunicarVi ufficialmente \mbox{l'impegno} preso
		per la realizzazione del prodotto da Voi commissionato, denominato:
		\begin{center}
			\textbf{\textit{Butterfly}: monitor per processi CI/CD}
		\end{center}
		In allegato saranno inclusi i seguenti documenti:
		\begin{itemize}
			\item \Doc{Analisi dei Requisiti v1.0.0}
			\item \Doc{Piano di Qualifica v1.0.0}
			\item \Doc{Piano di Progetto v1.0.0}
			\item \Doc{Norme di Progetto v1.0.0}
			\item \Doc{Glossario v1.0.0}
			\item \Doc{Studio di Fattibilità v1.0.0}
			\item Verbali interni:
			\begin{itemize}
				\item \textit{VI\_22-11-2018}
				\item \textit{VI\_26-11-2018}
				\item \textit{VI\_17-12-2018}
				\item \textit{VI\_07-1-2019}
			\end{itemize}
			\item Verbali esterni:
			\begin{itemize}
				\item \textit{VE\_12-12-2018}
			\end{itemize}
      	\end{itemize}

		Il team di sviluppo stima di consegnare il prodotto entro il 17/05/2019.
		Il costo preventivato è di \mbox{\euro11 890.00},
		come è propriamente riportato nel documento \textit{Piano di Progetto v1.0.0}.

		Si riportano qui i componenti del gruppo AlphaSix:
		\begin{center}
			\begin{tabularx}{0.4\textwidth}{Xc}
				\textbf{Nominativo} & \textbf{Matricola} \\
				\toprule
				Laura Cameran & 1143488 \\
				Nicola Carlesso & 1123257 \\
				Samuele Gardin & 1143807 \\
				Timoty Granziero & 1123442 \\
				Matteo Marchiori & 1143333 \\
				Ciprian Voinea & 1143057 \\
				\bottomrule
			\end{tabularx}
		\end{center}

		\par Siamo a Vostra completa disposizione per eventuali chiarimenti.
	\end{addmargin}

	%\vspace{2\parskip} % Extra whitespace for aesthetics
	\closing{Distinti saluti,}
	%\vspace{2\parskip} % Extra whitespace for aesthetics

	%\ps{P.S. You can find additional information attached to this letter.} % Postscript text, comment this line to remove it

	%\encl{Copyright permission form} % Enclosures with the letter, comment this line to remove it

	%----------------------------------------------------------------------------------------

	\end{letter}
 
\end{document}