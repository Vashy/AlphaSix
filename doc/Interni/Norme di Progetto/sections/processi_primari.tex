\section{Processi primari}\label{PP}

    \subsection{Processo di fornitura}\label{PP:Fornitura}	%istanziare Management Process, Infrastructure Process e Improvement Process

        \subsubsection{Scopo}\label{PP:Fornitura:Scopo}
		La sezione corrente ha lo scopo di riportare le attività principali che impegneremo ad attuare al fine di
		diventare \gloss{fornitori} per la proponente \II.

		\subsubsection{Ricerca delle tecnologie}
		Ognuno di noi approfondirà la propria conoscenza su tecnologie e \gloss{framework} vari, in modo da discutere insieme quali sono
		quelle di interesse comune e di utilità per il progetto. Questo prevede poi l'auto-formazione di quelle scelte in comune accordo
		per acquisirne una buona padronanza.

        \subsubsection{Studio di fattibilità}\label{PP:Fornitura:SdF}
        In quest'attività viene prodotto il documento \Doc{\SdFv} al fine di analizzare ogni \gloss{capitolato} e scegliere quale contratto accettare.
        Nello specifico, il documento in ogni sezione contiene:
        	\begin{itemize}
        		\item \textbf{Descrizione generale}: breve descrizione del capitolato.
        		\item \textbf{Obiettivo finale}: obiettivo da raggiungere.
        		\item \textbf{Tecnologie coinvolte}: elenco delle tecnologie direttamente coinvolte esplicitate nel capitolato.
        		\item \textbf{Valutazione conclusiva}: giudizio finale del team di sviluppo.
        	\end{itemize}

        \subsubsection{Preparazione in vista della revisione}
		In questa attività preparemo tutto il materiale necessario per il buon superamento della revisione: 
		\begin{itemize}
			\item I vari documenti, i verbali e la lettera di presentazione per quanto riguarda la documentazione in ingresso
			\item I diagrammi dei package e il codice necessari per la realizzazione del \gloss{Proof of Concept}
			\item Le diapositive per l'esposizione
		\end{itemize}

    \subsection{Processo di sviluppo}\label{PP:Sviluppo}

		\subsubsection{Scopo}\label{PP:Sviluppo:Scopo}
		Lo sviluppo consiste nell'affrontare le attività volte a produrre il software richiesto dal proponente.
		Per una corretta implementazione di questo \gloss{processo}, è necessario:
		\begin{itemize}
			\item Fissare degli obiettivi di sviluppo
			\item Realizzare un prodotto che sia conforme a:
			\begin{itemize}
				\item \gloss{Requisiti} definiti dal proponente
				\item Test definiti dalle norme di \gloss{qualità}
			\end{itemize}
		\end{itemize}
		Lo standard ISO/IEC 12207:1995\footnote{Riferirsi alla voce ``ISO/IEC 12207'' in \S\ref{rifnorma}} definisce il processo di
		sviluppo quel processo contenente tutte le attività relative al prodotto finale, quali:
		\begin{itemize} % [noitemsep]
			\item Analisi dei requisiti
			\item Progettazione
			\item Codifica
			\item Integrazione ed installazione
		\end{itemize}


        \subsubsection{Analisi dei requisiti}\label{PP:Sviluppo:AdR}
		Gli Analisti si occupano di redigere l'\Doc{\AdRv}, composta come segue:
		\begin{itemize}
			\item Descrizione generale del prodotto
			\item Modellazione concettuale del \gloss{sistema} tramite la definizione dei vari \gloss{casi d'uso}
			\item Classificazione e tracciamento dei requisiti
		\end{itemize}

		\paragraph{Denominazione dei requisiti}\label{PP:Sviluppo:AdR:DenominazioneRequisiti}
		Ogni requisito che è stato individuato durante l'analisi presenta il seguente identificativo univoco:
		\begin{center}
			\texttt{R[Numero][Tipo][Priorità]}
		\end{center}

		\begin{itemize}
			\item \textbf{R}: si riferisce a requisito.
		 	\item \textbf{Numero}: corrisponde ad un numero che cerca di seguire la struttura del documento ed è progressivo. Inizia da 1.
		 	\item \textbf{Tipo}: segnala la tipologia di requisito che può essere di:
		 	\begin{itemize}
		 		\item \textbf{F}: funzionalità, che ha a che vedere con le funzionalità del sistema software.
		 		\item \textbf{Q}: qualità, che riguarda tecniche ad hoc.
		 		\item \textbf{V}: vincolo, proposto da \II.
		 	\end{itemize}
	 		\item \textbf{Priorità}: indica il grado di urgenza per il soddisfacimento di un requisito, come:
	 		\begin{itemize}
	 			\item \textbf{0}: opzionale, di grado basso e solo marginalmente utile.
	 			\item \textbf{1}: desiderabile, di medio livello quindi non strettamente necessario ma che dà valore aggiunto.
	 			\item \textbf{2}: obbligatorio, di grado alto quindi irrinunciabile per il \gloss{committente} e impossibile da tralasciare.
	 		\end{itemize}
		\end{itemize}
	
		Esempio: \texttt{R2Q1} indica il secondo requisito di qualità ed è desiderabile.
	

		\paragraph{Casi d'uso}\label{PP:Sviluppo:AdR:CasiUso}
		% descrivere cos'è?
		Un caso d'uso è una tecnica che identifica i requisiti funzionali descrivendo le interazioni tra il sistema di riferimento e un utente ad esso esterno.\\
		Ogni caso d'uso che si vuol descrivere presenta:
		\begin{itemize}
		 	\item \textbf{Codice}: per l'identificazione del tipo:
		 	\begin{itemize}
		 		\item \texttt{UC[Numero]}: per un caso principale (e.g. per il primo caso d'uso avremo \texttt{UC1}).
		 		\item \texttt{UC[Numero].[Numero]}: per un sotto caso (e.g. per il primo figlio del primo caso d'uso avremo \texttt{UC1.1}).
		 	\end{itemize}
		 	\item \textbf{Titolo}: denominazione del caso d'uso, possibilmente breve.
		 	\item \textbf{Attori primari}: tutti gli \gloss{attori} primari coinvolti.
		 	\item \textbf{Attori secondari}: opzionale, tutti gli attori secondari coinvolti.		 	
		 	\item \textbf{Descrizione}: per spiegare più nel dettaglio le azioni che vengono compiute.
		 	\item \textbf{Precondizione}: per rappresentare lo stato del sistema nell'istante precedente.
		 	\item \textbf{Postcondizione}: per rappresentare lo stato del sistema l'istante successivo.
		 	\item \textbf{Scenario principale}: contenente la serie di azioni da compiere numerate nell'ordine in cui vengono compiute.
		 	\item \textbf{Estensioni}: opzionale, per azioni inerenti a scenari alternativi come d'eccezione o errore.
		\end{itemize}

        %\subsubsection{Progettazione}\label{PP:Sviluppo:Progettazione}	%PIÙ AVANTI
        %Contenuto generico

		%\paragraph{Obiettivi}\label{PP:Sviluppo:Progettazione:Obiettivi}
		%Contenuto
		 
		\subsubsection{Diagrammi UML}\label{PP:Sviluppo:UML} %SOLO di attività e sequenza PIÙ AVANTI

		\paragraph{Diagrammi dei casi d'uso}\label{DiagrammiCasiUso}
		Per i diagrammi dei casi d'uso utilizziamo lo strumento \gloss{Draw.io} e il linguaggio \gloss{UML} 2.0. Essi descrivono la visione di un utente
		esterno al sistema e non danno nessun dettaglio implementativo. I \gloss{componenti} contenuti in questo tipo di diagrammi sono:
		\begin{itemize}
			\item \gloss{Attore}
			\item Casi d'uso
			\item Relazioni, che possono essere di:
			\begin{itemize}
				\item \textbf{Associazione}: è sempre presente e rappresenta una comunicazione diretta dell'attore con il caso d'uso.
				\item \textbf{Inclusione}: è una funzionalità comune che coinvolge più casi d'uso, in cui ogni istanza del primo esegue
					necessariamente il secondo. Tutte le inclusioni vengono sempre eseguite dall'utente.
				\item \textbf{Estensione}: aumenta le funzionalità di un caso d'uso coinvolgendone un altro. L'esecuzione di quest'ultimo interrompe il precedente, ma non è necessariamente detto che tutte le estensioni vengano eseguite.
				\item \textbf{Generalizzazione}: avviene tra casi d'uso o tra attori e rappresenta delle modifiche alle caratteristiche di base.
			\end{itemize}
		\end{itemize}
		Per la specifica di un caso d'uso si fa riferimento al paragrafo \S\ref{PP:Sviluppo:AdR:CasiUso}.

        %\paragraph{Codifica}\label{PP:Sviluppo:Codifica} %PIÙ AVANTI

        \subsubsection{Strumenti di base}\label{PP:Sviluppo:Strumenti}

	    \paragraph{Ambiente di sviluppo}\label{PP:Sviluppo:Strumenti:AmbienteSviluppo}
	    Nei paragrafi successivi vengono riportate le componenti software utilizzate da ogni membro del team per lo sviluppo del progetto.


	    \subparagraph{Sistema operativo}\label{PP:Sviluppo:Strumenti:AmbienteSviluppo:SistemaOperativo}
	    Abbiamo deciso di usare, come sistema operativo, GNU/Linux, in particolare una qualsiasi distribuzione basata su Ubuntu 18.04.
