
\section{Processi di supporto}\label{PS}

	\subsection{Documentazione}\label{PS:Documentazione}


		\subsubsection{Implementazione}\label{PS:Documentazione:Implementazione}

			\paragraph{Template}\label{PS:Documentazione:Implementazione:Template}
			Prima di iniziare a redigere i documenti, è stato creato un \gloss{template} per \LaTeX \ (\S\ref{LaTeX})
			contenente tutte le impostazioni grafiche condivise tra i vari documenti, per sfruttare il riutilizzo
			del codice e semplificare enormemente la manutenzione dei sorgenti.\par
			Nello specifico, è presente un file per ognuna delle seguenti utilità:
			\begin{itemize}
				\item \gloss{Layout} delle pagine
				\item \gloss{Macro} personalizzate volte a semplificare l'utilizzo di strutture o comandi ricorrenti
				\item Codice per la generazione della prima pagina
					(struttura definita in \S\ref{PS:Documentazione:Struttura:Frontespizio})
				\item Diario delle modifiche
			\end{itemize}

			\paragraph{Ciclo di vita dei documenti}\label{PS:Documentazione:Implementazione:CicloVita}
			Durante il suo ciclo di vita, ogni documento potrà trovarsi in una delle seguenti fasi:
			\begin{itemize}
				\item \textbf{Redazione}: fase che inizia con la creazione del documento e dura fino alla sua ultima approvazione.
					Il \Res\ assegna ai \gloss{redattori} le varie sezioni di ogni documento da redigere, i quali aggiorneranno la versione nel diario delle modifiche
					come normato in \S\ref{Versionamento}.
				\item \textbf{Verifica}: il documento entra in questa fase nel momento in cui i redattori hanno terminato la stesura del lavoro loro assegnato
					segnalandolo al \Res, che a sua volta assegnerà ai Verificatori la verifica della qualità del prodotto, secondo quanto riportato nelle norme di verifica.
					Essi potranno approvare il documento oppure notificare il \Res\ su eventuali errori o incongruenze emerse durante la fase di verifica, che provvederà
					a riassegnare il lavoro.
				\item \textbf{Approvazione}: fase che inizia dall'accettazione del documento da parte dei Verificatori nella fase di verifica. Spetta al \Res\
					l'approvazione ufficiale del documento, seguita dal rilascio di una \gloss{major release}.
			\end{itemize}

		\subsubsection{Struttura}\label{PS:Documentazione:Struttura}

			\paragraph{Frontespizio}\label{PS:Documentazione:Struttura:Frontespizio}
			La prima pagina di ogni documento, eccezione fatta per i verbali (in cui saranno presenti informazioni ridotte), sarà caratterizzata da:
			\begin{itemize}
				\item Logo e nome del gruppo
				\item Titolo del documento
				\item Informazioni sul documento:
					\begin{itemize}
						\item Versione documento
						\item Data di creazione e ultima modifica
						\item Nominativo dei Redattori
						\item Nominativo dei Verificatori
						\item Nominativo del \Res
						\item Destinazione d'uso
						\item Destinatari del documento
						\item Contatto del gruppo
					\end{itemize}
				\item Breve descrizione del documento
			\end{itemize}

			\paragraph{Storico delle versioni}\label{PS:Documentazione:Struttura:StoricoVersioni}
			La pagina che segue il frontespizio contiene lo storico delle versioni del documento, in cui ogni aggiunta o modifica significativa ha
			comportato un incremento di versione. Ogni riga contiene, a partire da sinistra:
			\begin{itemize}
				\item Il numero della versione nel formato espresso in \S\ref{Versionamento}
				\item Una breve descrizione delle modifiche apportate
				\item Il ruolo dell'autore che ha apportato la modifica
				\item Il nominativo dell'autore
				\item La data di modifica
			\end{itemize}
			La chiave primaria della tabella è il numero di versione ordinata in senso decrescente, in modo che la versione più vecchia sia
			l'ultima riga della tabella.

			\paragraph{Indice}\label{PS:Documentazione:Struttura:Indice}
			In ogni documento, esclusi i verbali, è presente un indice contenente tutte le sezioni, sottosezioni e paragrafi. I numeri di sezioni, sottosezioni,
			e paragrafi sottostanti saranno separati da un punto (ad esempio, 1.4.1).\par
			Saranno eventualmente presenti un indice delle
			figure e un indice delle tabelle, assenti in caso non ci siano tabelle o figure nel documento.\par
			I valori degli indici partono da 1.

			\paragraph{Contenuto}\label{PS:Documentazione:Struttura:Contenuto}
			La struttura di ogni pagina presenta:
			\begin{itemize}
				\item Intestazione con:
				\begin{itemize}
					\item A sinistra, logo di \emph{\gruppo}
					\item A destra, nome del capitolato e documento corrente
				\end{itemize}
				\item Piè pagina con:
				\begin{itemize}
					\item A sinistra, nome e mail di riferimento del gruppo
					\item A destra, numero della pagina corrente
				\end{itemize}
			\end{itemize}


		\subsubsection{Design}\label{PS:Documentazione:Design}

			\paragraph{Norme tipografiche}\label{PS:Documentazione:Design:NormeT}
			Le norme tipografiche qui di seguito elencate sono state decise in modo che ogni membro di \gruppo\ concorra a mantenere una forma coerente e univoca
			per tutti i documenti redatti.

			\subparagraph{Stile del testo}\label{PS:Documentazione:Design:NormeT:StileTesto}
			\begin{itemize}
				\item \textbf{Corsivo}: solo per i nomi dei documenti citati.
				\item \textbf{Maiuscolo}: la prima lettera per
				\begin{itemize}
					\item tutte le parole appartenenti ai nomi dei documenti tranne gli articoli
					\item i nomi dei ruoli
				\end{itemize}
			\end{itemize}

			\subparagraph{Elenchi puntati}\label{PS:Documentazione:Design:NormeT:ElenchiPuntati}
			\begin{itemize}
				\item \textbf{Simboli di livello}: un pallino nero per il primo livello, un trattino per il secondo livello.
				\item \textbf{Punteggiatura}: nessuna punteggiatura alla fine di una frase, tranne nel caso in cui sia presente una descrizione.
					In quel caso la descrizione è preceduta dai due punti ``:'' e termina con un punto ``.''.
				\item \textbf{Grassetto}: solo se è presente una descrizione, allora sono in grassetto tutte le parole prima dei due punti ``:''.
			\end{itemize}

			\subparagraph{Altri formati testuali comuni} \label{PS:Documentazione:Design:NormeT:AltriFormati}
			\begin{itemize}
				\item \textbf{Orari}: \texttt{HH:MM} secondo la norma ISO 8601 nel formato 24 ore dove:
				\begin{itemize}
					\item \texttt{HH} indica le ore, da 00 a 23
					\item \texttt{MM} i minuti, da 00 a 59
				\end{itemize}
				\item \textbf{Date}: \texttt{DD-MM-YYYY} formato adottato in Europa dove:
				\begin{itemize}
					\item \texttt{DD} indica il numero del giorno, da 01 a 31
					\item \texttt{MM} il mese, da 01 a 12
					\item \texttt{YYYY} l'anno
				\end{itemize}
				\item \textbf{Nota a piè pagina}: serve ad inserire elementi aggiuntivi, come osservazioni o riferimenti a parti interne al documento,
				che sono utili alla comprensione del testo, ma se inseriti all'interno del discorso ne interromperebbero la lettura rendendola meno scorrevole.
			\end{itemize}


			\paragraph{Elementi grafici}

			\subparagraph{Figure}
			Ogni immagine inserita nei documenti deve sempre essere centrata rispetto al foglio e adeguatamente separata dal testo. Deve inoltre essere
			accompagnata da una breve \gloss{caption} che permetta al lettore di capire esattamente che cosa sta guardando. È presente nell'indice l'elenco
			delle figure che raccoglie la lista di tutte le immagini presenti.

			\subparagraph{Tabelle}
			Come per le figure, ogni tabella sarà accompagnata da una caption e sarà della dimensione del testo, o se più piccola, centrata.
			Tutte le tabelle saranno raccolte nell'elenco delle tabelle.\par
			Saranno presenti due tipologie di tabelle:
			\begin{itemize}
				\item \textbf{Semplici}: tabelle standard senza uno stile particolare, in cui le celle sono separate da bordi neri (evitare, ove non risulta necessario,
					le righe verticali).
				\item \textbf{Complesse}: tabelle con un'alternanza di colori tra le righe delle celle (grigio e bianco) e senza bordi verticali.
					Le celle sono separate orizzontalmente da una corretta spaziatura e allineamento e verticalmente dall'alternanza dei due colori.
					La riga dell'\gloss{header} può essere bianca o di un grigio più scuro in base al contesto, con il testo che può essere in grassetto.
			\end{itemize}


		\subsubsection{Produzione}

			\paragraph{Suddivisione dei documenti}

			\subparagraph{Documenti interni}
			Sono considerati interni documenti quali \Doc{\SdF} e \Doc{\NdP} che non sono visibili ad entità esterne,
			ma solo ad \gruppo.

			\subparagraph{Documenti esterni}
			Questi documenti sono ufficiali e approvati direttamente dal \Res\ e comprendono, per esempio: \Doc{\PdP},
			\Doc{\PdQ} e \Doc{\AdR}. Si chiamano esterni perché accessibili al committente.

			%\subparagraph{Glossario} 
			% TODO: già scritto nell'introduzione, eliminare? L: anche sì
			%Il documento denominato \Doc{\Gl}  raccoglie in ordine alfabetico tutti i termini utilizzati che necessitano di una spiegazione più approfondita.
			%Sono identificabili all'interno degli altri documenti da un font differente e una G a pedice la prima volta che appaiono.


			\subparagraph{Verbali}
			Questi documenti vengono redatti quando \gruppo\ tiene delle riunioni o ci sono incontri esterni, per esempio con \II. Vengono redatti da una singola persona
			e presentano tutti le stesse sezioni:
			\begin{itemize}
				\item \textbf{Informazioni incontro}: lista delle informazioni principali riguardanti la riunione quali luogo, data, orario, ordine del giorno, ecc\dots
				\item \textbf{Argomenti}: lista dei principali argomenti trattati con descrizione di cosa si è discusso nel dettaglio.
				%TODO: da riempire se viene aggiunto altro
			\end{itemize}


			\paragraph{Strumenti di supporto}

			\subparagraph{\LaTeX} \label{LaTeX}
			Per la stesura della documentazione è stato scelto di usare il linguaggio di markup \LaTeX \ perché presenta molti vantaggi, tra i quali:
			\begin{itemize}
				\item Supporta nativamente il versionamento, essendo un linguaggio compilato
				\item Supporta la modularità, rendendo più facile organizzare un documento dividendone logicamente i vari moduli
				\item Permette il riutilizzo del codice tramite l'uso di macro già pronte o personalizzate, oppure includendo lo stesso sorgente in punti diversi
					(ad esempio come fatto in \S\ref{PS:Documentazione:Implementazione:Template})
				\item Gestisce automaticamente indici e riferimenti
			\end{itemize}

			\subparagraph{TexStudio/Visual Studio Code}
			TexStudio e Visual Studio Code sono i due ambienti di sviluppo scelti dal team per stilare la documentazione.
			TexStudio è un \gloss{IDE} nativo per l'utilizzo di \LaTeX. Visual Studio Code è un editor intelligente moderno (praticamente al pari di un IDE) che, tramite
			estensioni, permette il supporto di praticamente ogni linguaggio, tra cui \LaTeX.
			Entrambi permettono una rapida compilazione e un'istantanea visualizzazione dell'anteprima del PDF prodotto, oltre agli altri vantaggi che ogni IDE offre,
			tra cui: suggerimenti e completamenti automatici delle parole chiave, ricerca intelligente (eventualmente tramite \gloss{regexp}) controllo ortografico della
			lingua italiana o inglese e così via. Più informazioni sono reperibili sui rispettivi siti ufficiali, i cui link sono presenti in \S\ref{rifinfo}.
			% \subparagraph{Visual Studio Code}

			\subparagraph{GanttProject} % TODO: Rivedere?
			GanttProject è un programma gratuito dedito alla formazione dei \gloss{diagrammi di Gantt}. Permette di creare task e milestone, organizzare le task
			in lavoro strutturato a interruzioni, disegnare i vincoli di dipendenza tra di esse e molte altre utilità, generando automaticamente il relativo diagramma.
			Per maggiori informazioni, si rimanda alla fonte ufficiale (consultare \S\ref{rifinfo}).

			\subparagraph{Draw.io}
			Draw.io è un'applicazione web in grado di creare diagrammi UML, di Entità-Relazionale, di flusso e molto altro. Il motivo che ha portato AlphaSix a scegliere
			questo strumento è la sua perfetta integrazione con \gloss{Google Drive}, oltre al suo ottimo livello di intuitività. Questo vantaggio permette
			di poter condividere i diagrammi creati tra tutti i collaboratori in ogni momento e in modo automatico.
			Per maggiori informazioni, visualizzare la fonte ufficiale (\S\ref{rifinfo}).

			\subparagraph{Indice di Gulpease}
			Per il calcolo dell'\gloss{Indice di Gulpease}, è stato creato uno script ad hoc che, preso in input un file PDF, produce in output l'indice Gulpease.
			È stato scelto un indice con un valore che sia compreso tra 50 e 60 per i documenti che seguiranno le norme qui definite.

		\subsubsection{Mantenimento}

			\paragraph{Versionamento} \label{Versionamento}
			Tutti i documenti redatti supporteranno il versionamento, in modo da essere univoci e rendere disponibile la possibilità di consultare versioni precedenti
			in qualsiasi fase del loro ciclo di vita.
			Il modello di versionamento adottato segue lo schema \gloss{change significance}. La versione di un file è espressa secondo la notazione
			\begin{center}
				\texttt{vX.Y.Z}
			\end{center}
			\indent dove:
			\begin{itemize}
				\item \texttt{X} indica il numero di versione principale. Inizia da 0 e viene incrementato ogni volta che il \Res\ approva il documento, determinando
					una major release.
				\item \texttt{Y} indica il numero di versione secondario, contatore delle fasi di verifica effettuate dal \Ver\ superate positivamente. Inizia da 0.
					Viene riportato a zero ad ogni incremento della \texttt{X}.
				\item \texttt{Z} è l'indice di modifica minore, incrementato ogni volta che viene effettuato un aggiornamento inferiore, quale l'aggiunta di una sezione
					o correzioni grammaticali/sintattiche di un certo peso. Viene azzerato ad ogni incremento della \texttt{Y} o della \texttt{X}.
			\end{itemize}

			\texttt{X}, \texttt{Y} e \texttt{Z} hanno dominio $[0,+\infty)$, possono assumere pertanto un valore $> 9$.

			\paragraph{Continuous Integration}
			Per quanto riguarda la stesura dei documenti, verrà adottato il principio di \gloss{continuous integration}, che sarà tuttavia limitato in questo periodo
			a sincronizzarsi il prima possibile con il repository remoto (non essendoci
			una vera e propria build o dei test da effettuare), sia per quanto riguarda il \gloss{fetch} che per quanto riguarda il \gloss{push}.

			Questo serve a rendere più remota possibile la probabilità di incappare nell'\gloss{integration hell}.


			\paragraph{Nomenclatura}

			\subparagraph{Verbali}	\label{NomenclaturaVerbali}
			I Verbali  possono essere interni oppure esterni, nel caso in cui il team incontri gli esponenti di \II.
			Il nominativo del file in cui sono formalizzati è il seguente:
			\begin{itemize}
				\item \texttt{VI\_dd-mm-yyyy.pdf} per i verbali interni
				\item \texttt{VE\_dd-mm-yyyy.pdf} per i verbali esterni
			\end{itemize}
			dove dd-mm-yyyy è la data in cui sono stati tenuti, nel formato descritto nel paragrafo \S\ref{PS:Documentazione:Design:NormeT:AltriFormati}.

			\subparagraph{Documenti vari}
			Saranno presenti due tipologie di file: file interni al team e file esterni.
			\begin{itemize}
				\item La prima categoria include moduli di \LaTeX\ contenenti le varie sezioni, che non verranno mai esposti esternamente. Questi file verranno
					denominati usando la convenzione \texttt{snake\_case.tex}, dove snake\_case è il nome della sezione o modulo.
				\item La seconda categoria include i file \texttt{.tex} principali che produrranno i PDF da consegnare al committente. Essi verranno denominati
					con la convenzione \texttt{CamelCase\_vX.Y.X.tex} / \texttt{CamelCase\_vX.Y.Z.pdf}, dove CamelCase sarà il nome del documento generico mentre
					\texttt{vX.Y.Z} sarà la versione che identifica univocamente il documento come descritto in \S\ref{Versionamento}.
			\end{itemize}

	\subsection{Verifica}

		\subsubsection{Scopo}
		Questa sezione vuole descrivere come il team di sviluppo esegue la fase di verifica, per capire se i prodotti e i processi sono conformi a quanto ci si attende.
		La fase di verifica serve per stabilire se è opportuno o meno procedere alla fase successiva del progetto, se ciò non fosse possibile sarà necessario il
		ritorno ad una fase stabile del progetto per poi da lì ripartire prendendo in considerazione i risultati della precedente verifica.

		%\subsubsection{Aspettative} %da mettere nel PdQ

		\subsubsection{Descrizione}
		Ogni processo e prodotto deve essere valutato in modo quantificabile attraverso metriche apposite, quando possibile, e stabilendo il risultato che si vuole raggiungere.

		Come indicato dal \textit{Ciclo di Deming} nel \Doc{\PdQ} all'appendice \S A, nel momento in cui tale risultato sarà raggiunto, se esso non è il migliore,
		servirà come ``base'' per alzare il livello di qualità di quel processo o prodotto.

		I risultati ottenuti nella fase di verifica sono riportati all'appendice \S B del \Doc{\PdQ}, in questo modo, confrontandoli con gli esiti attesi,
		è possibile valutare un miglioramento per i vari processi e prodotti che viene riportato nell'appendice §C sempre del \PdQ.

		\subsubsection{Metriche}
		La denominazione delle metriche è già stata descritta in \S\ref{Classificazione metriche}, qui viene descritto il loro funzionamento.
			\paragraph{Metriche per i documenti}
				\subparagraph{MPD001 Indice Gulpease}
				Si tratta di un indice di leggibilità dei documenti in lingua italiana. A differenza degli indici per le altre lingue, questo si basa sulla
				lunghezza delle parole in lettere e non in sillabe. L'indice si calcola:

				\[89+\dfrac{300\times n_{\text{frasi}}-10\times n_{\text{lettere}}}{n_{\text{parole}}}\]

				\textbf{Metrica}: il risultato della formula è interpretato nel seguente modo

				\begin{itemize}
					\item \textbf{<80}: documento  difficile da leggere per chi ha la licenza elementare;
					\item \textbf{<60}: documento  difficile da leggere per chi ha la licenza media;
					\item \textbf{<40}: documento difficile da leggere per chi ha un diploma superiore;
				\end{itemize}

				Nel momento in cui avviene un commit all'interno di repository, in automatico si avvia uno script che analizza tutti i documenti in pdf
				per valutarne l'indice Gulpease. I risultati vengono poi riportati in un apposito file di testo per verificarne la qualità ed un possibile miglioramento.

				\subparagraph{MPD002 Correttezza ortografica}
				Gli errori ortografici possono essere segnalati dallo strumento di Controllo Ortografico presente in \textit{TexStudio}.

				\textbf{Metrica}: il numero di errori ortografici presenti nel documento.

			\paragraph{Metriche per i processi}
				\subparagraph{MPR001 Varianza della programmazione} % TODO: (sezione con l'organigramma) -> ?
				Nella sezione (sezione con l'organigramma) nel \Doc{\PdP} con l'organigramma sono stabilite le \gloss{baseline} e le scadenze di consegna dei vari prodotti.
				Nonostante il tempo di \gloss{slack} che ogni fase possiede, è possibile che delle date non vengano rispettate causa incidenti di vario tipo che sono
				analizzati nella sezione (sezione dell'analisi dei rischi) del \Doc{\PdP}.

				\textbf{Metrica}: vengono contati i giorni di ritardo nel rispettare una scadenza.

				\subparagraph{MPR002 Varianza dei costi} % TODO: (sezione preventivo) -> ?
				All'interno della sezione (sezione preventivo) nel \Doc{\PdPv} è indicato il costo approssimativo del progetto.
				In corso d'opera possono presentarsi dei problemi che richiedano un aggiunta di costo in termini di $\frac{\text{tempo}}{\text{persona}}$.
				Lo scopo del preventivo infatti è quello di fare una stima non definitiva dei costi. Per verificare se il preventivo viene
				rispettato viene fatto periodicamente un consuntivo di periodo.

				\textbf{Metrica}: viene misurata la differenza conteggiata in \euro\ tra il costo finale e il preventivo.
				Viene seguito tale schema per la tariffa oraria dei vari ruoli del team di sviluppo:

				\begin{itemize} % TODO: tabellina?
					\item \textbf{Responsabile}: \euro\ 30
					\item \textbf{Amministratore}: \euro\ 20
					\item \textbf{Analista}: \euro\ 25
					\item \textbf{Progettista}: \euro\ 22
					\item \textbf{Programmatore}: \euro\ 15
					\item \textbf{Verificatore}: \euro\ 15
				\end{itemize}

				\subparagraph{MPR003 Aderenza agli standard}
				Per misurare e verificare i processi sono stati scelti degli standard di qualità descritti nel \Doc{\PdQv} che possono offrire una valutazione quantitativa.

				Gli standard scelti sono:

				\begin{itemize}
				\item \textbf{ISO/IEC 15504}: ogni processo attivato verrà classificato e valutato secondo gli attributi assegnati ai vari livelli di qualità.
					Per ogni attributo verrà infine indicata una percentuale di quanto il processo rispetti l'attributo, potendo infine capire nel complesso quanto
					quel processo riesca a superare un dato livello di maturità;
				\item \textbf{Ciclo di Deming}: nella fase migliorativa del processo sarà data particolare attenzione nel non iniziare una fase del ciclo di Deming
					senza aver finito completamente le fasi precedenti;
				\end{itemize}

				\textbf{Metrica}: il livello di maturità descritto nell'ISO/IEC 15504 alla appendice §A del \Doc{\PdQv}.

				\subparagraph{MPR004 Frequenza commit nella repository}
				Per mantenere aggiornate le versioni dei prodotti è necessario che ogni membro del team effettui un \gloss{commit} ad ogni sua modifica significativa.
				Così facendo, in caso di errori, è possibile tornare ad una versione stabile del progetto.

				\textbf{Metrica}: numero minimo di commit effettuati in un giorno lavorativo dall'intero team di sviluppo.


				\subparagraph*{Metriche sui requisiti e i rischi}
				Dato che non è fisso il numero dei requisiti di un progetto, sono state scelte una serie di metriche dove il valore ottimale da raggiungere è sempre uguale,
				lo zero. È stato scelto di contare il numero di requisiti non soddisfatti invece che il contrario. Lo stesso ragionamento è valido per quanto
				riguarda i rischi che possono verificarsi nel corso del progetto.

				Gli obiettivi che si vogliono raggiungere attraverso tali metriche possono essere stabiliti solo a progetto concluso.

				\subparagraph{MPR005 Requisiti obbligatori non soddisfatti}
				Per adempire completamente alla richiesta del cliente, serve individuare tutti i requisiti presenti nella sua richiesta, impliciti, espliciti, diretti e
				derivati. Alcuni sono imprescindibili, detti obbligatori, e il loro soddisfacimento determina la buona riuscita del progetto.

				\textbf{Metrica}: numero dei requisiti obbligatori non soddisfatti.

				\subparagraph{MPR006 Requisiti desiderabili non soddisfatti}
				I requisiti desiderabili non sono necessari, ma offrono un valore aggiunto al progetto.

				\textbf{Metrica}: numero dei requisiti desiderabili non soddisfatti.

				\subparagraph{MPR007 Requisiti opzionali non soddisfatti}
				Tali requisiti devono essere adempiuti sono nel momento in cui tutti i requisiti obbligatori saranno soddisfatti.
				Possono essere concordati col cliente in corso d'opera.

				\textbf{Metrica}: numero dei requisiti opzionali non soddisfatti.

				\subparagraph{MPR008 Rischi non previsti avvenuti} % TODO: Rivedere l'ultima frase
				Nell'analisi dei rischi presente nella sezione §2 del \Doc{\PdPv}, sono presenti i rischi ritenuti possibili per i quali è proposta una soluzione.
				Possono presentarsi anche rischi non previsti in tale analisi. Questi devono essere il meno possibile perché la loro soluzione sarà decisa al momento causando ritardi all'interno della programmazione.

				\textbf{Metrica}: numero di rischi non previsti avvenuti nel corso dell'intero progetto.

				\paragraph{MPR009 Frequenza controllo prodotti}
				I documenti e i prodotti software hanno bisogno di una verifica frequente, commisurata in base al numero di modifiche che vengono apportate.

				Chi esegue la modifica deve controllare ciò che ha fatto prima di poter ufficializzare, mentre la verifica fatta dal \Ver\ deve essere fattasolo nel momento in cui è
				stato raggiunto un numero significativo di modifiche, per evitare di spendere troppe risorse in questa fase.

				\textbf{Metrica}: numero massimo di modifiche apportate ai prodotti prima che vengono effettuate senza una verifica dal parte del \Ver.

			\paragraph{Analisi statica}
			L'analisi statica verifica solo i prodotti che non sono o non possono essere eseguiti, come i documenti.

				\subparagraph{Analisi dei documenti}
				L'analisi statica per i documenti si limita a valutare come e con che contenuti questi vengono scritti.

				Per l'analisi dei documenti, vengono utilizzate:

				\begin{itemize}
					\item \textbf{MPD001}
					\item \textbf{MPD002}
				\end{itemize}

			\paragraph{Analisi dinamica}
			L'analisi dinamica valuta il comportamento dei prodotti in esecuzione, verificando se restituiscono i risultati attesi e se operano nel modo stabilito.

			Vengono presi in esame i prodotti software e i processi.

				\subparagraph{Analisi dei processi}
				Per analizzare i processi vengono usati gli standard sopra elencati. Ad ogni fase del processo verranno valutati gli attributi richiesti secondo l'ISO 15504
				e in che misura questi sono stati rispettati, e in che fase del Ciclo di Deming il processo si trova.

				Per l'analisi dei processi vengono utilizzate:

				\begin{itemize}
					\item \textbf{MPR001 Varianza della programmazione}
					\item \textbf{MPR002 Varianza dei costi}
					\item \textbf{MPR004 Frequenza commit nella repository}
					\item \textbf{MPR009 Frequenza controllo prodotti}
					% \item \textbf{MPR010 Norme di progetto non seguite}
				\end{itemize}


				%\paragraph{...Tutti i test}	%UTILE??

				%\paragraph{Strumenti}

				%\subparagraph{Verifica di documentazione}
				%Google Docs?

				%\subparagraph{Integrazione continua} % Jenkins? Più avanti?
