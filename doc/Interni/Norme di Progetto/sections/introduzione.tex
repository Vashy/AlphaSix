\section{Introduzione}

    \subsection{Premessa}
    Il \gloss{documento} che segue verrà prodotto incrementalmente al presentarsi della necessità di redigere nuove \gloss{norme}.
    Per questo motivo, non è da considerare al pari di un documento completo (e.g. la parte relativa alla codifica non ci sarà fino
    al presentarsi di quella necessità).

    \subsection{Scopo del documento}
    Il presente documento ha l’obiettivo di mettere in chiaro le norme, le convenzioni e le tecnologie che verranno adottate da \gruppo\ durante lo svolgimento del \gloss{progetto}. Ogni membro del team \`e tenuto ad osservarlo rigorosamente, per mantenere consistenza ed omogeneit\`a in ogni aspetto, durante tutta la durata del progetto.\par
    %Questo documento verr\`a redatto incrementalmente, in base alle esigenze che verranno incontrate durante lo sviluppo del progetto stesso.

    \subsection{Scopo del prodotto}

%%| Ex Norme di Progetto |%%
% Il prodotto che \gruppo\ si incarica di realizzare è Butterfly: un \gloss{tool} di supporto alle figure di	sviluppo di aziende di software
% (non solamente quella committente). Questo applicativo permette di incanalare le notifiche dei vari strumenti utilizzati nel percorso di
% \gloss{CI/CD} (come \gloss{Redmine}, \gloss{GitLab}, ecc.) di un software e, tramite un \gloss{Broker} (\gloss{Apache Kafka} in questo caso),
% spedirli alla persona interessata tramite canale di comunicazione preferito scelto da quest’ultimo (email, \gloss{Telegram}, \gloss{Slack}, ecc).

% \vspace{1cm}

%%| Ex Analisi dei Requisiti |%%
Lo scopo del \gloss{prodotto} è creare un \gloss{applicativo} per poter gestire i messaggi o le segnalazioni provenienti da diversi prodotti per la realizzazione di software,
come \gloss{Redmine}, \gloss{GitLab} e opzionalmente \gloss{SonarQube}, attraverso un \gloss{Broker} che possa incanalare questi messaggi e distribuirli a strumenti come
\gloss{Telegram}, e-mail e opzionalmente \gloss{Slack}.\par
Il software dovrà inoltre essere in grado di riconoscere il \gloss{Topic} dei messaggi in input per poterli inviare in determinati canali a cui i
destinatari dovranno iscriversi.\par
\`E anche richiesto di creare un canale specifico per gestire le particolari esigenze dell'azienda. Dovrà essere in grado, attraverso la lettura di
particolari	\gloss{metadati}, di reindirizzare i messaggi ricevuti al destinatario più appropriato.

% \vspace{1cm}

%%| Ex Piano di Qualifica |%%
% Il prodotto finale consiste in uno strumento in grado di ricevere messaggi o segnalazioni da vari tipi di servizi per la produzione software chiamati
% \gloss{producer} (e.g. \gloss{GitLab}, \gloss{Redmine} e \gloss{SonarQube}), per poterli poi incanalare verso altri servizi chiamati \gloss{Consumer}
% atti a notificare gli sviluppatori (e.g. \gloss{Slack}, \gloss{Telegram} e Email).\par    
% L'applicazione sarà inoltre capace di organizzare le segnalazioni suddividendole per topic a cui i vari utenti dovranno iscriversi per esserne notificati.
% Nel caso in cui il destinatario dovesse segnalare di non essere disponibile, l'applicativo deve reindirizzare il messaggio verso la persona di competenza
% più prossima. 

% \vspace{1cm}

%%| Ex Piano di Progetto |%%
% Il prodotto che \gruppo\ si incarica di realizzare è Butterfly: un tool di supporto alle figure di sviluppo in aziende che producono software (non
% solamente quella del committente).
% Questo applicativo permette di incanalare le notifiche dei vari strumenti utilizzati nel percorso di \gloss{CI} e \gloss{CD} (come Redmine,
% GitLab, ecc.) di un software e, tramite un \gloss{broker} (\gloss{Apache Kafka} in questo caso), spedirli alla persona interessata tramite
% il canale di comunicazione preferito scelto da quest'ultimo (email, Telegram, Slack, ecc.).


    \subsection{Glossario e documenti esterni}
Al fine di rendere il documento più chiaro possibile, i termini che possono assumere un significato ambiguo o i riferimenti a documenti esterni
avranno delle diciture convenzionali:

\begin{itemize}
    \item \textbf{D}: indica che il termine si riferisce al titolo di un particolare documento (ad esempio \Doc{\PdPv});
    \item \textbf{G}: indica che il termine si riferisce ad una voce riportata nel \Doc{\Glv} (ad esempio \gloss{Redmine}).
\end{itemize}


\subsection{Riferimenti}

    \subsubsection{Normativi}	\label{rifnorma}
    \begin{itemize}
    	\item ISO 8601: \url{https://it.wikipedia.org/wiki/ISO\_8601#Orari}
    	\item ISO/IEC 12207: \url{https://en.wikipedia.org/wiki/ISO/IEC_12207}
    	\item Formato della data: \url{https://it.wikipedia.org/wiki/Formato\_della\_data}
    \end{itemize}

    \subsubsection{Informativi}	\label{rifinfo}
    \begin{itemize}
        \item Descrizione dei ruoli di progetto: \\\url{https://www.math.unipd.it/~tullio/IS-1/2018/Progetto/RO.html}
        \item Visual Studio Code: \url{https://code.visualstudio.com/docs}
        \item TexStudio: \url{https://www.texstudio.org/}
        \item GanttProject: \url{https://www.ganttproject.biz/}
	\end{itemize}

	\paragraph*{Link delle documentazioni, utili alla formazione}
	\begin{itemize}
		\item \LaTeX\ : \url{https://www.latex-project.org/help/documentation/}
		\item Redmine : \url{https://www.redmine.org/guide}
		\item GitLab : \url{https://docs.gitlab.com/}
		% \item SonarQube : \url{https://docs.sonarqube.org/latest/}
		\item Apache Kafka : \url{https://kafka.apache.org/documentation/}
		\item \gloss{Telegram} : \url{https://core.telegram.org/}
		\item Slack : \url{https://api.slack.com/}
		\item Docker : \url{https://docs.docker.com/}
		\item \gloss{The Twelve-Factor App} : \url{https://12factor.net/it/}
	\end{itemize}
