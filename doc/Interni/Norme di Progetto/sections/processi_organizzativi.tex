\section{Processi organizzativi}

    \subsection{Gestione del progetto}\label{GestioneProgetto}

	    \subsubsection{Scopo}
	    L'attività di gestione del progetto consiste nel delineare:
	    \begin{itemize}
	    	\item I \gloss{processi} di progetto
	    	\item Le \gloss{risorse} a essi necessarie
	    	\item I costi per la loro esecuzione
	    	\item I \gloss{task} di risorsa umana
	    	\item La verifica delle attività di processo
	    \end{itemize}

        \subsubsection{Pianificazione della qualità}\label{PianificazioneQualità}
        %Per cercare la qualità nei prodotti è necessario che vengano misurate diverse caratteristiche di questi ultimi.
        Per la ricerca della qualità nei prodotti è necessario che ne vengano misurate diverse caratteristiche associate.
        Per ogni caratteristica devono essere fissati un livello minimo e un livello ottimale che questa deve avere.
        In tale sezione, viene descritto il modo di utilizzo sia delle metriche che degli strumenti utili ad ottenere qualità nei processi, nei documenti e nei prodotti.
        Gli strumenti utilizzati possono essere software, descritti nelle \Doc{\NdPv}, o standard di qualità, descritti nel \Doc{\PdQv}.\\
        Gli obiettivi che ci si pone di raggiungere sono descritti nel \Doc{\PdQv}.

        \paragraph{Classificazione dei processi}
        I processi analizzati nel \Doc{\PdQv} sono classificati nel seguente modo:

        \begin{center}
            \texttt{PROC[ID] [Nome]}
        \end{center}

        \begin{itemize}
            \item \textbf{ID}: un numero incrementale per indicare in modo univoco il processo.
            \item \textbf{Nome}: una breve frase per indicare la funzione del processo.
        \end{itemize}

        \paragraph{Classificazione degli obiettivi}
        Gli obiettivi per la qualità (Q) concordati dall'\Amm\ e dal \Ver, descritti nel \PdQ, avranno il seguente codice identificativo:

        \begin{center}
            \texttt{Q[Tipo][ID] [Nome]}
        \end{center}

        \begin{itemize}
            \item \textbf{Tipo}: indica il la tipologia dell'oggetto di cui viene valutata la qualità, e questa può essere:
            \begin{itemize}
                \item \texttt{PR}: processi.
                \item \texttt{PD}: prodotti di documentazione.
                \item \texttt{PS}: prodotti software.
            \end{itemize}

            \item \textbf{ID}: ogni tipo di obiettivo possiede una lista ordinata attraverso un numero incrementale di tre cifre.
            \item \textbf{Nome}: offre un'informazione più chiara dell'obiettivo attraverso una breve frase.
        \end{itemize}

        Ad esempio:

        \begin{itemize}
            \item QPD001 Leggibilità del testo
        \end{itemize}


        \paragraph{Classificazione delle metriche}\label{Classificazione metriche}
        Ogni obiettivo della qualità deve, per quanto possibile, essere collegato ad una metrica, anch'essa scelta dall'\Amm\ e dal \Ver. Questo per valutare
        quantitativamente il raggiungimento o meno degli obiettivi stabiliti.

        Le metriche (M) verranno classificate nel seguente modo:

        \begin{center}
            \texttt{M[Tipo][ID] [Nome]}
        \end{center}

        \begin{itemize}
            \item \textbf{Tipo}: indica la tipologia di quello su cui viene applicata la metrica, può essere:
            \begin{itemize}
                \item \texttt{PR}: processi.
                \item \texttt{PD}: prodotti di documentazione.
                \item \texttt{PS}: prodotti software.
            \end{itemize}

            \item \textbf{ID}: ogni tipo di obiettivo possiede una lista ordinata attraverso un numero incrementale di tre cifre.
            \item \textbf{Nome}: offre un'informazione più chiara dell'obiettivo attraverso una breve frase.
        \end{itemize}

        Ad esempio:

        \begin{itemize}
            \item MPD001 Indice Gulpease
        \end{itemize}

        Quando sarà possibile, la metrica e l'obiettivo di qualità collegati fra loro avranno lo stesso Tipo e ID.

        Ad esempio:

        \begin{table}[H]
            \begin{detailtable}{\textwidth}{YYY}
                \rowcolor{gray!30}
                \thead{Obiettivo} &
                \thead{Metrica} &
                \thead{Valore desiderato}\\\toprule
                % \hline
                QPD001 Leggibilità del testo & MPD001 Indice Gulpease & 50-60\\
                % \hline
                \rowcolor{gray!15}
                \multicolumn{3}{>{\hsize=\dimexpr3\hsize+4\tabcolsep}X}{\textbf{Descrizione}: quanto il testo è leggibile e comprensibile a livello sintattico lo stabilisce l'\gloss{indice di Gulpease} e una verifica da parte del \Ver.}
            \end{detailtable}
            \caption[Metrica Indice di Gulpease]{Metrica dell'Indice di Gulpease}%, usata per il calcolo della qualità della leggibilità del testo}
        \end{table}

        \paragraph{Classificazione dei test}\label{ClassificazioneTest}
        Ogni test viene classificato identificato univocamente da tale codice:

        \begin{center}
            \texttt{T[Tipo][ID]}
        \end{center}

        \begin{itemize}
            \item \textbf{T}: si riferisce a ``Test''.
            \item \textbf{Tipo}: la tipologia a cui il test appartiene che, seguendo il modello a V\footnote{\url{https://en.wikipedia.org/wiki/V-Model_(software_development)}}, può essere:
            \begin{itemize}
                \item \textbf{V}: validazione.
                \item \textbf{S}: sistema
                \item \textbf{I}: integrazione.
                \item \textbf{U}: unità.
            \end{itemize}
            %per la corrispondenza con i requisiti, non vanno bene le "tre cifre"
            \item \textbf{ID}: numero incrementale che rispetta una struttura gerarchica.
        \end{itemize}

        % COLORI
        \newcommand{\TNI}{{\color{gray}\textbf{NI}}}
        \newcommand{\TI}{{\color{blue}\textbf{I}}}
        \newcommand{\TNS}{{\color{red}\textbf{NS}}}
        \newcommand{\TS}{{\color{green}\textbf{S}}}

        Le tabelle che raccolgno i test di una determinata tipologia presentano i campi:
        \begin{itemize}
            \item \textbf{Codice}: comprendente il codice identificativo del test.
            \item \textbf{Test}: descrive cosa il test deve verificare.
            \item \textbf{Stato}: indica lo stato del test e può essere:
            \begin{itemize}
                \item \TNI: non implementato.
                \item \TI: implementato ma non ancora avviato.
                \item \TNS: avviato e fallito.
                \item \TS: avviato e superato.
            \end{itemize}
        \end{itemize}

    Ad esempio:

    \newenvironment{VTtable}[1][1]{%
        \renewcommand*{\arraystretch}{#1}%
        \renewcommand\theadfont{\bfseries}%
        \oldtabularx%
    }{\endoldtabularx}
    \newcounter{tv}
    \newcommand{\addtotv}{\stepcounter{tv}TV\thetv}
    \begin{table}[H]
        \begin{VTtable}[1.7]{\textwidth}{cXc}
            \rowcolor{\tablegray}
            \textbf{Codice} & \centering\textbf{Test} & \textbf{Stato} \\\toprule
            \addtotv & Verifica la segnalazione dell'apertura di una issue in Redmine \dots
            & \TNI \\
            \bottomrule
        \end{VTtable}
    \end{table}


		\subsubsection{Pianificazione di attività}

			\paragraph{Scopo}
			\Res\ e \Amm\ insieme si occupano di redigere il \PdP\ che tratta generalmente:
			\begin{itemize}
				\item \textbf{Modello di sviluppo}: per la decisione del modello di sviluppo che si intende adottare.
				\item \textbf{Analisi dei rischi}: in cui si identificano i vari possibili rischi e le strategie per evitarli o mitigarli.
				\item \textbf{Partizione del carico di lavoro}: che si occupa di delineare un preciso calendario di avanzamento e la rotazione dei ruoli.
				\item \textbf{Prospetto economico}: per la stima dei costi relativa alle risorse umane.
			\end{itemize}

			\paragraph{Obiettivo}
			L'obiettivo che ci poniamo di raggiungere tramite il \PdP\ è organizzare il lavoro con \gloss{efficenza} ed \gloss{efficacia},
			prevedendo anticipatamente le varie attività da svolgere e determinando dei precisi periodi di tempo da dedicare ad ognuna di esse.\par
			Più nello specifico, il \Doc{\PdPv} contiene:
			\begin{itemize}
				\item Introduzione e scopo
				\item Ciclo di vita con esposizione del modello di sviluppo
				\item Analisi dei rischi con annessa valutazione e classificazione
				\item \gloss{Pianificazione} delle attività in una visione temporale
				\item Prospetto economico del personale durante le attività
				\item \gloss{Preventivo} di ore e costi
				\item \gloss{Organigramma} comprendente i componenti del team di sviluppo
				\item Attualizzazione dei rischi verificatisi nel corso del progetto
			\end{itemize}

			\paragraph{Ordine di esecuzione}
			L'insieme di azioni che indicano in quale modo procedere sono riportate di seguito, in ordine:
			\begin{enumerate}
				\item Individuazione di tutte le attività da svolgere basandosi sulle varie revisioni da affrontare
				\item Riconoscimento dei vari problemi in cui è possibile incorrere nel corso del progetto, con conseguente analisi
					approfondita di ognuno essi. Questo comporta l'identificazione di:
				\begin{itemize}
					\item Probabilità di verificarsi
					\item Gravità della situazione
					\item Strategie da seguire per l'accertamento
					\item Contromisure da adottare per la risoluzione
				\end{itemize}
				\item Report dei problemi incorsi durante lo svolgersi delle attività del progetto, con soluzione adottata
				%si può ampliare
				\item Ordinamento delle attività tramite diagrammi di \gloss{Gantt} creati secondo alcuni fattori importanti da tenere
				in considerazione, quali:
				\begin{itemize}
					\item Dipendenze tra attività
					\item Sequenzialità
					\item Possibilità di parallelismo
					\item Andamento del progresso
					\item Margine di \gloss{slack time} per consentire l'ammortizzamento di possibili rallentamenti
				\end{itemize}
			\item Stima delle risorse secondo la \gloss{metrica} tempo/persona e le \gloss{milestone} nel tempo, pianificando all'indietro
			\item Assegnazione delle risorse umane alle attività secondo una precisa rotazione dei ruoli
			\end{enumerate}


			\paragraph{Metriche di pianificazione} \label{ProcessiOrganizzativiMetriche}

			\begin{itemize}
				\item MPR001 Varianza della pianificazione
				\item MPR002 Varianza dei costi
			\end{itemize}

			La denominazione delle metriche è descritta a \S\ref{Classificazione metriche}.

			\subparagraph{MPR001 Varianza della pianificazione}
			Nel documento \Doc{\PdPv}, sono stabilite le \gloss{baseline} e le scadenze di consegna dei vari prodotti.
			Nonostante il tempo di slack che ogni fase possiede, è possibile che delle date non vengano rispettate causa incidenti di vario tipo.
			\[\dfrac{\sum_{i=1}^{n} |x_i|}{n} \qquad x_i=\text{numero di ore di variazione rispetto al preventivo per l'i-esimo ruolo.}\]


			\textbf{Metrica}: viene indicato il numero di ore di variazione presenti nel momento della verifica rispetto al preventivo effettuato per un preciso periodo. Questo viene svolto per ogni ruolo del progetto ed infine fatta la somma delle variazioni in valore assoluto. Per misurare il numero di ore di varianza viene usato lo strumento Toggl in \S\ref{toggl} per misurare le ore di lavoro effettive. I valori vengono infine confrontati col preventivo.

			\subparagraph{MPR002 Varianza dei costi}
			All'interno del \Doc{\PdPv} è indicato il costo approssimativo del progetto.
			In corso d'opera possono presentarsi dei problemi che richiedono un'aggiunta di costo in termini di $\frac{\text{tempo}}{\text{persona}}$.
			Lo scopo del preventivo è infatti fare una stima non definitiva dei costi. Per verificare se il preventivo è rispettato faremo periodicamente
			un consuntivo di periodo.

			\textbf{Metrica}: viene misurata la differenza conteggiata in \euro\ tra il costo finale e il preventivo.
			Viene seguito tale schema per la tariffa oraria dei vari ruoli del team di sviluppo:

			\begin{table}[H]
				\centering
				\begin{paddedtablex}[1.7]{\textwidth / 2}{Xc}
					\textbf{Ruolo} & \thead{Costo orario} \\
					Responsabile & \euro\ 30 \\
					Amministratore & \euro\ 20 \\
					Analista & \euro\ 25 \\
					Progettista & \euro\ 22 \\
					Programmatore & \euro\ 15 \\
					Verificatore & \euro\ 15 \\
				\end{paddedtablex}
				\caption{Costo orario per ruolo}
			\end{table}


			\paragraph{Classificazione dei rischi}\label{ClassificazioneRischi}
			Le tipologie in cui suddividere i rischi sono state prese dal libro Software
			Engineering\footnote{Riferirsi alla voce ``Software Engineering, 10° edizione'' in \S\ref{rifinfo}} e possono essere di tipo:
			\begin{itemize}
				\item \textbf{Organizzativo}: dovuto alla gestione di persone che hanno diverse responsabilità all'interno del progetto.
				\item \textbf{Personale}: riguarda le conoscenze, i tempi e la \gloss{formazione} personale.
				\item \textbf{Requisiti}: ha a che fare con il numero di requisiti che può variare nel corso dello sviluppo del progetto. %tempo.
				\item \textbf{Strumentale}: per l'utilizzo e la performance degli strumenti hardware.
				\item \textbf{Tecnologico}: per problemi riguardanti l'utilizzo e le funzionalità degli strumenti software.
			\end{itemize}

			A ciascun rischio viene assegnato un codice identificativo in modo da essere facilmente riconoscibile e per comprenderne le generalità
			(classe, probabilità e severità), per poi non doverle cercare nelle tabelle che comprendono tutti i rischi del progetto analizzati.

			Questo codice è composto da:

			\begin{center}
				\texttt{[Tipologia][ID]-[Gravità][Probabilità][Classe]}
			\end{center}

			Nel caso dell'attualizzazione dei rischi, vengono aggiustati i valori di gravità,
			probabilità, classe e al codice viene aggiunta in coda la data in cui si è verificato il problema, in modo da tenere una traccia cronologica.

			Il codice in tal caso diventa:

			\begin{center}
				\texttt{[Tipologia][ID]-[Gravità][Probabilità][Classe]:[Data]}
			\end{center}


			Nel caso i rischi non si siano verificati ma ne venga riconsiderata la probabilità, allora vengono posti in elenco in seguito ai rischi effettivamente
			riscontrati, con il vecchio codice per tipologia e id e con i valori aggiornati per gravità, probabilità e classe.

			I valori che possono assumere sono:

			\begin{itemize}
				\item \textbf{Tipologia}:
				\begin{itemize}
					\item \textbf{O}: organizzativo.
					\item \textbf{P}: personale.
					\item \textbf{R}: requisiti.
					\item \textbf{S}: strumentale.
					\item \textbf{T}: tecnologico.
				\end{itemize}
				\item \textbf{ID}: numero progressivo di tre cifre (001 - 999).
				\item \textbf{Gravità}:
				\begin{itemize}
					\item \textbf{0}: accettabile.
					\item \textbf{1}: tollerabile.
					\item \textbf{2}: inaccettabile.
				\end{itemize}

				\item \textbf{Probabilità}:
				\begin{itemize}
					\item \textbf{0}: bassa.
					\item \textbf{1}: media.
					\item \textbf{2}: alta.
				\end{itemize}

				\item \textbf{Classe}: ci si riferisce ai livelli di rischio.
				\begin{itemize}
					\item \textbf{0}: basso.
					\item \textbf{1}: medio.
					\item \textbf{2}: alto.
				\end{itemize}

				\item \textbf{Data}: data in cui si è verificato il problema.
			\end{itemize}

			Ad esempio con P001-021 si può intuitivamente capire che si tratta del primo rischio relativo al personale, di gravità accettabile,
			probabilità alta e un valore di classe medio.\\
			Invece P002-122:2019-01-13 è il secondo rischio attualizzato relativo al personale, di gravità tollerabile, probabilità alta, valore di classe alto e si è verificato in data 2019-01-13.

            \subparagraph{MPR008 Rischi non previsti avvenuti}
            La denominazione delle metriche è descritta a \S\ref{Classificazione metriche}.


            Nell'Analisi dei rischi presente nel \Doc{\PdPv}, sono presenti i rischi ritenuti possibili per i quali è proposta una soluzione.
            Possono presentarsi anche rischi non previsti in tale analisi. Questi devono essere il meno possibili (nulli) perché la loro soluzione sarà decisa al momento causando ritardi all'interno del progetto.

            \textbf{Metrica}: numero di rischi non previsti avvenuti nel corso dell'intero progetto.

			\paragraph{Ruoli di progetto}
			I ruoli\footnote{Riferirsi alla voce ``Descrizione dei ruoli di progetto'' in \S\ref{rifinfo}.} adoperati per lo sviluppo del progetto sono:
			\begin{itemize}[noitemsep]
				\item Analista
				\item Progettista
				\item Responsabile
				\item Amministratore
				\item Programmatore
				\item Verificatore
			\end{itemize}

		% \subsubsection{Controllo del progetto}

			\subsubsection{Monitoraggio del progetto}

			\paragraph{Monitoraggio dell'esecuzione dei processi}
			Ogni processo verrà controllato periodicamente durante tutta la sua esecuzione in modo da non intralciare il \gloss{way of working},
			mediante misurazioni associate a metriche descritte dai processi di verifica, controllate tramite l'utilizzo di strumenti automatizzati.
			Le metriche adottate sono, per il processo considerato, indicatori di efficacia dei prodotti rispetto ai requisiti di funzionalità e qualità,
			stabiliti nel \Doc{\PdQv} e nell'\Doc{\AdRv}.\\
			Risultano validi indicatori per la valutazione di:
			\begin{itemize}
				\item Aderenza al way of working
				\item Stato di avanzamento del processo rispetto alla pianificazione
				\item Identificazione dei problemi
				\item Eventuali ripianificazioni
			\end{itemize}

			\paragraph{Procedure di comunicazione}
			%Per coordinare il team e il committente sullo stato del progetto sono state stabilite le seguenti norme:
			Per la coordinazione del team e le comunicazioni con il cliente sullo stato del progetto, abbiamo stabilito le seguenti norme:
			\begin{itemize}
				\item Per le comunicazioni interne utilizzeremo gli strumenti segnalati in \S\ref{PianificazioneCoordinamento},
					in particolare con l'utilizzo di \gloss{Slack}.
				\item All'occorrenza, fisseremo riunioni per le questioni più importanti, concordando:
					\begin{itemize}
						\item Data, ora e luogo
						\item Un ordine del giorno da discutere
						\item La persona incaricata ad appuntare il contenuto della riunione per poter poi redigere formalmente il verbale dell'incontro
					\end{itemize}
				\item Per le comunicazioni esterne verrà utilizzata prevalentemente la comunicazione via email e in caso di necessità si concorderà con l'azienda per riunioni via Hangouts o in un luogo prestabilito come quanto segnalato in \S\ref{ConivolgimentoAcquirente}.
			\end{itemize}


    		\paragraph{Gestione dei problemi emersi}
			Problemi emersi durante l'esecuzione dei processi verranno segnalati tramite \gloss{ticket}, utilizzando il sistema integrato di
			\gloss{GitHub} come descritto in dettaglio in \S\ref{GitHub}. Questo ci permetterà di tracciare i problemi insorti durante le attività
			e di pianificare la risoluzione degli stessi.\par
			Per segnalazioni minori, verranno spesso utilizzati dei tag commentati nei \gloss{sorgenti}.
			\begin{samepage}
				Saranno nella forma:
				\begin{center}
					\texttt{[TAG]: [Descrizione]}
				\end{center}
			\end{samepage}
			I tipi di tag che utilizzeremo sono:
			\begin{itemize}
				\item \textbf{TODO}: per segnalare la presenza di un lavoro da fare o lasciato a metà.
				\item \textbf{FIXME}: per segnalare la presenza di una correzione da effettuare su del lavoro già svolto.
				\item \textbf{NOTE}: per aggiugere una nota su cui porre particolare attenzione.
			\end{itemize}

		\paragraph{Monitoraggio delle ore di orologio}
		È necessario distinguere le ore di orologio dalle ore di calendario. Le ore di orologio sono ore effettive di lavoro, in
		cui non c'è alcuna perdita di tempo. Le ore di calendario sono ore in cui non si conta solamente il lavoro effettivo
		ma anche tutte le distrazioni. A noi interessa monitorare le ore di orologio, e le ore rendicontate pertanto sono da
		considerare ore di orologio.

		\subsubsection{Chiusura dei processi}

    		\paragraph{Archiviazione dei prodotti}
			I prodotti che, una volta terminate le attività e i processi per completarli, soddisfano le aspettative in termini di qualità,
			verranno archiviati in apposite \gloss{repository}. Alla versione \textit{2.0.0} del presente documento, saranno archiviati
			i file \LaTeX\ per la generazione dei relativi PDF, i sorgenti Python per il Proof of Concept e gli eventuali file \gloss{JSON}.

    		\paragraph{Archiviazione delle misurazioni}
			Oltre ai prodotti archivieremo le metriche, definite e classificate nel presente documento, con i limiti di accettazione riportati nel \Doc{\PdQv}.
			Il calcolo delle metriche verrà effettuato su ciascun prodotto nel momento del suo inserimento nella repository, ove possibile; tali risultati
			o anomalie verranno opportunamente archiviati.


    	\subsubsection{Strumenti organizzativi}\label{PianificazioneCoordinamento}

    		\paragraph{Slack}
			Slack è uno strumento di collaborazione nato appositamente per coordinare il lavoro tra i team, permettendo la comunicazione in tempo
			reale e mettendo a disposizione molte altre utilità indispensabili. Tra queste, c'è la possibilità di dividere il \gloss{workspace}
			in vari canali specifici, marcare parole o frasi importanti con diversi stili (grassetto, corsivo, codice, barrato), fissare i messaggi
			importanti, aprire una discussione sotto ogni messaggio per non creare troppi messaggi sul canale, ecc \dots

			Il workspace usato è stato suddiviso in diversi canali in base alle esigenze del periodo di lavoro.
			Nel periodo relativo alla stesura della versione attuale del documento, i vari canali sono:
			\begin{itemize}
				\item \textbf{\# general}: canale in cui verranno discusse tematiche generali riguardanti il progetto e le sue attività.
				\item \textbf{\# Analisi dei Requisiti}: per discussioni inerenti alla stesura del documento \AdR. Nello specifico verranno discussi i requisiti e i casi d'uso.
				\item \textbf{\# Piano di Progetto}: per discussioni inerenti la pianificazione, la suddivisione del lavoro, la valutazione dei rischi, la consuntivazione e altre attività volte alla redazione del \PdP.
				\item \textbf{\# Piano di Qualifica}: per discussioni inerenti principalmente la qualità di processo e di prodotto, riportate nel \PdQ.
				\item \textbf{\# Norme di Progetto}: per discussioni riguardanti le norme che il team di sviluppo adotterà e che ogni membro sarà tenuto a rispettare. Esse saranno riportate nelle \NdP.
				\item \textbf{\# Studio di Fattibilità}: canale in cui verrà discussa la fattibilità di ogni capitolato proposto, convergenti nello \SdF.
				\item \textbf{\# git}: canale in cui un \gloss{bot} correttamente configurato manderà una notifica ogni volta che verrà effettuato un \gloss{commit}, oppure alla chiusura o apertura di una \gloss{issue} nella repository principale.
				\item \textbf{\# latex}: per discussioni riguardanti gli aspetti tecnici di \LaTeX.
				\item \textbf{\# random}: per discussioni off-topic del team.
				\item \textbf{\# Proof of Concept}: per discussioni inerenti alla realizzazione del Proof of Concept.
			\end{itemize}

    		\paragraph{GitHub}\label{GitHub}
			GitHub fornisce un \gloss{Issue Tracking System} per ogni repository che ospita. Questo strumento offre più funzionalità di quanto potrebbe sembrare,
			poich\'e permette anche di trattare le issue come delle task, marcandole con una opportuna label. Infatti, questo sistema offre le seguenti funzionalità:
			\begin{itemize}
				\item Aprire una issue, attribuendone un titolo e una descrizione più dettagliata. Ogni issue è legata a un ID generato automaticamente in maniera
					incrementale (e.g. \texttt{\#32}), al quale sarà possibile fare riferimento nei messaggi di commit.
				\item Creare delle milestone, e aggregare ad esse le issue.
				\item Assegnare delle label alle issue, per marcarne la natura. Ad esempio, una issue marcata come \gloss{bug} sarà un problema da risolvere, una
					issue marcata come \gloss{enhancement} sarà un miglioramento o una task. È possibile inoltre creare delle label personalizzate.
				\item Assegnare una issue a s\'e stesso o a un collaboratore.
				\item Supporto di \gloss{markdown} esteso, che permette di
				formattare le pagine delle issue in modo estremamente funzionale.
			\end{itemize}


			\paragraph{Toggl}\label{toggl}
			Come strumento di monitoraggio delle ore di orologio usiamo Toggl\footnote{Riferirsi alla voce ``Toggl'' in \ref{rifinfo} per maggiori informazioni},
			un software che è disponibile come applicazione web, applicazione
			desktop e applicazione mobile.\par
			Tramite questo strumento è possibile cronometrare il tempo che si spende nello svolgere una determinata task, ed è possibile organizzarsi in un team
			in modo che le ore dedicate a un progetto siano condivise.\par
			Per maggiori informazioni, riferirsi alla fonte ufficiale.

			%\paragraph{Redmine} % ? da vedere piu avanti.
			% Per ora GitHub per issue-task tracker ~Tim




	\subsection{Formazione}\label{Formazione}

		La formazione avviene tramite studio autonomo dei \gloss{framework} menzionati da \II\ durante la presentazione del capitolato e incontri interni.

		\subsubsection{Piano di formazione}
		Il piano di formazione prevede lo studio di:
		\begin{itemize}
			\item Tecnologie e metodologie per la realizzazione del Proof of Concept\footnote{Vedere \S\ref{rifinfo} per i link alla documentazione.}:
			\begin{itemize}
				% \item \LaTeX
				\item Python 3
				\item Apache Kafka
				\item Redmine
				\item GitLab
				\item Telegram API per l'invio di messaggi su Telegram tramite un bot dedicato
				\item Protocollo SMTP per l'invio di email tramite uno script Python
				\item Docker
				\item Rancher
				\item CherryPy
				% \item SonarQube
				% \item Slack
				% \item API Rest
				% \item The Twelve-Factor App
			\end{itemize}
			\item Possibilità d'integrazione tra le tecnologie sopra elencate
		\end{itemize}
