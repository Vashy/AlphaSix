\section{Processi Organizzativi}

    \subsection{Gestione del progetto}

	    \subsubsection{Scopo}
	    L'attività di gestione del progetto consiste nel delineare:
	    \begin{itemize}
	    	\item I processi di progetto
	    	\item Le \gloss{risorse} a essi necessarie
	    	\item I costi per la loro esecuzione
	    	\item I \gloss{task} di risorsa umana
	    	\item La verifica delle attività di processo
	    \end{itemize}

		%TODO: da mettere?
	    % \subsubsection{Istanziazione dei processi}
		% Derivano dagli standard...
		
		\subsubsection{Pianificazione di attività}

			\paragraph{Scopo}
			\Res\ e \Amm\ insieme si occupano di generare il \PdP\ che tratta generalmente:
			\begin{itemize}
				\item \textbf{Modello di sviluppo}: per la decisione del modello di sviluppo che si intende adottare.
				\item \textbf{Analisi dei rischi}: in cui si identificano i vari possibili rischi e le strategie per evitarli o mitigarli.
				\item \textbf{Partizione del carico di lavoro}: che si occupa di delineare un preciso calendario di avanzamento e la rotazione dei ruoli.
				\item \textbf{Prospetto economico}: per la stima dei costi relativa alle risorse umane.
			\end{itemize}

			\paragraph{Obiettivo}
			L'obiettivo che ci si pone di raggiungere tramite il \PdP\ è organizzare il lavoro con \gloss{efficenza} ed \gloss{efficacia} prevedendo anticipatamente le varie attività da svolgere e determinando dei precisi periodi di tempo da dedicare ad ognuna di esse. \par
			Più nello specifico, il \Doc{\PdPv} contiene:
			\begin{itemize}
				\item Introduzione e scopo
				\item Ciclo di vita con esposizione del modello di sviluppo
				\item Analisi dei rischi con annessa valutazione e classificazione
				\item Pianificazione delle attività in una visione temporale
				\item Prospetto economico del personale durante le attività
				\item Preventivo di ore e costi
				\item Organigramma comprendente i componenti del team di sviluppo
			\end{itemize}

			\paragraph{Ordine di esecuzione}
			L'insieme di azioni che indicano in quale modo procedere sono riportate qui di seguito in ordine:
			\begin{enumerate}
				\item Individuazione di tutte le attività da svolgere basandosi sulle varie revisione da affrontare
				\item Riconoscimento dei vari problemi che in cui è possibile incorrere nel corso del progetto con conseguente analisi approfondita di ognuno essi. Questo comporta l'identificazione di:
				\begin{itemize}
					\item La probabilità di verificarsi
					\item La gravità della situazione
					\item Le strategie da seguire per l'accertamento
					\item Le contromisure da adottare per la risoluzione
				\end{itemize}
				%si può ampliare
				\item Ordinamento delle attività tramite diagrammi di Gantt creati secondo alcuni fattori importanti da tenere in considerazione quali:
				\begin{itemize}
					\item Dipendenze tra attività
					\item Sequenzialità
					\item Possibilità di parallelismo
					\item Andamento del progresso
					\item Margine di slack per consentire l'ammortizzamento di possibili rallentamenti
				\end{itemize}
			\item Stima delle risorse secondo la metrica tempo/persona e le \gloss{milestone} nel tempo, pianificando all'indietro
			\item Assegnazione delle risorse umane alle attività secondo una precisa rotazione dei ruoli.
			\end{enumerate}
		
			\paragraph{Classificazione dei rischi}
			Le tipologie in cui suddividere i rischi sono state prese dal libro Software Engineering\footnote{Software Engineering, 10° edizione, Pagina 648, Capitolo 22: Project management} e sono:
			\begin{itemize}
				\item \textbf{Organizzativo}: dovuto alla gestione di persone che hanno diverse responsabilità all'interno progetto.
				\item \textbf{Personale}: riguarda le conoscenze, i tempi e la formazione personale.
				\item \textbf{Requisiti}: ha a che fare con il numero di requisiti che può variare nel corso dello sviluppo del progetto. %tempo.
				\item \textbf{Strumentale}: per l'utilizzo e la performance e degli strumenti hardware.
				\item \textbf{Tecnologico}: per problemi riguardanti l'utilizzo e le funzionalità degli strumenti software.
			\end{itemize}
			
			A ciascun rischio viene assegnato un codice identificativo in modo da essere facilmente riconoscibile e per comprenderne le generalità (classe, probabilità e severità), per poi non doverle cercare nelle tabelle che comprendono tutti i rischi del progetto analizzati.
			
			Questo codice è composto da:
			
			\begin{center}
				\texttt{[Tipologia][ID]-[Gravità][Probabilità][Classe]}
			\end{center}
			
			I valori che possono assumere sono:
			
			\begin{itemize}
				\item \textbf{Tipologia}:
				\begin{itemize}
					\item \textbf{O}: organizzativo.
					\item \textbf{P}: personale.
					\item \textbf{R}: requisiti.
					\item \textbf{S}: strumentale.
					\item \textbf{T}: tecnologico.
				\end{itemize}
				
				\item \textbf{ID}: numero progressivo di tre cifre (001 - 999);
				\item \textbf{Gravità}:
				\begin{itemize}
					\item \textbf{0}: accettabile.
					\item \textbf{1}: tollerabile.
					\item \textbf{2}: inaccettabile.
				\end{itemize}
				
				\item \textbf{Probabilità}:
				\begin{itemize}
					\item \textbf{0}: bassa.
					\item \textbf{1}: media.
					\item \textbf{2}: alta.
				\end{itemize}
				
				\item \textbf{Classe}:
				\begin{itemize}
					\item \textbf{0}: basso.
					\item \textbf{1}: medio.
					\item \textbf{2}: alto.
				\end{itemize}
			\end{itemize}
			
			Ad esempio con P001-021 si può intuitivamente capire che si tratta del primo rischio relativo al personale, di gravità accettabile, probabilità alta e un valore di classe medio.
			
			\paragraph{Ruoli di progetto}
			I ruoli\footnote{Per la descrizione completa vedere link ``Descrizione dei ruoli di progetto'' in \S\ref{rifinfo}.} adoperati per lo sviluppo del progetto sono:
			\begin{itemize}[noitemsep]
				\item Analista
				\item Progettista
				\item Responsabile
				\item Amministratore
				\item Programmatore
				\item Verificatore
			\end{itemize}

		\subsubsection{Pianificazione della qualità}\label{Pianificazione qualita}
		%Per cercare la qualità nei prodotti è necessario che vengano misurate diverse caratteristiche di questi ultimi.
		Per la ricerca della qualità nei prodotti è necessario che vengano misurate diverse caratteristiche associate a questi.
		Per ogni caratteristica devono essere fissati un livello minimo e un livello ottimale che questa deve avere.
		In tale sezione viene descritto il modo di utilizzo sia delle metriche che degli strumenti utili ad ottenere qualità nei processi, nei documenti e nei prodotti.
		Gli strumenti utilizzati possono essere software, descritti nelle \Doc{\NdPv}, o gli standard di qualità, descritti nel \Doc{\PdQv}.\\
		Gli obiettivi che ci si pone di raggiungere sono descritti nel \Doc{\PdQv}.

			\paragraph{Classificazione dei processi}
			I processi analizzati nel \Doc{\PdQv} sono classificati nel seguente modo:

			\begin{center}
				\texttt{PROC[ID] [Nome]}
			\end{center}

			\begin{itemize}
				\item \textbf{ID}: un numero incrementale per indicare in modo univoco il processo.
				\item \textbf{Nome}: una breve frase per indicare la funzione del processo.
			\end{itemize}

			\paragraph{Classificazione degli obiettivi}
			Gli obiettivi per la qualità (Q) concordati dall'\Amm\ e dal \Ver\ descritti nel \PdQ\ avranno il seguente codice identificativo:

			\begin{center}
				\texttt{Q[Tipo][ID] [Nome]}
			\end{center}

			\begin{itemize}
				\item \textbf{Tipo}: indica il la tipologia dell'oggetto di cui viene valutata la qualità, e questa può essere:
				\begin{itemize}
					\item \texttt{PR}: processi;
					\item \texttt{PD}: prodotti che risultano essere documenti;
					\item \texttt{PS}: prodotti software;
				\end{itemize}

				\item \textbf{ID}: ogni tipo di obiettivo possiede una lista ordinata attraverso un numero incrementale di tre cifre;
				\item \textbf{Nome}: offre un'informazione più chiara dell'obiettivo attraverso una breve frase;
			\end{itemize}

			Ad esempio:

			\begin{itemize}
				\item \textbf{QPD001 Leggibilità del testo}
			\end{itemize}


			\paragraph{Classificazione delle metriche}\label{Classificazione metriche}
			Ogni obiettivo della qualità deve, per quanto possibile, essere collegato ad una metrica, anch'essa scelta dall'\Amm\ e dal \Ver. Questo per valutare
			quantitativamente il raggiungimento o meno degli obiettivi stabiliti.

			Le metriche (M) verranno classificate nel seguente modo:

			\begin{center}
				\texttt{M[Tipo][ID] [Nome]}
			\end{center}

			\begin{itemize}
				\item \textbf{Tipo}: indica la tipologia di quello su cui viene applicata la metrica, può essere:
				\begin{itemize}
					\item \texttt{PR}: processi;
%					\item \texttt{PD}: prodotti che risultano essere documenti;
					\item \texttt{PD}: prodotti di documentazione;
					\item \texttt{PS}: prodotti software;
				\end{itemize}

				\item \textbf{ID}: ogni tipo di obiettivo possiede una lista ordinata attraverso un numero incrementale di tre cifre;
				\item \textbf{Nome}: offre un'informazione più chiara dell'obiettivo attraverso una breve frase;
			\end{itemize}

			Ad esempio:

			\begin{itemize}
				\item \textbf{MPD001 Indice Gulpease}
			\end{itemize}

			Quando sarà possibile, la metrica e l'obiettivo di qualità collegati fra loro avranno lo stesso Tipo e ID.

			Ad esempio:

			\begin{table}[H]
				\begin{detailtable}{\textwidth}{YYY}
					\rowcolor{gray!30}
					\thead{Obiettivo} &
					\thead{Metrica} &
					\thead{Valore desiderato}\\\toprule
					% \hline
					QPD001 Leggibilità del testo & MPD001 Indice Gulpease & 50-60\\
					% \hline
					\rowcolor{gray!15}
					\multicolumn{3}{>{\hsize=\dimexpr3\hsize+4\tabcolsep}X}{\textbf{Descrizione}: quanto il testo è leggibile e comprensibile a livello sintattico lo stabilisce l'indice Gulpease e una verifica da parte del \Ver}
				\end{detailtable}
				\caption[Metrica Indice di Gulpease]{Metrica dell'Indice di Gulpease}%, usata per il calcolo della qualità della leggibilità del testo}
			\end{table}

		\subsubsection{Controllo del progetto}

			\paragraph{Monitoraggio del progetto}

			\subparagraph{Monitoraggio dell'esecuzione dei processi}
			Ogni processo verrà controllato periodicamente durante tutta la sua esecuzione in modo da non intralciare il \gloss{way of working},
			mediante misurazioni associate a metriche descritte dai processi di verifica, controllate tramite l'utilizzo di strumenti automatizzati.
			Le metriche adottate sono, per il processo considerato, indicatori di efficacia dei prodotti rispetto ai requisiti di funzionalità e qualità,
			stabiliti nel \Doc{\PdQv} e nell'\Doc{\AdRv}.\\
			Risultano validi indicatori per la valutazione di:
			\begin{itemize}
				\item Aderenza al way of working
				\item Stato di avanzamento del processo rispetto alla pianificazione
				\item Identificazione dei problemi
				\item Eventuali ripianificazioni
			\end{itemize}

			\subparagraph{Procedure di comunicazione}
			%Per coordinare il team e il committente sullo stato del progetto sono state stabilite le seguenti norme:
			Per la coordinazione del team e le comunicazioni con il committente sullo stato del progetto, sono state stabilite le seguenti norme:
			\begin{itemize}
				\item Per le comunicazioni interne verranno utilizzati gli strumenti segnalati in \S\ref{pianificazione e coordinamento}, in particolare con l'utilizzo di \gloss{Slack}.
				\item All'occorrenza, verranno fissate riunioni per le questioni più importanti concordando:
					\begin{itemize}
						\item Data, ora e luogo
						\item Un ordine del giorno da discutere
						\item La persona incaricata ad appuntare il contenuto della riunione per poter poi redigere formalmente il verbale dell'incontro
					\end{itemize}
				\item Per le comunicazioni esterne verrà utilizzata prevalentemente la comunicazione via email ed in caso di necessità si concorderà con l'azienda per riunioni via Skype o in un luogo prestabilito. % HELP WANTED fissare riunioni con l'azienda?
			\end{itemize}


    		\paragraph{Gestione dei problemi emersi}
			Problemi emersi durante l'esecuzione dei processi verranno segnalati tramite \gloss{ticket}, utilizzando il sistema integrato di Github come descritto in dettaglio in \S\ref{Github}. Questo permette di tracciare i problemi insorti durante le attività e di pianificare la risoluzione degli stessi.
			Per segnalazioni minori, verranno spesso utilizzati dei tag commentati nei \gloss{sorgenti} \LaTeX.
			\begin{samepage}
				Saranno nella forma:
				\begin{center}
					\texttt{[TAG]: [Descrizione]}
				\end{center}
			\end{samepage}
			I tipi di tag utilizzati dal team sono:
			\begin{itemize}
				\item \textbf{TODO}: per segnalare la presenza di un lavoro da fare o lasciato a metà.
				\item \textbf{FIXME}: per segnalare la presenza di una correzione da effettuare su del lavoro già svolto.
			\end{itemize}
		
    		% \paragraph{Riportare lo stato di avanzamento del progetto}
			% vedere Breaking Bug, io non riesco a capire il senso di questo paragrafo e lo toglierei ~ Tim

		\subsubsection{Chiusura dei processi}

    		\paragraph{Archiviazione dei prodotti}
			I prodotti che, una volta terminate le attività e i processi per completarli, soddisfano le aspettative in termini di qualità,
			verranno archiviati in apposite \gloss{repository}. Alla versione \textit{1.0.0} del presente documento, gli unici prodotti archiviati saranno i file \LaTeX\ per la generazione dei relativi PDF.

    		\paragraph{Archiviazione delle misurazioni}
			Oltre ai prodotti vengono archiviate le metriche, definite e classificate nel presente documento, con i limiti di accettazione riportati nel \Doc{\PdQv}.
			Il calcolo delle metriche verrà effettuato su ciascun prodotto nel momento del suo inserimento nella repository, ove possibile; tali risultati o anomalie verranno opportunamente archiviati. % FIXME: Archiviate? Come? Dove?

    	\subsubsection{Strumenti di pianificazione e coordinamento}\label{pianificazione e coordinamento}

    		\paragraph{Slack}
			Slack è uno strumento di collaborazione nato appositamente per coordinare il lavoro tra i team, permettendo la comunicazione in tempo
			reale, e mettendo a disposizione molte altre utilità indispensabili. Tra queste, c'è la possibilità di dividere il \gloss{workspace} in vari canali specifici, marcare parole o frasi importanti con diversi stili (grassetto, corsivo, codice, barrato), fissare i messaggi importanti, aprire una discussione sotto ogni messaggio per non creare troppi messaggi sul canale, ecc \dots
			
			Il workspace usato dal team è stato suddiviso in diversi canali in base alle esigenze del periodo di lavoro. 
			%Nel periodo attuale, i vari canali sono:
			Nel periodo relativo alla stesura della versione attuale del documento, i vari canali sono:
			\begin{itemize}
				\item \textbf{\# general}: canale in cui verranno discusse tematiche generali riguardanti il progetto e le sue attività.
				\item \textbf{\# Analisi dei Requisiti}: per discussioni inerenti alla stesura del documento \Doc{\AdR}. Nello specifico verranno discussi i requisiti e i casi d'uso.
				\item \textbf{\# Piano di Progetto}: per discussioni inerenti la pianificazione, la suddivisione del lavoro, la valutazione dei rischi, la consuntivazione e altre attività volte alla redazione del \PdP.
				\item \textbf{\# Piano di Qualifica}: per discussioni inerenti principalmente la qualità di processo e di prodotto, riportate nel \PdQ.
				\item \textbf{\# Norme di Progetto}: per discussioni riguardanti le norme che il team di sviluppo adotterà e che ogni membro sarà tenuto a rispettare e riportate nelle \NdP.
				\item \textbf{\# Studio di Fattibilità}: canale in cui verrà discussa la fattibilità di ogni capitolato proposto, convergenti nello \SdF.
				\item \textbf{\# git}: canale in cui un bot correttamente configurato manderà una notifica ogni volta che verrà effettuato un commit, oppure alla chiusura o apertura di una \gloss{issue} nella repository principale.
				\item \textbf{\# latex}: per discussioni riguardanti gli aspetti tecnici di \LaTeX.
				\item \textbf{\# random}: per discussioni off-topic del team. %aka SPAM
			\end{itemize}

    		\paragraph{GitHub}\label{Github}
			GitHub fornisce un servizio di \gloss{Issue Tracking System} per ogni repository che ospita. Questo strumento offre più funzionalità di quanto potrebbe sembrare, poich\'e permette anche di trattare le issue come delle task, marcandole con una opportuna label. Infatti, questo sistema offre le seguenti funzionalità:
			\begin{itemize}
				\item Aprire una issue, attribuendone un titolo e una descrizione più dettagliata. Ogni issue è legata a un ID generato automaticamente in maniera incrementale (e.g. \texttt{\#32}), al quale sarà possibile fare riferimento nei messaggi di commit.
				\item Creare delle milestone, e aggregare ad esse le issue.
				\item Assegnare delle label alle issue, per marcarne la natura. Ad esempio, una issue marcata come \gloss{bug} sarà un problema da risolvere, una issue marcata come \gloss{enhancement} sarà un miglioramento o una task. È possibile inoltre creare delle label personalizzate.
				\item Assegnare una issue a un collaboratore, che sarà tenuto a correggerla.
%				\item Supporto di \gloss{markdown} esteso, che permette di formare delle pagine per le issue di aspetto delizioso e allo stesso tempo estremamente funzionali.
				\item Supporto di \gloss{markdown} esteso, che permette di 
				formattare le pagine delle issue e allo stesso tempo estremamente funzionali.			
			\end{itemize}

			%\paragraph{Redmine} % ? da vedere piu avanti.
			% Per ora Github per issue-task tracker ~Tim



	\subsection{Formazione}

		La formazione di ogni membro di AlphaSix avviene tramite studio autonomo dei \gloss{framework} menzionati da \II\ durante la presentazione del progetto e incontri con \gruppo.

%TODO: aggiungere gloss se non sono state menzionate prima queste tecnologie
		\subsubsection{Piano di formazione}
		Il piano di formazione prevede:
		\begin{itemize}
			\item Lo studio della documentazione\footnote{Vedere ''Link delle documentazioni utili alla formazione'' in \S{1.5.2}} di tecnologie e metodologie quali:
			\begin{itemize}
				\item \LaTeX 
				\item Redmine 
				\item GitLab 
				%\item SonarQube
				\item Apache Kafka
				\item Telegram 
				%\item Slack 
				\item Docker 
				\item API Rest
				\item The Twelve-Factor App
			\end{itemize}
			\item Lo studio delle possibilità d'integrazione tra di esse
		\end{itemize}
