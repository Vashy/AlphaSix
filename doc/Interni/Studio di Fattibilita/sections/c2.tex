\section{Colletta - C2} \label{c2}
    \subsection{Descrizione generale}
    Il progetto Colletta proposto dall'azienda Mivoq prevede la realizzazione di una piattaforma di collezione dati contenente piccoli esercizi di grammatica.
    In essa si identificano tre attori principali: insegnanti, allievi e sviluppatori. L'insegnante fornisce gli esercizi di grammatica che vengono svolti dagli allievi.
    I dati derivanti da queste interazioni tra insegnati e allievi possono in seguito essere utilizzate dagli sviluppatori per migliorare il sistema di riconoscimento delle frasi.

    \subsection{Obiettivo finale}
    L'obiettivo del progetto è creare un'applicazione (Web o mobile) con una struttura come quella descritta precedentemente,
    in cui la raccolta dei dati avviene in modo implicito tramite il solo utilizzo da parte degli utenti. Questi dati possono essere impiegati successivamente per la produzione di
    servizi utili basati sull'apprendimento automatico.

    \subsection{Tecnologie coinvolte}
    La scelta delle tecnologie viene lasciata libera, ma vengono consigliate:
        \begin{itemize}
            \item \gloss{Firebase} o altri servizi esistenti per l'immagazzinamento dei dati
            \item Software open-source per lo svolgimento degli esercizi:
            \begin{itemize}
                \item \gloss{Hunpos} 
                \item \gloss{Freeling}.
            \end{itemize}
        \end{itemize}

    \subsection{Valutazione conclusiva}
    Il capitolato appena descritto non è stato scelto da \gruppo\ perché tocca solo marginalmente l'ambito del \gloss{machine learning}.
    Tratta solamente di una piattaforma per la raccolta dati, motivo per cui non risulta particolarmente accattivante.
    La mancanza di vincoli non dà un'idea precisa di come sarebbe possibile operare per sviluppare il progetto.

    % TODO: machine learning da glossarizzare
