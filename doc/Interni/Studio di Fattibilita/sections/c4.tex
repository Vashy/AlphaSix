\section{MegAlexa - C4} \label{c4}

    \subsection{Descrizione generale}
    Il capitolato propone lo sviluppo di una piattaforma
    dedita alla creazione di una \gloss{routine} in grado di eseguire una sequenza
    di Skill per Alexa, l'assistente vocale di Amazon.

    \subsection{Obiettivo finale}
    Nello specifico, \`e richiesto lo sviluppo di una piattaforma multilingua Web e
    mobile (\gloss{iOS} o \gloss{Android}) che sia in grado di
    creare \gloss{workflow} personalizzati per Alexa creati dagli utenti. Ogni utente
    deve aver la possibilit\`a di poter nominare i propri workflow senza collidere con
    quelli di altri utenti (e.g. un workflow chiamato "Buongiorno" dell'utente A \`e
    diverso dal workflow "Buongiorno" dell'utente B).

    \subsection{Tecnologie coinvolte}
    L'azienda consiglia di utilizzare:
    \begin{itemize}
    	\item Linguaggi per la piattaforma Web:
    	\begin{itemize}
            \item \gloss{HTML5}, \gloss{CSS3} (\gloss{Bootstrap}), Javscript
                per il \gloss{frontend}
            \item Node.js per il \gloss{backend}.
    	\end{itemize}
        \item \textbf{\gloss{Kotlin}/\gloss{Swift}}: linguaggi per lo sviluppo delle app
            rispettivamente per Android e iOS.
        \item \textbf{\gloss{Amazon Web Services} (AWS)}: servizio scelto per l'hosting del software e del database.
            Nello specifico, saranno utilizzati:
            \begin{itemize}
                \item \gloss{Amazon API Gateway}
                \item \gloss{AWS Lambda}
                \item \gloss{Amazon Aurora Serverless}.
            \end{itemize}
    \end{itemize}

    \subsection{Valutazione conclusiva}
    Dall'analisi fatta dal team è stato deciso di non sviluppare questo capitolato per l'elevata mole di lavoro prevista.
    Non è semplice inoltre gestire le varie routine per gli utenti oltre che sviluppare un'applicazione Web e mobile nativa multilingua.
    Nonostante questo le tecnologie di AWS e di workflow di assistenti vocali risultano interessanti.
