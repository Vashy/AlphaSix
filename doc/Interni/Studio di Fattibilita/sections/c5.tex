\section{P2PCS - C5} \label{c5}
    \subsection{Descrizione generale}
    Il quinto capitolato propone la creazione di un'applicazione Android in grado di gestire una piattaforma di car sharing \gloss{peer to peer}.

    \subsection{Obiettivo finale}
    \`E richiesto lo sviluppo di un sistema software che consenta ad un utente la condivisione della propria auto. Tale condivisione si potrà
    avere solo una volta che l'operatore avrà inserito nel calendario i giorni in cui non utilizzerà il mezzo.
    In questo modo si lascia così la possibilità di poter dare le chiavi della propria autovettura in mano ad un'altro utente del servizio.

    \subsection{Tecnologie coinvolte}
    L'azienda consiglia di utilizzare:
        \begin{itemize}
        \item \gloss{Google Maps}
        \item \gloss{Google Cloud Platform}
        \item \gloss{Henshin movens platform}
        \item \gloss{Octalysis}
        \item Node.js.
    \end{itemize}

    \subsection{Valutazione conclusiva}
    Fin da subito, l'idea di ``condividere la propria macchina con altre persone amiche o meno'' (cit. da capitolato) non ha suscitato interesse
    e motivazione ad AlphaSix. Inoltre, la tecnologia Octalysis sembra vincolare troppo l'andamento del progetto: ne viene richiesto un
    uso stringente. Il risultato è un'applicazione che, secondo il parere del team di sviluppo, sfrutta il \gloss{gamification} per convincere gli utenti.
    Dopo queste considerazioni AlphaSix ha deciso di spostare i propri interessi verso altri capitolati.
